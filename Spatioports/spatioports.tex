\documentclass{article}
\usepackage[french]{babel}
\usepackage[T1]{fontenc}
\usepackage[left=2cm,right=2cm,top=2cm,bottom=2cm]{geometry}
\usepackage{fancyhdr}
\pagestyle{fancy}
\chead{Spatioports}
\usepackage{multicol}

\usepackage{fontspec}
\defaultfontfeatures{Ligatures=TeX}
%\setmainfont[Mapping=tex-text]{Sitka Display}
\usepackage[small,sf,bf]{titlesec}

\usepackage{graphicx}
\usepackage{xcolor}
\usepackage{sectsty}

\definecolor{DarkGreen}{HTML}{384d3e}
\definecolor{PureWhite}{HTML}{FFFFFF}
\definecolor{DarkRed}{HTML}{6e272d}
\definecolor{DarkGold}{HTML}{a48e3b}

\sectionfont{\color{DarkGreen}}
\subsectionfont{\color{DarkRed}}
\subsubsectionfont{\color{DarkGold}}

\begin{document}

\title{\vspace{-0.5cm}{\Huge Spatioports} \vspace{-1cm}}

\date{}

\maketitle

\section*{Procédures d'atterrissage et de décollage}
\subsection*{Fréquence d'Adressage aux Vaisseaux}
La FAV est une fréquence radio standard sur laquelle le contrôle spatial fournit diverses informations locales comme : profil planétaire de base, capacité d'accueil actuelle du spatioport, fréquences du contrôle spatial, couloirs d'approches, formalités spécifiques, convois, incidents et conditions astrographiques pouvant perturber le trafic, etc\ldots La FAV est mise à jour au minimum une fois par jour ou dès que cela s'avère nécessaire. Il ne s'agit pas d'une fréquence de communication et elle fonctionne en sens unique donc inutile de tenter de communiquer avec un processeur de base qui débite un message préenregistré à tous les vaisseaux entrant dans l'espace local. La loi punit très sévèrement les petits malins qui tentent de balancer des messages personnels sur la FAV ou de la brouiller. Il ne s'agit pas non plus d'une fréquence d'urgence.

\subsection*{Contrôle Spatial -- Atterrissage}
Si vous ne connaissez pas leur fréquence locale, la FAV vous l'indiquera. Pensez à toujours vous caler dessus lorsque vous approchez à moins de 50 unités spatiales de votre destination.\\

Lorsque le contrôle spatial vous appelle, le minimum est de décliner votre nom (ou celui de votre capitaine si vous n'êtes que le pilote) et le nom de votre navire. Votre transpondeur de bord fournit de lui-même l'immatriculation du navire et les renseignements que vous venez de communiquer verbalement mais c'est une vieille tradition.\\

Si le contrôle local est laxiste ou peu soupçonneux, il ne vous reste plus qu'à suivre leurs instructions pour l'atterrissage. Ne déviez pas de votre trajectoire prévue si vous voulez éviter d'attirer l'attention. Sur les mondes sous loi martiale, avec un trafic spatial intense ou des conditions climatiques dangereuses, les risques ne se limitent pas à une amende et un sermon du contrôle spatial\ldots \\

Le contrôle spatial a toute latitude pour vous interroger avant de vous autoriser à atterrir. Les questions typiques concernent votre précédente escale, votre cargaison, la présence ou non à bord de passagers à la biologie incompatible avec les conditions locales etc\ldots Il est courant de procéder à une vérification de votre signature transpondeur avec les bases de données locales et le contrôle spatial peut vous obliger à subir une inspection des douanes, des services sanitaires ou des forces de police locales avant de vous poser ou dès l'atterrissage.\\

Pratiquement tous les mondes civilisés n'autorisent pas les atterrissages en dehors des spatioports officiels. Si vous possédez un yacht, que votre port d'attache est la planète sur laquelle vous comptez atterrir et que vous possédez une piste privée officielle, vous êtes autorisé à y atterrir mais devez quand même suivre les instructions du contrôle spatial pour l'approche et les éventuelles inspections. Idem si votre navire est affilié à une corporation possédant ses propres installations portuaires, sauf si elle a conclu des arrangements spécifiques avec le gouvernement local.\\

Il peut vous être demandé par le contrôle de laisser les commandes à un système de guidage automatique pour votre atterrissage et un refus de votre part entrainera généralement un refus de vous laisser atterrir de la leur\ldots \\

Si vous arrivez dans un vaisseau sur le point d'exploser ou qui est en train de partir en morceaux sous vos bottes, le contrôle vous indiquera les procédures d'atterrissage d'urgence.

\subsection*{Contrôle Spatial -- Décollage}
Si vous n'êtes pas poursuivi par les autorités alors que vous comptez partir, la procédure veut que vous demandiez l'autorisation de décoller avant de quitter le sol. Si elle vous est accordée, vous devez emprunter le couloir aérien qui vous est affecté ou laisser le système de guidage automatique vous prendre en charge jusqu'à l'altitude ou vous serez autorisé à reprendre les commandes. Sauf ordre du contrôle, urgence technique ou médicale, vous n'êtes pas autorisé à sortir de votre vecteur de départ ou à demeurer dans la zone d'approche (50 unités spatiales) du monde ou de la station que vous venez de quitter. Dans tous les cas d'urgence, vous désobéissez aux consignes du contrôle spatial à vos risques et périls\ldots

\subsection*{Taxes Portuaires}
Si vous avez affaire à un spatioport de classe Stellaire ou Impériale, elles sont automatiquement déduites d'un de vos comptes en banque que vous devez obligatoirement fournir au contrôle durant les procédures d'arrivée et avant d'obtenir l'autorisation de vous poser. Ce type de spatioport est en effet présent sur des mondes ou les principales banques interstellaires ont des succursales et la plupart des capitaines sensés disposent d'au moins un compte en banque pour les frais divers dans un établissement dont l'enseigne est répandue dans toute la galaxie.\\

Sur les spatioports de moindre importance, les taxes portuaires sont prélevées à l'atterrissage par un représentant des autorités locales. Il faut alors préciser combien de jours l'on compte rester sur la planète et payer d'avance.\\

Notez que sur les mondes contrôlés par des gouvernements peu scrupuleux ou des cartels du crime, les taxes portuaires peuvent s'avérer très très abusives mais qu'il est impossible d'y échapper sans se retrouver avec un délit d'évasion fiscale sur le dos au minimum, voire pire pour les mondes ou la seule loi appliquée est celle du blaster.

\subsection*{Taxe de Stationnement}
Elle est payable à la journée (une fraction de journée compte pour une journée complète). Elle peut aller de 10 à 150 crédits par jour selon l'importance du port et la fiscalité locale, voire plus. Sur quelques mondes dépourvus d'industrie hôtelière destinée à une clientèle de spationautes désargentés, on ne voit aucune objection à ce que vous dormiez à bord de votre vaisseau. Pareil pour les endroits où l’on n’aime guère les étrangers. Vous trouverez sur ce type de monde tous les commerces et débits de boisson nécessaires dans la zone portuaire, à des prix prohibitifs, et serez probablement accueillis à l'extérieur à coups de cailloux.\\

La plupart des mondes ont cependant mis au point diverses astuces afin d'encourager l'équipage à profiter de l'hébergement local et à enrichir les petits commerçants. A titre d'exemple, on trouve des taxes d'entrée et de sortie du spatioport (10 à 30 crédits par passage et par personne, tous les commerces et tous les bars étant bien évidemment de l'autre côté du portail\ldots), une majoration forfaitaire de la taxe portuaire (de 50 à 300 \%) pour "nuisances inhérentes à un hébergement sur site", des forfaits spéciaux repas/boisson/lit dans certains établissements à proximité, des rabais auprès des mécaniciens du port sur recommandation de l'hôtelier, etc\ldots évidemment, les tenanciers du coin accordent des remises collectives très intéressantes aux équipages qui ont la gentillesse de débarquer en troupe pour vider leur bar et occuper ensuite toutes leurs chambres.\\

A l'opposé, les ports de la Bordure dépourvus de service de sécurité adéquat tolèrent sans peine que l'on puisse laisser un homme de quart à bord pour éviter que le vaisseau tombe entre de mauvaises mains. Il y a même des ports particulièrement mal fréquentés ou l'on majore la taxe de stationnement si vous ne prenez pas cette précaution\ldots

\subsection*{Taxe de Maintenance}
Elle est facultative mais il faut la payer d'avance pour en bénéficier. Tous les spatioports ne sont pas capables de fournir cette prestation. Cette taxe inclut les prestations suivantes : 

\begin{itemize}
	\item Vérification de base de l'extérieur du navire : présence de myosis sur la coque, fissures ou rayonnements suspects etc\ldots (Sauf si un contrôle sanitaire vous a été imposé). Les frais de réparation éventuels restent à votre charge mais pas le diagnostic.
	\item Remise à niveau de l'énergie et des consommables : dans la limite des stocks disponibles et jusqu'à concurrence de votre capacité maximale. Vous spécifiez le type de consommables souhaités (en fonction des espèces accueillies à bord) et on vous livre le tout dans les 24 heures. Si cela n'est pas possible, le prix à payer sera calculé au prorata de la livraison et la différence vous sera remboursée.
\end{itemize}

Calcul de la taxe de maintenance : \\

Taxe de Base (varie de 10 à 50 Cr selon les ports) $\times$ nombre total de personnes à bord $\times$ nombre de journées de voyage ($=$ de consommables utilisées)\\

Si par exemple le Millenium Falcon arrive sur Corellia après douze jours sans escale et seulement son équipage à bord, la remise à niveau sera de 10 crédits (taxe de base sur Corellia) $\times$ 2 (personnes embarquées) $\times$ 12 (jours de voyage depuis le dernier "plein") $=$ 240 crédits.\\

Il s'agit bien évidemment de consommables de base : air recyclé, filtres standard, nourriture synthétique assimilable par de nombreuses espèces etc\ldots si vous voulez quelque chose d'un peu moins spartiate et fonctionnel, doublez le prix de la taxe de base. Si enfin vous ne voulez que des articles luxueux ou haut de gamme, quadruplez la taxe de base.\\

La Taxe de Base représente la fiscalité locale, la politique commerciale des autorités, l'isolement relatif du port mais aussi la facilité à obtenir en grandes quantités les consommables susceptibles d'être achetés par des équipages provenant de centaines de mondes différents. \\

La remise à niveau de vos réserves d'énergie est incluse dans la Taxe de Maintenance. Si vous possédez un système spécial (voir Ajouts) vous permettant de refaire le plein d'énergie vous-même, réduisez la valeur totale de la Taxe de Maintenance de moitié puisque seuls les consommables vous seront facturés.\\

Par contre, les navires de taille inférieure à la classe Corvette ne disposent pas normalement de systèmes permettant la fabrication ou le réapprovisionnement en consommables (ateliers à filtre, cultures hydroponiques, synthétiseurs d'eau et d'air\ldots) et ce genre de système requiert énormément de place sur les navires capable de les accueillir et de les faire fonctionner.

\section*{Classes de Spatioport}
Les informations qui suivent sont des généralités et considérer que tous les spatioports entrent dans cette classification sommaire est à la fois inexact et insuffisant. Les évènements et bouleversements locaux peuvent également réserver bien des surprises.

\subsection*{Champ d'atterrissage}
\begin{itemize}
	\item Contrôle spatial : balise
	\item Inspection : pardon ?
	\item Antenne du BoSS : mais oui\ldots
	\item Taxe de stationnement : à la discrétion du propriétaire, le plus souvent dans les 5 à 10 Cr
	\item Taxe de Maintenance : pratiquement jamais
	\item Hébergement : au village le plus proche, s'il existe
	\item Pièces détachées : pratiquement jamais et rarement ce qu'il vous faut
	\item Ateliers de réparation : retroussez vos manches, la boite à outils est sous votre fauteuil. \\
\end{itemize}

Dans le meilleur des cas, ce type "d'installation" se limite à une piste de ciment. Le reste du temps, il faut se contenter d'une surface relativement plane pour atterrir. Il n'y a pas de contrôle spatial, ni de douanes et vous aurez de la chance s'il existe ne serait-ce qu'un hangar dans lequel vous pourrez travailler sur votre vaisseau à l'abri des intempéries. Si par miracle il existe un prestataire local assurant la remise à niveau de votre énergie et vos consommables, ses tarifs seront sans doute intéressants mais il ne faudra pas trop attendre de lui.

\subsection*{Services limités}
\begin{itemize}
	\item Contrôle spatial : balise et tour de contrôle
	\item Inspection : aléatoire selon les mœurs locales, souvent aucune
	\item Antenne du BoSS : rare
	\item Taxe de stationnement : 10 à 30 Cr le plus souvent
	\item Taxe de Maintenance : très rare, souvent 40 à 50 Cr de base et les stocks sont limités
	\item Hébergement : ça peut se faire
	\item Pièces détachées : ça dépend de votre navire et de ceux qui passent par-là d'habitude
	\item Ateliers de réparation : oui, mais il faudra probablement vous y mettre aussi \\
\end{itemize}

Ce type d'endroit ressemble déjà un peu plus à un spatioport : il y a une tour de contrôle des hangars et des entrepôts à louer etc\ldots malheureusement, les places sont peu nombreuses et le plus souvent occupés par les navires des habitués. De même, les rares fournitures disponibles sur place sont le plus souvent réservées ou monopolisées par les capitaines qui font régulièrement escale dans ce port. L'hébergement et diverses autres attractions pas trop lamentables peuvent se trouver sur place mais le plus souvent, il faudra parcourir quelques kilomètres dans la nature avant d'y parvenir. L'équipe assurant le service du port n'est pas assez nombreuse pour couvrir la totalité du quadrant et il est rare de trouver autre chose qu'un contrôleur et un vigile à moitié endormis lorsqu'on arrive au beau milieu de la nuit. La planète dispose généralement de techniciens capables selon la condition standard d'effectuer des travaux pas trop inédits sur votre navire mais ils n'habitent pas forcément la porte à côté.

\subsection*{Classe Standard}
\begin{itemize}
	\item Contrôle spatial : oui
	\item Inspection : occasionnelle et le plus souvent de routine
	\item Antenne du BoSS : oui
	\item Taxe de stationnement : 10 à15 Cr en général
	\item Taxe de Maintenance : 30 à 70 Cr de base
	\item Hébergement : oui
	\item Pièces détachées : oui
	\item Ateliers de réparation : oui \\
\end{itemize}

Un spatioport de classe standard est homologué et cela se voit. Le service est assuré en permanence, il est rare de ne pas y trouver de quoi refaire le plein, un certain nombre de petites entreprises locales ou d'antennes corporatistes ont leurs bureaux à proximité et recherchent des navires pour des trajets ponctuels. Si vous avez de la chance, vous n'aurez pas à placer votre navire sur la liste d'attente pour d'éventuelles réparations\ldots sinon, comptez le double du temps nécessaire parce que leurs équipes seront sur un tas d'appareils à la fois. La qualité est rarement décevante par contre, sauf si vous déboulez en pleine bourre et manquez de la plus élémentaire politesse.

\subsection*{Classe Stellaire}
\begin{itemize}
	\item Contrôle spatial : oui
	\item Inspection : presque toujours une inspection de routine
	\item Antenne du BoSS : toujours
	\item Taxe de stationnement : 50 à 100 Cr
	\item Taxe de Maintenance : 20 à 25 Cr de base
	\item Hébergement : oui, pour toutes les bourses et tous les goûts
	\item Pièces détachées : oui, les principaux constructeurs ont même des stocks importants
	\item Ateliers de réparation : oui, fréquemment des chantiers de construction également \\
\end{itemize}

On trouve toujours ce genre de port sur les mondes situés au carrefour de plusieurs lignes commerciales et même sur certains mondes dont l'économie dépend principalement de l'exportation massive de denrées, minerais ou produits locaux. On peut y trouver pratiquement tous les services désirés avec une certaine variété au niveau des prix proposés. Se procurer des pièces ou louer les services de mécaniciens est en général très facile parce que plusieurs constructeurs navals ont des filiales à proximité ou participent directement à l'administration du port avec les autorités locales.

\subsection*{Classe Impériale}
\begin{itemize}
	\item Contrôle spatial : la totale
	\item Inspection : systématique et souvent assez fouillée
	\item Antenne du BoSS : toujours
	\item Taxe de stationnement : 90 à 150 Cr
	\item Taxe de Maintenance : 10 à 15 Cr
	\item Hébergement : on n'a que l'embarras du choix
	\item Pièces détachées : pareil
	\item Ateliers de réparation : chantiers navals de première importance, mécaniciens indépendants, guildes techniques etc\ldots \\
\end{itemize}

Autrefois nommés "spatioports de classe galactique", les ports de classe Impériale sont de véritables villes vouées à l'accueil des navires venus d'autres mondes. Il y a des capitaines qui ont passé toute leur vie à faire escale dans des ports de ce type sans jamais avoir mis un pied en dehors de leur enceinte. La loi et l'ordre y sont assurés avec toute la méticulosité et la rigueur nécessaire. Pratiquement toutes les administrations et les banques ayant la moindre importance y sont représentées. Même les appareils les plus exotiques ou les modèles uniques pourront être réparés par un des innombrables ateliers du port. 


\end{document}
