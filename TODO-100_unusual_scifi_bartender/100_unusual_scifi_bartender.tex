\documentclass{article}
\usepackage[french]{babel}
\usepackage[T1]{fontenc}
\usepackage[left=2cm,right=2cm,top=2cm,bottom=2cm]{geometry}
\usepackage{fancyhdr}
\pagestyle{fancy}
\chead{100 barman insolites de science-fiction}
\usepackage{multicol}

\usepackage{fontspec}
\defaultfontfeatures{Ligatures=TeX}
%\setmainfont[Mapping=tex-text]{Sitka Display}
\usepackage[small,sf,bf]{titlesec}

\usepackage{graphicx}
\usepackage{xcolor}
\usepackage{sectsty}

\definecolor{DarkGreen}{HTML}{384d3e}
\definecolor{PureWhite}{HTML}{FFFFFF}
\definecolor{DarkRed}{HTML}{6e272d}
\definecolor{DarkGold}{HTML}{a48e3b}

\sectionfont{\color{DarkGreen}}
\subsectionfont{\color{DarkRed}}
\subsubsectionfont{\color{DarkGold}}

\begin{document}

\title{\vspace{-0.5cm}{\Huge 100 barman insolites de science-fiction} \vspace{-1cm}}

\date{}

\maketitle

\begin{enumerate}
	\item Un gigantesque mille-pattes s’enroulant le long du bar, plusieurs sections de son corps sont au service de différents clients.
	\item Une masse de petits robots non coordonnés tentent de servir des boissons aux clients. Ils renversent des boissons, brisent des verres et font généralement un mauvais travail de service, mais réussissent d'une manière ou d'une autre grâce à la force du nombre.
	\item Un voyageur dans le temps qui obtient les boissons en entrant dans un petit portail à l'arrière du bar et en commandant les boissons dans d'autres bars à des heures et des dimensions différentes en tant que faux visiteur. Il peut vous offrir n'importe quel verre pour le double du prix d'origine. Cela pourrait aussi altérer la réalité de façon permanente à cause du paradoxe du voyage dans le temps.
	\item Un grand ordinateur relié à un distributeur automatique de boissons, mécanisme de collecte d'argent qui sert joyeusement les boissons des clients tout en émettant des bips incessants. Lumières clignotantes incluses.
	\item Skliff - Amiboïde de six pieds de haut avec une légère teinte violette. Skliff a la fâcheuse habitude d'absorber accidentellement les boissons qu'il sert. Bien qu'il ne puisse pas parler, Skliff a la réputation d'être un grand auditeur et plusieurs habitués insistent pour dire qu'il leur a donné d'excellents conseils. Quelqu’un a même affirmé que Skliff a sauvé son mariage.
	\item Graelg - Une énorme créature, un peu comme une baudroie terrestre. L’appât de Graelg est un humanoïde voluptueux. Graelg entendit le tumulte du bar et utilisa les égouts pour se déplacer en dessous et placer son appât. Le leurre se déplace derrière le bar et a suffisamment de liens avec le cerveau du Graelg pour faire la conversation et recevoir des ordres, malheureusement pas assez pour tout arranger. C'est bien pour le Graelg parce que les humanoïdes qui sont attirés par le leurre suffisamment pour ne pas se soucier des erreurs le suivront en bas jusqu'au sous-sol et à la bouche du Graelg.
	\item Terrence - D'une hauteur de 2m75, Terrence est un scarabée humanoïde de l'Exoplanète 648-A89. Il est couvert d'une armure lisse, bleue-noire, chitineuse, et de poils fins de la même couleur. Terrence peut servir quatre plateaux de boissons à la fois, et les autres serveurs lui demanderont de l'aider pour transporter un fût, un tonneau ou une caisse de bouteilles particulièrement lourde. Terrence agit aussi comme videur du bar, mais ceux qui travaillent avec lui savent que, malgré ses mandibules intimidantes et son exosquelette, Terrence est un amour.
	\item Thompson n'est pas le propriétaire du bar et n’y travaille pas. Il était client ici il y a longtemps, et un jour, il a décidé d'aider en ramenant un verre vide d'une table voisine au bar. En chemin, quelqu'un lui a commandé un verre et il s'est dit : "Je crois que je pourrais lui apporter... " et alors qu'il revenait pour servir ce client, quelqu'un d'autre a demandé une commande de rondelles d'oignons. Thompson a essayé de trouver le personnel de service, le personnel de cuisine et le propriétaire, mais en vain. Depuis, il est au service des clients. Il garde tous les pourboires qu'il gagne, mais il garde un œil attentif sur les livres de comptes, et garde tout l'or qu'il reçoit des boissons et de la nourriture dans des barils dans la cave pour quand les propriétaires reviendront éventuellement, s’ils reviennent. Il se dit : "Si j'avais une taverne et que j'étais indisposé, j'espère qu'une bonne âme me couvrirait pendant un certain temps" et il travaille donc ici depuis 12 ans. Il ne prend jamais de salaire et ne vit que de ses pourboires. Il garde le décor de la taverne tel qu'il était le jour où il s'est arrêté pour prendre un verre il y a 12 ans. C'est un vieil homme gentil, et personne ne sait vraiment qu'il n'en est pas le propriétaire. Il ne corrige pas ceux qui le supposent, de peur que quelqu'un ne profite de la taverne. Il nettoie tous les soirs tout seul, verrouille la porte, rentre chez lui, et le matin est le premier de retour en s'assurant que les verres sont nettoyés, que les planchers sont balayés, que la bière est froide et que la taverne Navet et Figue fonctionne bien, jour après jour, jusqu'au retour de ses propriétaires.
	\item Un barman qui devient très offensé par la couleur jaune.
	\item James est une brique gris métallisé de 2m15 de haut et presque 1m de largeur. Il est le barman pour le bar Incognito dans la septième bordure, quatrième rayon. Le bar lui-même n'a rien d'inhabituel, avec ses bières de qualité mauvaise à moyenne et ses cocktails peu ingénieux. La présence scénique de James, cependant, fait que le bar est bondé tous les soirs. Ceux qui espèrent boire un verre peuvent s'exclamer : "James ! James ! James !" et si suffisamment de personnes le font simultanément, les lumières du bar clignoteront et s'éteindront un moment. Lorsque les lumières se rallument, les chanteurs trouveront la boisson qu'ils désirent dans un verre devant eux. Essayer de garder une lampe de poche sur James empêchera le tour de fonctionner, et les clients irritables sont prompts à décourager les gens qui essaient de découvrir le secret. Les prix sont clairement marqués et les paiements se font sur l'honneur, car James est le seul employé. Ceux qui n’honorent pas leur facture, cependant, se réveillent souvent avec une visite à domicile de James lui-même qui cherche à récupérer son dû.
	\item Un humain complètement générique qui a très peur et est confus à cause de tous les clients bizarres que son bar reçoit.
	\item Une pieuvre dans un imperméable. Personne ne sait pourquoi elle est là, mais peut-être que le manager pensait que c'était un psychopathe ou une sorte d'extraterrestre. Il ne comprend aucune langue et ne sert qu'une bière générique... dans un verre gluant.
	\item Un chien, c'est un bon garçon. Il va chercher n'importe quelle boisson que vous demandez et les gens le payent par respect. Il possède et dirige l'établissement lui-même.
	\item Gabbo ressemble à un être humain, excepté pour sa peau vert métallique et sa bouche exceptionnellement large. Il n'est pas non plus techniquement masculin, mais il insiste sur le fait que c'est "assez proche". Il a un vice pour les petits paris, et a donné beaucoup trop de boissons gratuites à cause de cela. Son patron le virerait, mais les habitués l'adorent !
	\item Lin-Sabah : Elle est lépidoptériste à temps partiel (personne qui étudie ou collectionne les papillons) et a partiellement modifié génétiquement son corps pour communiquer avec ses animaux domestiques. Elle possède un bar pour financer ses recherches, et ses insectes prennent les commandes et livrent les boissons quand elle est occupée avec d'autres clients. Les papillons sont pour le matin et les papillons de nuit pour la nuit. Ses yeux sont nuageux et elle a de petites ailes et des antennes. Lin laisse ses animaux de compagnie essayer des tâches pour tester leurs limites, mais elle est très sur la défensive par rapport aux clients qui blessent accidentellement ou non ses animaux de compagnie. Si un client en tue un ou fait du grabuge, elle les presse jusqu'à ce qu'ils partent. Elle parle constamment à ses animaux de compagnie, donc elle parle presque toujours à quelqu'un à un moment donné.
	\item Sinhestara (Sin en abrégé) est une grande femme verte avec une paire d'antennes sur la tête et trois seins qu'elle affiche avec des pastilles rose vif et éclatantes. Sa moitié inférieure est longue, ressemblant à un serpent, et nue. En fait, les seuls vêtements qu'elle porte vraiment sont ses bracelets en cuir standard autour de ses quatre poignets. C'est une experte multitâche avec ses quatre bras. Ses yeux sont sombres et nébuleux, sans pupille distincte. Ses cheveux noirs sont maintenus dans une queue de cheval sauvage, à l'exception de sa frange de plumes. Celle-ci émet une lueur rouge à la lumière. Elle allongeait ses S quand elle parle. Elle-même n'a pas la langue fourchue, c'est juste une habitude qu'elle a prise dans sa langue maternelle ou son dialecte. Elle est confiante et ouverte.
	\item Un arthropode octopédale de la taille d'un enfant, qui crie continuellement des insultes à ses clients. Ses insultes font souvent référence à l'anatomie de son espèce, si bien qu'il est très rare que les sentiments de quelqu'un soient blessés. Des boissons décentes.
	\item Le "barman" est un concierge, qui dirigera les clients vers les pods et collectera les paiements. Le vrai bar est en réalité virtuelle accessible par les pods. Le pod modifiera la composition chimique de son corps (ou exécutera un script de modification de comportement temporaire pour tout client robotique) et l'intoxiquera lorsqu'il boit des boissons en RV. Reroll pour le barman du pub VR. Le goût et la force perçus des boissons varient en fonction du coût.
	\item Celui qu'on appelle le barman est une anémone féloorienne des mers de Telvar-1c. Il s'agit d'un monticule de chair gris-bleu de quatre mètres de haut, hérissé de plus de trois douzaines de tentacules et noyé dans un grand anneau de pierre de mer légèrement poreuse, qui sert de bar. Tous les 17 ans, une nouvelle pierre doit être expédiée au bar, et le barman y déménage, car il consomme lentement les minéraux qu'elle contient pour se nourrir. Sept de ses tentacules se terminent par de grands yeux avec des pupilles en forme de 'Y', chacune d'une couleur différente. Elle a appris au fil des siècles à contrôler individuellement les multiples membranes nictitantes que possède chaque œil, ce qui lui permet une certaine interaction avec ses clients, au-delà de la simple fabrication de boissons. Les autres tentacules se ramifient en une myriade de minuscules bras, qu'il utilise pour faire son commerce, en utilisant un grand anneau suspendu au plafond du bar, contenant un nombre impressionnant de bouteilles fixées la tête en bas. L’anneau contient également un système de brumisation pour garder la peau humide et hydratée. Bien qu'il ne puisse pas parler, il peut entendre et reconnaît toujours les clients d'un clin d'œil amical lorsqu'il prépare les boissons qu'ils demandent. Si l'on examinait de près la sélection des ingrédients, on trouverait quelques bizarreries. Un flacon bleu à faible luminescence marqué "nostalgie", un flacon vert dont le contenu semble bouillonner en permanence et dont l'étiquette n'est en aucune langue connue. Il y en a un avec un liquide qui semble rester immobile sous la forme d'une vague déferlante, étiquetée "joyeux retour à la maison dans les derniers jours de l'été". Un flacon noir de jais avec l'étiquette inquiétante de "Le dévoreur", et bien d'autres.
	\item Ni'Mora est un esprit de ruche très jeune. Ils ont quatre corps ; leur type de corps préféré est le bipède sensible. Ils travaillent ici à cause de leur aversion à forcer d'autres êtres sensibles à s'unir contre leur gré. Ni'Mora est l'esprit enfantin de K'naxx, un esprit de ruche qui traverse sept planètes. Ni'Mora est toujours à la recherche de nouveaux corps, et est une équipe chirurgicale bon marché, ils peuvent guérir même la mort jusqu'à 24 heures après quand ils ont la permission d'assimiler un être.
	\item Un robot humain qui prépare des boissons en consommant chaque ingrédient séparément, puis enlève un verre de l'intérieur de son abdomen qui contient la boisson désirée faite parfaitement. À l'occasion, un petit morceau de métal, comme une rondelle ou un boulon, peut se trouver dans la vitre, au grand embarras du robot.
	\item Vous-désirez-quelque-chose est un automate mécanique avec un bras distributeur de boissons gazeuses/alcoolisées. Théoriquement, il est programmé pour demander : "Est-ce que vous désirez quelque chose ?". Un bogue dans le codage laisse tomber le "Est-ce que" et il indique simplement qu'il s'agit d'une commande. Le programmeur a été tué en jouant au poker sur Alpha-7, et il a utilisé un très bon mot de passe.
	\item Très excentrique. Ça a l'air humain. Organes, membres, et tout. Aucune caractéristique supplémentaire apparente. Il insiste pour qu'on l'appelle "Le Barman". Il y a une photo du Barman à l'extérieur de la Taverne, mais de temps en temps, lors des visites de retour, il a l'air différent. Toujours humain mais très distingué. Une ou deux fois, c'était en fait une FEMME. Tant de clients semblent vénérer ce barman. Vous entendez parfois son nom prononcé, par peur ou vénération, dans les coins les plus reculés du monde.
	\item Un petit fœtus portant un petit costume de barman fait sur mesure. Il flotte dans un globe de verre en lévitation, entouré d'orbes de verre plus petits qui orbitent autour de lui. Les boissons ne sont jamais commandées, mais si quelqu'un veut avoir un verre, on peut le lire sur le bar, avec le prix ajouté à sa note. Chaque boisson est fabriquée par télékinésie, les différents orbes se déplacent, saisissent les boissons qu'ils contiennent et les mélangent. Sa boisson signature est le "Jeu pour Enfant", une boisson multicolore aux paillettes tourbillonnantes, qui évoque un sentiment de sécurité et de satisfaction, généralement suivi par des souvenirs d'enfance heureux lorsque cela est possible.
	\item Le barman est un petit humanoïde d'une espèce méconnaissable, mais il semble très, très vieux. Il porte un peignoir terne et s'assoit solennellement sur une chaise derrière le bar, à côté d'une glacière. Pendant que vous commandez votre boisson, il la sort de la glacière avant que vous ayez fini de parler. C'est parfaitement fait. Il prépare immédiatement une autre boisson et la place dans la glacière, prête pour le prochain client.
	\item Le barman est un nuage dense de gaz nocif avec 5 orbes luminescents colorés flottant à l'intérieur. Le gaz est télépathique et parle dans votre esprit avec un accent ambigu, les orbes clignotant pendant qu'il "parle". Il matérialise votre boisson désirée à partir de rien et il est délicieux, mais le nuage a l'habitude d'éclater en morceaux.
	\item Le barman semble relativement normal, bien qu'il ait quelque chose de suspect. Lorsque vous prenez vos pièces de monnaie pour payer la boisson, vous vous rendez compte que le paiement requis est manquant. Il est impossible qu'il ait pu les atteindre.
	\item Ce robot barman a Tellement. De. Bras. Malheureusement, toutes ses boissons sont toxiques pour les humains.
	\item En atteignant le bar, il y a un nuage dense, qui brille et se déplace de façon irrégulière. Lorsque vous commandez une boisson, le nuage se condense près d'une tasse vide et une "pluie" de la boisson que vous avez commandée tombe dans la tasse. Les nappages et la coupe sont déplacés par une forte force du vent. En s'emparant de la boisson, le nuage passe devant votre oreille et des bruits aléatoires provenant de l'autre côté de la pièce sont apparemment portés par le vent pour former les mots "merci".
	\item Une créature limace à l'air négligé est assise derrière le bar. Les 12 boissons au menu sont celles dont vous n'avez jamais entendu parler. Sur commande, la limace soulève son corps et expose 12 seins. L'un des seins est placé dans un appareil qui le presse en faisant jaillir des liquides de différentes couleurs pour chacun d'eux. L'action a l'air dégoûtante, mais la boisson à l'air et un goût incroyable.
	\item Buford, un grizzli apprivoisé, n'a pas d'opinion bien arrêtée sur la politique locale. Il vous parlera de la façon dont les microbrasseurs de la 16e vague se sont tiré une balle dans le pied en essayant de faire pression contre les levures sensibles. Les blagues sur les paniers de pique-nique sont un excellent moyen de se faire bannir du bar à vie.
	\item Exodus est une intelligence artificielle et techno-prophète. Du moins, c'est ce qu'il prétend, et puisqu'il semble toujours savoir pourquoi vous êtes là, qui vous rencontrez et ce que vous aimeriez boire, c'est difficile de discuter. Son avatar (un homme aux yeux fous qui a tendance à dire 'Lo' et 'Unto' plus que quiconque) n'est jamais vu que dans le bar, mais il semble toujours savoir ce qui se passe dehors.
	\item Après que son visage ait été mutilé lors d'un déversement de produits chimiques sur Orléans VII, Edmée Jackson-Salazar a réussi à s'en sortir avec un dédommagement impressionnant qu'elle a consacré à deux choses : une façade holoprojection sur mesure qui couvre ses traits avec un simulacre de son ancien visage le plus crédible, et le bar où elle sert maintenant à boire. Edmée n'a aucun amour pour les entreprises qui sucent la vie de l'ouvrier, mais personne ne peut dire si les rumeurs sur son implication avec divers groupes de cette persuasion sont exactes ou non. Si vous vous sentez courageux, on dit que la tequila maison d'Edmée utilise de l'agave d'Orléans, une souche interdite pour ses propriétés psychotropes.\item The bar tender appears to be a normal human despite his purple skin tone. He seems like he is ignoring you to clean a glass whenever you approach him, but if you order a drink his visage flutters for a moment and the beverage is set in front of you, as he continues to work off the smudge on the glass that he just can't seem to get.
	\item \textbf{Suite à traduire}
	\item Un être stellaire se tenant derrière un écran protecteur, lequel émet les ingrédients nécessaires à la préparation de votre boisson, en effectuant la fusion de l’hydrogène qu’il consomme comme nourriture. Les éléments produits sont ensuite collectés et versés dans votre chope.
	\item Une douzaine de clones du même type, répartis tout au long de la journée, racontant les mêmes blagues et plaisanteries à tous les clients.
	\item Un monstre de pierre ramasse plusieurs gemmes et les réduit en une fine poudre qu’il applique sur le rebord d’une margarita sophistiquée. Elles luisent d’une légère iridescence.
	\item Une jolie extraterrestre dans un bar miteux déclare qu’elle aimerait « vérifier ton taux de midichloriens ». Quoi que cela veuille dire.
	\item Un robot doté de bras et de mains capables de se diviser à l’infini en appendices de plus en plus petits pour servir tout le monde en même temps.
	\item Poser la paume de votre main sur le bar et penser à votre commande fait apparaître votre boisson, qui sort d’une ouverture proche dans le comptoir.
	\item Un cerveau immobile dans un bocal lit vos pensées à votre approche, puis prépare votre boisson par télépathie et la fait glisser vers vous.
	\item Une paire de jumeaux ajoutant chacun un ingrédient à votre cocktail, en alternance.
	\item Une puce intégrée dans la table capte les commandes de votre groupe, et un plateau flottant apporte les verres.
	\item Un robot monotone mais très sarcastique qui se moque de tout ce que les clients commandent.
	\item Pas de barman. Tout est automatisé.
	\item Un extraterrestre qui ne comprend ce que vous dites que si vous parlez en rimes.
	\item Un homme des années 1950. Il n’est pas sûr de savoir comment il est arrivé là.
	\item Un être gélatineux qui sécrète chaque boisson depuis un réservoir interne différent.
	\item Une paire de tentacules surgissant d’un portail. Vous ne voyez jamais ce qu’il y a de l’autre côté.
	\item Un arbre qui fait pousser chaque boisson. Vous cueillez vos verres comme des fruits. Ses branches recouvrent tout le plafond.
	\item Une équipe de minuscules extraterrestres de moins de 13 centimètres de haut.
	\item Un humain avec un tatouage en alphabet extraterrestre le long de son avant-bras, qu’il affirme vouloir dire « Incassable ». Toute personne capable de lire cette langue y voit en réalité : « Serveurs de traduction actuellement hors ligne, veuillez réessayer plus tard ».
	\item Le « bar » est en réalité une bande de terre surélevée. Lorsqu’une boisson est commandée, les plantes nécessaires à sa fabrication (orge pour la bière, raisin pour le vin, etc.) poussent en accéléré, puis sont rapidement récoltées et transformées par des machines. L’ensemble du processus dure environ cinq minutes.
	\item Autrefois un sbire anonyme servant un empire maléfique récemment renversé ; désormais un raté amer et généralement ivre. Il porte encore les lambeaux de son ancien uniforme. Il servira à contrecœur ce que vous demandez, dans la limite de son (très restreint) stock, mais s’il vous reconnaît comme quelqu’un ayant participé à la chute de son ancien employeur, il pourrait bien sortir son arme contre vous.
	\item Un agrégat non-sentient de nanomachines, formant un prisme rectangulaire d’environ deux mètres. L’agrégat reconnaît les commandes de boissons et peut reconfigurer n’importe quelle matière en un liquide correspondant à la commande passée. Veillez simplement à ne rien laisser d’important à portée de main, sinon vous pourriez bien finir par boire votre pistolet sous la forme d’un shooter de whisky.
	\item Skudge : une créature semblable à un gremlin, haute d’environ trente centimètres, qui se déplace dans le bar sur une plateforme tirée çà et là par un système étonnamment complexe de poulies et de leviers. Skudge est un génie mécanique et peut probablement réparer n’importe quel gadget endommagé que vous lui donnez, contre paiement. Mais n’espérez pas un résultat « comme neuf »…
	\item Une énorme sphère de verre lévite dans les airs, remplie d’un tourbillon coloré de gaz. Lorsqu’une commande est passée, certains gaz commencent à tomber comme de la pluie à l’intérieur de la sphère et s’accumulent au fond, qui se gonfle et forme alors une gigantesque goutte de pluie de la taille d’une pinte, laquelle tombe dans le verre en attente.
	\item Zenith ! – Il a un corps humanoïde en métal avec pour tête un haut-parleur des années 1940 (du genre avec la façade en grillage). Il « voit » grâce à l’antenne radio qui dépasse de son crâne et les clients ont remarqué qu’il connaît par cœur la disposition du bar ; d’où les rumeurs selon lesquelles il serait en réalité aveugle.
	\item Un barman en forme de machine à pince, et vous devez payer et jouer pour obtenir vos boissons. Mais cela reste à l’intérieur d’un bar.
	\item Une masse de tentacules noires surgissant du plancher pour collecter les boissons, prendre et livrer les commandes, nettoyer les tables et expulser les clients turbulents. Ils communiquent via un tableau noir.
	\item Un humanoïde métamorphe télépathe qui fouille les souvenirs des gens afin d’adopter un visage avec lequel le client se sentirait à l’aise. Si les choses deviennent trop intenses, ses traits commencent à se mélanger car il ne peut pas changer assez vite. Si vous restez au bar après la fermeture, vous verrez sa véritable forme : une personne androgyne extrêmement pâle, émaciée, sans la moindre caractéristique distincte — si ce n’est précisément son absence de traits.
	\item Plusieurs Roombas futuristes qui foncent dans les pieds des gens puis jurent en utilisant des symboles holographiques façon Q-Bert. Il n’y a pas vraiment de « barman », mais chaque table dispose d’un écran tactile qui est garanti de dysfonctionner d’une manière ou d’une autre.
	\item Les bouteilles sont toutes animées et possèdent un emplacement où vous pouvez scanner une puce de crédit. Elles vous racolent pour consommer, de la même manière que des danseurs ou danseuses dans un club de strip-tease. Chaque bouteille est programmée avec sa propre personnalité, et les habitués choisissent leur boisson préférée en fonction de cette personnalité plutôt que du goût.
	\item Une créature si étrange, si extraterrestre, que pour préserver la santé mentale de ses clients, elle pose un tableau baroque représentant un roi sur son « visage », les yeux découpés, afin de la rendre plus accessible.
	\item Un homme (?) très poilu, dont de longues vagues rouges couvrent presque tout le corps. La seule chose visible est un long nez proéminent et des bras noueux et ridés. Lorsqu’il prépare une boisson, il rapproche le verre de lui, derrière la couche de poils, avec tous les robinets, ingrédients et outils nécessaires. En un temps étonnamment court, il vous tend le verre. Étonnamment, la boisson est parfaite. Plus surprenant encore, pas un seul poil ne tombe dedans.
	\item Le barman (homme), un client du bar (homme) et une mère enceinte (femme) sont en réalité la même personne à différents stades de sa vie de voyageur temporel.
	\item L’alcool sentient – Vous criez ce que vous voulez, et les bouteilles appropriées commencent à trembler. Un mince filet de liquide s’élève, serpentant, hors de la bouteille et se tord pour aller dans un verre. La rumeur dit que quiconque ne paie pas écope d’une énorme gueule de bois qui dure jusqu’à ce que la dette soit réglée.
	\item Le barman porte un débardeur lâche qui laisse voir une bonne partie de sa peau. Vous remarquez que ses tatouages ne sont pas de simples encres : ils bougent. L’étrave du navire pirate bondit sur les vagues, l’écume éclaboussant la figure de proue en sirène. « Maman », décédée en 1993, serre un peu plus fort son enfant dans ses bras et lui dégage une mèche des yeux. Le méchant en robe noire tranche la main du héros avec un sabre-laser rouge flamboyant, avant que le héros ne trébuche hors de la plateforme.
	\item Personne ne connaît son nom, mais il semble ne servir que de l’eau, et fusille tout le monde du regard jusqu’à ce qu’ils la boivent. La seule raison pour laquelle il reste en activité est que le seul autre bar dans un rayon de 100 années-lumière ne sert que des sels métalliques mélangés à divers isotopes radioactifs.
	\item Krema Soada, une magnifique femme extraterrestre qui adore chanter et danser en préparant les boissons. Elle ne demande jamais ce que vous voulez mais, à chaque fois, le verre qu’elle envoie plaît au client. Elle adore parler d’elle, tant que vous ne lui demandez pas pourquoi elle est devenue un homme maintenant ?
	\item Un barman humain qui a survécu à une malheureuse rencontre avec des radiations alors qu’il servait dans la force galactique. Bien qu’il ait survécu indemne, sans séquelles visibles, les dommages causés à son ADN furent tels qu’il n’est plus légalement classé comme humain, et il a donc été déchu de sa place dans la force ainsi que de sa citoyenneté. À cause de cela, il a passé la dernière décennie en errance, travaillant à des petits boulots (comme ici, dans ce bar), et il se montre ouvertement hostile envers les autres « humains légaux ».
	\item Six-Doigts : un homme sans orbites oculaires, doté d’un large front qui s’étend de la racine des cheveux jusqu’au milieu du nez (d’où son nom : six doigts de front). Nul ne sait comment, mais il prépare toujours des boissons uniques pour ses clients, sans jamais répéter la même combinaison.
	\item Un humain avec un seul bras robotique tient le bar. Quelques minutes après votre arrivée, son service se termine et il est remplacé par un robot doté d’un seul bras humain.
	\item Une série de barmans qui imitent tous le tout premier barman ayant ouvert l’endroit, il y a des décennies.
	\item Une série de distributeurs automatiques de sodas qui vous permettent de payer pour un buffet d’alcools illimité, provenant de différentes planètes et royaumes. Une excellente affaire si vous parvenez à en profiter dans le (court) laps de temps prévu.
	\item Une station de rafraîchissements, semblable à celles des parcs d’attractions, qui vaporise une substance alcoolisée (en brume) absorbée par la peau dès que vous entrez dans le bar. À partir de là, il n’y a pas de barman, mais plusieurs videurs et un hôte qui vous explique que vous serez complètement ivre en moins de deux minutes.
	\item Un énorme slime capable de prendre n’importe quelle forme à volonté. Tout le bar est en réalité son corps, et il propose des boissons fabriquées à partir des molécules réarrangées de ses déchets. Il ne demande pas d’argent, mais se nourrit plutôt des bactéries de vos pieds. Quand vous marchez, vous sentez une légère succion, comme si vous traversiez de la boue épaisse.
	\item Plusieurs moustiques-robots injectent des substances illicites aux clients, selon leurs « besoins ». Un ingénieur, installé derrière un ordinateur à l’arrière, contrôle leurs actions et maintient la fête en vie grâce à une surveillance constante et des injections stratégiques.
	\item Le bar n’a aucun personnel ; tout est en libre-service. Vous vous servez directement aux pompes, et les verres comptabilisent ce que vous y versez pour vous facturer en conséquence. Les failles du système apparaissent lorsqu’un homme visiblement ivre trébuche jusqu’à la pompe et commence à se verser la bière directement dans la bouche.
	\item La Première Miséricorde – un ange archétypal, avec des ailes, des yeux doucement lumineux, et des arias célestes murmurant autour d’elle. Elle encourage tous ceux qui entrent à renoncer à l’alcool et essaie de promouvoir les boissons non alcoolisées. Néanmoins, le bar reste toujours bondé.
	\item Un clone de l’actuel Président de la Fédération Galactique. Adore parler politique.
	\item Chyrossis de Nyx – un cyclope de trois mètres de haut, vêtu d’un smoking. Il porte une massue dorée, marquée des mots « La Direction » en lettres flamboyantes, suspendue au-dessus du bar.
	\item Un casque est fourni à votre entrée dans le bar ; vous êtes « transporté » dans un speakeasy des années 1920, avec une serveuse grande gueule nommée Sweet Marie qui sert les verres. On ne sait pas s’il s’agit d’une création virtuelle élaborée ou d’un véritable voyage temporel. Les boissons sont toutes fidèles à l’époque.
	\item Un humain séduisant et charismatique, nommé Frankie Ghullet, tient le bar. Mais en observant attentivement, on s’aperçoit qu’il est en réalité enchaîné au comptoir par de lourds fers aux pieds, avec juste assez de chaîne pour lui permettre de se déplacer derrière le bar. Si on l’interroge, il répond généralement en plaisantant qu’il vaut mieux régler son addition.
	\item Un grand miroir se dresse au bout du bar. Chaque client s’y voit préparer lui-même la boisson commandée et se la servir. Il s’avère que les gens se montrent naturellement généreux en pourboires… lorsqu’ils se les donnent à eux-mêmes.
	\item Une sculpture de glace en train de fondre lentement, mais mouvante, sert les boissons pré-refroidies. L’eau de fonte s’écoule dans une grille ; une fois que la sculpture est trop fondue pour servir correctement, elle se glisse d’elle-même dans un congélateur et est remplacée par une nouvelle.
	\item Le Barman – Inexplicablement, personne n’est capable de décrire son apparence, ni même de se souvenir de lui. En fait, personne ne se rappelle non plus ce qu’il a bu dans ce bar. Et pourtant, ils ont bien consommé – leurs relevés de paiement en témoignent. Mais tout à propos de ce lieu reste… flou.
	\item Tom le Coléreux – le fantôme d’un pirate du XVIIIe siècle. Le bar est sculpté dans une partie du gaillard arrière que Tom arpentait autrefois. Après avoir hanté l’épave de son navire, puis le bar, pendant des siècles, il s’est lassé du hantement et s’est intéressé au métier de barman. Aujourd’hui, il est devenu aussi incontournable au bar que son vieux gaillard arrière.
	\item Le Gobbo Furtif – Un barman incroyablement discret, qui adore remplir les verres quand personne ne regarde. Il est considéré comme impoli de commander explicitement une boisson ; à la place, on s’assoit avec un verre vide et on commente simplement ce qu’on aime boire. Le verre se remplit peu après, accompagné du bruit de petits pas pressés et de ricanements malicieux. Le paiement se fait de la même façon.
	\item Un oiseau haut en couleur, semblable à un casoar, se tient au bar, la tête légèrement penchée. Il ne réagit pas immédiatement aux commandes, mais pousse un cri rauque après environ quatre secondes. Quelques instants plus tard, un oisillon sort de l’arrière-salle, roulant un œuf contenant la commande de boisson à l’intérieur.
	\item Gidzet ressemble à un nain de jardin, jusqu’au chapeau, et marmonne sans arrêt entre ses dents ; il alterne des phrases adressées aux clients avec son bavardage inaudible. Il semble connaître tout le monde par leur nom, et une oreille attentive reconnaîtra parfois celui de l’interlocuteur dans ses marmonnements – mais rien de plus.
	\item Les Sorcières Maudites d’Abason IV – Le bar est sombre et brumeux, et des bruits de créatures ailées résonnent dans l’obscurité au-dessus. Quatre vieilles femmes à la peau verte, ridées, gèrent l’endroit, remuant un immense chaudron noir au-dessus d’un feu violacé. Chaque commande est servie à partir de ce chaudron, mais correspond parfaitement à ce qui a été demandé. Les quatre sorcières gloussent en se traînant d’un client à l’autre, mais elles sont en réalité des interlocutrices agréables et des contacts utiles. Quiconque recueille suffisamment d’informations apprend un fait fascinant : quatorze de ces sorcières ont épousé des clients et quitté le bar. Une fois la cérémonie de mariage terminée, elles se sont transformées en magnifiques jeunes femmes ; les couples ont vécu heureux et confortablement, et chacun a eu cinq enfants destinés à devenir de grands héros. Après chaque mariage, une nouvelle sorcière est apparue, maintenant le nombre à quatre en permanence.
	\item Un lapin anthropomorphe plutôt miteux, nommé « Le Lièvre de Septembre », bien que les habitués l’appellent « Sep ». Il garde derrière le bar une photo de lui avec son cousin, plus célèbre. Toutes les boissons sont servies dans d’immenses tasses à thé.
	\item Une grande statue se dresse derrière le bar. En l’observant de près, on remarque qu’elle est couverte de minuscules échafaudages et de cordages ; de petites personnes courent dessus en tous sens, actionnant des leviers et criant des ordres dans une langue incompréhensible. La statue bouge par à-coups pour préparer et servir les commandes.
	\item Just Jack – Le barman est un humain mince et nerveux, aux yeux fuyants et sauvages. Il prend chaque commande avec un petit rire nerveux. Il prépare les boissons en plaçant soigneusement les verres, puis en perçant sauvagement les bouteilles avec un couteau papillon, le regard brillant et un sourire démoniaque aux lèvres. Les bouteilles se referment d’elles-mêmes après avoir libéré la quantité nécessaire. Personne n’a jamais oublié de payer dans ce bar.
	\item Une main humaine détachée, semblable à La Chose de La Famille Addams. Elle verse les boissons avec une grâce et une rapidité incroyables. Elle ne parle pas, mais comprend manifestement tout.
	\item Une gigantesque version en 3D d’un sprite 8-bits sert au bar. Elle semble se déplacer de façon saccadée et émet des sons rappelant une console Game Boy en versant les boissons.
	\item Un étrange extraterrestre ressemblant à une pieuvre occupe tout le plafond. Il sert les clients en faisant passer les boissons le long de ses nombreux tentacules.
	\item Un petit canon pulvérisateur est posé sur le bar. Lorsqu’une boisson est commandée, il calcule les vecteurs et propulse le liquide dans l’air directement dans le verre posé sur la table. Il affiche une précision de 99,986 %.
	\item Un DJ rétro-futuriste dans le style des années 1980 fait office de barman. Le mixeur à boissons est un gigantesque synthétiseur.


\end{enumerate}

\end{document}
