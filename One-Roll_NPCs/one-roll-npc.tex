\documentclass{article}
\usepackage[french]{babel}
\usepackage[T1]{fontenc}
\usepackage[left=2cm,right=2cm,top=2cm,bottom=2cm]{geometry}
\usepackage{fancyhdr}
\pagestyle{fancy}
\chead{One-Roll NPCs}
\usepackage{multicol}

\usepackage{fontspec}
\defaultfontfeatures{Ligatures=TeX}
\setmainfont[Mapping=tex-text]{Sitka Display}
\usepackage[small,sf,bf]{titlesec}

\usepackage{graphicx}
\usepackage{xcolor}
\usepackage{sectsty}

\definecolor{DarkGreen}{HTML}{384d3e}
\definecolor{PureWhite}{HTML}{FFFFFF}
\definecolor{DarkRed}{HTML}{6e272d}
\definecolor{DarkGold}{HTML}{a48e3b}

\sectionfont{\color{DarkGreen}}
\subsectionfont{\color{DarkRed}}
\subsubsectionfont{\color{DarkGold}}

\begin{document}

\title{\vspace{-1cm}{\Huge One-Roll NPCs} \vspace{-1cm}}

\date{}

\maketitle

\section*{Utilisation}
Utilisez ceci comme point de départ. Posez toujours des questions : Qui, quoi, quand, où, pourquoi. Extrapolez et embellissez vos réponses, et inspirez-vous de votre histoire. Comme toujours, si un lancer ne vous va pas, changez-le !


\begin{multicols}{2}
	\section*{D4 -- Disposition à votre égard}
	\begin{enumerate}
		\item S'oppose à vous
		\item Ne vous aime pas, mais vous aideras\ldots si vous y mettez le prix.
		\item Vous aime bien, mais il ne vous aidera pas gratuitement.
		\item Vous soutient
	\end{enumerate}
	\section*{D6 -- Alignement}
	\begin{enumerate}
		\item Bon
		\item Loyal
		\item Neutre
		\item Chaotique
		\item Mauvais
		\item Choisissez
	\end{enumerate}
	\section*{D8 -- Quelle classe ou profession (ensemble de compétences) ?}
	\begin{enumerate}
		\item Barde -- Animateur
		\item Ecclésiastique -- Acolyte
		\item Druide -- Professionnel (boulanger, forgeron, fermier, etc.)
		\item Combattant -- Soldat
		\item Paladin -- Gardien, tuteur
		\item Ranger -- Chasseur
		\item Voleur -- Criminel
		\item Sorcier -- Érudit
	\end{enumerate}
	\section*{D10 -- D'où vient-il/elle ?}
	\begin{enumerate}
		\item Où vous êtes en ce moment (local)
		\item Une région voisine
		\item Le même endroit d'où vient un PJ
		\item Un endroit lointain
		\item Une guilde locale
		\item Un lieu exotique
		\item Une grande île
		\item Une ville souterraine ou sous-marine
		\item Les ombres\ldots
		\item Relancez, et cet endroit n'existe plus !
	\end{enumerate}
	\section*{D12 -- Que fait-il/elle en ce moment ?}
	\begin{enumerate}
		\item Recherche d'un PJ
		\item Cherche de quelque chose
		\item De passage pour aller autre part
		\item Ce pour quoi il/elle est entra\^{i}né(e) à faire.
		\item Fuis / se cache de quelqu'un / quelque chose
		\item Apporte un message
		\item Formation (lui/elle-même ou quelqu'un d'autre)
		\item Transporte des personnes
		\item Tue quelqu'un, ou tente de tuer quelqu'un.
		\item Vole quelque chose, ou tente de voler quelque chose.
		\item Ach\^{e}te / vend quelque chose, ou tente de le faire
		\item Enqu\^{e}te sur quelque chose
	\end{enumerate}
	\section*{D20 -- Ce PNJ sait/connait\ldots (volontairement vague, définir qui, quoi, où, etc.)}
	\begin{enumerate}
		\item quelqu'un qui sait quelque chose. Lancez à nouveau pour découvrir quoi
		\item où quelqu'un a été enlevé
		\item qui a pris quelqu'un
		\item qui est le bouc émissaire
		\item pourquoi personne n'en parle
		\item comment faire disparaître quelqu'un
		\item comment entrer dans cet endroit
		\item quand ça va arriver
		\item qui était le vrai tueur / voleur
		\item VOTRE secret
		\item plus sur le monstre qu'il ne le devrait.
		\item comment obtenir ce qu'ils veulent de vous
		\item où il est caché
		\item la véritable identité de la personne
		\item qui l'a
		\item qui le/la veut
		\item ce que vous avez fait dans la dernière ville
		\item qui fait le suivi de vos actions
		\item qui sont ces gens qui viennent d'arriver en ville
		\item Beaucoup de secrets ! Relancez 1d4+1 secrets, ignorant les lancés de 20
	\end{enumerate}
\end{multicols}


\end{document}
