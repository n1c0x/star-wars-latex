\documentclass[twoside]{article}
\usepackage[french]{babel}
\usepackage[T1]{fontenc}
\usepackage[left=2cm,right=2cm,top=2cm,bottom=2cm]{geometry}
\usepackage{fancyhdr}
\pagestyle{fancy}
\fancyfoot[LE]{\thepage}
\fancyfoot[RO]{\thepage}
\fancyfoot[C]{}
%\usepackage{anyfontsize}

\usepackage{fontspec}
\defaultfontfeatures{Ligatures=TeX}
%\setmainfont[Mapping=tex-text]{Sitka Display}
\usepackage[small,sf,bf]{titlesec}

\usepackage{graphicx}
\usepackage{xcolor}
\usepackage{sectsty}

\definecolor{DarkGreen}{HTML}{384d3e}
\definecolor{PureWhite}{HTML}{FFFFFF}
\definecolor{DarkRed}{HTML}{6e272d}
\definecolor{DarkGold}{HTML}{a48e3b}

\sectionfont{\color{DarkGreen}}
\subsectionfont{\color{DarkRed}}
\subsubsectionfont{\color{DarkGold}}

\begin{document}

\title{\vspace{8cm}{\Huge Cadre de campagne} \\ \vspace{2cm} \includegraphics{../img/logo}}

\date{}

\maketitle
\clearpage
\tableofcontents
\clearpage


\section{La Galaxie}
\subsection{Histoire galactique}
\textit{Nota Bene : il s'agit d'un très bref condensé de l'histoire galactique telle qu'elle est enseignée sous le règne de Palpatine. La datation est calculée en utilisant l'année de son couronnement comme référence (les chronologies publiées par Lucas et autres prennent l'année de Star Wars IV comme an 1).
Du point de vue des PJs, la chronologie des évènements présentés ici est correcte mais leur exactitude ne l'est que lorsqu'ils n'entrent pas en contradiction avec les dogmes de l'Ordre Nouveau. Au fur et à mesure de l'avancement de la campagne, les découvertes réalisées par les PJs amenderont et étendront cette chronologie et permettront de mettre en évidence certaines choses volontairement passées sous silence ou déformées par les chroniqueurs officiels.}

\subsubsection*{Env -- 1 000 000 d'années}
Plusieurs races intelligentes entament leur évolution et atteignent la sapience à peu près à cette période, notamment les Hutts.  Des races plus anciennes voyagent déjà entre les étoiles depuis des millénaires mais leurs civilisations tomberont dans l'oubli ou dans la décadence bien avant la naissance de la République. On trouve encore sur certains mondes les ruines de fabuleuses Portes Stellaires qui reliaient les mondes des anciennes races entres eux mais mêmes leurs descendants dégénérés ne savent plus comment s'en servir.
\subsubsection*{Env -- 500 000 ans}
Aube de la civilisation humaine sur plusieurs planètes : Coruscant, Corellia\ldots 
Les Devaroniens découvrent le vol spatial
\subsubsection*{Env -- 100 000  ans}
La plupart des  "nouvelles races" développent à leur tour le vol spatial subluminique. C'est le début de la première grande phase d'expansion interstellaire de l'histoire connue. 
\subsubsection*{Env -- 40 000 ans}
Construction des premiers droïdes et aube de l'intelligence artificielle 
\subsubsection*{Env -- 28 000 ans}
Après de nombreuses guerres civiles, les Hutts parviennent à rendre Varl, leur monde natal, inhabitable et le quittent dans un gigantesque exode qui durera des millénaires à des vitesses inférieures à celle de la lumière. (Les Hutts n'ont jamais admis cette version des faits et prétendent que leur système a été dévasté par de nombreux cataclysmes\ldots)\\

A peu près à la même époque, la Cité Galactique voit le jour sur Coruscant. La Cité Galactique est la première enclave multiraciale dans cette région de l'espace et le point culminant de la période d'urbanisation de la planète qui n'est déjà plus qu'une immense métropole ravitaillée par les mondes voisins. De nombreux contacts sont pris par le biais de messages radio subluminiques. 
\subsubsection*{-- 27 500 ans}
Des colons humains originaires de Coruscant atteignent après plusieurs siècles de voyage subluminique la planète Alderande et s'y installent. Les théories sur le subespace permettent de fabriquer les premiers émetteurs subspatiaux dont la portée ne dépasse pas quelques années lumières mais qui permettent une communication instantanée.\\

A cette époque, le seul moyen de voyager plus vite que la lumière réside dans une poignée de Portes Stellaires encore en activité mais qui tombent en panne les unes après les autres.
\subsubsection*{-- 27.000 ans}
La découverte de plusieurs gaz hautement énergétiques et la conception de batteries portables à grande capacité entra\^{i}ne l'apparition des premières armes à rayon portables, les blasters. 
\subsubsection*{-- 26.000 ans} 
Les Hutts s'emparent de la planète Kintan par la force, grâce à un appareil de traduction instantanée de leur invention, ils parviennent à établir le Traité de Vontor qui réduit effectivement les peuples de Kintan en esclavage. Les Verpines, une race insectoïde de nomades spatiaux d'origine inconnue, décident de s'établir dans la Ceinture d'Astéroïdes de Roche. 
\subsubsection*{-- 25.110 ans}
Le tyran connu sous le nom de Xym le Despote commence depuis la planète Dellalt une campagne de conquête des systèmes voisins et son monde d'origine devient vite un véritable joyau alors que les tributs des mondes soumis s'y entassent.
Début de plusieurs guerres civiles sur Kintan. 
\subsubsection*{-- 25.080 ans}
Une coalition Hutt défait Xym le Despote. La planète Dellalt tombe rapidement dans l'oubli alors que les divers mouvements unificateurs qui donneront naissance à l'Ancienne République commencent à faire sentir leur influence à grande échelle. 
\subsubsection*{-- 25.000 ans}
Les Corelliens développent les premiers hypermoteurs réellement fiables et relativement bon marché : le voyage plus vite que la lumière est une réalité. Gràce aux communications subspatiales, ils vendent leur concept, totalement divergent des théories antérieures, à plusieurs autres mondes parmi lesquels Coruscant, Duro  et Alderande. La phase d'unification galactique conna\^{i}t une accélération brutale alors que des éclaireurs s'élancent depuis les mondes du Noyau sur deux voies hyperspatiales relativement sures qui formeront par la suite deux grandes routes commerciales : la route perlemienne et la route corellienne.
Aidés par les corelliens, le peuple Duro perfectionne la navigation hyperspatiale et ses membres deviennent rapidement des navigateurs réputés.\\

La planète Coruscant est instrumentale dans cette phase d'expansion galactique et c'est à partir de sa langue déjà assez répandue dans les mondes proches que l'on forme le Basic qui deviendra en quelques siècles la langue galactique. \\

La République Galactique na\^{i}t ainsi que son Sénat et la première Constitution Galactique. \\

Les civilisations du Noyau entrent en contact avec les Hutts. 
\subsubsection*{-- 24.960 ans}
La Reine Rana des Duro meurt mais son fils Kassid maintient sa politique d'exploration spatiale. Les Duro sont de plus en plus nombreux à se lancer dans l'exploration ou à abandonner leur monde pour des stations spatiales. Durant cette période, un vaisseau d'exploration probablement Duro ou Corellien tombe aux mains de la race barbare et agressive des Aqualish, de la planète Ando, qui commencent aussitôt à en reproduire la technologie. 
\subsubsection*{-- 24.955 ans}
Les Aqualish ont terminé la construction d'une flotte d'invasion de bric et de broc et tentent de s'emparer de plusieurs systèmes voisins. Ils sont presque aussitôt écrasés par les toutes jeunes armées de la République mais intégrant indirectement à la civilisation galactique, de nombreux soldats Aqualish devenant casseurs de pouces, mercenaires et lutteurs en raison de leur tempérament agressif.
\subsubsection*{-- 24.800 ans}
Les Hutts découvrent la magnifique planète Evocar. Ils concluent alors des arrangements commerciaux avec les indigènes primitifs et leur vendent de la technologie en échange de terres exploitables. Lorsque les natifs d'Evocar finissent par réaliser que les Hutts possèdent la majeure partie de leur planète et sont en train d'acquérir le reste, ils font appel à la République mais on ne trouve aucune faille dans les contrats des Hutts (dont la réputation déjà bien établie fait un nouveau plongeon) et les natifs sont évacués vers une lune proche. \\

Les Hutts décident de raser toutes les cités et constructions d'origine et renomment leur acquisition Nal Hutta (ce qui signifie dans leur langue "Glorieux Joyau") en hommage à sa beauté qu'ils transforment rapidement en fonction de leurs propres canons esthétiques, rasant les superbes forêts evocariennes pour laisser la place à des complexes industriels et à des marécages putrides ou ils aiment à se promener. 
\subsubsection*{-- 24.500 ans}
La lune ou ont été transportés les Evocains d'origine acquiert rapidement un rôle primordial dans les réseaux criminels Hutts comme port d'attache pour les vaisseaux de leurs employés. Elle est sauvagement urbanisée par les Hutts et devient une sorte de réplique miniature puante et décadente de Coruscant qui sera connue par la suite sous la dénomination de Nar Shadda, "la Lune des Contrebandiers". Les derniers Evocains se réfugient dans les profondeurs du métroplexe planétaire et dégénèrent en quelques générations. 
\subsubsection*{-- 24.000 ans}
Une vague d'expansion et de colonisation tout autour des principales routes commerciales de l'époque ouvre à l'exploitation la région des Colonies (les Anciennes Colonies à l'époque actuelle). Les premiers réseaux de communication par hyper-radio sont établis. 
\subsubsection*{-- 20.000 ans}
La plupart des prestigieuses corporations d'industrie navale comme Kuat Drive Yards, la Corporation Technique Corellienne et Propulsions Rendili sont fondées à cette époque. \\

Le peuple Etti, une ethnie humaine caractérisée par une minceur extrême, s'unit pour acheter plusieurs vaisseaux colonies afin de quitter définitivement les territoires de la République avec lesquels les désaccords politiques sont trop nombreux et que les Etti considèrent comme "trop tyrannique". Ils empruntent des routes nouvellement découvertes pour aller s'installer dans un coin perdu de la galaxie qui sera quelques milliers d'années plus tard donné aux grandes compagnies et deviendra le Secteur Corporatif. \\

La Cité Galactique de Coruscant devient la capitale culturelle de la République (après 5000 ans d'existence en tant que capitale administrative) et une Bibliothèque Sénatoriale recensant pratiquement tous les ouvrages de la galaxie est ébauchée. \\

Le Registre de la République comme à être compilé et décrit brièvement toutes les cultures et races qui sont membres de ce gouvernement interstellaire. 
Les premières lois de la République commencent à être appliquées sur l'ensemble de son territoire. \\

Le cycle annuel de Coruscant devient le Temps Galactique Standard : 60 secondes standard forment une heure, 24 heures forment un jour, cinq jours forment une semaine, sept semaines forment un mois, l'année standard compte douze mois (350 jours) trois semaines fériées (15 jours) et trois jours de célébrations et commémorations officielles.\\

Un ordre mystique, les Jedi, devient le bras exécutif du Ministère de la Justice de la République.\\

Avec l'accroissement des échanges commerciaux, les premières mégacorporations interstellaires voient le jour. 
\subsubsection*{-- 18.000 ans}
Le Bureau Officiel des Services Spatiaux (BoSS) est formé. Il s'agit d'une des plus anciennes institutions de la galaxie qui existe encore à ce jour. Le BoSS collecte, compile et archive toutes les informations sur les navires spatiaux civils enregistrés, leurs propriétaires, leur armement, leurs manifestes de cargaison. Il a aussi pour tâche de veiller à ce que les cartes stellaires soient à jour et le contenu de ses banques de données est régulièrement téléchargé par les services des astroports, les services de police et pratiquement toutes les agences privées ou gouvernementales accréditées. A la longue, le BoSS deviendra une institution à part entière complètement séparée des gouvernements locaux ou galactique. 
\subsubsection*{-- 15.100 ans}
La race Anomide développe ses propres hypermoteurs dont certains éléments deviendront très répandus dans la galaxie en quelques décennies. Les Anomides ne se lanceront jamais dans l'exploration spatiale et l'on considère qu'ils ont développé cette technologie par curiosité plus que pour des raisons pratiques. 
\subsubsection*{-- 13.000 ans}
Le peuple de la planète Ailon fonde la Garde Nova et toute la planète entame une nouvelle phase culturelle alors qu'une philosophie militariste est instaurée et que na\^{i}t ainsi la première armée rassemblant tous les individus d'un peuple sans aucune exception. Durant les 13.000 années qui vont suivre, la tradition martiale extrêmement dure de la  Garde Nova d'Ailon lui permettra d'être reconnue comme une armée sans égale. 
\subsubsection*{-- 12 700 ans}
Quelques-unes des plus anciennes familles nobles de l'histoire humaine, en désaccord avec la République depuis sa naissance, établissent des colonies dans un secteur spatial et abandonnent leurs anciens foyers derrière elles. Elles donneront naissance aux Douze Royaumes de l'Etendue. 
\subsubsection*{-- 12.500 ans}
L'esclavage devient une pratique commune dans la société du peuple Twi'lek. 
\subsubsection*{-- 12.000 ans}
Plusieurs systèmes sont colonisés par les armées de la république afin de servir de dépôts de ravitaillement et de bases éloignées. Ord Mantell et Ord Pardron sont les plus connus (peu de gens se souviennent que le préfixe Ord était l'abréviation administrative d'Ordonnance Regional Depot : Dépôt de Matériel Régional). 
\subsubsection*{-- 11.500 ans}
Le plus grand schisme de l'Ordre Jedi se produit et pendant près d'un siècle il est déchiré par les luttes internes. 
\subsubsection*{-- 10.000 ans}
Le HoloNet est développé : des centaines de milliers de relais hyperspatiaux et une technologie révolutionnaire permettent désormais de relier de manière instantanée pratiquement tous les mondes de la République. A cette époque, le coût de l'entretien du HoloNet est tel qu'il était en pratique réservé aux grandes corporations et au gouvernement central. 
Un important mouvement de rébellion contre la République est démantelé et les survivants déportés sans espoir de retour. 
\subsubsection*{-- 9970 ans}
L'ordre Jedi développe une nouvelle arme : le sabre-laser. 
\subsubsection*{-- 8000 ans}
Malgré l'opposition de ses pairs, le Sénateur Elrood parvient à faire financer un programme de colonisation d'un nouveau secteur. Lorsque le principal monde visé se révèle riche en ressources minières et avec un climat agréable, les colons décident de le baptiser ainsi que le secteur du nom de leur bienfaiteur. 
\subsubsection*{-- 7308 ans}
Le jeune noble Shei Tapani prend le contrôle des Douze Royaumes. L'Etendue est rebaptisée "Empire de Tapani" mais devient par la suite connue sous le nom de Secteur Tapani alors que Shei Tapani rejoint la République et que les anciens royaumes deviennent des maisons nobles subordonnées à leur nouveau souverain. 
\subsubsection*{-- 7000 ans}
La librairie de Veeshas Tuwan sur Arcania est fondé par un ordre de savants indépendants de la République. Elle deviendra par la suite un gigantesque centre et est aujourd'hui utilisée par l'Empire comme étant une des archives historiques et scientifiques les plus fiables 
\subsubsection*{-- 5977 ans}
Deux factions Jedi rivalisant pour le pouvoir commencent à s'affronter de manière violente et plusieurs mondes sont rapidement pris dans la tourmente 
\subsubsection*{-- 5867 ans}
Le conflit entre Jedi prends fin et sera connu par la suite comme "le siècle des ténèbres". 
\subsubsection*{-- 4987 ans}
La planète Carida devient le siège d'une des plus prestigieuses académies militaires de la République et par la suite de l'Empire.  
\subsubsection*{-- 4980 ans}
Durant un conflit périphérique sans importance, une flotte de la république utilise une arme secrète qui provoque l'implosion de l'étoile Denaari en nova, détruisant du coup tout le système et la presque totalité de la flotte républicaine. Le secret de cette arme ne sera jamais retrouvé. 
\subsubsection*{-- 4177 ans} 
Deux nouveaux conflits secouent l'ordre Jedi et entra\^{i}nent des millions de morts dans le Massacre de Gank et le Cataclysme de Vultar 
\subsubsection*{-- 4130 ans} 
Le groupe pirate des Maraudeurs de Lorell s'installe dans le secteur isolé d'Hapes et fonde une nouvelle alliance, le Consortium d'Hapes. Le Consortium devient une fraternité masculine dans laquelle les femmes ont le plus souvent un rôle négligeable 
\subsubsection*{-- 4030 ans} 
La République éradique pratiquement tous les leaders du Consortium d'Hapes ce qui permet aux femmes de prendre le pouvoir. La première Reine-Mère d'Hapes est couronnée. 
\subsubsection*{-- 4000 ans} 
Des explorateurs intrépides parviennent à établir la majeure partie des routes actuelles de la Région d'Expansion jusqu'à présent peu exploitée et qui devient rapidement synonyme d'opportunité et d'enrichissement. 
\subsubsection*{-- 3980 ans} 
La Grande Révolution des Droïdes : pratiquement tous les droïdes se rebellent sur la planète Coruscant mais ils sont finalement vaincus par la République. Les maraudeurs mandaloriens, un groupe humain sous le commandement du Seigneur Mandalore, commencent à faire parler d'eux. Ils acquièrent rapidement la réputation d'être les guerriers les plus redoutables de leur époque.
\subsubsection*{-- 3976 ans} 
Le seigneur de la guerre Mandalore attaque le système Impératrice Teta mais il est tué durant un affrontement avec les forces de la République. Un de ses subordonnés prend le contrôle et organise la retraite. Les mandaloriens continueront a hanter la République pendant des milliers d'années mais tomberont graduellement dans l'oubli. 
\subsubsection*{-- 3960 ans} 
La Région d'Expansion est placée sous le contrôle des grandes corporations en raison de ses considérables richesses. Durant quelques siècles, la main-mise totale des corporations empêchera le Sénat de découvrir le mécontentement de leurs employés et des populations indigènes mais les corporations seront finalement incitées à déplacer progressivement leurs intérêts vers un secteur de l'espace nouvellement ouvert à l'exploitation loin du Noyau, le futur Secteur Corporatif. 
La République fonde le Corps Républicain des Eclaireurs. 
Les progrès en informatique rendent les calculs hyperspatiaux suffisamment fiables pour que l'antique système des hyper-balises devienne obsolète et les voyages bien plus aisés. 
\subsubsection*{-- 2976 ans} 
L'Amas d'Hapes coupe tous contacts avec le reste de la galaxie et ferme ses frontières 
\subsubsection*{-- 2000 ans} 
Fondation de Incom Corporation, qui deviendra un fabriquant réputé de chasseurs stellaires. 
\subsubsection*{-- 1561 ans} 
Le Secteur Tapani subit plusieurs bouleversements lorsque le système impérial hérité de son fondateur est renversé par des réformateurs républicains. Les maisons nobles parviennent cependant à demeurer au pouvoir mais l'établissement d'une nouvelle route spatiale la même année (le Raccourci Shapani) permet  à des mondes isolés du secteur de participer plus activement à l'économie galactique et à devenir indépendants, formant ainsi les Mondes Libres. 
\subsubsection*{-- 700 ans} 
Le Secteur Corporatif voit finalement officiellement le jour. Un bon nombre des grandes corporations installées autrefois dans la Région d'Expansion ont partiellement abandonné leurs anciennes sphères d'influence pour s'installer dans une région ou malgré les nombreuses restrictions placées par la République, elles savent pouvoir faire des affaires comme elles l'entendent. En effet,  elles n'auront pas à souffrir des taxes gouvernementales sectorielles ou planétaires, pas plus qu'elles n'auront à s'opposer aux intérêts de populations indigènes primitives dont le nouveau Secteur Corporatif est totalement dépourvu. Rapidement, les compagnies du Secteur Corporatif réalisent des profits monstrueux et concluent de nombreux arrangements réduisant la compétition interne. \\

Une épidémie mortelle frappe des centaines de mondes dont les populations sont décimées ou retombent dans la barbarie. 
\subsubsection*{-- 500 ans} 
La planète Kuat conna\^{i}t une brève période de célébrité alors que son plus gros conglomérat, Kuat Drive Yards, parvient à racheter Alderande Royal Engineers et Core Galaxy Systems, devenant ainsi le plus grand constructeur naval de l'époque. La noblesse de Kuat commence à être connue dans la galaxie pour ses goûts sybaritiques et quelques traditions datant de cette époque, notamment l'usage d'anneaux contenant du poison pour régler certains problèmes politiques. Les nobles kuati mettent aussi en place le système de  recrutement des telbun, de jeunes hommes et femmes aux antécédents génétiques connus qui sont élevés et  éduqués afin de servir de consorts aux nobles et de tuteurs aux enfants de leur union avec leur ma\^{i}tre/ma\^{i}tresse. Les Telbun sont privés de tout pouvoir ou droit officiel et ne peuvent ainsi nuire aux mariages politiques conclus entre membres de la noblesse qui contournent du même coup les problèmes de consanguinité dans leurs lignées.\\

Un auteur corellien anonyme écrit  "Le Roi Filou" qui est encore considéré actuellement comme un des monuments de l'art dramatique sur pratiquement tous les mondes humains de la galaxie. 
\subsubsection*{-- 320 ans} 
De nombreuses compagnies des secteurs centraux de la galaxie fondent la Fédération du Commerce, impulsée par le peuple Neimoidien. Durant les trois siècles suivants, la Fédération du Commerce deviendra un groupe de pression de plus en plus puissant au point que ses représentants obtiendront le droit de siéger au Sénat Galactique.\\
  
Durant une expédition cartographique depuis Ithor, le vaisseau d'exploration Telarion découvre plusieurs mondes habitables dans un coin isolé de la galaxie avant de dispara\^{i}tre corps et bien. Les premiers colons qui viennent s'installer dans la région d'où proviennent ses derniers rapports décident de baptiser le premier monde qu'ils colonisent "Telar" en l'honneur du navire disparu. Sur le plan administratif, cette région de l'espace est baptisée Secteur Telarion. 
\subsubsection*{-- 280 ans} 
En traversant une phase expansionniste de son histoire, le peuple Bothan acquiert une réputation galactique de manipulateurs obsédés par l'espionnage et les secrets susceptibles de ruiner la réputation de leurs ennemis. Les sombres machinations entre clans bothans rivaux sont telles aux yeux des autres peuples que rapidement l'expression "faire les choses à la manière bothane" devient synonyme d'utiliser la trahison, le chantage et la manipulation pour parvenir à ses fins. 
\subsubsection*{-- 270 ans} 
Les colons du Secteur Telarion découvrent l'Enclave Gree, une région de l'espace peuplée par une des Anciennes Races qui préfère demeurer dans l'isolement et poursuivre sa lente décadence. 
\subsubsection*{-- 230 ans} 
La planète Adarlon de l'amas de Minos (un des secteurs les plus éloignés jamais explorés à ce jour) est colonisée par une collection d'humains excentriques et marginaux alors qu'elle est pratiquement dépourvue de toute ressource naturelle significative.  Rapidement, Adarlon devient un producteur d'holovidéos dramatiques très demandés dans la galaxie. 
\subsubsection*{-- 232 ans} 
Le sénateur Valorum devient chancelier suprême du Sénat. Durant les deux siècles suivant, plusieurs de ses descendants occuperont cette position prestigieuse jusqu'à la décadence finale de la Vieille République. 
\subsubsection*{-- 220 ans} 
Le capitaine Terris Antal parvient à établir une route praticable reliant le Secteur Telarion à Ord Mantell et Wroona.
\subsubsection*{-- 100 ans} 
Jusqu'à cette époque, le seul moyen de préserver un individu du passage du temps était de le congeler dans la carbonite, processus qui n'était pas dépourvu de risques. Le Secteur Corporatif développe les premiers caissons de stase nettement plus pratiques et fiables. Ils deviennent rapidement d'usage courant dans certaines institutions pénitentiaires.\\
  
Plusieurs centaines de firmes industrielles de taille moyenne basées dans les régions centrales de la galaxie et qui ne peuvent investir pleinement dans le Secteur Corporatif commencent à se réunir pour trouver un moyen d'assouplir les lois républicaines jugées trop restrictives. Rapidement, les principaux fabricants de droïdes, d'électronique grand public, de véhicules et d'équipement militaire dans leurs rangs décident de former la Techno Union qui leur permettra non seulement d'obtenir plus d'influence auprès de la République mais aussi et surtout s'assurer la protection mutuelle des membres face aux grandes compagnies, aux firmes impliquées dans le Secteur Corporatif et à d'autres alliances puissantes comme le Cartel Bancaire Intergalactique ou la Fédération du Commerce.
\subsubsection*{-- 62 ans} 
Naissance de Palpatine

\subsubsection*{-- 26 ans} 
Un des plus grands échecs technologiques de l'histoire : deux cents cuirassés de guerre républicains sont modifiés afin de réduire leurs équipages (qui passent de 16204 à seulement 2204 hommes par navire) grâce à des circuits esclaves et un réseau informatique interne qui relie également les navires entres eux et à leur vaisseau amiral, le Katana. Lors du vol d'essai, un problème informatique imprévu frappe le Katana qui bondit soudainement dans l'hyperespace avec le reste de la flotte à sa suite, pour ne jamais réappara\^{i}tre.

\subsubsection*{-- 25 ans} 
Maintenant membre du Sénat Galactique, Palpatine parvient à faire financer l'envoi de milliers de sondes robots à destination du C\oe ur Galactique, considérant que ce secteur de la galaxie mérite une véritable exploration et qu'il regorge potentiellement de ressources. Bien qu'à l'origine simple représentant de la planète Naboo, Palpatine commence ainsi sa véritable entrée sur la scène politique.

\subsubsection*{-- 12 ans} 
Le Sénat passe la résolution BR-0371 qui impose de lourdes taxes sur les routes commerciales des Bordures Moyenne et Extérieure, malgré les pressions exercées par la Fédération du Commerce. Le public apprend rapidement que la corruption généralisée dans les hautes sphères de la République a provoqué plusieurs gouffres budgétaires qu'il faut combler rapidement et que la résolution BR-0371 est considérée comme le meilleur moyen d'y parvenir. Le Sénateur Palpatine fait plusieurs discours dans lesquels il déplore cet état de fait mais appuie néanmoins avec beaucoup de vigueur cette décision, ce qui n'échappe pas à la Fédération du Commerce. \\

Peu de temps après, la Fédération établit un blocus autour de Naboo. Après un mois de blocus, alors que la situation devient dramatique, la Reine Amidala parvient à obtenir du grand chancelier Valorum une promesse d'intervention alors que le Sénat s'embourbe dans les méandres bureaucratiques. Cependant, Valorum est lui-même rattrapé par plusieurs accusations de fraude et doit démissionner. 
Finalement, alors que la Fédération tente un débarquement en force sur Naboo, les sénateurs parviennent malgré la corruption rampante à se mettre d'accord pour élire Palpatine à leur tête. Les dirigeants de la Fédération responsables du blocus sont rapidement interpellés et celui-ci est levé. \\

Une des premières décisions du nouveau Chancelier Suprême est de sanctionner l'usage de droïdes de combat dans le blocus de Naboo. Les droïdes de ce type sont désormais interdits et les modèles existants sont détruits ou reprogrammés afin de ne pas pouvoir causer de mal à un être pensant. \\

Plusieurs puissants groupements d'intérêts déjà fort occupés à faire du lobbying se constituent de manière officielle ou tentent de rallier d'autres puissances mineures à leurs vues. \\

De nombreux magnats industriels, sénateurs, dirigeants militaires et simples citoyens décident de créer le COMPORN (Comité pour la préservation de l'Ordre Nouveau) afin de soutenir le programme de réformes envisagé par le Chancelier Suprême Palpatine.  \\

Dans le même temps, les principales corporations impliquées dans le Secteur Corporatif se réunissent sous la bannière de la Ligue Corporatiste Galactique, dont le porte-parole est le Baron Orman Tagge. La Ligue n'est que le dernier des lobbies financiers à voir le jour durant les années de décadence de la République et il s'oppose à des groupements plus traditionnels déjà bien implantés. \\

La Fédération du Commerce, le Cartel des Banques et la Techno-Union commencent à discuter des menaces qui pèsent sur leurs intérêts à cause de la Ligue Corporatiste Galactique, du COMPORN et des résolutions du Sénat. \\

Dooku, un des plus grands Maitre Jedi de son époque, quitte l'Ordre et démissionne pour retourner chez lui sur la planète Serenno ou il récupère son titre de Comte et une partie conséquente de la fortune familiale. Il est le vingtième Maitre dans l'histoire de l'Ordre à procéder ainsi et son buste rejoint celui des dix-neuf autres "traitres". (Les bustes des Vingt étaient placés dans le grand hall des Archives Jedi sur Coruscant pour que les chevaliers n'oublient pas ceux qui avaient choisi durant les millénaires de les quitter pour des raisons diverses. Afin qu'ils puissent méditer sur la place de l'individu au sein de l'Ordre et de la galaxie). 

\subsubsection*{-- 11 ans} 
Malgré le soutien du COMPORN et de nombreuses planètes, le Sénat (certainement inspiré par la Fédération du Commerce) fait abroger la résolution BR-0371, ce qui représente un camouflet pour la faction progressiste alignée avec Palpatine. Les gouffres budgétaires ne pouvant être comblés, l'économie galactique s'enfonce lentement dans la crise alors que de nombreux ministères subissent des coupes financières dramatiques. \\

En dépit d'un considérable battage médiatique, le Projet Vol Extérieur, activement défendu par l'ordre Jedi et visant à envoyer une expédition hors de la galaxie vers des amas stellaires proches, est un échec retentissant lorsque le navire dispara\^{i}t corps et bien pratiquement dès le début de son voyage.\\

Le Comte Dooku commence à se faire connaitre comme un des principaux porte-paroles des mécontents et des indépendantistes et fonde le Mouvement Séparatiste. Il se procure la voile hyperluminique de son vaisseau personnel auprès d'un trafiquant opérant dans l'Enclave Gree.

\subsubsection*{-- 10 ans} 
La crise économique s'aggrave. Les forces de la République et l'ordre Jedi sont forcés d'intervenir constamment alors que les affrontements au Sénat incitent de nombreux mouvements indépendantistes répandus dans toute la galaxie à prêcher la sécession, souvent de manière violente. La piraterie conna\^{i}t de nouveaux records alors que la situation se dégrade et que des systèmes se déclarent par centaines en faveur des Séparatistes.\\

Les Séparatistes semblent ne pas être impliqués dans divers actes de violence ou de terrorisme mais leur puissance croissante inquiète le Sénat.\\

L'inventeur Wallex Blissex réalise le premier star destroyer de classe Victoire qui s'avère nettement moins couteaux que les cuirassés républicains et dont la production en cha\^{i}ne commence presque aussitôt.\\

Dooku rencontre Darth Sidious et décide de le rejoindre. Il devient Darth Tyrannus 

\subsubsection*{-- 9 ans} 
Darth Sidious se fait passer pour le maitre Jedi Syfo Dias et incite les kaminoans à produire des clones à partir de Jango Fett, sélectionné par Dooku.

\subsubsection*{-- 6 ans} 
Un groupe de Jedi renégats sème la terreur dans le système Bphassh mais finit par être éliminé.\\

Le Mouvement Séparatiste devient la Confédération des Systèmes Indépendants et les tensions diplomatiques avec la République (dont ils n'ont pas encore officiellement fait scission) se multiplient.

\subsubsection*{-- 3 ans} 
La corporation Sienar propose à la République un nouveau modèle de chasseur stellaire à courte portée, fiable et économiquement peu couteux, le Twin Ion Engines (T.I.E).

\subsubsection*{-- 2 ans} 
La République découvre l'existence des armées de la Confédération des Systèmes Indépendants et leur quartier général sur Géonosis. Une force militaire importante assistée par plusieurs centaines de Jedi se rend sur place et livre la première bataille des Guerres Cloniques. (Voir Star Wars - L'Attaque des Clones pour tous les détails)\\

Le conflit prend rapidement de l'ampleur. La République s'appuie sur ses membres et les légions clonées des Kaminoans tandis que la Confédération fait appel à des armées de droïdes. Rapidement, mercenaires, droïdes renégats, forces militaires indépendantes compliquent la situation et chacun des deux belligérants finit par avoir de plus en plus souvent recours aux soldats clonés plus fiables que les droïdes. \\

Une des plus importantes batailles de la Guerre des Clones a lieu dans le système Dreighton. Le système devient le théâtre d'évènements mystérieux après la bataille : les vaisseaux civils qui passent par Dreighton signalent des échos radar anormaux, des apparitions subites de navires spatiaux en plein affrontement avec un ennemi invisible et un certain nombre de pannes inexplicables. Certains navires disparaissent corps et biens et la zone devient rapidement connue sous le surnom de Triangle de Dreighton. La plupart des pilotes sains d'esprits préfèrent perdre quelques jours à la contourner plutôt que d'y pénétrer.

\subsubsection*{-- 1 an}
La Guerre des Clones coûtant des sommes monstrueuses, le TIE devient rapidement un substitut bon marché aux autres modèles et le chasseur standard des forces républicaines. \\

Lassés de voir que malgré la crise galactique, les sénateurs continuent à utiliser flottes, armées et agents pour leurs complots personnels, les directeurs des quatre services de renseignements de la République décident lors d'une conférence secrète de coordonner leurs informations afin d'éviter d'agir les uns contre les autres en servant les intérêts de factions du Sénat. Peu de temps après, cet accord prend forme de manière plus définitive lorsque les quatre agences fusionnent afin de former l'Ubiquetorat. Les sénateurs sont mis devant le fait accompli : les services de renseignements ne suivront désormais que le pouvoir exécutif (Le Chancelier Palpatine) ou un vote sénatorial et n'accepteront plus aucun ordre venant d'une faction ou d'un individu.\\

Des preuves accablantes accusent certains Jedi d'avoir une part de responsabilité dans la Guerre des Clones, ce qui amène les premières purges dans leur rang, ordonnées par le Sénat. Les coupables sont déportés mais en raison de leurs pouvoirs, la majorité doit être tuée alors qu'ils s'opposent aux forces de l'ordre. 
Plusieurs incidents impliquant des soldats clones donnent à penser que leur fiabilité et leur obéissance ne sont pas totales. (Voir annexe 5.1.5 à ce sujet). La République semble prendre enfin l'avantage sur la Confédération des Systèmes Indépendants alors que plusieurs compagnies alliées à la Confédération décident de rejoindre le camp républicain grâce à l'influence de la Ligue Corporatiste Galactique. \\

De nombreux centres de clonage sont détruits ou volontairement fermés sur ordre du Sénat.

\subsubsection*{Année du Couronnement -- An 1 du Calendrier Impérial}
Palpatine, appuyé par la majorité des militaires, des grandes corporations et des sénateurs, est couronné Empereur. Alors que les groupes terroristes manipulés par certaines factions du sénat et l'ordre Jedi frappent sur Coruscant et ailleurs, L'Empereur décide de faire appel à toutes les forces loyalistes pour écraser les renégats. Les chevaliers Jedi, qui ont survécu à la Guerre des Clones en causant des milliards de morts (y compris parmi les rangs de leur ordre ou peu de gens connaissaient leur véritable rôle), sont impitoyablement pourchassés. Ils sont considérés comme mortellement dangereux en raison des dégâts qu'ils ont causés à plusieurs occasions durant les dix dernières années.\\

Coruscant cesse d'être la Cité Galactique pour devenir le Centre Impérial.\\

Les derniers affrontements des Guerres Cloniques se soldent par la défaite de la Confédération. Les principaux groupes corporatistes qui en étaient membres sont démantelés et leurs actifs distribués aux compagnies de la Ligue Corporatiste et autres factions alliées à Palpatine.\\

Le Capitaine Alater Brind découvre l'existence des Barabels sur Barab Prime, malgré les tentatives de dissimulation de la société Safaris Planétaires.\\

L'HoloNet est nationalisé et placé sous le contrôle direct de l'empire, de nombreux secteurs de ce réseau doivent être fermés d'urgence alors que les responsables découvrent qu'il sert à véhiculer des virus informatiques et de la propagande mensongère réalisés par les adversaires de l'Ordre Nouveau. Les anciennes familles nobles du Noyau sont divisées et certaines complotent contre l'Empire. Palpatine ordonne alors leur arrestation et plus de 300.000 nobles se voient graciés à condition d'accepter l'exil sur la planète Eliad de la Bordure Extérieure. Les autres prennent lâchement la fuite ou sont condamnés et exécutés. Les familles nobles restées fidèles se voient attribuer une partie conséquente des possessions des tra\^{i}tres, le reste étant vendu pour permettre la reconstruction d'un gouvernement galactique malmené par la trahison, la corruption et la guerre. (Bien évidemment, même si certains nobles avaient profité des derniers soubresauts de la République pour augmenter leur puissance et étaient devenus hostiles à une tutelle quelconque, bon nombre d'entre eux furent éliminés ou exilés pour laisser la place à des gens plus en phase avec les objectifs de Palpatine et/ou son idéologie).   

\subsubsection*{An 2 du Calendrier Impérial}
L'idéologie de la Haute Culture Humaine, qui reconna\^{i}t la supériorité de l'espèce humaine sur les autres espèces et son rôle primordial dans l'histoire galactique, commence à se répandre et cimente l'Ordre Nouveau.\\

L'organisation administrative de l'Empire (Gouverneurs Sectoriels, Moffs\ldots) commence à prendre forme et le pouvoir glisse des mains des sénateurs vers celles des administrateurs impériaux.\\

L'ancien chasseur T.I.E est réformé et laisse la place au nouveau modèle de la série.\\

Lira Blissex, fille du concepteur du star destroyer de classe Victoire, épouse le gouverneur Wessex et réalise les plans du prototype de star destroyer de classe Imperator.\\

La planète Alderande, traumatisée par les horreurs des Guerres Cloniques, décide d'abandonner l'usage des armes et toutes celles qui ont servies à sa défense sont détruites.\\

L'Empereur nomme le fidèle Sate Pestage au rang de Grand Vizir et lui confie le Sceau Impérial ainsi que l'organisation de l'administration de la Maison de l'Empereur. La construction du Palais Impérial commence, sa magnificence devant surpasser celle de l'ancien Sénat et symboliser la renaissance de la galaxie. Un incident regrettable impliquant le navire personnel du Gouverneur Tarkin sur la planète Ghorman cause la mort imprévue de plusieurs dizaines de civils. Bien que le pouvoir central tente de faire la lumière sur cette affaire et innocente rapidement Tarkin qui devient Moff, des groupuscules rebelles se servent du "Massacre de Ghorman" pour tenter de saborder l'Ordre Nouveau.\\

Des rapports alarmants, indiquant que durant les deux derniers siècles l'esclavage s'est multiplié partout dans la galaxie, incitent l'Empereur à ordonner le Décret Impérial A-SL-4557-607.232 qui légalise certaines des organisations esclavagistes les plus respectables afin de donner davantage de contrôle à l'Empire sur ces activités. Les autres organisations sont considérées comme devant être éliminées à tout prix.\\

Le peuple reptilien Trandoshan longtemps en guerre contre le peuple Wookie apporte la preuve que ses adversaires comptent procéder à l'extermination de la race Trandoshan et entendent faire sécession de l'Empire. Les autorités imposent alors la loi martiale sur la planète Kasshyyk (le monde natal des Wookies) et déportent les criminels démasqués afin de leur faire effectuer des travaux d’intérêt commun pour la galaxie.\\

Le Corps des Éclaireurs est réformé et devient le Corps Impérial d'Exploration. Sa mission évolue également : autrefois voué à découvrir de nouveaux mondes exploitables et colonisables, le CIE voit cette orientation dispara\^{i}tre car l'Empire considère qu'elle représente des dépenses inutiles et que les populations malmenées par la Guerre des Clones doivent plutôt être aidées à reconstruire leurs foyers et non à partir à l'autre bout de la galaxie. Néanmoins, les opérations de cartographie des nouveaux systèmes continuent avec le souci constant de découvrir et évaluer toute civilisation inconnue pouvant représenter un allié ou un ennemi potentiel pour la civilisation galactique.\\

Des maraudeurs inconnus parviennent à franchir les boucliers de la pacifique planète Caamas et à ravager sa surface. Les seuls Caamasi qui survivent sont ceux qui n'étaient pas là lors de l'attaque qui ne laisse aucun témoin. La galaxie est frappée par l'horreur de cet acte barbare.

\subsubsection*{An 3 du Calendrier Impérial}
Face à de nouvelles menaces de séditions et des rumeurs croissantes selon lesquelles des fugitifs Jedi seraient derrière certains mouvements terroristes, l'Empereur donne leur forme définitive aux groupes de chasseurs de Jedi et fonde ainsi l'Ordre des Inquisiteurs Impériaux.\\

Le plus fidèle serviteur de l'Empire, le Seigneur Darth Vader, est forcé de raser la colonie de la planète Talaesa qui servait de camp d'entra\^{i}nement secret à des fanatiques hypnotisés par les pouvoirs d'un ma\^{i}tre Jedi.\\

La Ligue Corporatiste Galactique qui veut développer le Secteur Corporatif obtient l'autorisation de l'Empire d'en étendre les frontières sur plusieurs milliers de systèmes inhabités autour de l'espace qu'il occupait. Un gouvernement central prend définitivement forme : la Corporate Sector Authority (CSA) qui est également une corporation à part entière. La naissance de la CSA rend caduque l'existence de la Ligue Corporatiste Galactique qui s'auto-dissout. De nombreuses compagnies du Noyau (y compris d'anciens membres de la Techno Union ou du Cartel Bancaire Intergalactique qui soutenaient la Confédération)  aident à la création de la CSA et des taxes colossales commencent à affluer dans les caisses du trésor impérial. Le pouvoir central décide d'accorder à l'Autorité du Secteur Corporatif le droit d'assurer la défense du Secteur Corporatif en échange des considérables apports financiers et industriels qu'il fournit à la reconstruction globale.\\

Longtemps considérée comme un lieu de débauche et de propagande mensongère, la planète Adarlon de la Bordure Extérieure est soumise à de sévères restrictions afin que ses programmes d'holovision soient rectifiés et cessent de prêcher la sédition et le vice.

\subsubsection*{An 5 du Calendrier Impérial}
Le premier Grand Inquisiteur, Lord Torbin, est tué par un droïde assassin. Plusieurs de ses subordonnés sont élevés au rang de Grands Inquisiteurs et décident se partager les responsabilités de ce corps.

\subsubsection*{An 10 du Calendrier Impérial}
Une race inconnue originaire de l'Espace Sauvage, les Lortan, attaque plusieurs systèmes de la Bordure Extérieure, exterminant sans pitié la population d'une douzaine de systèmes stellaires.
L'Empire découvre la planète Froz mais ses indigènes étant déterminés à ne pas se laisser envahir, leur monde est massivement bombardé et la presque totalité de leur espèce anéantie. 

\subsubsection*{An 11 du Calendrier Impérial}
L'armada Lortan est décimée par la Marine Impériale dans une série d'engagements particulièrement violents. Les colonies conquises par les Lortan sont rapidement libérées mais l'on ne parvient pas à découvrir leur point d'origine exact. Plusieurs expéditions armées sont envoyées dans le secteur de l'Espace Sauvage dont ils sont soupçonnés provenir.

\subsubsection*{An 14 du Calendrier Impérial}
Un commando terroriste lâche un virus mortel sur la planète Falleen qui fait plus de 200.000 morts en quelques heures. Le Seigneur Darth Vader est obligé d'ordonner l'oblitération de la ville infectée pour éviter que le virus ne s'étende à l'ensemble de Falleen et ne tue des milliards de victimes. (Le virus en question s'est en fait échappé d'un centre de recherches impérial)

\subsubsection*{An 15 du Calendrier Impérial}
De hauts responsables de la Corporation Incom sont reconnus coupables de malversations mais parviennent à échapper à l'Empire en obtenant l'assistance des terroristes rebelles en échange des plans du tout nouveau prototype de chasseur d'Incom : le T-65 X-Wing. La corporation Incom est nationalisée sur le champ mais les plans du X-Wing ne sont jamais retrouvés

\subsubsection*{An 16 du Calendrier Impérial}
Le Moff Sarn Shild, responsable de la Bordure Extérieure, annonce une grande campagne contre la criminalité, visant en particulier les organisations Hutts.
Une escadre impériale envoyée pour bombarder  Nar Shadda (le principal port de l'espace Hutt) est défaite par une coalition de contrebandiers et de pirates. Darth Vader exécute l'Amiral Greenlanx lorsqu'il découvre que celui-ci avait touché des pots-de-vin pour envoyer ses vaisseaux à leur perte.\\

Accusé d'avoir des responsabilités dans le désastre de cette campagne, le Moff Sarn Shild se suicide. Le Grand Moff Wilhuff Tarkin est placé en charge de toutes les opérations dans la Bordure Extérieure.

\subsubsection*{An 17 du Calendrier Impérial}
Alors que les premiers contacts semblaient prometteurs, les habitants de la planète Mon Calamari exterminent brutalement tous les représentants de l'Empire sur leur monde. Loin des zones contrôlées par l'Empire, Mon Calamari devient un havre pour les renégats, les rebelles et les criminels de toutes sortes. L'Empereur Palpatine décide de remettre à plus tard sa conquête car les troubles se multiplient dans l'espace impérial.\\
	
Le gouverneur Zrie Prakis qui officiait sur Rhynnal est promu Moff du Secteur Telarion.

\subsubsection*{An 18 du Calendrier Impérial}
À l’âge de 16 ans, la princesse Leia Organa d'Alderande devient le membre le plus jeune du Sénat, rapidement célèbre pour ses discours anti-impériaux.\\

Un autre sénateur sécessionniste, Mon Mothma de Chandrila, est déclarée coupable de trahison. Les forces impériales chargées de l'arrêter sur son monde natal se heurtent à la population civile manipulée, ce qui entra\^{i}ne des milliers de morts et permet à la tra\^{i}tresse de s'échapper.\\

Des troubles civils importants amènent l'Empire à déclarer la loi martiale sur la planète Cordell.

\subsubsection*{An 19 du Calendrier Impérial}
Le Bureau de la Sécurité Impériale (BSI) dévoile le rôle fondateur de Mon Mothma dans la naissance d'un nouveau mouvement terroriste : l'Alliance pour la Restauration de la République, dont les déclarations pompeuses se répandent partout dans la galaxie et qui devient rapidement connue sous le nom d'Alliance Rebelle. Les opérations terroristes frappant civils et militaires se multiplient. Plusieurs mondes sont la proie des flammes alors que les rebelles tentent d'en prendre le contrôle et se heurtent à la population et aux forces impériales.\\

Le nouveau sénateur de Chandrila, Omonda, prêche la sécession avec encore plus de vigueur que son prédécesseur. Elle est arrêtée et exécutée tandis que l'Empire établit un blocus autour de Chandrila afin d'éviter la guerre civile.\\

Un programme militaire intensif est mis en \oe uvre pour lutter contre la rébellion.\\

La multiplication des raids pirates et "rebelles" perturbe le commerce galactique et entra\^{i}ne la faillite de nombreuses corporations de petite taille, notamment dans le domaine de la robotique.\\

La rébellion obtient un nouveau chasseur, l’Aile A, réalisé par deux tra\^{i}tres à l'Empire : l'ancien général Jan Dodonna et le concepteur du vieux destroyer de classe Victoire, Walex Blissex.

\subsubsection*{An 20 du Calendrier Impérial}
Le sénateur corellien Garm Bel Iblis meurt dans un attentat à la bombe, imputé à l'Alliance Rebelle\\

Ysanne Isard, fille du directeur des Renseignements Impériaux, découvre que son père complote pour renverser l'Empereur. Le tra\^{i}tre est arrêté, inculpé et exécuté. Ysanne Isard prend sa succession à la tête des services secrets de l'Empire.\\

L'Empereur dissout le Sénat. \\

La planète Alderande est détruite et la galaxie découvre l'existence de l'Etoile Noire. Selon le Haut Commandement Impérial, les Alderaniens se préparaient à utiliser des armes bactériologiques et chimiques sur des mondes voisins, se dissimulant derrière leur apparent pacifisme. (Il est vrai qu'Alderande soutenait l'Alliance Rebelle sur le plan financier et en lui fournissant également des informations ainsi que des fournitures médicales et de la nourriture mais jusqu'à leur anéantissement, les Alderaniens refusèrent de construire ou vendre des armes quelles qu'elles soient). \\

L'Etoile Noire est détruite par l'Alliance Rebelle. Peu de gens croient qu'un simple garçon de ferme à bord d'un chasseur stellaire ait pu y parvenir mais l'espoir prenait dans la galaxie. \\

La Marine Impériale établit un blocus autour du système de Yavin mais, redoutant de tomber sur une mystérieuse arme secrète (la version officielle de la cause de la perte de l'Etoile Noire), elle s'abstient de donner l'assaut au système et doit se contenter de raids épisodiques et d'escarmouches avec des bâtiments rebelles. \\

La hiérarchie militaire impériale est amputée de nombreux officiers compétents avec la perte de la station de combat. Plusieurs mois de réorganisation, des campagnes de propagande interne et des purges s'avèrent indispensables. \\

Crix Madine abandonne le projet Storm Commando et rejoint l'Alliance. \\

La rébellion éclate sur la planète Dentaal, provoquant l'évacuation du Gouverneur et de son entourage. Rapidement, une mystérieuse épidémie s'étend sur ce monde qui finit par se soumettre à nouveau à l'Empire. \\

L'Empereur nomme un de ses fidèles, Ardus Kaine, Grand Moff de la Bordure Extérieure. \\

Les chantiers navals de Fondor (dans le Secteur Tapani) redoublent d'effort pour terminer le projet le plus secret qu'ils aient eu en charge : le Super Star Destroyer Executor. \\

La Flotte Rebelle parvient à quitter le système de Yavin malgré le blocus impérial et commence sa longue quête d'une nouvelle base.\\

L'Executor est placé sous le commandement de Darth Vader et une escadre est assemblée pour l'assister dans sa recherche de la flotte de l'Alliance. \\

Dans le Secteur Telarion un groupe d'agents rebelles découvre le Vangel, un destroyer républicain de classe Victoire II partiellement automatisé disparu durant les Guerres Cloniques (son équipage victime d'une arme bactériologique dérobée à un labo de la Confédération l'avait programmé pour aller se dissimuler près de la Nébuleuse Itani afin que la contagion ne s'étende pas et qu'une mission de secours puisse analyser le virus et lui trouver un antidote mais l'évolution du conflit tourna différemment et la mission du Vangel fut oubliée). Bien que le Vangel soit en mauvais état et nécessite trop de personnel pour opérer à pleine efficacité, le commandement sectoriel de l'Alliance décide néanmoins de le remettre en état et surtout d'utiliser ce qui reste de ses stocks de fournitures pour alimenter les forces rebelles dans le Secteur Telarion et plusieurs systèmes voisins. La mystérieuse arme bactériologique est détruite après analyse. 


\subsection{Principales divisions géographiques}
Avec un diamètre dépassant les 120.000 années lumières et plus de 100 millions d'étoiles, la Galaxie est loin d'être entièrement connue à ce jour. L'espace autrefois contrôlé par l'Ancienne République et désormais aux mains de l'Empire ne compte en proportion que quelques millions de mondes habités et plusieurs milliers d'espèce intelligentes. Pour être plus clair, la grande majorité de la galaxie reste à explorer.\\

Au sein même de l'espace connu, il demeure de nombreuses zones mal cartographiées ou presque inexplorées car en marge des routes commerciales ou difficilement accessibles par l'hyperespace.\\

Du temps de l'Ancienne République, la galaxie connue était découpée en Secteurs, chaque Secteur représentant cinquante mondes habités. Les secteurs du Noyau, à proximité du centre galactique ou les étoiles sont assez proches les unes des autres et les civilisations fondatrices de la République anciennes, étaient donc de taille relativement petite par rapport à leurs homologues de la périphérie plus dispersés et en cours de colonisation. Mais avec les millénaires, de nouvelles planètes autrefois inexploitables devinrent des colonies à part entières et cette division administrative bien qu'encore utilisée n'est plus aussi fiable sur un plan numérique.\\

Les cartographes contemporains retiennent cependant ce découpage sectoriel et les secteurs eux-mêmes sont inclus dans des ensembles plus vastes, les Régions. Le C\oe ur Galactique

\subsubsection{Le C\oe ur Galactique}
Le centre de la galaxie est presque inexploré malgré sa proximité avec les principaux mondes habités, principalement parce que la densité d'étoiles en formation, de corps célestes divers et de perturbations électromagnétiques ou solaires y est telle que la navigation hyperspatiale est extrêmement hasardeuse. Le plus souvent, on préfère contourner le C\oe ur plutôt que le traverser lorsque l'on veut passer d'un bord à l'autre. L'Empire a cependant mis sur pied un programme pour trouver des routes viables dans le C\oe ur Galactique. 

\subsubsection{Les Mondes du Noyau}
C'est dans cette région que la civilisation humaine fit son apparition et c'est aussi dans ce coin de la galaxie que l'on utilisa les premiers hyperpropulseurs. La relative proximité des étoiles dans le Noyau a fait qu'avant même cette découverte technologique révolutionnaire, des échanges entre systèmes proches avaient déjà eu lieu et plusieurs empires éphémères y avaient existé. A l'heure actuelle, les Mondes du Noyau sont parmi les plus densément peuplés et les plus riches de la civilisation galactique.\\

Les Mondes du Noyau, à de très rares exceptions près, représentent aussi le plus puissant bastion de l'Empire. Leurs habitants s'intéressent très peu aux régions périphériques et l'Empire n'a guère d'intérêt à leur montrer les nombreuses exactions qu'il y commet quotidiennement. C'est donc au sein de ces mondes que l'Empire Galactique recrute la majorité de ses soldats, de ses officiers, de ses administrateurs et ou sa politique est la plus favorablement accueillie car les Mondes du Noyau ont généralement un statut relativement privilégié dans la structure impériale et subissent peu de contraintes directes la plupart du temps. Certains ont même grandement profité de la politique impériale. La civilisation humaine est originaire du Noyau et certaines de ses plus anciennes traditions y sont encore pratiquées.

\subsubsection{Les Anciennes Colonies}
Comme son nom l'indique, cette région a été le théâtre de la première vague d'expansion interstellaire depuis le Noyau. On appelle toujours les mondes de cette région "les colonies" bien qu'elles aient dans leur grande majorité développé leur propre indépendance financière, politique et militaire depuis des millénaires. Il se peut que dans un avenir lointain les civilisations vieillissantes du Noyau soient lentement absorbées par les Colonies relativement plus "jeunes". De fait, à l'heure actuelle les Colonies ont pour seule différence réelle avec le Noyau d'avoir été peuplées ultérieurement.

\subsubsection{La Bordure Intérieure}
Autrefois, lors des premières étapes de son exploitation, la Bordure Intérieure était simplement "la Bordure", et elle fut pendant longtemps une région frontalière. Lorsque ses principaux mondes devinrent des planètes industrialisées avec une démographie galopante et qu'elle dut accueillir les surplus de population des Colonies et du Noyau, elle se lança dans l'exploitation de ce qui était appelé à l'époque "la bordure étendue" et devait devenir par la suite La Région d'Expansion. Depuis l'avènement de l'Empire, la Bordure Intérieure a été mise en coupe réglée et l'on y assiste à beaucoup d'exactions par rapport à ce qui se passe dans les Colonies et le Noyau. Par voie de conséquence, beaucoup de gens ont déménagé vers des mondes plus périphériques mais cela n'a pas encore eu de répercussions directes sur la démographie. Les gouverneurs impériaux peuvent en effet pressurer à volonté les riches mondes de la Bordure Intérieure sans avoir à compter avec l'influence du gouvernement central ou des puissantes familles nobles des planètes du Noyau.

\paragraph{Le Consortium d'Hapes}
Un groupe d'étoiles comptant soixante-trois planètes habitées aux ressources naturelles incroyables, le Consortium est, depuis plusieurs siècles, synonyme d'isolationnisme, de frontières hermétiques et de rumeurs fabuleuses sur ses richesses comme les gemmes arc-en-ciel, les arbres de sagesse et les armes de contrôle psychique. Hapes serait dirigé par une Reine-Mère mais comme le Consortium n'a aucun échange commercial ou culturel avec le reste de la galaxie et que ses navires (les fameux Dragons d'Hapes) détruisent sans pitié tous ceux qui font intrusion dans son espace, personne ne saurait confirmer cela avec certitude. L'Empire veille sur son coté de la frontière avec l'espace du Consortium mais ne semble pas du tout intéressé par ce qui se passe à l'intérieur.

\subsubsection{La Région d'Expansion}
Lorsqu'elle fut ouverte à la colonisation il y a plusieurs millénaires, le Sénat fit l'expérience de laisser les grandes corporations investir dans l'exploitation de ce secteur de la galaxie riche en métaux lourds et qui vit l'apparition d'innombrables chantiers navals. Malheureusement, le temps montra que les populations locales et les colons n'avaient guère d'affinités avec les grands trusts et que la République éprouvait de grandes difficultés à influencer la politique locale. Il fut finalement décidé d'aménager le statut de cette région : la République serra la vis et les grandes compagnies se virent offrir ce qui allait devenir par la suite le Secteur Corporatif en guise de compensation. Cela ne fut pas trop difficile à négocier puisque les considérables ressources minières de la Région d'Expansion commençaient déjà à s'épuiser après des millénaires d'exploitation totale. Durant les derniers siècles, la région connut une longue et grave crise économique mais l'économie locale parvint à prendre un second souffle en réorientant son activé commerciale. Désormais, la plupart des bénéfices significatifs dégagés dans l'Expansion sont en effet produits par la reconversion de nombreux systèmes stellaires en  "systèmes portuaires". L'affrètement des flottes civiles et militaires est devenu une spécialité dans l'Expansion qui forme la zone intermédiaire entre les mondes du centre et les régions en plein développement. L'Empire est plutôt apprécié dans cette région car les gros contrats d'affrètement militaires pour les convois de ravitaillement et de transport de pièces détachées ont évité à de nombreuses planètes de sombrer dans le marasme économique. Les anciennes corporations nées sur place ou dans le Noyau y sont encore très présentes mais elles ont beaucoup réinvesti dans le Secteur Corporatif.


\subsubsection{La Bordure Moyenne}
Région encore assez peu peuplée, la Bordure Moyenne demeure par bien des aspects une zone frontalière ou les gens ont conservé la mentalité des peuples situés loin des centres de décision et disposant de colossales ressources inexploitées. De nombreux secteurs de la Bordure Moyenne sont encore inexplorés et totalement en friche et la piraterie y est très intense même si l'émergence progressive de centres économiques importants donne à penser que la situation sera très semblable à celles des mondes de la Bordure Intérieure d'ici quelques siècles.

\paragraph{L'Espace Hutt}
Pratiquement à cheval sur la Bordure Moyenne et la Bordure Extérieure, cette partie de la galaxie est contrôlée par les puissants clans Hutts, une race tristement célèbre pour sa cruauté et son opportunisme qui l'a amené à prendre une part active dans la criminalité galactique. Malgré sa xénophobie avérée, l'Empire n'exerce qu'une présence des plus symboliques dans l'Espace Hutt, sans doute le résultat d'inavouables arrangements. Peu d'honnêtes gens se rendent sur les mondes contrôlés par les Hutts s'ils ont le choix\ldots ceux-ci ont d'ailleurs tendance à considérer que tous ceux qui entrent dans leur espace sont hostile s'ils ne sont pas affiliés à un de leurs clans ou à l'Empire\ldots

\subsubsection{Les Territoires de la Bordure Extérieure}
Les immenses étendues stellaires qui forment la Bordure Extérieure reflètent bien la diversité d'une zone frontalière : quelques mondes richissimes côtoyant des systèmes entiers à peine habités, opportunisme, piraterie, civilisations étrangères ayant leur propre sphère d'influence locale, hordes de pirates et de chasseurs de primes, etc. La présence impériale y est plus diffuse que partout ailleurs mais la loi y est aussi appliquée de manière beaucoup plus radicale et la plupart des individus sensés sont armés. Le plus inoffensif cargo opérant dans la Bordure Extérieure est armé de telle manière qu'il serait pratiquement considéré comme un appareil de pirates dans des régions plus "civilisées". L'esprit farouchement volontaire et indépendant de nombreux citoyens de la Bordure Extérieure, le nombre considérable de hors la loi et de pirates qui y résident ou y opèrent compliquent beaucoup la vie des responsables impériaux mais c'est aussi une région ou les jeunes officiers qui font leurs preuves peuvent espérer une progression rapide dans la hiérarchie.

\paragraph{Secteur Corporatif}
Malgré le strict contrôle exercé à l'origine par la République sur cette région de l'espace qui devait devenir le Secteur Corporatif, de  nombreuses corporations du Noyau qui avaient mis à sac la Région d'Expansion décidèrent d'investir dans cette région isolée qui offrait cinq avantages : un grand nombre de routes hyperspatiales extrêmement pratiques (qui avaient permis une exploration intensive du secteur alors que le reste de la Bordure Extérieure était encore terra incognita), de nombreuses ressources naturelles vierges de toute exploitation, une seule autorité régulatrice (la République), aucune taxe gouvernementale locale et surtout l'absence remarquable de vie intelligente sur les 10.000 systèmes explorés (qui permettrait d'éviter la plupart des conflits d'intérêts qui étaient survenus dans la Région d'Expansion).\\

Après des siècles d'existence, le Secteur Corporatif devint un des principaux atouts de l'Empire lors de la construction de la machine de guerre impériale. Palpatine accepta alors la proposition des grandes corporations d'établir une autorité centrale autonome sur le Secteur à la place de la tutelle républicaine. En échange, l'Empire reçoit taxes, matières premières et produits finis en grandes quantités, le tout sans avoir à assurer l'administration et la défense de la région la plus productive de la galaxie.\\

Le "gouvernement" du Secteur Corporatif est en fait une sorte de super-conglomérat, la Corporate Sector Authority (CSA). La CSA assure toutes les facettes administratives, financières et militaires du Secteur. En acceptant que certains de leurs bâtiments et actifs locaux soient nationalisés au sein de la CSA, les corporations membres deviennent en échange actionnaires du super-trust (à hauteur de leur investissement) et touchent donc des dividendes conséquents. Des centaines de corporations participent à la CSA et des milliers opèrent dans le secteur sous le contrôle d'une autorité qui leur convient parce que toutes ses lois et réglementations sont axées vers un seul but : le bénéfice.

\subsubsection{Au-delà de l'Espace Connu}

\paragraph{L’Espace Sauvage}
La presque totalité des étoiles à la frontière de la Bordure Extérieure n'ont jamais été visitées par une sonde ou un vaisseau éclaireur. En dehors de sectes isolationnistes, de quelques entrepreneurs originaux (ou inconscients) et des éclaireurs indépendants ou affiliés à l'Empire, la civilisation originaire du Noyau est quasi-inexistante dans cette partie de la galaxie, la phase d'explosion coloniale s'étant progressivement tarie durant les derniers siècles. Les éclaireurs impériaux sillonnent cependant de plus en plus souvent les routes de cette région, en quête de civilisations technologiquement avancées qui pourraient constituer des menaces ou des atouts pour l'Empire.

\paragraph{Les Régions Inconnues}
Comme leur nom l'indique, il n'existe pratiquement aucune carte des millions de systèmes qui se trouvent au-delà des derniers avant-postes de la civilisation galactique. En dehors de quelques expéditions menées par des fous et d'une ou deux "colonies perdues" dont on ignore ce qu'elles ont bien pu aller faire si loin, on ignore presque tout de ces étendues presque infinies qui forment encore la grande majorité de la galaxie. Même les meilleurs pilotes qui opèrent dans la Bordure Extérieure ou l'Espace Sauvage n'ont pas grand-chose de concret à dire sur les Régions Inconnues. De temps en temps, une légende, une rumeur provient de ces régions inexplorées mais rares sont ceux qui s'y aventurent et encore plus rares ceux qui partent à leur recherche lorsqu'ils disparaissent.

\subsection{Légendes}
Voici quelques-unes des plus significatives parmi les millions de légendes et de rumeurs qui courent depuis des millénaires dans toute la galaxie. Ce ne sont pas forcément les plus connues mais elles donnent une idée assez précise de ce que l'amateur pourrait trouver en cherchant des choses de ce genre. Certaines s'inspirent de faits historiques avérés, d'autres sont probablement complètement fictives et un bon nombre sont sans doute très éloignées des faits qui leur ont permis de voir le jour. Comme de juste, le pouvoir central a parfois jugé nécessaire d'ajouter ses propres travaux à une légende dérangeante ou de la rendre méconnaissable\ldots

\subsubsection{Les sorcières de vie}
Il existe des femmes à la beauté surhumaine et à la richesse incroyable qui vivent au sein de la civilisation galactique. On dit que ces sorcières sont immortelles et qu'elles recherchent les hommes aisés. Celui qu'elles épousent bénéficie pendant quelques années d'une santé et d'une acuité d'esprit exceptionnelles avant de mourir brutalement, laissant sa veuve un peu plus riche et libre de choisir une nouvelle victime.

\subsubsection{La planète Exo}
La planète Exo est depuis longtemps une légende que plusieurs civilisations humaines et non humaines connaissaient avant d'entrer en contact les unes avec les autres. Il s'agirait d'un monde idyllique regorgeant des trésors et de la sagesse d'une civilisation disparue depuis longtemps. Durant l'Ancienne République, il y a plusieurs milliers d'années, un prophète auto-proclamé finança la construction de gigantesques vaisseaux-colonies a bord desquels embarquèrent des centaines de milliers d'adeptes de sa croyance.\\

On retrouva certains d'entre eux dérivant dans l'espace à la suite d'accidents fatals, quelques-uns firent demi-tour à la suite d'une mutinerie, une poignée fut redécouverte longtemps après sur des mondes qui n'avaient rien d'idyllique mais ou les adeptes avaient été forcés d'atterrir et la plupart disparurent sans laisser de traces. Peut-être l'un de ces navires  est-il finalement arrivé à trouver Exo, à moins que cette planète n'existe que dans les légendes.

\subsubsection{Le trésor de Xim}
Sur la planète Dellalt, le despote légendaire Xim avait enterré dans des caveaux indétectables par la technologie la plus moderne des monceaux de richesses pillées durant son règne, qui attendent encore celui qui les découvrira malgré les innombrables fouilles effectuées depuis la fin du tyran, il y a plus de 250 siècles.

\subsubsection{Le monde des spectres}
Il existe un monde où seules des entités spectrales habitent, invisibles aux mortels et foncièrement cruelles, elles guettent l'explorateur imprudent qui se pose sur leur planète pour le détruire. Les anciennes légendes des Hutts parlent de ce monde qui se trouverait dans leur espace mais de toute manière, qui irait explorer l'espace des Hutts à la recherche d'une mort certaine ?

\subsubsection{Nuniok Dak}
Plusieurs peuples qui sont depuis longtemps des voyageurs spatiaux impénitents comme les Duros ou les Sullustains disent que dans l'espace demeure le démon Nuniok Dak qui se manifeste parfois en rendant les cartes illisibles ou en faussant les repères d'astronavigation, condamnant les navires à une errance sans fin qui les fera sauter d'une étoile à l'autre sans jamais retrouver un système connu ou une planète habitable. Les légendes sur les vaisseaux fantômes que l'on rencontre à l'occasion dans les profondeurs reculées de l'espace sont certainement basées sur celle-ci.

\subsubsection{Les Anciennes Guerres}
Il y a plusieurs dizaines de milliers d'années, longtemps avant que naisse la République, des civilisations déjà anciennes se livrèrent à des conflits titanesques, détruisant des planètes, éteignant des étoiles, décimant des peuples entiers jusqu'à leur extermination presque totale. Les descendants de ces combattants sont depuis longtemps retournés à la barbarie et certains ont même oublié jusqu'à l'intelligence mais dans des systèmes mal connus, sur des mondes isolés, dans des champs d'astéroïdes inexplorés, leurs arsenaux automatiques continuent de fabriquer des armes surpuissantes désormais capables de combattre sans leurs maitres. Un jour, un signal malencontreux, une découverte fortuite, une conjonction stellaire programmée depuis des millénaires réactiveront l'un de ces arsenaux et par réaction en chaine tous les autres, enflammant à nouveau toute la galaxie dans une guerre déjà terminée depuis longtemps. D'ailleurs, on dit qu'il arrive encore parfois qu'une planète reculée soit attaquée par un navire inconnu, dirigé par des systèmes automatiques qui cherchent à anéantir une cible disparue.

\subsubsection{Les Siths}
Une race non-humaine ancienne et capable d'utiliser des pouvoirs de domination mentale terrifiants, les Siths furent vaincus par l'Ordre Jedi et la République après de longs siècles de lutte dans l'ombre sans que la plupart des gens soupçonnent leur existence. Certains disent qu'ils se sont réfugiés dans une nébuleuse inaccessible, d'autres qu'ils se cachent au-delà de l'espace connu ou encore qu'ils habitent au c\oe ur de la Gueule, cet amas de trous noirs scientifiquement impossible que l'on trouve près de Kessel. Et un jour, lorsque leurs agents dans la galaxie jugeront que le moment est venu, ils reviendront.

\subsection{l'épice}
Sous le terme d'Épice, on rassemble un certain nombre de substances d'origine organique ou minérale qui ont les caractéristiques communes suivantes :

\begin{itemize}
	\item elles sont utilisées sous forme cristalline ou poudreuse
	\item on peut les utiliser par inhalation ou oralement (et parfois même par injection)
	\item elles génèrent des phénomènes d'accoutumance et de dépendance (à partir d'un certain dosage)
	\item elles procurent des sensations, voire des capacités inhabituelles à leurs utilisateurs (à partir d'un certain dosage)
	\item elles ont aussi des vertus culinaires, médicinales ou autres qui font qu'un usage restreint ou à doses très minimes peut s'avérer intéressant et moins dangereux, voire inoffensif
	\item elles font effet sur la presque totalité des espèces intelligentes mammifères et un grand nombre d'autres espèces.
	\item elles sont difficiles à trouver (généralement on ne les trouve que sur un seul monde à l'état naturel) et trop onéreuses ou impossibles à synthétiser
	\item la réglementation empêche la plupart des gens d'y avoir accès légalement et leur contrebande est donc florissante (et sévèrement punie).
\end{itemize}

Comme on peut le voir, l'Épice, ou plus exactement les épices, représentent le débouché commercial le plus intéressant en matière de produits stupéfiants de la galaxie.\\

Parmi les centaines de substances qui peuvent donc être rassemblées sous la dénomination "épices", les quatre plus connues dans la galaxie sont les suivantes.

\subsubsection{Le Glitterstim}
La fameuse épice extraite des mines de Kessel. Le Glitterstim à l'état naturel se présente sous forme de brins souples parsemés de cristaux noirs sensibles à la lumière du spectre visible (les mineurs de Kessel opèrent dans l'obscurité ou aux infrarouges pour éviter d'activer les cristaux et ainsi de les rendre inutilisables). Les cristaux de Glitterstim sont délicatement retirés des brins (les cristaux réduits en poudre avant raffinage sont nettement moins efficaces) avant d'être traités et conditionnés. Le Glitterstim est le plus souvent conditionné dans des fioles ou emballé dans des cylindres de papier opaque. Pour s'en servir, il suffit d'exposer les cristaux à la lumière quelques secondes (le temps qu'ils deviennent d'un beau bleu luminescent) et d'avaler la dose ou de l'inhaler. Si on n'utilise qu'une fraction de dose, le Glitterstim possède des propriétés euphorisantes et toxiques légères qui s'accroissent au fur et à mesure que l'on s'approche d'une dose standard bien que la sensation d'euphorie demeure relativement faible.\\

Ce qui rend le Glitterstim particulièrement recherché est le fait qu'à partir d'un dosage standard, il confère pendant quelques secondes à quelques minutes (selon les personnes) une sorte d'intuition télépathique à son utilisateur qui devient capable de focaliser son esprit sur celui d'une créature proche et de lire ses pensées. Un usage trop fréquent de Glitterstim sous son dosage standard rend fréquemment l'utilisateur définitivement aveugle et/ou mentalement instable et certaines espèces ne peuvent pas supporter cette substance. L'épice s'accumule en fait progressivement dans les tissus de manière résiduelle  et même une utilisation régulière de fractions de doses (pour bénéficier uniquement des propriétés euphorisantes) peut générer à la longue des troubles de la vue ou du comportement définitifs.\\

A l'heure actuelle, l'Empire contrôle la production et la vente de Glitterstim par le biais de la Compagnie des Épices de Kessel, une émanation du pouvoir impérial chargée d'administrer la planète-prison et dont les seuls clients sont l'Empire lui-même et certaines familles nobles ou corporations pharmaceutiques. Comme on peut s'en douter, la contrebande de Glitterstim est florissante et le "Run de Kessel" est un des trajets les plus dangereux et les plus lucratifs que puisse faire un contrebandier entreprenant. Le Glitterstim est utilisé en psychiatrie mais surtout par les interrogateurs impériaux (qui lui préfèrent cependant les "bonnes vieilles méthodes" qui ont l'avantage d'être moins nuisibles à la santé de celui qui les emploie\ldots). Sur le marché noir, un certain nombre d'escrocs et de négociateurs s'en servent également ainsi que des amants fortunés pour partager une intimité accrue.

\subsubsection{Le Ryll}
Un minéral extrait du sol de la planète Ryloth (le berceau de la race Twi'lek). Le Ryll se présente comme une poudre minérale grisâtre généralement utilisée après raffinage par injection, inhalation ou sous forme d'emplâtres. Il possède de nombreuses vertus médicinales analgésiques et cicatrisantes. Le Ryll est également un hallucinogène puissant et même à doses médicinales, il génère facilement accoutumance et dépendance si on l'administre sans précautions. Cette épice est normalement vendue uniquement par des circuits commerciaux ciblant les hôpitaux, médecins et services médicaux de la galaxie. Le Ryll entre également en infimes proportions dans la composition d'un grand nombre de drogues de synthèse. 

\subsubsection{Le Carsunum}
Une épice de couleur noire originaire de la planète Sevarcos et extrêmement rare dans le sous-sol de ce monde. Sur un plan purement scientifique, le Carsunum n'a guère de valeurs médicinales ou autres mais c'est un euphorisant extrêmement puissant qui procure une sensation de force et de bien-être pendant plusieurs heures. La "descente" est cependant nettement moins agréable (dépression, agressivité, troubles physiologiques variés\ldots) et les syndromes de manque ou de sevrage très fréquemment mortels. Curieusement, bien que l'Empire monopolise l'exportation du Carsunum qui est illégal en dehors de son utilisation comme composant dans certains médicaments et certaines drogues militaires, il existe un marché clandestin florissant de cette épice en dehors des toxicomanes.
En effet, la légende veut qu'un ordre de guérisseurs oublié ait autrefois utilisé le Carsunum pour élaborer des remèdes à de grandes épidémies qui ont dévasté plusieurs mondes. Un certain nombres de gens voient donc cette épice comme une sorte de panacée miraculeuse ou de symbole religieux et l'achètent afin de la conserver "en cas de besoin" ou comme porte-bonheur. Même dans les riches mondes du Noyau ou l'Empire exerce toute sa puissance, certains personnages influents portent autour du cou une chaine avec une petite fiole remplie de Carsunum censée écarter la maladie.

\subsubsection{L'Andris}
Également extraite du sol de Sevarcos, cette épice sous sa forme naturelle est de couleur beige. Elle est, depuis des millénaires, utilisés comme agent préservateur et condiment dans la nourriture de nombreuses espèces. Une fois raffiné à hauteur de 25\%, l'andris se présente sous la forme de fins cristaux blancs et est encore plus efficace pour assaisonner la nourriture ou aider à sa préservation conjointement avec le froid. Des centaines d'espèces utilisent quotidiennement de l'Andris 25  dans la préparation de leur repas quotidien, une simple pincée pouvant relever le gout d'un plat familial. Avec un seuil de raffinement supérieur à 25\% (très difficile à obtenir sans les chimistes et équipements adéquats), l'Andris acquiert des propriétés euphorisantes  (et toxiques) de plus en plus puissantes, un seuil de pureté de 100\% étant souvent mortel. L'Andris raffiné à 25\% est en vente libre mais l'épice brute fait l'objet d'une contrebande intensive et les réseaux de distribution d'Andris 40, 50 ou même 100 sont innombrables.\\

Les toxicomanes utilisent généralement l'Andris par voie orale ou par inhalation. L'andris est assez peu stable sur un plan moléculaire et il est pratiquement impossible d'utiliser de l'Andris 25 afin de lui faire subir un second raffinage le transformant en drogue, ce qui oblige donc la plupart des trafiquants à rechercher de l'Andris brut qui pourra être raffiné à leur convenance.


\subsection{Minerais, matériaux, gaz et alliages}

\begin{description}
	\item[Baradium] Un minerai lourd extrêmement dangereux à manipuler. On ne s'en sert que comme explosif et il est un composant essentiel d'outils de destruction comme les détonateurs thermaux, divers types de missiles et certaines grenades offensives.;
	\item[Carbonite] Un alliage dont les composants sont souvent trouvés dans la même écosphère ce qui facilite grandement sa fabrication. La Carbonite à très basse température est utilisée pour le transport de certains gaz ou de matières organiques congelés dans des blocs faciles à manier. Il s'agit certainement de la méthode de conservation la plus fiable, tout au moins tant qu'on n'essaye pas de s'en servir sur des êtres vivants, ce qui nécessite des manipulations assez délicates.
	\item[Chanlon] Un métal très dense et rare qui est parfois employé tel quel mais sert surtout à composer des alliages extrêmement résistants.
	\item[Cortheum] Minéral cristallin employé pour la construction des photorécepteurs de droïdes et autres systèmes d'acquisition optiques. Existe aussi sous forme gazeuse, le Corthel, parfois raffiné et utilisé comme gaz à blaster de faible rendement.
	\item[Durelium] Un des métaux utilisés pour les parois des réacteurs ou des chambres à combustion
	\item[Hfredium] Un des métaux les plus communs utilisés sous forme d'acier pour la fabrication de coques de navires spatiaux
	\item[Hyperbarides] Une famille de métaux particulièrement résistants à la chaleur et aux radiations. La plupart des turbolasers lourds dont on équipe les stars destroyers, les croiseurs ou les stations spatiales militaires ont une gaine d'hyperbarides. Le processus de raffinage des hyperbarides bruts est particulièrement polluant au niveau toxique et radioactif. Malheureusement, il dégage des perturbations électromagnétiques telles que l'on est obligé d'utiliser une main d'\oe uvre organique bon marché, les droïdes spécialement protégés coutant bien trop cher ou étant jugés moins économiques. On se sert également des hyperbarides pour les tuyères d'échappements des vaisseaux spatiaux.
	\item[Lommite] Un minéral transparent extrêmement résistant utilisé pour la fabrication du Transparacier. La planète Elom en est la principale source mais on peut trouver de la lommite sur de nombreux mondes.
	\item[Permabéton] Un des principaux matériaux de construction employés dans la galaxie, il s'agit essentiellement de béton ordinaire dont la structure est renforcée par des composants cristallins. Le permabéton est bien évidemment très résistant mais forme aussi un bon isolant sonore et thermique. Il accepte de nombreuses teintures plus ou moins complexes sans le moindre problème.
	\item[Phrik] Un des métaux les plus utilisés dans la fabrication d'armures individuelles, notamment pour sa légèreté. Officiellement, l'exploitation du Phrik est illégale et l'Empire nationalise toutes les firmes qui sont soupçonnées d'y procéder.
	\item[Phobium] Un alliage à base de Durelium utilisé pour les noyaux énergétiques des plus grands navires ou stations spatiales.
	\item[Quadrillium] Un métal brut incontournable dans la fabrication de pratiquement toutes les coques de navire de la galaxie. Les navires construits en quadrillium pur prennent une teinte bleu-gris facilement reconnaissable (un grand nombre de TIE entrent dans cette catégorie).
	\item[Tibanna] Le gaz de Tibanna est un élément essentiel de la technologie des blasters. En effet, c'est par l'excitation de ses molécules que le faisceau d'énergie d'une arme de ce type est produit. De nombreux gaz ont des propriétés analogues à celles du Tibanna mais il est le seul à n'avoir aucun besoin d'être raffiné. On le trouve en abondance sur Bespin et en traces significatives dans l'atmosphère d'autres mondes gazeux.
	\item[Transparacier] Un composé de verre et de Lommite, le transparacier est utilisé pour les hublots et cockpits de navires spatiaux ainsi que dans de nombreux autres domaines grâce à sa grande résistance et à sa capacité induite à foncer en fonction de l'intensité lumineuse.
\end{description}


\section{L'Empire}

\subsection{L'Idéologie Impériale}
En la résumant à l'extrême et en la débarrassant de toutes les fioritures des propagandistes, l'idéologie de l'Ordre Nouveau peut être énoncée de la manière suivante. 

\begin{itemize}
	\item La galaxie doit être administrée par une autorité forte et dépourvue de la corruption de la "vieille" République.
	\item Seul l'Empereur Palpatine a su éviter le pire et restaurer la grandeur de la civilisation galactique
	\item L'Ordre Nouveau doit s'étendre et son influence finira par embrasser tous les mondes de la galaxie
	\item Lutter contre L'Ordre Nouveau est synonyme de lutter contre la volonté de l'Empereur sans lequel la civilisation galactique aurait disparue. Lutter contre la volonté de l'Empereur est puni par la mort.
	\item Les instances de la République (y compris l'Ordre Jedi désormais éradiqué) sont des survivances héréditaires qui ont permis à des générations de gens médiocres de s'accrocher au pouvoir et de tromper les peuples. Ces survivances archaïques doivent cesser.
	\item Enfin, toutes les espèces intelligentes peuvent rejoindre l'Empire et s'y intégrer mais en raison de leur nombre et des sacrifices qu'ils ont consentis pour permettre son existence, les humains se doivent d'être les tenants de l'Ordre Nouveau et ses principaux piliers.  En effet, contrairement aux autres espèces, les humains ont des origines diverses (Coruscant, Corellia, Chandrila\ldots) et sont actuellement la race la plus impliquée à tous niveaux dans la politique et l'économie galactique. Ils sont historiquement destinés à perpétuer la grandeur de la civilisation galactique. 
\end{itemize}

Sur un plan pratique, cette idéologie fasciste s'exprime de plusieurs manières.

\subsubsection{Endoctrinement}
Par le biais des médias bien sûr mais aussi des organisations de jeunesse, des chantiers collectifs "d'intérêt public", des rallyes politiques constants, etc\ldots Plus un monde embrasse l'idéologie impériale et plus ses structures politiques sont apparemment laissées intactes tout en étant remaniées en profondeur et en douceur. Ainsi, il existe une multitude de régimes "démocratiques" dans la galaxie qui sont en fait de simples paravents à l'autorité impériale et dont le quotidien n'a plus de "démocratique" que le nom. Lorsque l'Empire opère à visage découvert, les choses sont souvent encore plus tranchées. Bien évidemment, tous ceux qui contestent, ne serait-ce que verbalement l'Ordre Nouveau, sont des terroristes en puissance, surtout s'ils sont non-humains. Dans le meilleur des cas, ils sont dépeints comme des pions manipulés par les poseurs de bombes et qui doivent être "rééduqués" par le travail.

\subsubsection{Nivellement culturel}
À moyen terme (quelques dizaines d'années tout au plus), l'Empire entend bien faire progressivement disparaitre la plupart des croyances et philosophies incompatibles avec sa façon de voir les choses. A l'heure actuelle, bien que toutes les races soumises à Palpatine soient ciblées, les humains sont plus particulièrement concernés. Outil privilégié de l'Empereur, l'espèce humaine sous toutes ses variantes doit devenir un instrument homogène, stérile, désincarné et prêt à le servir partout et à tout moment. Quelques souverains fantoches et des festivals ou monuments pas trop gênants seront conservés sous prétexte de "préserver le capital culturel humain" mais dans le fond, un modèle unique sera lentement mais surement mis en place et il servira de justification à la croissance de la fameuse "Haute Culture Humaine" qui soudera l'humanité sous une bannière unique en lui faisant "prendre conscience de sa destinée historique".

\subsubsection{Ségrégation}
L'idéologie de Palpatine tire parti de tous les vieux démons qui sommeillent dans chaque être intelligent. En mettant en avant la puissance d'une humanité "supérieure", il incite ainsi les autres races à s'y soumettre tout en laissant à leurs représentants les plus opportunistes la possibilité de devenir de fidèles seconds. Ainsi, les uniformes des différentes branches de l'Ordre Nouveau sont arborés uniquement par les humains mais de nombreux "auxiliaires précieux" de toutes les races les secondent et espèrent bien se tailler leur part du gâteau impérial. Ce phénomène permet à ces opportunistes d'acquérir un statut un peu privilégié tout en garantissant au moyen d'un cercle particulièrement vicieux leur fidélité : leur statut n'étant que le résultat de leur dévouement et leur origine non-humaine étant susceptible de les faire "redescendre en bas" à tout moment, seuls un plus grand dévouement et une obéissance accrue peuvent leur permettre de demeurer dans cette situation confortable. Comme toutes les idéologies totalitaires, l'Ordre Nouveau se nourrit des ambitions des faibles et les broie dans ses rouages en les incitant à donner toujours plus pour conserver des "privilèges" payés en fidélité, en argent et souvent en sang\ldots\\

Certaines races qui sont restées passives face à l'expansion impériale ou qui ont su se rendre utiles bénéficient de tels aménagements de manière collective. Par exemple, la corporation sullustaine SoroSuub est un des principaux fournisseurs de l'Empire et par voie de conséquence, le peuple sullustain est rarement l'objet de mesures discriminatoires officielles en dehors des "mesures de base". A l'opposé, les Mon Calamari qui ont pris les armes sont presque systématiquement emprisonnés ou abattus à vue dès qu'ils s'aventurent dans l'espace impérial.
Il est également significatif que parmi les nombreuses femmes humaines qui portent l'uniforme et agissent avec dévouement pour l'Empire, très peu parviennent à se hisser à des postes d'importance significative dans l'Armée ou la Marine. Les choses sont un peu moins tranchées dans le COMPORN, les Renseignements ou la plupart des administrations mais l'Empire demeure fondamentalement un régime machiste qui met en parallèle pouvoir, puissance militaire et virilité. Dépourvu des paravents de la propagande et des incohérences inhérentes à tout système politique, l'Empire apparait tel qu'il est en réalité : une machine de guerre et de conquêtes.

\subsubsection{Féodalisme}
L'Ordre Nouveau distingue toujours les Forts et les Faibles. Les premiers sont censés "protéger" les seconds qui en échange leur "donnent" le droit de régenter leur vie. Palpatine sait qu'il peut compter sur beaucoup de monde en dehors même de la hiérarchie impériale pour appliquer cette idée. En tête de la liste, on trouve les familles nobles et les grandes corporations (et assez souvent les familles nobles qui possèdent les corporations). Celles qui font la preuve de leur dévouement se voient autorisées à régenter comme elles le veulent leur petit coin d'espace ou leur entreprise\ldots formant une multitude de vassalités au sein desquelles n'importe quel petit tyran héréditaire ou n'importe quel industriel mégalomane peut exercer son pouvoir à sa guise tant qu'il obéit (au moins sur le papier) aux lois impériales. Comme tout système féodal, l'Empire repose sur le principe d'une autorité supérieure qui fixe des règles générales et peut intervenir à tout moment pour faire rentrer les vassaux dans le rang et, contrairement à de nombreux régimes féodaux, l'Empire a effectivement les moyens d'agir de la sorte. Tous ces petits nobliaux et magnats ne sont donc en fait que des extensions du pouvoir central, quoi qu'ils puissent en penser. En cas de divergences ou d'un manque prononcé de zèle, il est toujours aisé de nationaliser un conglomérat ou de faire exécuter un noble afin de laisser la place à un héritier plus obéissant ou une autre famille plus ambitieuse. Ainsi, Palpatine sait que les complots et la concurrence commerciale suffisent presque toujours à assurer ses arrières au niveau de ses "fidèles alliés". Même la gigantesque puissance financière du Secteur Corporatif paye sa dime à l'Empereur et ne perd jamais de vue qu'un grand nombre de compagnies qui ne sont pas dans ses rangs donneraient beaucoup pour s'emparer des marchés qu'elle détient\ldots

\subsubsection{Répression}
L'Empire repose sur un principe simple au niveau de la liberté et de l'ordre : l'Empereur à toujours raison. Ceux qui prétendent le contraire doivent être "persuadés", "rééduqués" ou "neutralisés". Tous les représentants du pouvoir impérial sont censés parler pour l'Empereur et ils sont donc aussi infaillibles que lui, tout au moins aux yeux des masses. Étant donné qu'il existe des groupes dissidents terroristes, certains s'en prenant même à des cibles civiles, il est aisé d'assimiler tous les contestataires et les rebelles (y compris ceux de l'Alliance) à ces "traitres". L'Empire peut briser une personne, un peuple, un monde, et à peu de chose près une galaxie entière\ldots même sur les mondes les plus fidèles, en dehors des fanatiques et des naïfs la plupart des gens agissent en fonction d'un axiome très simple : j'obéis aux ordres parce que je n'ai pas envie de savoir ce qu'il advient à ceux qui ne le font pas\ldots bien qu'ils ne soient pas l'instrument le plus utilisé par le pouvoir impérial,  les Soldats de Choc à l'armure blanche sont un des plus efficaces sur le plan de l'intimidation : anonymes, apparemment dépourvus d'émotions, totalement dévoués, impitoyables et mortels, ces soldats sont l'incarnation vivante de la mort en marche et beaucoup de gens qui n'ont rien à se reprocher se sentent coupables rien qu'en les croisant dans la rue\ldots

\subsubsection{Esclavage}
L'esclavage a toujours existé, même durant les périodes les plus glorieuses de l'Ancienne République. Désormais, il est officiellement cautionné dans certaines limites. En effet, tout gouvernement dont la philosophie est basée sur la conquête et la puissance armée a besoin d'une main d'\oe uvre bon marché et abondante qui est à la fois un moyen de production et aussi un moyen de dissuasion (même un ouvrier sous-payé et exploité sur Coruscant sait ainsi de manière vague qu'il y a pire que son sort\ldots).\\

Pour obtenir cette main d'\oe uvre, l'Empire dispose de deux sources essentielles : la justice impériale et les conquêtes.\\

La loi impériale encourage donc l'usage de l'indenture : le coupable doit purger une peine de travaux forcés jusqu'à ce qu'il ait remboursé sa dette à la société. La fermeté et la partialité des tribunaux impériaux garantissent un ample choix de "coupables" qui devront travailler jusqu'à leur dernier jour pour "purger leur peine". Même dans le cas de peines de courte durée, il n'est pas rare que le condamné meure à force de surmenage ou à cause des normes de sécurité quasi-inexistantes sur la plupart des chantiers pénitentiaires. À cet égard, les mines d'épice de Kessel sont tristement réputées même s'il existe de nombreux autres chantiers pénitentiaires impériaux.\\

Dans le cas de nombreuses races non-humaines, la justice impériale est encore plus sévère et certaines (comme les Wookies par exemple), se sont vu coller l'étiquette "espèce vassale". Cela signifie que de par leur naissance, tous les représentants d'une telle race doivent obéissance à leurs maitres humains. Alors que des peuples comme les Sullustains ou les Duros sont laissés relativement tranquilles et ne sont pas l'objet d'une discrimination publique, les Wookies et bien d'autres ont été massivement déportés hors de leur planète natale pour servir dans les chantiers impériaux ou être vendus à des corporations peu scrupuleuses qui payent bien évidemment des taxes impériales sur leurs "acquisitions" ainsi que sur les bénéfices que leurs travailleurs forcés leur permettent d'obtenir. Ainsi, tout le monde (en dehors de l'esclave) y trouve son compte financièrement et idéologiquement. Certaines espèces récemment découvertes ont même été déclarées "non-sapientes" et aucune loi n'empêche un humain (ou qui que ce soit d'autre d'ailleurs) d'exploiter, de maltraiter ou de tuer un "animal"\ldots

\subsection{Structure de l’Empire Galactique}
La création de Palpatine n'est pas le premier gouvernement totalitaire dans l'histoire de nombreux peuple de la galaxie mais aucune des dictatures qui l'ont précédé n'a pu étendre sa tyrannie à une telle échelle et son influence s'accroit chaque jour bien au-delà des frontières de l'Ancienne République. Des milliards d'êtres intelligents participent volontairement à cette expansion continue, par intérêt, par naïveté ou par conviction mais le nombre des victimes de l'Empire Galactique est encore bien plus grand. Les populations de dizaines de planètes ont été décimées et déportées, des races non-humaines réduites à l'esclavage ou vouées à l'extermination, des centaines de mondes bombardés, des milliers de cités rasées en quelques décennies.\\

La machine de guerre impériale est l'outil de son créateur et n'a que deux raisons d'être : conquérir et soumettre. Pour y parvenir, Palpatine a su jouer avec les pouvoirs en place lors de son ascension tout en créant les instruments qui lui manquaient pour y parvenir. Depuis son palais impérial de Coruscant si grand qu'il pourrait être une cité à part entière, le pouvoir de Palpatine s'étend partout par le biais des légions d'individus ambitieux, fanatiques ou tout simplement terrorisés qui le servent. Ce pouvoir s'appuie sur plusieurs fondations :

\subsubsection{Les Conseillers Impériaux}
Il s'agit des membres du gouvernement impérial, des super-ministres qui doivent leur réussite politique à l'Empereur. La plupart d'entre eux vivent dans le luxe et l'affichent d'une manière si flagrante que les sénateurs corrompus de la République auraient l'air presque modestes en comparaison. Tout leur pouvoir dépend de la bonne volonté de Palpatine qui a rapidement montré qu'il aimait diviser pour mieux régner. Ils sont donc à la fois fonctionnaires, courtisans et comploteurs, \oe uvrant sans cesse pour se rendre indispensables tout en surveillant leurs collègues. Certains d'entre eux, comme le grand vizir Sate Pestage ou Crueya Vandron, ont un pouvoir colossal alors que d'autres ne sont rien d'autres que des intriguant qui considèrent que le simple fait d'être près de l'Empereur est la récompense ultime, même si leurs pouvoirs effectifs sont ridicules. Le plus redouté en dehors de Pestage est le Seigneur Darth Vader, l'homme à tout faire de l'Empereur.

\subsubsection{Le Sénat Impérial}
Le Sénat Impérial est tout ce qui reste de l'ancien Sénat de la République mais ses pouvoirs déjà restreints lors du couronnement de Palpatine se voient diminués un peu plus chaque jour. Actuellement, il n'est en pratique qu'une chambre de représentation des mondes de l'Empire, chaque sénateur représentant une planète ou un système habité mais la plupart des décisions étant prises sans que cette instance soit consultée. La plupart des sénateurs sont soit des rescapés de l'ancien Sénat, soit des ambitieux, soit des hommes de paille mais c'est encore le seul endroit de la galaxie ou l'on peut s'opposer publiquement à la volonté de l'Empereur sans se retrouver systématiquement devant un peloton d'exécution\ldots à condition de savoir choisir ses mots. Parmi les sénateurs les plus virulents à l'encontre de l'empire, on trouve Bail Organa d'Alderande (et depuis peu sa fille Leia qui siège à sa place). Auparavant, la digne Mon Mothma de Chandrila était considérée comme la voix des peuples mais accusée de haute trahison, elle a dû prendre la fuite. Elle fut remplacée par sa jeune collègue Canna Omonda  qui a tenté de reprendre le flambeau mais qui fut rapidement arrêtée et exécutée alors que des stars destroyers impériaux établissaient un blocus autour de Chandrila. Certains pensent que le Sénat ne tardera pas à être dissout ou à faire l'objet d'une vaste purge.

\subsubsection{Les Moffs}
Les Moffs sont les fonctionnaires territoriaux de l'Empire. Chacun d'eux à la charge d'un Secteur parmi la multitude qui forment l'Empire. Alors que l'ancienne république donnait une certaine marge de contrôle aux gouvernements locaux, les Moffs sont virtuellement à la tête de l'administration civile et militaire dans le Secteur dont ils ont la charge. Ils n'ont de comptes à rendre qu'à l'Empereur et à ses Conseillers.  Des rangs du corps des Moffs ont été tirés une poignée de Grands Moffs chargés en plus de leurs taches sectorielles de superviser certains problèmes à une échelle plus vaste. Les Grands Moffs sont en fait un peu les "administrateurs spéciaux" que l'Empereur délègue pour faire face à certains problèmes. L'un des plus célèbres est Wilhuff Tarkin qui dirige le secteur d'Eriadu mais qui a aussi en charge de coordonner la lutte contre les groupes rebelles sur l'ensemble des Territoires de la Bordure Extérieure.

\subsubsection{Les Gouverneurs}
Subordonnés aux Moffs, les Gouverneurs ont en charge l'administration d'un monde, voire d'un système. Les mondes les plus loyaux à l'Empire ont pu conserver leur gouvernement d'origine et le Gouverneur Impérial est bien souvent là pour collecter les impôts et veiller à ce que les autorités locales suivent la voie tracée par l'Empereur. A l'opposé, les mondes les plus hostiles qui ont dû être soumis par la force ou la menace ont vu leur gouvernement démantelé, leurs dirigeants emprisonnés ou exécutés et le Gouverneur Impérial est devenu la seule autorité légale à laquelle ils puissent s'adresser mais surtout à laquelle ils doivent obéissance.

\subsubsection{Le COMPORN (COMité pour la Préservation de l'ORdre Nouveau)}
Le COMPORN est la machine politique de l'Empire, l'instrument qui doit permettre à l'Empereur de faire régner son idéologie sur la galaxie. Afin de répandre cette idéologie, le COMPORN possède plusieurs branches répandues dans tout l'espace impérial parmi lesquelles certaines sont devenues célèbres. 
\begin{description}
	\item[Le GroupSA (Groupe des Sub-Adultes)] La machine d'endoctrinement destinée à permettre l'éducation des enfants puis des adolescents dans le respect des préceptes de l'Ordre Nouveau. Plus tard, ces jeunes patriotes dévoués deviennent des citoyens modèles et un nombre non négligeable entre au service direct de l'Empire, voire du COMPORN lui-même. Le GroupSA organise des séminaires, des campagnes humanitaires, des manifestations politiques, des camps de jeunesse et toute autre forme d'activité pouvant valoriser l'idéologie de l'Ordre Nouveau et aider à l'implanter au sein de la jeune génération.
	\item[La Coalition pour le Progrès] Est un vaste organisme bureaucratique qui a pour tâche de surveiller divers domaines d'activité et de s'assurer de leur conformité avec l'idéologie impériale par le biais de ses départements Art, Éducation, Justice, Commerce et Sciences. Elle fournit des rapports à l'Empereur, aux Moffs, aux Gouverneurs et aux Conseillers ainsi qu'à divers autres départements du COMPORN.
	\item[La Coalition pour l'Amélioration] Est le pendant de la précédente. Lorsqu’un système stellaire s'écarte sensiblement des critères établis, ses experts analysent la situation et mettent en \oe uvre les mesures administratives, économiques et politiques nécessaires. Si la situation dépasse certains seuils d'alerte, l'affaire est traitée en tandem avec l'Armée ou la Marine Impériale\ldots
	\item[Le BSI (Bureau de la Sécurité Impériale)] La police politique de l'Empire, chargée de repérer les "déviationnistes" et les "traitres" afin de procéder à leur incarcération, rééducation ou élimination. Les agents du BSI sont craints et hais partout dans l'Empire, y compris au sein des gens qui portent l'uniforme impérial. Leur mission les amène à surveiller, interroger et parfois "rééduquer" ou éliminer les citoyens de l'Empire soupçonnés de déviationnisme, y compris les gradés ou les fonctionnaires importants.
	\item[La CompForce] L'organe paramilitaire chargé de défendre les installations du COMPORN et ses responsables. Les soldats de la CompForce sont souvent comparés à des robots sans humanité et leur entrainement est si sévère que pratiquement 80\% des candidats échouent (dont 22\% de manière fatale). Lorsque le COMPORN décide de prendre en main l'ensemble d'une opération de "prévention" ou de "rééducation" à grande échelle sans passer par les militaires, la CompForce et les agents de terrain du BSI en sont souvent les exécutants.
\end{description}

\subsubsection{Les Renseignements Impériaux}
Les divers services de renseignements de la République ont été refondus en une organisation unique qui étend ses tentacules et envoie ses agents partout dans la galaxie. Il est de notoriété publique que les Renseignements et le BSI n'ont pas de bons rapports. Les agents de terrain des Renseignements ont fait la preuve de leur terrifiante efficacité que ce soit au niveau du contre-espionnage, de la déstabilisation politique ou de missions nettement plus sinistres. Ysanne Isard, surnommée "C\oe ur de Glace", est la femme la plus puissante de l'Empire en tant que Directeur des Renseignements, poste dont elle a "hérité" en se débarrassant de son propre père qui en était le précédent titulaire et qui s'apprêtait à trahir l'Empereur.

\subsubsection{La Marine Impériale}
D'après la rumeur, durant les six premiers mois de son règne, Palpatine a plus dépensé sur le plan militaire que la République durant toute son histoire et le résultat est particulièrement visible lorsque l'on regarde la Marine Impériale, un corps d'élite empreint de nombreuses traditions plurimillénaires. Si la plupart des citoyens de l'Empire qui voyagent dans l'espace ont le plus souvent affaire aux navires des Douanes Impériales, il n'est pas rare de rencontrer les corvettes, croiseurs et cuirassés lourds de la Marine et ses célèbres destroyers stellaires. La Marine possède des pilotes de chasse qui comptent parmi les meilleurs de la galaxie et qui sont aux commandes des redoutables chasseurs TIE, les engins les plus maniables construits à ce jour. Ses artilleurs, ses techniciens, ses soldats embarqués et ses officiers de pont sont également conscients de l'honneur qu'ils ont à servir dans la Marine qui  permet à l'Empereur de tendre la main partout dans les vastes domaines de son empire.

\subsubsection{L'Armée Impériale}
Par tradition les troupes terrestres n'ont jamais été vraiment considérées durant la République. Mais l'Empire a vu un véritable renouveau des unités d'infanterie, d'artillerie et de blindés. De nombreux mondes ont dû être soumis par la force et dans beaucoup de cas les bombardements orbitaux de la Marine auraient été trop destructeurs malgré leur précision. Les fantassins, les artilleurs, les tankistes et les pilotes de quadripodes savent que sans leurs efforts l'Empire serait bien moins grand qu'il ne l'est aujourd'hui. Certaines unités de l'armée placées sous l'autorité directe des Gouverneurs forment les forces de police de nombreux mondes.

\subsubsection{Les Soldats de Choc (Stormtroopers)}
A l'opposé des stars destroyers, l'Empire possède aussi un autre visage bien connu (et craint) des masses populaires, celui des troupes de choc en armure blanche. On trouve ces hommes partout : dans la garde des Gouverneurs et des Moffs, aux côtés des simples fantassins de l'Armée ou en tant que troupes d'abordage de la Marine. Personne ne sait exactement comment sont sélectionnés et entrainés ces hommes qui ne quittent pratiquement jamais leur armure et qui s'adressent les uns aux autres par leur matricule. Certains vont même jusqu'à dire qu'il s'agit de clones. Quoi qu'il en soit, leur dévotion rivalise avec celle de la CompForce à laquelle il faut ajouter un entrainement tactique et militaire très poussé qui a fait d'eux les soldats les plus redoutés de l'Empire. En leur sein, on trouve des unités spécialisées pour les opérations aquatiques, dans l'espace ou dans divers milieux hostiles. Lorsque les troupes embarquées de la Marine ne suffisent pas, lorsque les sections d'opérations spéciales de l'Armée ne font pas l'affaire, les généraux et les amiraux de l'Empire savent qu'ils peuvent encore compter sur les Soldats de Choc pour faire ce qu'il faut\ldots

\subsubsection{L'HoloNet Impérial}
Du temps de la République, l'HoloNet couvrait tous les mondes et permettait une communication instantanée d'un bout à l'autre de l'espace connu. Il suffisait d'habiter à proximité d'une des stations émettrices et surtout d'avoir assez d'argent pour payer les frais énormes engendrés par le fonctionnement et l'entretien du Réseau pour parler en direct à quelqu'un à l'autre bout de la galaxie. Peu de citoyens pouvaient se permettre ce genre de fantaisie mais les diplomates, les savants, les industriels et les militaires pouvaient ainsi échanger des informations sans délai.\\

Palpatine fit démanteler la majeure partie du plus vaste réseau de communication jamais conçu et les sommes d'argent ainsi dégagées servirent à financer sa machine de guerre. Ce qui reste de l’HoloNet est désormais sous le contrôle de l'Empereur, des Moffs et de la Marine bien que les puissantes familles nobles et les grandes corporations puissent occasionnellement y avoir accès. Ainsi, les armées impériales peuvent coordonner leur action et savoir instantanément ce qui se passe partout dans l'Empire tandis que les citoyens doivent se contenter d'attendre les nouvelles apportées par la radio subspatiale dont les ondes mettent beaucoup de temps à traverser l'espace en comparaison des faisceaux instantanés de l'HoloNet. L'avantage de l’HoloRéseau Impérial est ainsi double. \\

\begin{itemize}
	\item Permettre à l'Empire de conserver l'avantage des communications instantanées tout en facilitant la censure des informations. Les grandes corporations et certains gouvernements fidèles à Palpatine ont réussi à établir des versions plus réduites de réseaux de communications instantanées mais aucun d'eux ne dépasse les frontières d'un Secteur et la plupart sont d'accès aussi restreint que l'HoloNet Impérial.
	\item Par contre, s'il est pratiquement impossible pour un simple citoyen d'entrer en liaison avec le reste de la galaxie, il est quotidiennement bombardé par la propagande via l'HoloVision Impériale et autres médias galactiques dont la fidélité est acquise et qui fournissent aux médias locaux la quasi-totalité des informations galactiques connues du public.
\end{itemize}

\subsubsection{Le Corps Impérial d'Exploration}
Bien qu'il soit des plus réduits par rapport à l'ancien Corps des Éclaireurs de la République, le CIE dispose de nombreux vaisseaux et droïdes sondes. Ses équipes opèrent principalement dans l'Espace Sauvage et les Régions Inconnues mais sont aussi actives dans les coins les plus reculés de secteurs déjà colonisés mais mal cartographiés. Le Corps Impérial d'Exploration découvre un nouveau système stellaire toutes les 207 minutes. Malgré les monstrueuses ressources naturelles que cela peut représenter, la plupart de ces systèmes ne sont pas jugés intéressants à court terme. En effet, l'Empereur considère qu'étendre sans cesse son influence par le biais de la colonisation ne fera que renforcer les difficultés que ses agents connaissent déjà avec la Bordure Extérieure et les autres régions périphériques. L'espace que contrôle l'Empire regorge encore de ressources telles qu'il faudrait des millénaires pour les épuiser. \\

En dehors de quelques minéraux et gaz inconnus aux propriétés inédites, les rapports sur les vastes régions explorées par les agents de l'empire et les éclaireurs indépendants sont archivés et destinés à servir plus tard, dans quelques générations ou à être vendus à des corporations fiables qui ne rechigneront jamais à payer les taxes et à soutenir l'effort militaire impérial. Les navires et sondes du Corps d'Exploration ont en fait pour rôle essentiel de recenser les civilisations technologiquement développées en dehors de l'espace impérial. Lorsqu'une telle civilisation est découverte, l'Empire négocie (ou obtient par la force) le droit d'utiliser la main d'\oe uvre locale et ses connaissances technologiques. Les habitants de Mon Calamari sont un exemple célèbre de civilisation non-humaine possédant le vol spatial découverte par l'empire et soumise par ses armées. Nombreux sont les Calamari qui ont été réduits à l'esclavage dans les chantiers navals de l'empire. Même les civilisations plus primitives ne sont pas à l'abri des exactions impériales car leurs représentants sont souvent déportés vers les grands marchés d'esclaves.

\subsubsection{Les Inquisiteurs}
Qu'il s'agisse d'avoir affaire aux spécialistes du BSI ou à ceux des Renseignements, subir un interrogatoire impérial est une expérience que personne ne souhaite faire. Lorsque les talents des agents impériaux et de leurs droïdes spécialisés ne suffisent pas, ces spécialistes peuvent demander l'aide de véritables artistes dans le domaine de la torture : les Inquisiteurs. Les Inquisiteurs Impériaux ne sont pas rattachés à une branche de l'administration impériale mais sont un peu des agents spéciaux chargés de collecter des informations qui intéresseraient l'Empereur, ses Conseillers ou les Grands Moffs. Certains d'entre eux opèrent de manière analogue à celle des Moffs et exercent leurs "talents" à l'échelle d'un système ou d'un secteur bien défini alors que d'autres se déplacent en permanence et se rendent partout où l'Empereur estime que leur présence est nécessaire.\\

Ainsi, même si un Moff ou un Gouverneur peut solliciter les services d'un Inquisiteur pour briser un prisonnier particulièrement résistant ou bien entrainé, il peut aussi s'attendre à voir débarquer sans préavis un tel personnage chargé d'une mission précise ou tout simplement d'une "inspection de routine". Au début du règne de Palpatine, les Inquisiteurs se sont montrés extrêmement zélés dans le programme de recherche et d'élimination des Chevaliers Jedi et il leur arrive encore à l'occasion de découvrir un chevalier en fuite qui était parvenu à échapper à la vigilance des Gouverneurs et des Moffs. Le Haut Inquisiteur Tremayne est sans doute le plus influent des Inquisiteurs et certainement le plus redouté. Il parcourt l'ensemble de l'espace impérial, en particulier la Bordure Extérieure et l'Espace Sauvage.

\subsection{La loi et l’ordre}
Plusieurs organisations sont chargées de faire appliquer la loi impériale ou locale dans la galaxie. Certaines ont des missions assez floues et comme dans tous les systèmes depuis l'aube des temps, les problèmes de juridiction et la compétition entre agences demeurent à l'ordre du jour.\\

D'une manière générale, une organisation policière ou paramilitaire qui n'est pas une émanation du pouvoir impérial ne peut exister que si l'Empire y trouve un intérêt. Par exemples, dans le cas de systèmes dont la fidélité à l'Ordre Nouveau n'est plus à prouver ou au contraire dans des régions périphériques ou les forces impériales sont déjà surchargées par leurs activités anti-insurrectionnelles. Des agents de liaison du COMPORN (plus particulièrement le BSI et le BICE) font le lien entre les dirigeants impériaux locaux (gouverneurs, moffs\ldots) et les forces de l'ordre "indépendantes". Un certain nombre d'agents infiltrés du BSI ou des Renseignements Impériaux complètent le dispositif.

\subsubsection{Forces Locales}
Certains systèmes particulièrement fidèles se sont vus autorisés à disposer de services de polices et/ou de  navires spatiaux armés afin d'assurer le maintien de l'ordre et leur propre défense afin de permettre à l'Empire de déployer ses effectifs ailleurs. Généralement, les effectifs, l'armement, le nombre de navires capables d'hypersaut et leur tonnage sont limités par l'Empire. Les deux exemples les plus célèbres dans ce domaine sont la Sécurité Corellienne (CorSec) et surtout l'ESPO du Secteur Corporatif qui dispose également d'unités paramilitaires et même d'une flotte conséquente bien qu'un peu vieillotte comptant plusieurs cuirassés et destroyers de classe Victoire. Plusieurs grandes familles nobles possèdent également des forces armées significatives bien que négligeables à l'échelle de l'Empire, comme les dynasties unifiées de la planète Kuat, les Anciennes Familles du Secteur Senex ou les Sept Maisons du Secteur Tapani par exemple.


\subsubsection{Le Bureau des Investigations Criminelles de l'Empire (BICE)}
Cet organisme n'est pas exactement une agence de police mais un sous-département de la Coalition pour le Progrès du COMPORN.\\

Le BICE est chargé de valider les mises à prix des criminels et de coordonner l'activité de services de police locaux lorsque les personnes poursuivies changent de juridiction ou opèrent sur plusieurs juridictions simultanées. Les criminels politiques (les Rebelles\ldots)  sont généralement traités directement par le BSI et les Renseignements mais le BICE peut également intervenir si leur tête est mise à prix par les précédentes organisations ou pour des crimes de droit commun. Le BICE gère les gigantesques bases de données policières de l'Empire, tout au moins la partie qui concerne les crimes et délits de droit commun\ldots il est également chargé de superviser directement l'activité des Rangers Sectoriels. 

\subsubsection{Les Rangers Sectoriels}
Ils sont de fait la force de police la plus répandue dans l'Empire tout comme ils l'étaient sous l'Ancienne République.\\

Leur mission inclut toutes les facettes policières classiques ainsi que la lutte contre la piraterie, l'assistance aux naufragés et ainsi de suite. Les Rangers opéraient autrefois sous l'autorité du Ministère de la Justice en partenariat étroit avec les Chevaliers Jedi mais ils sont désormais sous le contrôle du BICE.  Ils ont supériorité juridictionnelle et droit de réquisition sur les autorités locales dès que l'on touche à des crimes validés par leur organisme de tutelle. Néanmoins, les effectifs des Rangers ont sensiblement diminué (surtout dans les secteurs les plus fidèles à l'Empire ou les agences locales ont une grande marge d'autorité) et ils n'ont pas de navires capables de faire face à des vaisseaux militaires de grande taille. Les Rangers sont surtout présents dans les régions centrales et la Bordure Intérieure. Les mondes des régions plus périphériques ont tendance à utiliser davantage les agences locales, à se voir imposer l'autorité de l'armée impériale, voire à dépendre exclusivement de quelques volontaires ou à être "policés" par des organisations criminelles\ldots

\subsubsection{Le Bureau de la Sécurité Impériale (BSI)}
La police politique de l'Empire est une organisation redoutée parce que ses agents peuvent à tout moment passer par-dessus l'autorité d'une autre agence civile ou militaire et utiliser ses ressources en ayant rarement des comptes à rendre (et jamais aux intéressés\ldots). Le BSI est obsédé par les "traitres" et il n'est jamais bon d'attirer son attention. Il possède tout l'arsenal technologique, logistique et juridique nécessaire pour accomplir sa mission.  

\subsubsection{Les Renseignements Impériaux}
Contrairement au BSI et aux autres organisations de police, les Renseignements sont bien évidemment dangereux par le fait que l'on ne sait pas ou leur influence exacte s'arrête puisque leurs agents opèrent rarement à visage découvert et qu'ils n'ont officiellement aucun pouvoir de police. Ils sont cependant susceptibles de recourir aux autres agences et organisations par réquisition bien que leurs pouvoirs soient en ce domaine moins étendus que ceux du BSI (une cause de rivalité supplémentaire) ou du BICE. De fait, ils préfèrent agir le plus souvent de manière indirecte.  

\subsubsection{Les Douanes Impériales}
Elles sont chargées bien évidemment de contrôler les cargaisons mais également de collecter les taxes auprès des transporteurs et de prêter assistance aux navires menacés par des pirates. Dans les secteurs les plus surs, la présence de la Marine Impériale est minimale et ce sont les Douanes qui remplissent une partie de ses missions. De même, dans certains systèmes de la Bordure Extérieure situés sur les quelques grands axes commerciaux, les Douanes sont en fait pratiquement les seules forces spatiales impériales présentes à l'exception de quelques navires de la Marine surtout chargés de défendre les bases avancées de l'Empire ou ses leaders locaux. Les Douanes Impériales sont placées sous le contrôle de la Coalition pour le Progrès tout comme le BICE.

\subsubsection{La Marine Impériale}
En plus de son activité militaire essentielle, la Marine est également autorisée à aborder tout navire qu'elle juge suspect et à poursuivre et éliminer les pirates de tous types. Dans les secteurs les plus anciens, la Marine du gouvernement central est encore auréolée d'une certaine aura de prestige pour ses traditions de lutte contre la piraterie, traditions encore vivaces à l'heure actuelle. D'un autre côté, les habitants de la Bordure Extérieure, de l'Espace Sauvage et de plusieurs mondes aux velléités d'indépendance ont une toute autre expérience des forces navales de l'Empire\ldots

\subsubsection{L'Armée Impériale}
Un Gouverneur ou un Moff peut accorder à l'armée impériale des pouvoirs de police spéciaux dans sa juridiction s'il ne dispose pas de forces locales ou s'il préfère les contrôler par une organisation parallèle. Il peut ainsi passer outre les forces de l'ordre locales et même les Rangers mais bien évidemment, pas le BSI\ldots certains moffs et gouverneurs ont ainsi constitué de véritables dictatures militaires bien plus rigides que les régimes les plus réactionnaires du Noyau.

\subsubsection{Les Stormtroopers}
Généralement, leur activité est purement militaire mais dans les mondes stratégiques de l'Empire ou les secteurs les plus turbulents de la Bordure Extérieure, on leur affecte également souvent des missions de police en parallèle ou à la place des forces de l'ordre locales (sur certains mondes, ils sont les seuls représentants de l'autorité impériale). La logique d'une telle décision est généralement limpide : faire régner l'ordre en faisant régner la terreur. Les Stormtroopers et l'Armée opèrent souvent en tandem dans ce genre de mission, d'autant plus facilement qu'ils doivent assez souvent procéder à une occupation militaire conjointe à la suite d'un débarquement en règle sur un monde peu enclin à adhérer de lui-même à l'Empire.\\

Presque toujours, lors d'un conflit de juridiction entre plusieurs organisations, l'Empire donne raison à celles qu'il a contribué à créer plutôt qu'à celles d'origine extérieure. Cette situation est renforcée par le fait que la plupart des organisations locales recrutent au sein de leur population ce qui implique un certain nombre de non-humains dans leurs services et des risques de sympathie avec les citoyens qui iraient à l'encontre des intérêts de l'Empire. Dans le doute, certains gouverneurs et moffs aiment à s'assurer la présence visible de l'Armée et si possible des Troupes de Choc\ldots\\

En cas de conflit entre organisations impériales (par exemple la rivalité bien connue entre les Renseignements et le BSI\ldots), les choses sont un peu plus\ldots disons\ldots intéressantes\ldots
\subsection{Les Renseignements Impériaux}
\subsubsection{Historique}
La bureaucratie de l'Ancienne République avait produit quatre agences gouvernementales chargées de recueillir et d'analyser les renseignements de manière plus ou moins officielle
\begin{itemize}
	\item L'Organisation de la Sécurité Républicaine
	\item Le Bureau de Renseignement Sénatorial
	\item Le Consortium Technologique Interstellaire
	\item Le Bureau des Acquisitions Spéciales de la Bibliothèque de la République 
\end{itemize}

Outre les rivalités traditionnelles entre services, les dirigeants de ces quatre agences découvrirent dans les dernières années de la République que de nombreux sénateurs et plusieurs factions corporatistes ou privées utilisaient la corruption et la désinformation afin de faire faire leur sale travail par leurs subordonnés, le plus souvent à l'insu des intéressés. Cela impliquait d'ailleurs un certain nombre d'opérations résolument hostiles menées contre les autres agences de renseignements sous des prétextes pour le moins fallacieux.\\

Alors que le Chancelier Suprême Palpatine affrontait les séparatistes peu de temps avant la Guerre des Clones, les directeurs de ces quatre agences décidèrent de se rencontrer en secret. Ils étaient las de la rivalité et de la corruption. Il leur fallait unir leurs ressources et leurs informations sous peine de ne jamais parvenir à jouer leur rôle alors que la crise menaçait.\\

Un comité secret de coordination vit le jour et c'est ainsi que naquit l'Ubiqtorat. Lorsque Palpatine proclama l'Empire et qu'il fit le ménage dans la bureaucratie corrompue, les membres de l'Ubiqtorat se rendirent auprès de lui et révélèrent leur existence. Les quatre agences originelles furent donc officiellement démantelées et sous la tutelle de l'Ubiqtorat les Renseignements Impériaux virent le jour.

\subsubsection{L'Ubiqtorat}
A la tête de l'Ubiqtorat, le Directeur des Renseignements est le ma\^{i}tre officiel de cet organisme tentaculaire et ne rend compte qu'à l'Empereur ou à certains de ses conseillers triés sur le volet. Il est le plus souvent le seul membre de l'Ubiqtorat dont l'identité soit connue, les autres demeurant dans l'anonymat. On pense que certains d'entre eux sont sans doute les directeurs de plusieurs branches des Renseignements mais dans l'absolu, n'importe qui pourrait siéger au sein de l'Ubiqtorat.\\

Pour communiquer avec le reste de l'organisation des Renseignements Impériaux, le comité directeur utilise des droïdes piégés, des courriers fanatisés ou des communications holonet haute sécurité. L'Ubiqtorat compte sans doute peu de membres en dehors de son comité directeur et la plupart d'entre eux n'ont certainement qu'un rôle purement administratif. Ils veillent à ce que les informations essentielles parviennent à leurs dirigeants et que les ordres de ceux-ci soient correctement transmis aux autres branches des R.I. Comme pour le comité directeur, les fonctionnaires de l'Ubiqtorat agissent certainement soit sous une couverture officielle dans une autre branche des Renseignements, soit dans le plus complet anonymat et sans lien apparent avec "la communauté du renseignement".\\

En clair et en résumé : l'existence de l'Ubiqtorat est connue mais on ne sait quasiment rien de son personnel, de ses dirigeants et de leurs installations.
En dessous de l'Ubiqtorat, la structure des R.I s'étend au niveau sectoriel et est rattachée au Moff local. Dans la mesure des moyens disponibles, toute ou partie des branches suivantes sont en place au niveau sectoriel. 

\subsubsection{Le Bureau de l'Organisation Interne (OrgInt)}
La mission de cette branche des Renseignements et de protéger l'ensemble de l'agence contre toute menace interne ou externe. Contrairement à ce que l'on pourrait croire, il n'a pas pour habitude d'organiser les purges et chasses aux sorcières chères aux agents du BSI et l'OrgInt jouit d'un certain prestige au sein des Renseignements. Ses représentants s'avèrent presque toujours courtois et extrêmement compétents. Le sous-département de la Sécurité Interne (SecInt) veille à la protection physique des bâtiments et du personnel des Renseignements tandis que le Contre-espionnage Interne (ConInt) surveille les activités du personnel et des agents.
 
\subsubsection{Le Bureau des Analyses}
Il centralise, classe et étudie les masses d'informations colossales qui lui parviennent par les canaux officiels et officieux. Il observe les comportements sociaux, les changements économiques, les avancées technologiques, etc. Il possède des subdivisions vouées à la surveillance des média, au décryptage, aux interrogatoires et même au contrôle des fréquences et émissions parasites afin de s'assurer qu'aucun message secret n'est transmis par ce biais.

\subsubsection{Le Bureau des Opérations}
C'est en son sein que l'on trouve la presque totalité des agents de terrain des Renseignements Impériaux. Les branches Surveillance, Contre-espionnage et Service Diplomatique permettent d'obtenir les informations et éventuellement de procéder à des arrestations ou des éliminations. Les agents ennemis capturés sont souvent confiés au service Interrogation du Bureau des Analyses afin d'être reprogrammés ou "retournés".\\

Enfin, le Bureau des Opérations dispose aussi de deux sous-sections dont on parle peu en raison de leurs missions particulièrement sinistres : Assassinats (dont les agents surentra\^{i}nés font la fierté des Renseignements) et Déstabilisation (qui va de la désinformation jusqu'au terrorisme de masse pour mettre à genoux les gouvernements que l'Empire souhaite soumettre sans faire appel à ses forces armées).

\subsubsection{Le Bureau des Renseignements}
Tire parti des informations collectées par le Bureau des Analyses et élabore à partir des données fournées des rapports et des suggestions qui aideront l'Ubiqtorat à définir objectifs et missions des différentes composantes de l'agence.

\subsubsection{Le Secteur Plexus}
Sa mission est d'assurer la sécurité des communications au sein des Renseignements Impériaux. Il surveille les canaux de l'HoloNet attribués aux Renseignements Impériaux ainsi que les grilles de communication locales afin d'empêcher qu'on espionne les communications secrètes ou qu'on les intercepte. En plus de l'HoloNet Impérial, le Secteur Plexus dispose de son propre réseau de communication qui fonctionne par le biais de vaisseaux-droïdes camouflés et capables de s'autodétruire. Ces petits appareils d'environ neuf mètres de long sont très rapides et font d'incessants allers retours entre deux systèmes précis. Ils émergent de l'hyperespace, reçoivent les communications attendues, transmettent en retour les données qu'ils transportent et repartent aussitôt vers le système d'où ils viennent tandis qu'un autre vaisseau-droïde assure la transmission jusqu'au système suivant. De cette manière, le Secteur Plexus dispose dans certaines régions de l'empire d'un vaste réseau de vaisseaux-courriers aux trajets parfaitement rodés ce qui minimise considérablement le temps nécessaire pour passer d'un système à l'autre ainsi que les risques d'accident de navigation

\subsubsection{Cellules de Système}
Lorsque c'est nécessaire, les responsables locaux des Renseignements peuvent assembler des équipes réduites (de 5 à 20 personnes le plus souvent) chargées de missions à long terme et composées d'agents de plusieurs branches distinctes. La Cellule de Système est alors implantée sur site, dispose d'un petit transceveur hyperspatial pour communiquer avec un agent de liaison et doit ensuite se débrouiller par ses propres moyens. Il existe plusieurs millions de cellules de système actives et la plupart ignorent totalement que d'autres cellules aux objectifs identiques aux leurs opèrent parfois à quelques kilomètres de leur planque. Occasionnellement, les responsables de l'opération peuvent décider de faire opérer de manière conjointe plusieurs cellules de systèmes qui ignoraient leur existence respective ou même en sacrifier une à l'insu des autres afin de leur "montrer la voie".

\subsubsection{Cellules de Crise}
Il ne s'agit pas de cellules de terrain mais bien de "bureaux temporaires" assemblés à la demande pour gérer une situation grave et inattendue. Les cellules de crise sont le plus souvent composées de membres du Bureau des Renseignements et d'un ou deux représentants d'autres départements selon les besoins. Ces cellules bénéficient d'un accès direct temporaire à l'Ubiqtorat et dans les cas les plus critiques, elles peuvent même avoir pour rôle de conseiller un Grand Moff ou l'Empereur lui-même. Une fois la crise résolue, la cellule est démantelée et ses membres retournent à leurs activités normales. Si un "suivi" est nécessaire, les branches adéquates des Renseignements se voient confier les missions correspondantes et une ou plusieurs cellules de système sont éventuellement créées. 

\subsubsection{Rectification}
Rectification n'est pas un service et n'appara\^{i}t dans aucun organigramme officiel. Lorsque l'Ubiqtorat estime qu'il a besoin de s'impliquer directement dans une affaire, il mandate un ou plusieurs agents de Rectification. Les agents de ce type sont l'élite de l'élite, ce que l'Empire a de mieux à sa disposition. Certains n'ont aucune existence légale, d'autres agissent au sein de l'Empire sous couverture mais tous sont aussi dévoués que compétents. En aucun cas un agent de Rectification n'agit en tant que tel. La plupart sont soit infiltrés dans une cellule active, soit font cavaliers seuls. Lorsque l'Ubiqtorat estime que les choses s'enveniment trop ou qu'il est nécessaire de veiller à certains détails sans que le reste des Renseignements Impériaux soit au courant, il fait appel à Rectification et explique ce qu'il attend de ses agents. Généralement, les choses se déroulent alors conformément aux attentes de l'Ubiqtorat et le reste de la galaxie continue à ignorer la vérité.

\subsection{Le Bureau de la Sécurité Impériale}
\subsubsection{Historique et Vue d'Ensemble}
Le Bureau de la Sécurité Impériale est le premier organisme de renseignement créé entièrement à l'initiative des alliés de Palpatine pour faire pendant aux vétérans des Renseignements Impériaux. Dès ses origines, il fit l'objet de considérables investissements financiers et ne tarda pas à devenir le principal débouché des adolescents les plus dévoués qui ne se destinaient pas aux carrières militaires après leur passage par le GroupSa, la branche du COMPORN dédiée à l'endoctrinement des jeunes générations.\\

Avec des effectifs de terrain supérieurs d'environ 70 fois à ceux des Renseignements Impériaux, le BSI demeure la plus grosse organisation de renseignements de l'Empire malgré son statut de "simple branche" du COMPORN. En fait, l'importance réelle du BSI est considérable et les autres branches du COMPORN ont depuis longtemps appris à prendre ses représentants avec des pincettes.\\

Le BSI est formé uniquement d'adeptes convaincus de l'Ordre Nouveau et représente la police politique de l'Empereur. Il préfère agir de manière très publique afin de prouver la puissance, l'omniprésence et surtout l'infaillibilité de l'Empire. Ses activités tournent surtout autour du maintien de l'ordre et il est aux yeux des propagandistes un outil de choix afin de mieux intimider les masses et bon nombre d'individus portant l'uniforme impérial. Pour toutes ces raisons, les Renseignements Impériaux considèrent le BSI comme un ramassis de dangereux fanatiques dépourvus de toute subtilité qui font plus de dégâts qu'autre chose. A l'opposé, le BSI quant à lui considère les R.I comme de dangereux opportunistes qui laissent parfois des tra\^{i}tres opérer trop longtemps et faire trop de dégâts sous prétexte que cela leur permettra de remonter à la source.\\

Ces différences de philosophie entra\^{i}nent également des différences de méthode. Alors que les R.I préfèrent l'usage des agents doubles, des taupes, de la désinformation et essayent autant que possible de ne jamais appara\^{i}tre au grand jour, le BSI quant à lui \oe uvre à travers la surveillance constante des populations, les rafles, les confessions publiques, la délation et les purges.

\subsubsection{Organisation}
Le BSI est dirigé par une Commission des Opérations installée dans son siège central sur Coruscant. La Commission exerce un contrôle constant et minutieux sur les activités de ses différentes branches partout dans l'Empire et missionne constamment des inspecteurs pour aller examiner de plus près les activités des branches sectorielles. Celles-ci sont officiellement rattachées auprès des Moffs locaux mais le BSI a également pour tâche de surveiller les fonctionnaires territoriaux de l'Empire et cette réalité n'échappe à aucun des intéressés\ldots

\subsubsection{Département Surveillance}
Il regroupe à lui tout seul environ 35\% des effectifs du BSI et la plupart des nouvelles recrues y font leurs premières années en attendant une affectation définitive. Surveillance veille à identifier les menaces potentielles avant de confier les dossiers intéressants au département Investigations. Surveillance s'appuie également sur un certain nombre d'indics professionnels, d'opportunistes et de simples citoyens tenus par le chantage.

\subsubsection{Département Investigations}
Ils sont chargés d'approfondir les enquêtes et éventuellement de procéder aux interpellations nécessaires. Investigations est donc la branche du BSI avec le plus haut profil public car autant que possible il est nécessaire de pouvoir donner aux propagandistes de la "matière d'\oe uvre". Ses agents ont autorité pour réquisitionner des membres des forces armées de l'Empire ou des agences de police locale pour les assister dans leurs missions à moins que le Gouverneur ou le Moff local s'y oppose en personne. Ce qui a rarement lieu\ldots les agents d'Investigations sont les membres du BSI qui ont également le plus souvent l'occasion de se frotter directement aux agents rebelles.

\subsubsection{Département des Affaires Internes}
Contrairement à ce que leur nom suggère, les Affaires Internes ne s'occupent pas uniquement de ce qui se passe dans le BSI mais bien dans l'ensemble du COMPORN dont même la toute puissante Commission Sélective n'est pas à l'abri de leurs investigations. Cette autonomie considérable témoigne bien du fait qu'aux yeux de l'Empereur, le BSI est un outil privilégié pour s'assurer la loyauté de ses subordonnés. Même au sein du Bureau de la Sécurité Impériale, les agents des Affaires Internes sont considérés comme des personnes à éviter dans la mesure du possible. Ils peuvent se montrer diaboliquement subtils dans leurs audits et leurs "enquêtes de routine" sont le plus souvent de simples paravents aux véritables chasses aux sorcières qu'ils organisent.

\subsubsection{Département Interrogation}
Contrairement au service homonyme des Renseignements Impériaux, Interrogation ne relâche jamais ceux qui lui sont confiés et se fait fort d'obtenir systématiquement des aveux complets de la part des suspects. Officiellement, Interrogation peut se vanter d'avoir un taux de succès de 100\% : tous les suspects qui passent entre ses mains s'avèrent coupables et font des aveux complets. Tout au moins, tous les suspects qui vivent assez longtemps pour signer les aveux en question\ldots

\subsubsection{Département Rééducation}
Il prend en charge les membres du COMPORN que les Affaires Internes ont interpellé à la suite de leurs investigations. Comme pour Interrogation, Rééducation peut s'enorgueillir d'un taux de réussite de 100\% et les personnes qui sont "confiées" à Rééducation en ressortent parfaitement aptes à reprendre leurs fonctions.\\

Généralement, ils sont cependant rétrogradés d'un rang ou deux car leur "stage" les transforme en personnes ternes et effacées, aux réactions émotionnelles parfois infantiles et qui souffrent occasionnellement de troubles de la mémoire ou de l'attention. Certaines rumeurs parlent même de "rééduqués" qui basculent dans la psychose plusieurs années après avoir repris leur vie normale.

\subsubsection{Département du Maintien de l'Ordre}
On fait appel à ses services lorsqu'il est peu souhaitable d'impliquer les forces de police, les militaires ou la CompForce dans une opération précise pendant laquelle un surcro\^{i}t de puissance de feu est cependant nécessaire. En fait, il s'agit de la branche action du BSI, en charge de toutes les opérations "lourdes". La plupart du temps, les agents de ce service sont missionnés pour assister Investigation ou parfois Affaires Internes. Le Département du Maintien de l'Ordre possède des fichiers très à jour qui permettent également au BSI de recruter des personnes extérieures pour effectuer certaines opérations particulièrement délicates qu'il est parfois obligé de mener de manière officieuse pour pouvoir par la suite monter des opérations officielles nettement plus crédibles et conformes à son image publique. 

\subsection{L'Armée Impériale}
\subsubsection{Histoire de l'Armée Impériale}
Durant les dernières décennies de la République Galactique, ses forces militaires étaient extrêmement réduites et si la Marine avait les plus grandes peines à surveiller l'espace républicain, le Sénat ne disposait plus de véritables forces terrestres capables d'intervenir partout dans la galaxie. Les millénaires d'expansion et de conflits avaient laissé la place à une lente et insidieuse décadence tandis que les frontières de la République s'avéraient de plus en plus fragiles. Les forces terrestres de la République devinrent disparates et la plupart des mondes représentés au Sénat obtinrent que les contingents de jeunes gens soient affectés aux forces de sécurité planétaires plutôt qu'envoyés à l'autre bout de la galaxie.\\

Peu de mondes voyaient encore l'intérêt d'entretenir de grandes armées et ceux qui s'étaient dotés de programmes militaires privilégiaient de plus en plus les armées automatisées, suivant l'exemple de la Fédération du Commerce qui alignait les effectifs terrestres les plus conséquents parmi les factions des dernières décennies de la République. Au niveau de la République dans son ensemble, aucun affrontement militaire important ni aucune guerre civile majeure ne s'annonçaient à l'horizon depuis des générations et les conflits d'intérêts se réglaient désormais pour l'essentiel dans les antichambres du Sénat ou des bâtiments corporatistes et gouvernementaux. Les opérations militaires étaient relativement rares et celles qui nécessitaient la présence de forces terrestres importantes étaient devenues vraiment exceptionnelles.\\

La Guerre des Clones fut un traumatisme pour de nombreuses populations qui durent subir le joug des forces mécanisées de la Confédération et ne durent leur sauvegarde qu'à l'intervention d'armées clonées par les kaminoans et au courage des dernières unités de l'armée républicaine en activité. De nombreuses voix se firent entendre pour que l'on réactive ou renforce à la hâte les programmes militaires et lorsque le conflit finit par s'apaiser, Palpatine sut tirer parti de ce sentiment d'insécurité pour bâtir la plus grande armée de l'histoire de la galaxie.\\

Les campagnes de recrutement remportèrent un succès considérable, de nombreux jeunes gens s'avérant désireux de prendre l'uniforme et de contribuer à restaurer la paix et la sécurité dans la galaxie. Très rapidement, ces légions de jeunes volontaires furent endoctrinées et déployées très loin de leurs mondes d'origine. Au lieu de protéger leurs proches ou leurs voisins, ces jeunes hommes et ces jeunes femmes se retrouvèrent à surveiller les populations d'autres planètes et à participer à l'invasion des mondes qui refusaient de rejoindre l'Empire. L'Empereur et ses alliés étaient parvenus à la fois à priver les mondes de l'ancienne république d'une partie de leurs forces vives et à augmenter dans le même temps la puissance du complexe militaire impérial. Les soldats de l'Armée Impériale ne furent jamais aussi fanatisés que les adolescents endoctrinés par le COMPORN ou les soldats de choc Stormtroopers mais ils participèrent pleinement à la politique de conquête et de répression de l'Empire. Très vite, le pouvoir impérial fut suffisamment fort pour instaurer des campagnes de conscription obligatoire auxquelles il s'avéra rapidement dangereux de s'opposer.\\

La grande majorité des soldats de l'Empire assimila sans guère de réticences la propagande continuellement déversée par les recruteurs, les sous-officiers et les agents de liaison du COMPORN. Aux craintes du passé on était parvenu à substituer un sentiment d'appartenance à la force militaire qui restaurerait l'ordre dans la galaxie en soumettant les dissidents aux idées proches des anciens confédérés indépendantistes, responsables de la guerre. Quant aux nombreux mondes nouvellement découverts aux marches de l'espace impérial, leurs populations non humaines devaient être assimilées pacifiquement ou par la force avant de développer une sympathie pour la cause des rebelles et des anarchistes, en tous genres. La surveillance constante des officiers politiques du BSI et la présence de nombreux contingents de Stormtroopers également chargés des opérations au sol permit d'étouffer dans l'\oe uf ou de réprimer sauvagement et rapidement toutes les velléités de contestation au sein des forces terrestres impériales.\\

\subsubsection{Structure de l'Armée Impériale}
L'unité de base de l'Armée Impériale est l'\textbf{Escouade} qui compte normalement neuf hommes. Toutes les escouades sont homogènes (escouades de fantassins, escouades de tireurs d'élite, d'artilleurs, de soldats du génie\ldots) et comptent huit soldats places sous l'autorité d'un sergent. Tous grades confondus, les sergents impériaux s'avérèrent les soldats les moins sensibles aux arguments qui amenèrent leurs collègues à déserter ou à se mutiner et cela tient directement au fait que leur recrutement et leur endoctrinement fut particulièrement soigné. Les stratèges impériaux comprirent en effet qu'il leur fallait s'assurer de la loyauté de chaque homme en étant au plus près de lui et puisqu'il était hors de question d'affecter des officiers politiques à chaque escouade de l'Empire, il revenait au sergent la commandant de ne jamais oublier certaines contingences d'ordre idéologique. Durant certaines opérations, un sergent était également autorisé à nommer un caporal temporaire parmi les hommes de son escouade, le chargeant de veiller à l'exécution correcte d'une partie du plan de bataille. Dans les unités de blindés ou d'artillerie, la discipline est aussi forte que l'esprit d'initiative est réduit alors qu'à contrario les tireurs d'élite ou les fantassins éclaireurs sont parmi les plus flexibles des soldats portant l'uniforme de l'Empire. Certaines unités s'avérèrent d'ailleurs capables d'obtenir des résultats équivalents et parfois même supérieurs à ceux des Scout Troopers ou des Storm Commandos. La notion d'escouade en ce qui concerne les véhicules blindés est sensiblement différente car il s'agit généralement de deux véhicules de même type travaillant en tandem. De même, une escouade d'éclaireurs se compose en fait de deux "lances" de cinq hommes chacune et compte non pas un mais deux sergents, chacun en charge de 4 hommes. Cette configuration est jugée préférable au niveau hiérarchique dans la mesure où les missions de reconnaissance impliquent une adaptabilité et des risques très supérieurs à ceux inhérents aux missions d'infanterie classique.\\

Quatre escouades peuvent être assemblées en \textbf{Peloton}, dirigé par un Lieutenant assisté d'un sergent-major et des sergents des escouades, ce qui fait un total de 38 hommes au minimum. Pour le tacticien, le peloton est l'unité de taille idéale sur le théâtre des opérations : réduite et donc facile à gérer, elle représente cependant une structure de commandement redondante (quatre sergents peuvent pallier à la mort du Lieutenant ou de son adjoint) qui peut facilement être déployée côte à côte avec une unité d'un autre type : fantassins avec une escouade de blindés, tanks en protection rapprochée d'une unité d'artillerie à grande portée et ainsi de suite. La nature essentiellement offensive des campagnes militaires impériales imposa presque naturellement la création du Peloton d'Assaut, formé de deux escouades de fantassins et de deux escouades de tireurs équipés d'armes lourdes. Cette configuration s'avéra à la fois adaptable aux contraintes du combat dans de multiples milieux et parfaitement efficace pour neutraliser les points de résistance et les camps retranchés durant les opérations d'invasion urbaine. Le Peloton blindé est composé de quatre ou cinq véhicules de même type ainsi que de leurs équipages mais ces effectifs peuvent être doublés dans le cas de blindés légers. Contrairement aux blindés déployés par paires (en "escouade" donc) ou seuls, les pelotons blindés se voient toujours adjoindre deux véhicules de transport qui convoient les pièces, munitions et carburant nécessaires ainsi que les équipes techniques du peloton. La flexibilité institutionnelle des pelotons fait que dans la pratique ils comptent rarement leurs effectifs officiels mais se voient souvent adjoindre une ou plusieurs escouades de spécialistes. Un peloton d'infanterie standard par exemple peut tout à fait se voir augmenté des effectifs d'une escouade d'éclaireurs et d'une escouade de tireurs d'élite si l'on prévoit de l'employer dans des missions de longue durée au c\oe ur d'un territoire ennemi hostile. A l'inverse, les pelotons assemblés pour les opérations spéciales sont fréquemment divisés en escouades mixtes chargées d'accomplir certains objectifs spécifiques dans le cadre de la mission rassemblant l'ensemble du peloton.\\

La \textbf{Compagnie} rassemble quatre pelotons placés sous la tutelle d'un capitaine secondé par quatre officiers d'état-major au grade de lieutenant. Chacun de ces quatre officiers a son rôle spécifique au sein de la compagnie : commandant adjoint, officier responsable de la maintenance, officier médical et officier technique. Ces lieutenants ont sous leurs ordres 23 hommes indépendants des pelotons qui comprennent 4 médecins, 8 intendants et 11 techniciens, formant à eux tous les services annexes. Au total, les effectifs des quatre pelotons, de l'état-major et des services annexes font d'une compagnie une unité aux effectifs théoriques de 152 personnes. Toujours en théorie, la Compagnie est la plus petite unité à laquelle puisse être affectée un contingent permanent de droïdes et ceux-ci sont normalement dédiés aux services annexes, leurs effectifs pouvant dépasser la centaine de droïdes\\

Au niveau de l'artillerie, la compagnie est dénommée "batterie" et ses effectifs sont de 175 hommes qui sont répartis soit en 16 pièces d'artillerie lourde, soit en 32 pièces d'artillerie légère. De tels effectifs qui sont en majorité dédiés à l'entretien des pièces font que les contingents de droïdes affectés aux batteries d'artillerie dépassent rarement les 70 unités.\\

Les compagnies blindées sont en fait des compagnies mixtes infanterie/véhicules d'attaque/artillerie sur répulseurs. Leurs effectifs varient de 200 à 232 hommes et comprennent quatre pelotons d'infanterie, trois pelotons blindés et un peloton d'artillerie. Différentes configurations sont possibles et peuvent modifier cette répartition selon le profil de mission de la compagnie blindée qui compte entre 20 et 40 chars ou véhicules équivalents (y compris bipodes impériaux).
Diverses autres configurations de compagnies sont possibles selon que l'on a affaire à une compagnie de percement, de garnison ou autre, l'articulation se faisant principalement au niveau des pelotons, de leurs effectifs et de leur composition.\\

Le \textbf{Bataillon} comprend sur le papier quatre compagnies dirigées par leurs capitaines respectifs sous les ordres d'un Major qui dispose d'un staff spécifique pour établir et faire fonctionner son QG, ce qui porte ses effectifs théoriques à 810 hommes dont 608 combattants. La composition exacte du bataillon varie cependant considérablement autour de ces chiffres statistiques car il est rare qu'il soit assemblé autour d'un seul type de troupes. Les Bataillons sont en fait généralement considérés comme des forces opérationnelles aux profils de missions spécifiques, ce qui impose donc l'affectation de compagnies et de pelotons en rapport avec ces profils. Les bataillons d'assauts sont parfaitement mixtes alors que les bataillons à répulsion surtout prévus pour occuper rapidement de vastes territoires en cours de pacification peuvent inclure jusqu'à 115 véhicules et 40 moto jets ainsi que leurs équipages en plus des 608 combattants de l'ordre de bataille standard. Le bataillon d'artillerie compte jusqu'à 96 pièces montées sur répulseurs alors que les 93 tanks et 55 landspeeders qui forment un bataillon entièrement blindé s'accompagnent généralement d'effectifs dépassant le millier d'hommes. Enfin, on assemble parfois des bataillons d'opérations spéciales qui ne comptent que 746 hommes mais dont les 544 combattants sont parfaitement capables de mener les opérations de raids stratégiques et d'invasion de QG propres à la guerre éclair.\\

Le \textbf{Régiment} est une unité systématiquement dotée d'un QG fixe dès qu'il est possible de l'établir sur le théâtre des opérations. Ce QG est non seulement le c\oe ur névralgique de la structure de commandement du régiment mais abrite aussi son hôpital et ses ateliers de réparations. La plupart des régiments sont assemblés autour de quatre bataillons. Si l'on considère l'exemple du régiment d'infanterie typique, deux de ces bataillons sont composés de fantassins, le troisième est un bataillon d'assaut mixte (infanterie/blindés/artillerie en proportions diverses) et le dernier entièrement composé de véhicules à répulseurs destinés à transporter l'ensemble du régiment. Au total, ce régiment type compte 3530 hommes dont 2558 combattants et dispose de 130 véhicules ainsi que de 200 droïdes Tous ces effectifs sont dirigés depuis le QG et placés sous la responsabilité de l'officier commandant du régiment, un lieutenant-colonel. Si le régiment d'infanterie ainsi décrit est considéré comme une unité parfaitement autonome, son pendant le régiment blindé n'est normalement jamais déployé seul quant à lui mais uniquement au sein d'une force plus vaste, le Groupe de Bataille.\\

Le \textbf{Groupe de Bataille} est dirigé depuis un QG qui nécessite à lui seul 274 personnes. L'état-major comprend 9 officiers dont 4 major et 5 lieutenants colonels qui rendent tous compte au Colonel en charge du Groupe. Comme sa dénomination l'indique, un Groupe de Bataille n'est pas considéré comme une unité de garnison ou de police mais comme une force d'attaque qui compte au minimum 14410 hommes, 10219 d'entre eux étant effectivement des combattants. Ces effectifs ne sont pas répartis en un nombre fixe de régiments, bataillons ou pelotons car le Groupe de Bataille est à l'image du peloton considéré comme une unité opérationnelle flexible. Dans le cadre d'opérations à grande échelle de l'Armée Impériale, certains "groupes de batailles auxiliaires" sont en réalité entièrement composés de volontaires de la CompForce (la division armées du COMPORN) ou d'une escadre de chasse/bombardement qui compte 40 appareils répartis en 10 ailes de 4 appareils (pour un total de 12 bombardiers, 4 Tie de reconnaissance et 24 Tie de chasse/soutien/couverture aérienne).\\

Le \textbf{Corps d'Armée} est souvent considéré comme la version supérieure du Groupe et existe là encore dans une optique essentiellement offensive. Le Corps est normalement intégré à une force spatiale dans le cadre d'une mission conjointe Marine/Armée qui vise à s'emparer d'un monde en neutralisant ses forces de défense spatiale avant de débarquer les forces d'invasion. Le Major Général qui commande un Corps de Bataille est bel et bien un général en campagne. Il a sous ses ordres des effectifs minimaux de 69 199 hommes dont 48 541 soldats. Plus de 2500 véhicules à répulseurs et au moins 371 blindés forment l'ossature de sa capacité de déplacement au sol. Enfin, son QG protégé par 152 soldats indépendants de la structure du Corps inclut 50 agents du Bureau de la Sécurité Impériale. 492 officiers et responsables de la logistique et au moins un officier des Renseignements Impériaux servant d'agent de liaison. Cette taille est considérée comme minimale et certains Corps d'Armée formés pour s'emparer de mondes très défendus sont bien plus considérables.\\

Sur le papier, un minimum de deux bataillons et jusqu'à quatre Corps peuvent être rassemblés et former une \textbf{Armée}. Cette unité placée sous le commandement d'un Général est considérée comme une fiction opérationnelle car sa nature est en fait fondamentalement administrative. Il est rare qu'une Armée soit déployée en totalité bien que n'importe laquelle de ses composantes puisse être envoyée au feu et que le Général puisse exercer son autorité sur ces subdivisions quand bien même la majeure partie de ses effectifs resterait en garnison. Le QG d'une armée comprend environ 1855 personnes et 4000 droïdes On lui rattache généralement près de 200 agents du BSI, 471 combattants de la CompForce et au minimum quatre agents de liaison des Renseignements Impériaux. Une armée est normalement considérée dans l'ordre de bataille impérial comme une force de garnison, d'occupation ou une division administrative d'une force d'invasion en charge d'un territoire planétaire qui peut aller de la métropole à l'ensemble d'un monde. De fait, il n'existe pas deux armées identiques en dehors de la composition standardisée de leurs Quartier Généraux.\\

Dans la même logique, l'\textbf{Armée de Système} est pour l'essentiel l'abstraction administrative qui rassemble toutes les forces terrestres de l'Empire dans un système stellaire. Placée là encore sous le contrôle d'un Général, l'Armée de Systèmes inclut des éléments de la CompForce, des Renseignements, du BSI ainsi que plusieurs bataillons de Stormtroopers. Le QG est considéré comme l'autorité militaire suprême des forces terrestres dans le système et rend compte directement à la fois au Moff local et à la hiérarchie de l'Armée Impériale.\\

L'échelon ultime de la structure de l'Armée Impériale est l'\textbf{Armée Sectorielle}. Le Maréchal de Surface qui la commande peut avoir des effectifs très variables sous ses ordres mais la configuration standard de l'ordre de bataille indique qu'une armée sectorielle comprend 1 180 309 hommes dont 774 576 combattants et qu'elle dispose de 66 640 véhicules à répulseurs et 13 992 blindés. Bien que les secteurs périphériques ou au contraire stratégiques de l'Empire soient dotés d'armées sectorielles aux effectifs d'une diversité considérable, ces chiffres peuvent être considérés comme une estimation moyenne des effectifs déployés par l'Armée Impériale dans l'ensemble d'un secteur pacifié. Ils ne tiennent pas compte des Bataillons et Corps d'Armée dédiés à l'occupation de mondes spécifiques, ni aux forces d'invasion en transit dans le secteur ou qui se préparent à soumettre les dernières poches de résistance. De même, l'affectation des Stormtroopers est réalisée indépendamment de la structure militaire et il y a des secteurs de l'Empire ou leurs effectifs sont dans les faits largement supérieurs à ceux de l'Armée Impériale.\\

Il n'existe pas de "grand maréchaux de l'Empire" qui seraient l'équivalent pour l'Armée des Grands Amiraux de la Marine. Les Moffs sont considérés comme les représentants directs de l'Empereur et en dehors de quelques individus au statut particuliers (conseillers impériaux, Grands Moffs, hauts responsables du BSI ou du COMPORN, etc\ldots), ils sont dans la pratique l'échelon suprême du commandement de l'Armée Impériale. Ils sont après tous les administrateurs territoriaux de l'Empire et en dehors des campagnes d'invasion réalisées conjointement avec la Marine, l'Armée Impériale ne possède pas (et ne nécessite pas d'après les théoriciens impériaux) un véritable commandement central unifié contrairement à la Marine dont la nature implique une mobilité et une coordination permanentes à l'échelle de la galaxie. Les promotions aux grades supérieurs (Major Général, Général, Maréchal\ldots) sont organisées directement par une commission composée des Moffs les plus influents, le plus souvent lorsqu'il était vivant en présence de l'Empereur ou d'un de ses portes paroles. Dans les faits, bon nombre de Moffs ou de Grands Moffs ont d'ailleurs également le titre honorifique de Maréchal et exerçaient directement le commandement des forces terrestres de leur juridiction\ldots

\subsection{La Marine Impériale}
\subsubsection{Histoire de la Marine Impériale}
L'expansion constante de la civilisation issue du Noyau et ses rencontres avec d'autres civilisations également en pleine croissance ont toujours rendu essentielle la présence d'une marine spatiale armée. En dehors même des conflits majeurs qui secouèrent la galaxie, la présence d'une multitude de groupes pirates, de renégats divers, de peuples hostiles dans un volume d'espace incommensurable et en perpétuelle augmentation n'ont jamais manqué de causer de nombreuses morts et pertes financières, sans parler d'une quantité appréciable de conflits de "basse densité".\\

Durant sa longue histoire, la République Galactique connut alternativement des périodes de course à l'armement effrénée et de calme plat. Entretenir et moderniser des flottes de combat dont le théâtre d'opérations s'étendait un peu plus chaque jour ne s'avéra pas toujours possible et si à certaines périodes de son histoire la République disposait de nombreuses forces spatiales modernes, il s'avéra à d'autres moments nécessaire de trouver des palliatifs divers : mercenariat, recyclage de navires antiques, incitations aux secteurs les plus riches à développer leurs propres forces de défense et ainsi de suite\ldots\\

Le dernier millénaire de l'histoire de la République ne connut guère de conflits majeurs et amena progressivement les décideurs de tous poils à restreindre lentement mais sûrement les crédits militaires. Petit à petit, les forces navales de la République se réduisirent et elle se mit à privilégier les solutions sénatoriales et diplomatiques à la plupart de ses problèmes. Peu avant la crise de Naboo, la République ne possédait plus de véritable marine spatiale à l'exception de quelques groupes d'unités affectés à des systèmes stratégiques. Les principales factions qui la composaient avaient développé leurs propres flottes spatiales dont certaines comme celle de la Fédération du Commerce étaient vraiment très conséquentes. Les rares cas de force majeure nécessitant une intervention militaire à grande échelle se réglaient dans une sous-commission ou dans les couloirs du Sénat lorsque plusieurs représentants de systèmes ou de groupes influents s'arrangeaient pour obtenir des faveurs conséquentes en échange de leur participation à l'effort militaire du moment. Les solutions diplomatiques étaient préférables non seulement pour des raisons économiques ou humanitaires mais aussi parce que le recours à la force armée impliquait de plus en plus souvent de donner encore plus de poids à des groupements d'intérêts qui ne prêtaient pas leurs forces sans avoir en échange des garanties sur certaines affaires les concernant\ldots certains Sénateurs virent bien les dangers que recélait cette évolution des choses mais les lobbys les plus influents étaient déjà fermement implantés dans le Sénat républicain et dans de nombreux gouvernements locaux.\\

À la longue, certaines factions devinrent des puissances militaires difficilement contrôlables et les ambitions indépendantistes trouvèrent là un terreau supplémentaire sur lequel fleurir. Quinze ans à peine avant la Crise de Naboo, quelques sénateurs républicains eurent un dernier sursaut et donnèrent leur aval à un projet qui devait permettre de moderniser la petite marine républicaine de plus en plus surclassée par les autres forces militaires. Cette tentative qui donna naissance à la Flotte Katana se solda par le désastre bien connu et sonna le glas des ultimes ambitions militaires de l'Ancienne République. Tout au moins, jusqu'à ce que la galaxie se retrouve plongée dans une guerre dévastatrice\ldots
La course à l'armement et les investissements monstrueux que provoqua la Guerre des Clones marquèrent un tournant significatif de l'histoire navale militaire. Comme il l'avait fait pour se constituer une armée terrestre sans égal dans l'histoire galactique, l'Empereur Palpatine sut tirer parti des craintes d'une population traumatisée par un conflit aussi bref que dévastateur et récupéra à son profit une bonne partie des arsenaux et des navires des belligérants. Pour "restaurer la paix dans la galaxie et s'assurer que des sursauts indépendantistes ou divers groupes de pirates ne menacent plus les populations", l'Empereur obtint l'appui des industriels et des militaires qui pour la plupart acceptèrent de le suivre sans rechigner.\\

Le programme d'armement qui suivit fut proprement gigantesque. L'Empire revendit des navires obsolètes à divers groupes alliés comme l'Autorité du Secteur Corporatif et les encouragea pendant un temps à renforcer leurs propres forces de défense tout en augmentant la pression fiscale et le pillage généralisé des mondes hostiles afin d'armer une flotte sans égal dans toute l'histoire de la galaxie. À la longue, les alliés impériaux purent conserver des escadres destinées à leur usage exclusif mais devaient dans le même temps contribuer au financement des escadres impériales. La nouvelle génération des destroyers de classe Impériale I devint rapidement le symbole de l'armada impériale, près de 25.000 de ces navires gigantesques furent mis en service depuis le couronnement de Palpatine et viennent appuyer une multitude de navires plus légers et des légions sans nombre de chasseurs spatiaux. Les secteurs jugés trop timorés ou potentiellement rebelles furent brisés et leur économie vampirisée par le programme d'armement de la Marine Impériale. L'Académie de la Marine Impériale dont le principal campus est basé sur Raythal a établi des antennes partout dans la galaxie, récupérant et assimilant souvent les anciennes institutions républicaines ou celles créées par des gouvernements locaux fortunés bien que certains parmi les plus fidèles aient été autorisés à conserver le contrôle de forces spatiales d'appoint réduites.
De tous les composants de l'Empire Galactique, la Marine Impériale est celui qui représente l'effort financier le plus gigantesque et c'est essentiellement pour alimenter cette machine monstrueuse que l'Empire procéda à des pacifications et des conquêtes sans nombre.\\

Le paradoxe de la Marine Impériale est que l'Empire ne l'a jamais déployé dans toute sa puissance et que fondamentalement, ce programme titanesque fut un énorme gaspillage.\\

En effet, bien que les escadres impériales soient en mesure de briser n'importe quel monde, n'importe quelle flotte spatiale dans un assaut frontal, l'Empire est trop étendu même pour ses légions de navires et le déploiement de la Marine n'est pas si rationnel qu'on pourrait le croire. Si des secteurs jugés importants ou dont les dirigeants ont l'oreille de l'Empereur se voient attribuer des groupes de combat comptant parfois plusieurs centaines d'unités lourdes, d'autres systèmes sont négligés. Plus grave encore, alors que l'état-major de la Marine n'aurait guère eu de problème à mobiliser rapidement une escadre capable d'anéantir un monde, l'Empereur jeta littéralement en l'air des sommes incroyables pour satisfaire une de ses lubies : les Etoiles de la Mort. Bien que le symbole de ces super armes soit fort et destiné à terroriser les ennemis de Palpatine, dans les faits elles étaient totalement inutiles car la puissance combinée de toutes les unités de la Marine déployées là où elles ne servaient à rien était plus que suffisante pour anéantir n'importe quel adversaire et le symbole d'escadres entières de destroyers s'abattant sur un monde aurait certainement été tout aussi frappant que celui d'une station de combat, aussi gigantesque soit-elle.
Pour aller à l'essentiel, bien qu'elle soit été dotée à tous les niveaux d'un grand nombre d'individus compétents voire même exceptionnels, la Marine Impériale s'avère être dans la pratique un mastodonte ingérable, un gouffre financier avec des effectifs colossaux qui dépassent tous les besoins prévisibles.\\

Deux ans avant la Bataille de Yavin, l'Empereur décida finalement de restructurer ses flottes et créa les Douze Grands Amiraux de l'Empire, veillant à susciter suffisamment d'antagonismes en leur sein pour qu'ils ne puissent constituer un front susceptible de le menacer. Certains de ces hommes étaient des plus compétents, d'autres des opportunistes ou des fanatiques, voire des fous dangereux et l'on considère qu'ils ne remplissent pas vraiment les attentes de l'Empereur\ldots. Ces personnalités antagonistes doivent aussi lutter contre les traditions d'excellence des académies de marine récupérées par l'Empire, des institutions souvent très anciennes qui forment les futurs officiers en les encourageant à un minimum d'initiative qui s'accommode mal de la bureaucratie gigantesque que forme la Marine Impériale. Cette situation déplait également aux censeurs politiques de Palpatine. Bien que la majorité des officiers de la Marine Impériale soient loyaux envers le régime dont ils portent l'uniforme, les capitaines de vaisseau ont de tous temps été des autocrates qui renâclent à subir les diktats des bureaucrates et des états-majors.\\

\subsubsection{Organisation de la Marine Impériale}
Surveiller l'ensemble de l'espace connu avec des dizaines de milliers de navires de tous types et de tous âges nécessite une flexibilité opérationnelle qui n'est pas toujours compatible avec les jolis schémas établis par les technocrates qui abondent dans tous les états-majors de la galaxie. Bien que des efforts d'harmonisation colossaux aient été entrepris, notamment par la standardisation des équipements embarqués comme les Chasseurs TIE, les bâtiments lourds sont de nature bien plus hétérogène. Et si les groupes sectoriels du centre de la galaxie possèdent des compléments de destroyers stellaires modernes conformes à l'ordre de bataille théorique, certains secteurs de la Bordure Extérieure doivent se contenter de vieux cuirassés recyclés ou de destroyers de classe Victoire en guise de vaisseaux amiraux ou parfois même de seuls navires lourds disponibles. 

\paragraph{Le Commandement de Vaisseau}
Capitaine\ldots un rêve vieux comme l'histoire de la civilisation galactique et qui est la transposition du rêve encore plus ancien des marins primitifs qui traversaient les océans de milliers de mondes dans un passé presque légendaire. Pour bien des officiers issus des académies militaires de tous les gouvernements de l'histoire galactique, obtenir le commandement de son propre vaisseau était pratiquement considéré comme l'aboutissement ultime d'une carrière. D'ailleurs, la plupart des traditions militaires acceptèrent que par principe de nombreux capitaines renoncent à diverses promotions ou insistent malgré les progrès technologiques pour demeurer sur la passerelle d'un vieux navire fidèle dont ils avaient été le seul ma\^{i}tre pendant des années. À cet égard, la Marine Impériale n'est pas différente de toutes les autres bien qu'elle "visse" davantage ses officiers. Chaque capitaine sait en effet que le Bureau de la Sécurité Impériale a placé au moins un de ses agents infiltrés parmi les officiers de son bâtiment\ldots un agent qui peut faire appel à des pouvoirs exceptionnels pour prendre le commandement au cas où le capitaine légitime ferait montre d'un comportement "suspect". Dans l'histoire de la Marine Impériale, certains de ces hommes n'hésitèrent pas à organiser des mutineries pour renverser un capitaine jugé "politiquement inapte", quitte ensuite à commander eux-mêmes l'exécution des mutins une fois la situation redevenue "politiquement correcte"\ldots

\paragraph{La Ligne}
La Ligne est l'équivalent dans la Marine de l'escouade terrestre. Si théoriquement une Ligne doit compter quatre navires de tonnages variés mais plus lourds que des chasseurs ou navettes, la pratique donne des unités comptant de deux à vingt navires. De fait, la Ligne est l'unité de déploiement collective la plus flexible dans l'ordre de bataille impérial et au sein d'une Escadre, les lignes théoriques sont souvent remaniées en fonction des nécessités du moment. Une ligne est placée sous le commandement d'un Capitaine de Ligne, c'est à dire le plus souvent du capitaine de vaisseau le plus expérimenté ou le plus décoré dont le navire fait office de vaisseau amiral.

\begin{description}
	\item[Ligne d'Attaque] Une configuration qui vise à affronter une autre force spatiale, si possible de tonnage et puissance de feu comparables. Une ligne d'attaque peut compter jusqu'à six frégates, corvettes ou croiseurs légers mais un destroyer de classe Impériale est considéré comme une ligne à lui seul.
	\item[Ligne d'Attaque Lourde] De quatre à huit croiseurs ou frégates, voire des navires de tonnage plus important y compris des destroyers. En théorie, une telle ligne est capable de soutenir le feu de n'importe quel autre type de ligne qu'on lui opposerait. Si possible, un assaut sur une planète faiblement défendue sera mené par une Ligne d'Attaque Lourde mais la taille gigantesque de la galaxie oblige souvent l'Empire à improviser avec ce qu'il avait sous la main pour pallier aux urgences. 
	\item[Ligne de Poursuite] Elle peut compter une dizaine de corvettes, d'avisos et autres navires de moyen tonnage dont les capacités subluminiques et hyperspatiales sont conséquentes. Le rôle d'une ligne de poursuite est de pilonner une force adverse en cours de retraite, afin d'éliminer les trainards, voire d'encercler des bâtiments trop endommagés pour fuir et les capturer. Éventuellement, si les trajectoires de saut hyperspatial des fuyards sont clairement identifiées, une ligne de poursuite peut se lancer sur leurs traces dans l'hyperespace mais l'Empire eut l'occasion d'apprendre à ses dépens que l'Alliance largement surclassée en termes numériques et matériels savait livrer des batailles destinées à attirer les forces impériales dans un piège\ldots les lignes de poursuites étaient également souvent déployées dans une zone où l'on soupçonnait la présence d'une base de rebelles ou de pirates afin de la localiser et de la harceler le temps qu'une force d'attaque plus conséquente survienne pour le gros nettoyage. L'Empire apprit également à se servir de ses propres lignes de poursuite pour amener des adversaires imprudents et qui avaient l'avantage à portée de lignes d'attaques dissimulées\ldots
	\item[Ligne de Reconnaissance] De deux à quatre appareils de faible tonnage (avisos, corvettes, navires éclaireurs impériaux\ldots) opérant seuls ou par paires. Leur fonction n'est pas d'engager le combat même avec un élément adverse isolé mais de localiser l'ennemi et de rendre compte. Lorsqu'une force impériale arrive dans un nouveau système hostile, il est également fréquent qu'une ou plusieurs lignes de reconnaissance soient détachés pour fouiller les champs d'astéroïdes, les géantes gazeuses et autres endroits ou l'ennemi pourrait dissimuler ses forces pendant que le gros des effectifs converge sur les objectifs principaux. Ce sont également des vaisseaux distribués en lignes de reconnaissance qui procèdent aux patrouilles avancées et aux missions de reconnaissance au long cours et en dehors des routes commerciales, ils sont souvent les premiers bâtiments impériaux auxquels sont confrontés les dissidents ou les peuples des systèmes inconnus.
	\item[Ligne d'Escarmouche] Composée de 15 à 20 bâtiments légers (avisos, patrouilleurs, navettes\ldots), elle a pour tâche de harceler l'ennemi afin de l'empêcher de se concentrer sur les navires lourds qui vont le pilonner. Lorsque l'Empire développa sa chasse spatiale avec ses légions de chasseurs TIE bon marché, le concept de ligne d'escarmouche fut peu à peu abandonné par la plupart des tacticiens impériaux mais les plus intelligents comprirent face à la grande diversités de chasseurs, navettes et cargos modifiés que comptaient les forces de l'Alliance qu'il fallait préserver de tels dispositifs plus polyvalents et fiables que des escadrilles de chasseurs légers dépourvus d'hyperpropulsion. Elles constituent également des forces d'escorte ou de défense intra système intéressantes.
	\item[Ligne de Torpillage] Une telle "ligne" est en fait constituée d'une Sphère à Torpilles qui se voit affecter une ou plusieurs autres lignes d'attaque ou d'escarmouche afin de la protéger et d'appuyer les assauts planétaires.
	\item[Ligne de Transport de Troupes] La configuration standard d'une force d'invasion planétaire fait appel à autant de vaisseaux d'escorte (généralement des croiseurs de classe Strike ou des Frégates) que de transports de troupes bien qu'en fonction des objectifs, les effectifs terrestres embarqués puissent varier considérablement. Une telle ligne compte une dizaine de bâtiments en général. Une ligne de transports de troupes opérant à proximité des principales bases impériales locales est également souvent appuyée par des navettes et autres navires légers hypercapables qui facilitent le déploiement des forces terrestres en plus des navires de la ligne elle-même. En réalité, les effectifs embarqués à bord des destroyers stellaires et des croiseurs de classe Strike en configuration transports de troupes rendaient souvent les Lignes de Transport de Troupes relativement inutiles en dehors des véritables assauts contre des mondes industrialisés et densément peuplés.
\end{description}


\paragraph{L'Escadre}
Comptant de 20 à 60 navires lourds ainsi que leurs appareils de soutien et chasseurs, l'Escadre de l'Ordre de Bataille Impérial en dit long sur le gouffre financier que représente la Marine Impériale puisque à peine un siècle avant son existence, un tel regroupement de navires méritait encore le qualificatif de "flotte" dans l'ancien ordre de bataille républicain. Des forces comme la Flotte Katana avec ses 250 cuirassés étaient considérées à l'époque comme gigantesques et dans la pratique, elles constituaient une exception et non une règle, la moyenne des flottes républicaines tournant autour de 20 à 50 navires. Une Escadre impériale est toujours commandée par un Amiral et représente en théorie la force la plus importante que l'on puisse déployer dans un seul système stellaire en dehors de circonstances exceptionnelles ou d'une campagne majeure.
\begin{description}
	\item[Escadre Légère] Composée en général d'une Ligne d'Escarmouche, une Ligne de Reconnaissance et deux Lignes d'Attaque. L'autre configuration de base implique un grand volume d'espace à couvrir avec une faible résistance attendue et se compose en conséquence de deux Lignes de Reconnaissance, une Ligne de Poursuite et une Ligne d'Escarmouche.
	\item[Escadre Lourde] Au minimum, un tel déploiement comporte deux Lignes d'Attaque Lourdes, une Ligne d'Attaque et une Ligne de Reconnaissance.
	\item[Escadre de Bataille] Configuration relativement polyvalente, elle compte normalement une Ligne de Poursuite (frégates et corvettes) et deux Lignes d'Attaque standard (croiseurs, cuirassés) ou même Lourdes (destroyers stellaires). Il est peu de forces spatiales dans un système non militarisé qui puissent résister à un tel déploiement qui représente une force de frappe terrestre et de combat spatial performante.
	\item[Escadre de Transport] Deux Lignes de Transport accompagnées d'une Ligne d'Attaque et d'une Ligne d'Escarmouche sont considérés comme formant une force suffisante pour soumettre un système stellaire à faible densité de population et dont les forces de défense n'ont rien de conséquent.
	\item[Escadre de Bombardement] Formée par deux Lignes de Torpillage, une Ligne d'Escarmouche et une Ligne de Poursuite, une escadre de ce type est capable normalement de briser n'importe quel bouclier de défense planétaire (elle compte deux Sphères à Torpilles\ldots) tout en pouvant repousser efficacement une chasse ennemie ou des forces de défense légères. Dans la pratique, utiliser deux Sphères à Torpilles dans un même système stellaire n'a encore jamais été fait et cette configuration reste théorique. De fait, un système stellaire suffisamment militarisé et développé pour nécessiter la présence de deux Sphères est aussi doté d'une flotte de défense convenable que la Marine Impériale devra briser avant tout. Une fois l'espace environnant ma\^{i}trisé par l'Empire, un monde isolé derrière ses boucliers de protection ne peut espérer grand-chose et une seule Ligne de Torpillage est suffisante pour anéantir ses défenses et permettre aux autres bâtiments qui ont détruit ses forces spatiales de pilonner la surface à loisir\ldots
\end{description}


\paragraph{La Force Spatiale}
Chargée de surveiller l'espace d'un système stellaire stratégique et de plusieurs systèmes proches, une telle force est placée sous le commandement d'un amiral qui porte le titre de Commodore. La tâche d'une Force Spatiale n'est pas une sinécure car elle doit souvent contrôler plusieurs centaines d'années-lumière d'espace dans les trois dimensions. En dehors des navires les plus lourds, l'essentiel des bâtiments ne possède pas d'accès à l'Holonet Impérial. Coordonner des navires en temps réel sur un tel volume d'espace demande donc des qualités de planificateur et de stratège certaines car il faut à la fois organiser des patrouilles régulières, établir des points de contrôle fixe et maintenir en disponibilité opérationnelle un contingent d'appareils constamment en mouvement. Dans certains secteurs de la Bordure Extérieure, l'Empire doit se contenter d'une Force Spatiale pour un secteur entier qui englobe parfois une zone qui représente largement de quoi occuper trois ou quatre Forces Spatiales.
Une Force de Spatiale est composées de plusieurs groupes secondaires et bien que leur répartition soit standardisée, les Commodores impériaux sont également censés modifier leur ordre de bataille en fonction des impératifs locaux ou tactiques.

\begin{description}
	\item[Force de Supériorité] L'objectif théorique majeur de la Marine Impériale est d'assurer la Supériorité Spatiale de l'Empire, définie comme suit "absence totale d'appareils hostiles dans l'espace orbital des mondes sous contrôle et élimination des activités ennemies dans l'ensemble de l'espace des systèmes contrôlés". Pour faire de cette idée une réalité, l'Empire considère donc qu'il doit disposer dans chaque système occupé de plusieurs navires disponibles et capables d'intervenir selon les besoins ponctuels. C'est cette priorité stratégique (et doctrinale\ldots) qui rend si gigantesque et coûteuse la Marine Impériale. Une Force de Supériorité compte en effet un minimum de trois Escadres de Bataille et une Escadre Légère. L'ordre de bataille impérial considère que trois des navires d'une Force de Supériorité doivent obligatoirement être des destroyers stellaires de classe Impériale. Dans la pratique, l'Empire atteint rarement un tel minima dans la Bordure (certains secteurs n'ont qu'un ou deux destroyers impériaux au total\ldots) alors que certains systèmes du Noyau sont nettement plus favorisés et comptent parfois plus d'une dizaine de destroyers affectés à une même Force de Supériorité.
	\item[Force d'Escorte] À l'encontre des Forces de Supériorité dont le rôle est à la fois offensif, défensif et dissuasif, les Forces d'Escorte ont pour tâche la surveillance des routes commerciales ou militaires ainsi que des mondes stratégiques. Leurs effectifs sont donc généralement affectés à des opérations de longue durée ou en rotations prévisibles bien qu'elles soient également chargées de poursuivre les pirates ou les rebelles qui tentent de s'en prendre aux objectifs qu'elles protègent. Une Force d'Escorte compte deux Escadres Lourdes et deux Escadres Légères.
	\item[Force de Transport] En dehors des plus grandes campagnes de conquête à long terme (quand l'Empire s'en prend à une nation stellaire comptant plusieurs systèmes industrialisés par exemple), il est inutile d'assembler une telle force qui compte deux Escadres de Transport renforcés d'une Escadre Légère.
	\item[Force de Maintenance] Là encore, rassembler une centaine de navires ateliers, de navires hôpitaux et de navires de récupération est presque toujours inutile en dehors des campagnes militaires les plus importantes. Le reste du temps, la Marine affecte un contingent de bâtiments de soutien directement auprès d'un Amiral et le laisse se débrouiller avec. Ces affectations sont des plus aléatoires et ont souvent lieu en dépit du bon sens ce qui n'a pas été sans conséquences sur le gaspillage monstrueux auquel se livre la Marine Impériale. Nombre de Récupérateurs en tous genres ont en effet fait fortune simplement en récupérant, reconditionnant et revendant à la Marine Impériale certains de ses propres "déchets" encore utilisables. À cet égard, la politique de gestion des ressources humaines sur le plan médical s'est révélée à peine plus performante que celle des ressources matérielles et nombre d'officiers supérieurs parmi les plus respectés de leurs hommes au sein de la Marine Impériale ont tout simplement conquis ce respect en accordant une vigilance particulière à l'entretien du matériel et aux soins accordés à leurs subordonnés.
\end{description}

\paragraph{La Flotte Spatiale}
Une Flotte Impériale compte au minimum quatre Forces Spatiales, dont une de Maintenance. Un tel déploiement représente un strict minimum de 160 navires de classe Aviso ou supérieure et peut en compter jusqu'à 5 fois plus. L'ordre de bataille impose qu'au moins six des navires d'une Flotte Spatiale soient des Destroyers de classe Impériale. En théorie, l'état-major de la Marine distingue des flottes de supériorité, de maintenance, d'attaque\ldots mais dans la pratique, de telles différences sont inapplicables. Chaque flotte se voit simplement attribuer une répartition d'escadres spécialisées conforme à ce que le commandement estime approprier aux objectifs de la dite Flotte. Pour les stratèges impériaux, une Flotte est une "disponibilité opérationnelle sectorielle" c'est à dire que ses effectifs peuvent avoir à intervenir n'importe où dans l'espace du secteur ou ils sont affectés, pour des missions de nature et de durée très variables. Une Flotte Spatiale est considérée comme suffisante pour assurer la supériorité spatiale de l'Empire dans un secteur classé comme "calme", c'est à dire ne comptant pas plus de seize planètes abritant des opposants en quantité significative dont quatre au maximum ouvertement hostiles.\\

L'état-major affecte également à certaines Flottes des forces spéciales de cales sèches, c'est à dire un contingent d'ateliers et de chantiers spatiaux. Lorsque l'Empire considère un système stratégique comme devant servir à ravitailler ou entretenir ses forces spatiales, la majeure partie des effectifs de la Flotte locale sont de fait dédiés au fonctionnement des chantiers et dépôts impériaux et elle compte alors jusqu'à trois Forces de Maintenance pour une Force de Supériorité Spatiale. Les systèmes stellaires de ce genre servent alors de zones de regroupement aux forces impériales à la veille de campagnes de longue durée et bon nombre de ces zones sont situés dans la Région d'Expansion.

\paragraph{Le Groupe Sectoriel}
Dans l'ordre de bataille standard, il n'existe normalement pas de groupe de navires impériaux plus important qu'un groupe sectoriel qui représente "l'ensemble des unités de la Marine Impériale affectées à un Secteur spatial". Un Groupe Sectoriel est commandé par un amiral en chef qui s'avère souvent dans la pratique être tout simplement le Moff local, à moins que la Marine soit souvent sur la brèche dans son secteur auquel cas un amiral surnuméraire est affecté à son service et gère pour lui ce genre de choses.
Dans sa configuration idéale, un Groupe Sectoriel ne compte pas moins de 1600 navires de classe Aviso ou supérieure dont 24 destroyers stellaires de classe Impériale. Bien que de nombreux secteurs de l'Empire Galactique aient pu bénéficier de tels effectifs, une analyse plus fine montre que la presque totalité de ces secteurs sont situés dans le Noyau, les Colonies ou plus rarement la Région d'Expansion. Comme à l'accoutumée, les secteurs périphériques sont lotis de manière beaucoup plus inégalitaire et souvent dépendante à la fois de l'intérêt stratégique du secteur pour l'état-major mais aussi de l'influence et des relations du Moff local. Ainsi, un secteur comme celui d'Elrood ne dispose en tout et pour tout que de deux destroyers impériaux et une dizaine de navires lourds de tous types alors que plus loin vers les Régions Inconnues, le secteur Kathol doit au Moff Kenton Sarne de disposer de plus d'une demi-douzaine de destroyers stellaires et d'une trentaine de navires secondaires.


\section{L'Alliance Rebelle}
\subsection{Généralités sur l'Alliance Rebelle }
\subsubsection{Les Objectifs de l'Alliance}
Comme son nom officiel l'indique, l'Alliance pour la Restauration de la République considère comme nécessaire le retour à une forme de gouvernement galactique global qui soit démocratique. Malheureusement, la propagande impériale qui jouait sur la décadence de l'Ancienne République et sur les actes de violence parfois gratuits des nombreux mouvements rebelles avait beau jeu d'utiliser les proclamations rebelles comme autant d'armes psychologiques à son service. Sur le plan politique, l'Empire assimila toujours l'Alliance soit à un mouvement anarchiste visant à instaurer le chaos général au service d'intérêts nébuleux, soit à un mouvement analogue à la Confédération des Systèmes Indépendants de triste mémoire. Outre la puissance de l'appareil de répression impérial, on peut aussi compter sa propagande intense et la peur d'un nouveau conflit galactique ancrée dans l'esprit de nombre de citoyens comme facteurs expliquant pourquoi l'opposition au régime de Palpatine fut souvent réduite à des idéalistes, des visionnaires ou des gens désespérés. Bien qu'elle possède sa propre monnaie, sa propre armée et son propre gouvernement, l'Alliance demeurait relativement inconnue de la majorité des gens qui ignoraient son existence ou ne la percevaient qu'à travers les mensonges impériaux. On doit cependant porter au crédit de millions d'individus courageux le fait que la vérité ait pu de manière plus ou moins discrète et risquée circuler et atteindre bien plus de gens que l'Empire ne l'aurait souhaité. Au final, un courant de sympathie latent mais bien réel envers la cause rebelle se constitua durant la guerre civile et contribua après la mort de l'Empereur à provoquer nombre d'évènements qui firent de la Nouvelle République une réalité.

\subsubsection{La Structure de l'Alliance}
On peut considérer qu'il existe trois branches principales dans l'Alliance. Chacune d'entre elles sera décrite de manière plus détaillée par ailleurs.

\begin{itemize}
	\item le Gouvernement de l'alliance est sa véritable instance dirigeante. Il rassemble des personnalités influentes des principaux mondes officiellement (comme c'est le cas de Mon Calamari) ou officieusement rattachés à l'Alliance (Kasshyyk ou Alderaan par exemple). Cette instance a pour double objectif à la fois de diriger l'Alliance elle-même mais aussi de servir de gouvernement de transition une fois l'Empire Galactique abattu. Son rôle sera alors d'organiser avec les représentants des mondes qui rejoindront la République restaurée un nouveau gouvernement galactique.
	\item Les Forces Armées sont à la fois l'outil le plus essentiel à la survie de l'Alliance mais aussi le plus délicat à utiliser car l'Empire Galactique et ses alliés possèdent des forces sans commune mesure avec les volontaires des armées rebelles. Cependant, l'Alliance possède une armée terrestre et une flotte spatiale qui ont su faire leurs preuves dans la multitude d'engagements directs contre les forces impériales malgré la supériorité technologique et numérique écrasante de l'ennemi.
	\item Les Services Secrets sont quant à eux la branche de l'Alliance dont le rôle est le plus difficile. Agents de terrain, sympathisants, agents de liaison avec divers mouvements locaux, informateurs et saboteurs forment une complexe nébuleuse qui permet à la fois de découvrir les intentions de l'ennemi et de lui fournir de fausses informations mais aussi d'obtenir du ravitaillement, de causer des dégâts conséquents dans la structure du complexe militaire impérial et parfois même de découvrir le moyen d'empêcher la victoire décisive de l'ennemi, comme ce fut le cas à Yavin. 
\end{itemize}

\subsubsection{Chronologie Historique}
S'il faut en croire les entretiens qu'eut Mon Mothma avec l'historien Arhul Hextrophon entre les batailles de Yavin et Hoth, l'Alliance pour la Restauration de la République eut des débuts plutôt incertains et faillit bien ne jamais voir le jour.\\

Dès le couronnement de Palpatine, il y eut des idéalistes ou des gens sensés pour se dresser contre lui. Lorsque des manifestations de protestations pacifiques eurent lieu contre certaines ordonnances impériales dans les mois qui suivirent la fin de la Guerre des Clones, la répression sauvage qui s'ensuivit fit également comprendre à beaucoup de gens qui étaient moins idéalistes ou sensés qu'il fallait faire quelque chose.\\

De nombreux mouvements de résistance virent le jour mais ils étaient animés par des amateurs et peu de vétérans de la guerre encore en âge de combattre avaient eu le droit de retourner chez eux. La plupart des militaires qui avaient soutenu les armées clones avaient en effet tout simplement été incorporés aux forces impériales. Nombre de ces soldats furent d'ailleurs les victimes des purges menées par l'Ordre Nouveau dans les mois et les années qui suivirent.\\

Les différents groupes d'opposants au nouveau régime étaient trop disparates, à tous niveaux. Certaines prônaient l'action violente, d'autres les campagnes de lobbying, d'autres encore se retrouvèrent rapidement trop occupées à leurs affrontements internes pour avoir un véritable impact sur l'Empire Galactique. Cependant, un grand nombre de gouverneurs de systèmes ou de moffs nouvellement mis en place rencontrèrent des difficultés dans l'exercice de leur mandat face à une opposition aussi disparate que résolue.\\

Durant les dernières années de la République, Mon Mothma qui venait juste d'entamer sa carrière de Sénatrice de Chandrila se fit rapidement conna\^{i}tre comme un brandon de discorde face au pouvoir sans cesse croissant du Chancelier Palpatine. Elle fut parmi les premiers politiciens du Sénat a sérieusement envisager une désobéissance civile générale contre le chancelier suprême et rapidement, elle se mit même à employer le mot de "révolution" en petit comité.\\

Bail Organa d'Alderaan qui devait être son plus fidèle allié par la suite fut parmi ses opposants à cette époque car il savait tous des intentions de la représentante de Chandrila. Durant leurs nombreuses rencontres privées à Cantham House (la résidence de Bail sur Coruscant), ils tombèrent d'accord sur de nombreuses choses mais Organa refusa toujours de soutenir un point de vue révolutionnaire. Il craignait que Mon Mothma ne fasse le jeu de ceux qui utiliseraient l'inquiétude causée par une tentative de révolution ou de révolte pour renforcer encore le pouvoir de Palpatine et de ses alliés. Bien qu'Organa ait été inquiet du développement de l'Armée Clone, il  n'était pas encore prêt à se dresser contre ce qu'il percevait encore comme le gouvernement légitime.\\

D'après Mon Mothma, trois évènements firent d'un coup basculer Bail dans le camp des opposants les plus résolus à Palpatine. Le premier de ces évènements est peu connu des historiens mais n'est pas dépourvu de conséquences. Les habitants de la planète Ghorman dans le Secteur Sern des Colonies décidèrent de s'opposer à ce qu'ils percevaient comme des taxes abusives, prétendument justifiées par l'effort de guerre imposé par le conflit contre la Confédération des Systèmes Indépendants. Ils décidèrent alors d'organiser un sitting pacifique sur la piste du spatioport ou le vaisseau de la minuscule marine républicaine rattachée au Sénat devait atterrir pour collecter les taxes. Le capitaine du vaisseau en question décida de se poser malgré tout et incinéra avec ses propulseurs plusieurs douzaines de citoyens de Ghorman. Non seulement cet officier ne fut pas sanctionné mais le Chancelier Palpatine obtint même qu'il ait une promotion parce qu'il avait "servi la République contre les traitres et les lâches". Ce capitaine fit encore parler de lui par la suite dans des circonstances bien plus célèbres. Il s'appelait Wilhuff Tarkin.\\

Le second évènement qui affecta Bail Organa est d'une portée beaucoup plus considérable. Ce fut la proclamation de l'Empire Galactique et le transfert définitif des pleins pouvoirs à Palpatine, privant ainsi le Sénat d'une bonne partie de ses prérogatives et surtout de son rôle effectif d'organe de gouvernement. Lié à cette proclamation, il faut également compter avec l'attaque du Temple Jedi de Coruscant dont le sénateur d'Alderaan fut témoin. Il eut en secret de longs entretiens avec les derniers survivants de l'Ordre Jedi, Yoda et Obi-Wan Kenobi, mais la disparition prématurée de Padmé Amidala qui aurait pu être une figure de proue pour les opposants au nouveau régime contribua à renforcer le désarroi du représentant d'Alderaan. Reconnaissant qu'il avait peut-être tenté trop longtemps de s'opposer diplomatiquement à une force considérablement plus puissante que lui sur le plan politique, il décida de reprendre contact avec Mon Mothma. Bien qu'il se soit toujours opposé à Palpatine, Bail Organa n'avait jamais affronté de face les partisans du Chancelier Suprême, contrairement à sa jeune collègue. Il avait donc des relations jusque dans les rangs de ses ennemis et cela lui fut plus utile qu'il ne l'aurait jamais soupçonné. Ainsi, lorsqu'il apprit que le tout nouveau Bureau de la Sécurité Impériale comptait arrêter la sénatrice de Chandrila, il parvint à la prévenir et elle s'échappa d'extrême justesse.\\

Livrée à elle-même, Mon Mothma sut mettre à profit les entretiens de Cantham House et les nouvelles idées de Bail Organa pour lancer un mouvement visant à unifier les opposants au nouveau régime.\\

Des années de fuite et d'efforts continus pour fédérer les différents mouvements rebelles finirent par porter leurs fruits lorsque trois des principales factions signèrent sur Corellia un document les engageant irréversiblement dans un mouvement commun. C'est ainsi que vit le jour officiellement l'Alliance Pour la Restauration de la République qui ne tarda pas à diffuser un document qui resterait par la suite essentiel dans le destin de la galaxie, la Déclaration Formelle de Rébellion.
Des millions d'exemplaires de ce texte furent distribués (parfois au prix de nombreuses morts) partout dans la galaxie et provoquèrent la première fissure perceptible dans l'édifice de la propagande impériale. Malheureusement, les gouvernements planétaires qui eurent le malheur de déclarer leur soutien à l'Alliance furent sauvagement réprimés par les forces impériales. Les Mondes Sécessionnistes, comme ils furent baptisés de manière éphémère, ne vécurent que quelques semaines hors de la tutelle impériale et cette liberté fut marquée par d'incessants combats jusqu'à ce que les forces de Palpatine les submergent.
Certains parvinrent cependant à confier à l'alliance de l'argent, des armes ou des soldats aguerris avant que la machine de guerre impériale ne les broie et c'est ainsi que l'Alliance obtint les fondations de son armée. De même, alors que la Déclaration Formelle avait été férocement censurée par l'Empire, la répression brutale qui s'abattit sur plusieurs dizaines de mondes contribua à alimenter la rumeur publique et fit bien plus pour la cause rebelle que ne l'auraient souhaité les impériaux.\\

Mon Calamari fut la seule planète qui parvint à rester libre durant cette période et ses habitants jouèrent à fond sur les trois seuls avantages dont ils disposaient, qui leur permirent de préserver leur monde. Les Mon Cals et leurs alliés Quarren possédaient les seuls bâtiments capables de tenir le feu des destroyers de l'Empire, leur système stellaire était isolé et surtout il était aisé d'en défendre les routes spatiales connues des impériaux. À la longue, Mon Calamari aurait certainement succombé si la Marine Impériale avait pu vraiment déployer sa force mais l'essor des mouvements d'insurrection suite à la fin des Mondes Sécessionnistes et par la suite de la destruction d'Alderaan lui obtint un sursis qui joua de manière déterminante par la suite.

\paragraph{La Bataille de Yavin}
Comparativement à ce que l'on aurait pu croire, l'Alliance livra très peu de véritables batailles navales ou terrestres et ses activités militaires tournèrent surtout autour de raids contre l'Empire et ses alliés ou d'actions de retardement visant à permettre l'évacuation d'une population civile ou d'une base rebelle menacées. La doctrine militaire rebelle a toujours prôné un fait essentiel : la disproportion des forces est telle entre l'Alliance et l'Empire Galactique que tout engagement majeur est potentiellement une défaite. Quand bien même une bataille navale ou terrestre serait une victoire, chaque navire, chaque bataillon perdu par l'alliance la prive d'une part conséquence de ses ressources alors que ses ennemis peuvent se permettre des pertes dix fois supérieures sans être aussi handicapés au final.

S'ils l'avaient pu, les rebelles auraient frappé l'Etoile de la Mort à leur convenance et pas parce qu'elle menaçait leur base principale de Yavin 4. Cependant, les retombées en termes de moral et de propagande de cet évènement furent gigantesques et entièrement bénéfiques à la cause rebelle.\\

Durant les mois qui suivirent Yavin, les forces impériales durent se réorganiser et se remotiver suite à cette défaite humiliante qui les privait d'un seul coup d'une arme considérée comme invincible, d'un outil de terreur majeur et de nombreux officiers compétents. Dans le même temps, une multitude de cellules rebelles se mirent à redoubler d'activité, stimulées par cette victoire. De même, la victoire de Yavin provoqua une vague d'engagement sans précédent dans l'Alliance et les mouvements non-affiliés au point que les troubles de l'ordre public et les actes de sabotage atteignirent des proportions inégalées pendant quelques mois.\\

Si l'on attache beaucoup d'importance au fait que l'Empire dépensa des sommes colossales pour construire le super-destroyer Executor et ses frères ainsi que la deuxième Etoile de la Mort, on oublie aussi toutes les ressources humaines et financières mobilisées pour identifier, traquer et anéantir les agents rebelles, leurs fournisseurs, leurs sympathisants.\\

Ainsi, la période qui suivit la Bataille de Yavin fut la plus dense en bouleversements du règne de Palpatine. Jamais l'Empire Galactique n'eut à affronter simultanément autant de groupes, cellules, commandos et maraudeurs rebelles qu'après la perte de sa station spatiale. Des efforts considérables furent déployés pour dissimuler aux populations les nombreux camouflets essuyés par les suivants de Palpatine et l'on aura sans doute jamais non plus la possibilité d'évaluer réellement combien de rebelles ou d'innocents furent massacrés en représailles dans toute la galaxie.\\

Malheureusement, cette période se caractérisa aussi par la nécessité accrue de dissimuler le Haut Commandement de l'Alliance. L'Alliance avait fait la preuve qu'elle était un danger au moins indéniable à défaut d'être sérieux pour l'Empereur. Coordonner toutes les activités rebelles devenait de plus en plus risqué alors que Darth Vader menait sa force d'attaque à la recherche des dirigeants rebelles et d'un certain Luke Skywalker. Un juste compromis entre discrétion et communication était nécessaire et l'on pensait l'avoir enfin trouvé dans le système de Hoth\ldots

\paragraph{La Bataille de Hoth}
De bien des manières, Hoth est à l'encontre de Yavin un exemple typique de la majorité des batailles que dut livrer l'Alliance : le dos au mur face à un ennemi terriblement supérieur en nombre qui finit par remporter la victoire. Cependant, sur le plan tactique, Hoth illustre aussi le fait qu'en dépit d'officiers talentueux, la machine de guerre impériale était en fait bien trop gourmande et inefficace et que ses victoires étaient surtout remportées grâce au fait qu'elle n'hésitait pas à gaspiller beaucoup de ressources pour un résultat assez ambigu. Sur un plan purement militaire, ne pas parvenir à capturer le haut commandement rebelle seulement protégé par un bouclier déflecteur, un millier de fantassins dans des tranchées et quelques véhicules de combat atmosphériques alors que l'on emploie pas moins d'une demi-douzaine de destroyers, un super-destroyer et des forces terrestres considérables est en effet plutôt pathétique. Cependant, du point de vue de l'Alliance, Hoth fut bel et bien un désastre : des hommes précieux et du matériel qui l'était à peine moins furent irrémédiablement perdus et plus grave encore, toute l'Alliance avait failli être décapitée d'un seul coup.

\paragraph{La Bataille d'Endor}
En surface, la seule offensive majeure réelle de la flotte rebelle contre l'Empire fut motivée par le désespoir et l'opportunisme : l'Alliance découvrit l'existence de l'Etoile de la Mort inachevée à Endor et surtout la présence de l'Empereur venu superviser les travaux. Pensant faire d'une pierre deux coups et encore traumatisés par les conséquences de Hoth, les leaders rebelles décidèrent de rassembler la quasi-totalité de leurs forces spatiales dans un raid particulièrement audacieux.
Comme on l'apprit ensuite, Palpatine avait tendu un piège aux rebelles et sans quelques retournements de situation imprévus, il aurait définitivement écrasé la flotte rebelle à Endor. Si cela avait eu lieu, les capacités opérationnelles des cellules rebelles n'auraient pas vraiment été handicapées directement mais un coup terrible aurait été porté au moral des opposants à Palpatine et la mise en service officielle de cette nouvelle Etoile de la Mort aurait certainement découragé la majorité des ennemis de l'Empire. Une défaite de l'Alliance à Endor n'aurait pas éteint le brasier de la rébellion mais aurait en tous cas complètement démoralisé les opposants de Palpatine tout en privant l'Alliance de ses forces spatiales ainsi que de sa crédibilité comme mouvement susceptible de mettre à bas le régime impérial. Fort heureusement, les rebelles sortirent vainqueurs de cette bataille et surent profiter au maximum du désarroi causé par la mort de l'Empereur et la destruction de sa station spatiale.\\

À l'issue d'Endor, Mon Mothma et ses conseillers déclarèrent (surtout dans un but de propagande afin de rallier des systèmes indécis à leur cause) la fin de l'Alliance pour la Restauration de la République et la naissance de l'Alliance des Planètes Libres. Quatre semaines plus tard, alors que l'insurrection s'étendait partout dans la galaxie, cette même Alliance des Planètes Libres cédait la place à la Nouvelle République dont le gouvernement combattit plus de trois ans avant de pouvoir s'installer sur Coruscant.

\subsection{La Flotte de l'Alliance Rebelle}
Avant la Bataille de Yavin, le Haut Commandement de l'Alliance avait peu de navires capables d'affronter les bâtiments impériaux et la plupart étaient en fait des vaisseaux déclassés, des navires civils reconvertis ou dans quelques cas des vaisseaux provenant de forces spatiales autonomes. La plupart de ces derniers provenaient de mondes qui avaient eu le tort de proclamer leur soutien à l'alliance lorsque celle-ci rendit public sa Déclaration Formelle de Rébellion. Les planètes natales de leurs équipages avaient été écrasées par les forces impériales et ceux qui avaient pu rejoindre l'Alliance avaient rarement eu la possibilité ou le temps d'emmener avec eux du ravitaillement, des armes ou des pièces de rechange en quantité utile.\\

Cette "marine" était bien dotée d'un commandement central mais sa structure était dispersée dans la multitude de réseaux locaux qui formaient l'Alliance et si à l'échelon local les bâtiments ou la chasse rebelles accomplirent bien des exploits, leur impact au niveau galactique était quasiment nul. Et la disproportion des forces était par trop gigantesque pour permettre autre chose que des coups d'éclats suivis de fuites éperdues. À titre d'exemple, la "marine rebelle" du Secteur Brak comptait en tout et pour tout une poignée de cargos, deux escadrilles de chasseurs et trois corvettes corelliennes CR-90, face à un Groupe Sectoriel complet et ses 25 destroyers de classe impériale !!\\

Les commandements sectoriels des  Secteur Tierfon ou Tapani étaient encore plus mal lotis et devaient se contenter de quelques escadrilles de chasseurs et de cargos à peine capables de tenir le feu d'une patrouille de TIE. Dans la Bordure Extérieure ou l'Empire déployait encore ses forces de manière inégale, la situation était parfois moins critique mais n'avait rien de bien réjouissant dans le meilleur des cas.\\

Le Haut Commandement disposait bien de quelques unités lourdes mais il s'agissait en fait de Croiseurs de Fret reconvertis pour la plupart et de quelques unités vestiges de la Guerre des Clones. Au mieux, la rébellion pouvait espérer aligner quelques Frégates d'Escorte Nebulon-B impériales et une poignée de corvettes des douanes volées à leurs ennemis ou dont les équipages s'étaient mutinés.\\

Cependant, la révolte réussie des habitants de Mon Calamari changea sensiblement la donne et peu avant Yavin, l'Alliance eut un nouveau problème sur les bras. Elle disposait désormais de chantiers navals en orbite de la planète récemment libérée qui pouvaient lui fournir des bâtiments ou réparer ceux qu'elle possédait mais ces mêmes chantiers devaient être protégés avec des forces conséquentes car l'Empire ferait son possible pour écraser Mon Calamari au plus vite. On apprit d'ailleurs par la suite que cette planète faisait partie des cibles prioritaires de la première Etoile de la Mort.\\

Pourtant, la destruction de la station spatiale à Yavin eut des effets extraordinaires sur le moral des opposants à l'Empire. Pour la première fois, l'alliance venait de remporter avec éclat une bataille navale contre un ennemi d'une puissance à peine imaginable, détruisant avec de simples chasseurs la station spatiale la plus puissante de la galaxie. Dans le même temps, l'Empire se retrouva d'un seul coup privé de nombreux officiers capables, démoralisé et surtout obligé de combattre avec acharnement la rumeur qui se répandait dans toute la galaxie : la plus puissante station de combat de l'Empire, capable d'anéantir une planète en quelques instants ou de vaincre une escadre de navires lourds sans difficultés, avait été détruite par une poignée de chasseurs stellaires et un garçon de ferme dont le nom n'était pas encore célèbre.\\

À la suite de Yavin, la multiplication des activités rebelles de la part de mouvements qui n'étaient pas toujours membres de l'Alliance obligea celle-ci à mener une politique sur deux axes : renforcer ses réseaux d'approvisionnement et se doter de véritables forces spatiales. Un audacieux programme visant à l'emploi de corsaires fut initié, en commençant par la frégate de mutins Orbite Lointaine qui captura des cargaisons d'importance non négligeable mais porta également plusieurs coups sérieux à l'ennemi au cœur même de l'espace impérial, dans les parages de la Coquille de Ringali. Dans le même temps, le Mon Calamari Ackbar très apprécié de son peuple et dont le génie stratégique était reconnu fut chargé de rassembler dans le système tous les bâtiments rebelles capables de s'y rendre dans un court délai. \\
C'est ce rassemblement et les premiers vaisseaux sortis des chantiers du monde natal d'Ackbar qui formèrent le noyau de la véritable marine rebelle. A ces forces s'ajoutaient les unités de ligne et groupes de chasse locaux qui n'avaient pu rejoindre le point de rendez-vous ou demeuraient sous la tutelle d'un commandement local. La structure des alliés  qui s'appuyait déjà considérablement sur les groupes rebelles locaux prit une nouvelle dimension et l'on réorganisa les rapports et les procédures reliant le Haut Commandement avec les Commandements Sectoriels. Cette réorganisation concerna autant les forces navales nouvellement créées que les forces terrestres et le génie des dirigeants rebelles fut de parvenir à harmoniser dans la plupart des cas les habitudes et les exigences de leurs partenaires sectoriels avec les impératifs de l'Alliance elle-même.\\

Durant les quatre années qui séparent Yavin de la Bataille d'Endor, Ackbar accomplit des miracles en ce qui concerne à la fois l'organisation de la flotte mais aussi ses tactiques. Comme on s'en doute, celles-ci étaient essentiellement orientées dans la perspective de harceler un ennemi possédant une supériorité numérique et technologique écrasante, tout en préservant au mieux les vaisseaux rebelles. Les batailles rangées étaient évitées autant que possible car, comme pour les conflits terrestres, chaque perte infligée par l'ennemi comptait de manière cruciale alors que l'Empire pouvait quant à lui se permettre de sacrifier des quantités colossales d'hommes et de matériel pour obtenir la victoire.\\

Bien que la tentation soit forte pour le haut commandement rebelle de se réfugier sur Mon Calamari après Yavin et d'y regrouper également sa flotte, même un stratège moins doué qu'Ackbar aurait aisément pu expliquer que cela serait le meilleur moyen de provoquer la perte de la rébellion. Même en rassemblant toutes ses forces, celle-ci ne pouvait en effet espérer survivre à un affrontement direct avec la machine de guerre impériale. Au lieu de concentrer ses forces autour d'un monde d'importance stratégique, l'Alliance devait entretenir le doute chez l'adversaire et le forcer à disperser ses forces. C'est pour cette raison essentiellement que l'on décida de ne pas déplacer le haut commandement sur Mon Calamari mais également de faire de la flotte de l'alliance une force en mouvement permanent. Ainsi, les dirigeants rebelles pensaient, avec raison, que l'Empire ne pourrait parvenir à décapiter ses opposants d'un seul coup. Si Mon Calamari tombait, la flotte et le haut commandement continueraient à mener le combat. Si la flotte était anéantie, certains de ses bâtiments en réchapperaient certainement et les chantiers calamariens pourraient en fournir de nouveaux. Et si par malheur le haut commandement était anéanti, à défaut d'une direction centrale les rebelles disposeraient toujours d'actifs militaires et de chantiers spatiaux.\\

L'Alliance veilla donc à ne pas dévoiler la localisation de ses dirigeants en les faisant accompagner de la flotte rebelle et lorsqu'ils finirent par s'installer sur Hoth, ils avaient déjà parfaitement rodées les procédures qui leur permettaient de rester en contact avec le reste de la rébellion en demeurant aussi discrets que possible. Comme on le sait, la découverte du QG de l'Alliance sur Hoth résulte d'ailleurs d'un concours de circonstances fortuit : si le seigneur Darth Vader n'avait pas correctement (ou intuitivement ?) interprété les informations parcellaires transmises par un certain droïde sonde, ses subordonnés n'auraient jamais ordonné d'eux-mêmes à l'Escadre de la Mort de se rendre sur place\ldots
La Bataille de Hoth montra clairement que les stratèges de l'Alliance avaient eu raison. Bien qu'elle ait été une défaite rebelle du point de vue strictement militaire, elle s'avéra en fait être une victoire des plus pitoyables pour la machine de guerre impériale. Une escadre de destroyers impériaux accompagnée d'un super destroyer n'avaient pas été capables de capturer un ennemi assiégé qui ne lui avait rien opposé de plus sérieux que des fantassins, des chasseurs légers, des airspeeders et de vieux transporteurs poussifs\ldots\\

Une fois encore, la puissance de la marine impériale s'était révélée insuffisante face aux tactiques ingénieuses des rebelles et à leurs improvisations audacieuses.
Pourtant, le Haut Commandement rebelle comprit qu'il n'avait gagné qu'un sursis et que d'une manière ou d'une autre, le conflit ne tarderait pas à prendre un tour radicalement différent, très probablement à l'avantage de l'Empire. Les leaders de la rébellion rejoignirent leur flotte itinérante et commencèrent à envisager des actions militaires plus offensives. Quand quelques mois plus tard les agents secrets bothans purent informer l'alliance de l'existence d'une seconde étoile de la mort à Endor et de la prochaine visite de l'Empereur sur place, le choix que firent les rebelles s'imposait pratiquement de lui-même.\\

Le haut commandement rebelle ordonna donc que sa flotte et les bâtiments de ses forces sectorielles disponibles se rassemblent dans le système de Sullust pour préparer une attaque en règle de l'Etoile de la Mort encore en construction. D'autres mouvements indépendants de l'Alliance, mais en bons termes avec celle-ci, furent également conviés à participer dans la mesure de leurs moyens à cette bataille décisive. On peut citer par exemple les combattants du système virgillien ainsi que le gouvernement de Dornea, qui envoyèrent respectivement à Sullust deux croiseurs de classe Quasar Fire modifiés et deux Canonnières de classe Braha'tok.
Du point de vue rebelle, une victoire à Endor avait en fait trois intérêts : elle permettrait d'éliminer ou de capturer l'Empereur, elle anéantirait la seconde Etoile de la Mort mais aussi, elle assurerait la réputation de la flotte rebelle dans son premier engagement naval majeur, ce qui représenterait un atout de poids dans la poursuite de la lutte contre les forces armées de l'Empire.\\

Bien que la Bataille d'Endor se soit révélée être un peu plus compliquée que prévu, ces trois objectifs furent pour l'essentiel atteints : Palpatine fut tué, l'Etoile de la Mort détruite et malgré les pertes, l'Alliance parvint même à capturer plusieurs bâtiments impériaux, se dotant ainsi de destroyers modernes. Quelques jours plus tard, l'Alliance Rebelle laissait la place pour un court mois à l'Alliance des Planètes Libres à laquelle succéderait la Nouvelle République. Mais déjà, les rescapés d'Endor assemblaient une force de combat pour foncer au secours d'un monde isolé attaqué par des étrangers inconnus, la planète Bakura.


\section{Autres groupes}
\subsection{Les Corporations}
Il existe des centaines de milliers de mondes habités dans l'espace connu et la plupart d'entre eux hébergent une multitude de compagnies de toutes tailles. En plus des simples particuliers, des petits artisans, des compagnies locales, des coopératives et ainsi de suite, d'innombrables sociétés vendent biens, produits et services à toute la galaxie et ont un rôle primordial dans l'économie galactique. Souvent regroupées en Guildes et Associations, les corporations de toutes tailles se livrent cependant à une compétition féroce à l'échelle de la galaxie.\\

Énumérer toutes ces compagnies est une tache gigantesque que même les départements du gouvernement central ont du mal à accomplir, chaque seconde qui passe étant le théâtre d'une multitude de créations/faillites/rachats/absorptions dans la galaxie. Les grandes places boursières disposent de systèmes informatiques plus puissants que bien des systèmes stellaires pour simplement parvenir à suivre les fluctuations des cours concernant les principaux acteurs économiques.

\subsection{Les Mégacorporations}
Les géants du monde des affaires sont de véritables empires miniatures. Dans toute l'histoire de la civilisation galactique, une multitude de gouvernements aux territoires gigantesques qui sont demeurés dans les légendes étaient en fait relativement petits face à des compagnies dont les produits se vendent dans des centaines de secteurs, dont les filiales sont répandues dans des milliers de systèmes, dont les employés comptent des centaines de millions d'individus appartenant à des dizaines d'espèces\ldots certaines de ces puissantes multistellaires sont d'ailleurs propriétaires de systèmes entiers, habitants inclus. D'autres ne vendent qu'un ou deux produits mais pratiquement toute la galaxie les achète.\\

Un bon nombre des mégacorporations ont des dirigeants héréditaires et plusieurs familles nobles des mondes du noyau ont même vu le jour par ce biais ou ont au contraire fondé certaines des grandes compagnies. Leurs titres boursiers sont hors de prix mais leurs dividendes particulièrement alléchants. Nous allons en citer ici quelques-unes parmi les plus connues du public.

\subsubsection{Banque du Noyau }
Une des grandes institutions financières du Noyau et des Colonies, cette compagnie possède plus de 3000 chaines bancaires distinctes sous sa coupe.  

\subsubsection{BlasTech Industries}
Leader incontesté dans le domaine de l'armement, depuis le pistolet-blaster au canon lourd en passant par les missiles. Contrairement à beaucoup de ses concurrents, la compagnie n'a jamais focalisé ses intérêts sur les contrats militaires exclusifs de l'Empire malgré la demande colossale, préférant diversifier ses sources de revenus en vendant ses produits partout où la demande existe.

\subsubsection{Chiewab Amalgamated Pharmaceuticals Company }
Un gigantesque conglomérat médical spécialisé dans le développement de nouveaux médicaments à partir de substances et organismes trouvés sur des mondes récemment découverts, Chiewab a obtenu les titres de propriété de plus de 600 mondes ou elle puise ses "matières premières". La compagnie a également développé des intérêts significatifs dans l'agro-alimentaire, le reste du marché médical, l'industrie lourde et bien évidemment l'exploration spatiale.

\subsubsection{Compagnie de Répulseurs Aratech}
Aratech est un des meilleurs fabricants de Fonceurs (motospeeders) de la galaxie, son modèle militaire 74-Z conservant la faveur de nombreux acheteurs après plusieurs dizaines d'années sur le marché

\subsubsection{Corporate Sector Authority (Autorité du Secteur Corporatif)}
Il s'agit en fait d'un "état-corporatiste" chargé d'administrer le Secteur Corporatif et dont les actionnaires sont les grandes compagnies qui ont accepté de "sacrifier" certains de leurs actifs pour créer une autorité de tutelle politique, financière et militaire dans une région de l'espace délimitée de plusieurs dizaines de milliers d'étoiles. Par les diverses mesures législatives dont elles bénéficient dans le Secteur Corporatif et les taxes récoltées par la CSA, les corporations actionnaires ont multiplié les profits, certaines ayant déjà dépassé un retour d'investissement de 360\% en se contentant d'encaisser les taxes des autres compagnies opérant dans le Secteur et en récoltant leur part des produits fabriqués et vendus par l'Autorité. Le système est fait de telle manière qu'il est plus intéressant pour une corporation de devenir membre du CSA (donc, de le financer et de lui céder certains actifs en échange d'actions et de part de bénéfices) plutôt que d'opérer indépendamment dans l'espace du Secteur Corporatif et bien que la libre entreprise soit le crédo officiel du Secteur Corporatif, le petit entrepreneur n'a guère de possibilités par rapport aux grands trusts qui eux-mêmes sont quelque peu désavantagés s'ils tentent d'apparaitre sous leur propre nom dans le Secteur plutôt que de récolter leur part du gâteau. Des milliers de corporations sont sur les listes d'attente pour rejoindre le club des membres de la CSA. Parmi celles décrites sur cette page, plusieurs (indiquées par un astérisque) en font déjà partie. Le Secteur Corporatif est la seconde puissance de la galaxie connue après l'Empire mais également son vassal le plus puissant. De nombreux produits sont fabriqués sous le logo CSA mais relativement peu sont vendus hors de ses frontières et cela convient aux corporations sponsors. Ainsi, elles disposent dans le Secteur d'une population de consommateurs avec une réglementation avantageuse pour elles et peuvent exporter directement vers l'Empire les fabuleuses ressources naturelles du Secteur ou les racheter à prix intéressant et les transformer en produits manufacturés portant leur propre sigle en dehors des frontières où règne la CSA (et même à l'intérieur dans certains cas).

\subsubsection{Corporation Technique Corellienne }
Une des industries les plus anciennes du Secteur Corellien, la CTC a réalisé des dizaines de modèles de navires spatiaux de faible et moyen tonnage qui ont traversé les siècles et dont les plus récentes versions sont extrêmement répandues, notamment les transports légers de la série YT et la Corvette Corellienne C-90 multi-usages. Certains navires plus anciens portent encore le label Corellian Stardrive, un concurrent absorbé il y a des siècles par la CTC qui est en tête du marché des constructions astronautiques avec Kuat Drive Yards et Sienar Astronautique. Contrairement à ses deux concurrents qui ont toujours recherché les gros contrats militaires, la CTC doit sa puissance à la fois à des facteurs historiques (les corelliens étant des voyageurs impénitents) et à ses navires civils ou multi-usage.

\subsubsection{Cybot Galactica }
Un des deux géants de l'industrie robotique, Cybot Galactica a conçu quelques merveilles parmi lesquelles la gamme des droïdes de protocole 6PO et plusieurs modèles de droïdes de surveillance, d'entretien et de construction. Son seul concurrent de taille est Industrial Automaton et les deux corporations se vouent une haine farouche depuis des siècles.

\subsubsection{Czerka}
Troisième fabricant d'armes de la galaxie après BlasTech et Merr-Sonn, Czerka fabrique toute la gamme des armes depuis l'artillerie orbitale jusqu'au blaster de poche avec des succès commerciaux très conséquents en matière d'armes blanches, notamment de vibrolames.

\subsubsection{Fabritech}
Bien qu'originaire de la Bordure Extérieure, cette compagnie est depuis près d'un demi-siècle un des principaux producteurs de senseurs et de systèmes de communication. Divers grands constructeurs navals (CTC, Rendili, Sienar\ldots) préfèrent équiper leurs produits de senseurs Fabritech plutôt que de développer à grand prix une gamme aussi fiable.

\subsubsection{Genetech Corporation}
Un trust de grande taille qui était à l'origine impliqué surtout dans la pharmaceutique. Genetech est restée célèbre pour avoir été une des premières corporations à employer massivement des droïdes en guise de personnel ouvrier, entrainant licenciements collectifs et hostilité envers les machines. Après des millénaires d'existence, Genetech continue à produire des médicaments mais s'est aussi diversifiée dans l'instrumentation médicale et la robotique.

\subsubsection{Golan Arms}
Principalement connue pour ses pièces d'artillerie et ses stations orbitales de défense, cette firme traverse actuellement quelques difficultés financières en raison de quelques problèmes lors de la signature de contrats impériaux qui l'ont privé d'une partie de sa meilleure clientèle.

\subsubsection{Incom Corporation}
Après pratiquement deux millénaires durant lesquels cette compagnie a littéralement dominé le marché des chasseurs stellaires et des navires monoplaces, Incom a subi récemment quelques revers de fortune. Tout d'abord, sa meilleure équipe de concepteurs a pris la tangente pour rejoindre la rébellion avec les plans d'un nouveau chasseur qui est devenu un atout majeur de l'Alliance : le T-65 X Wing. Ensuite, l'Empire rendu furieux par cette trahison a procédé à la nationalisation du trust et l'activité essentielle d'Incom  a été réorientée vers les contrats de maintenance et de customisation au détriment de la recherche pure.

\subsubsection{Industries Arakyd}
Bien que cette corporation produise également des armes lourdes et des vaisseaux spatiaux, c'est sur le marché des droïdes qu'elle réalise ses plus grands profits, notamment depuis quelques décennies au point qu'elle est parvenue à décrocher des contrats militaires avec l'Empire à la barbe de Cybot Galactica et Industrial Automaton. Plusieurs droïdes de sécurité ou d'exploration ont fait la célébrité de la compagnie, notamment le droïde sonde Vipère abondamment utilisé par la Marine Impériale.

\subsubsection{Industrial Automaton}
Résultat de la fusion plusieurs milliers d'années auparavant de deux grands consortiums, Industrial Automaton est devenu le seul concurrent à pouvoir rivaliser en diversité des modèles et en puissance de vente avec Cybot Galactica. Sa série des droïdes astromechs de type R, notamment la version R2, est des plus réputées. Ses rapports avec Cybot Galactica ont toujours frôlé la guerre ouverte.

\subsubsection{Kuat Drive Yards}
Sur la planète Kuat, d'anciennes familles nobles ont conclu il y a des millénaires des alliances permettant la naissance d'un des plus importants fabricants de navires de l'histoire galactique sous le contrôle de directeurs obtenant leur position par héritage. Parmi ses produits les plus célèbres, citons notamment la frégate d'escorte Nebulon-B mais surtout le star destroyer de classe Imperator, symbole redouté de l'Empire. Des filiales de KDY fabriquent également des véhicules militaires ou civils de tous types. La puissance financière de Kuat la place parmi les plus puissantes mégacorporations de la galaxie.

\subsubsection{Loronar Corporation}
Un conglomérat aux activités des plus diverses, ses deux branches les plus rentables sont Loronar Robotics et Loronar Defense Technologies (turbolasers lourds, croiseurs de classe Strike\ldots). Malgré ses ventes record, l'image de la corporation est cependant ternie depuis plusieurs décades par quelques procès retentissants résultants d'une longue suite de manœuvres douteuses envers les marchés financiers, la clientèle et ses propres employés.

\subsubsection{Merr-Sonn}
Second fabricant d'armes de la galaxie (derrière BlasTech), sa stratégie est de multiplier les produits à direction d'une clientèle essentiellement militaire et paramilitaire. Merr-Sonn produit ainsi de l'artillerie, des blasters individuels, des lance-missiles, des explosifs, des lunettes de visée, des armures, des armes blanches, etc\ldots Merr-Sonn a également développé une ligne de véhicules miniers ou de chantier.

\subsubsection{MicroThrust}
Le premier fabricant d'ordinateurs portables, de datapads et de systèmes d'enregistrement de la galaxie depuis plusieurs centaines d'années mais qui ne parvient pas à percer dans la robotique ou la macro-informatique

\subsubsection{Millenium Entertainment}
Le géant de l’industrie des médias, Millenium contrôle des chaines d’holovision de première importance comme Galaxy News Service avec des parts significatives dans TriNebulon Network, CoreDataFiles et une multitude de petites chaines locales.  

\subsubsection{Neuro-Saav Technologies}
Un des principaux conglomérats en matière d'électronique : senseurs, imagerie, communications, encryptage, scanners médicaux\ldots

\subsubsection{Rendili Propulsions Stellaires}
Un géant de l'astronautique dont la fondation remonte à l'aube de la République, Rendili a subi d'importants revers durant les derniers siècles et a fini par être supplantée par Kuat Drive Yards et Sienar Astronautique sur le marché inépuisable des contrats de la Marine Impériale. Sans sa gamme (vieillissante) de Cuirassés Stellaires et de star destroyers de classe Victoire, sa situation serait certainement encore plus désagréable.

\subsubsection{Santhe/Sienar}
Conglomérat de grande taille, Santhe/Sienar est surtout célèbre pour être le producteur, via sa filiale Sienar Astronautique, de la gamme des chasseurs stellaires T.I.E dont les premiers modèles datent des dernières années de la république et qui sont désormais omniprésents dans la flotte impériale. Par le biais de Sienar Astronautique qui est un des trois principaux constructeurs de vaisseaux de la galaxie, Santhe/Sienar est devenu un trust de première importance et sa branche Sienar Intelligence Systems est de plus en plus souvent en compétition avec Arakyd Industries pour les contrats de robotique militaire.

\subsubsection{SoroSuub}
Pratiquement toute l'économie du peuple Sullustain est sous la coupe de ce grand trust. SoroSuub produit localement pratiquement tout ce que l'on peut acheter sur Sullust et dans sa sphère d'influence depuis la nourriture jusqu'aux navires spatiaux mais la galaxie connait surtout ses blasters. L'armée impériale est le plus gros client de SoroSuub qui a, entre autre, développé pour elle le Stormtrooper One, un clone un peu moins cher de la carabine de série E-11 de Blastech.

\subsubsection{TaggeCo (la Compagnie Tagge)}
La famille Tagge est une ancienne famille noble avec des intérêts financiers conséquents dans l'industrie minière et l'épice qui ont régulièrement augmentés avec les siècles. L'actuel Baron, Olman Tagge, a passé plusieurs décennies à transformer tous ces atouts plus ou moins éparpillés en une puissante corporation qui s'est très rapidement retrouvée dans le peloton de tête de la finance galactique. Les deux atouts majeurs de la Compagnie Tagge, sont sa diversification extrême (finance, énergie, industrie minière, agroalimentaire, chaines de restaurants et fast-food, transport\ldots) et ses liens étroits avec le pouvoir impérial. Olman Tagge est en effet un supporter farouche de l'Ordre Nouveau et un de ses frères porte l'uniforme de la Marine Impériale. Bien que sa puissance financière soit moins visible que celle de Santhe/Sienar ou Kuat Drive Yards, la TaggeCo exerce une influence croissante sur l'économie galactique, Olman Tagge ayant également été parmi les fondateurs de l'actuelle administration du Secteur Corporatif, la CSA.

\subsubsection{TransGalMeg Industries Inc.}
Bien que nettement moins importante que TaggeCo, TransGalMeg (TGM) possède une structure analogue et une multitude de filiales impliquées dans l'astronautique, l'industrie minière et l'agriculture. À l'origine, TGM s'intéressait surtout à établir des colonies minières et à concevoir ses propres moyens de production et de transport mais ses dirigeants ont fini par comprendre qu'ils pourraient vendre certains produits sur des marchés porteurs et le résultat n'a pas tardé à leur donner raison.

\subsubsection{Ubrikkian Transports}
Un des principaux constructeurs de speeders de la galaxie, depuis les engins de combat aux barges de luxe destinées à survoler océans et déserts en passant par les speeders familiaux et les véhicules de course.

\subsubsection{Xizor Transport Systems}
XTS est le géant de l'affrétage et du transport de cargaisons dans toute la galaxie. Son dirigeant, le Falleen Xizor, est sans doute le non-humain le plus riche de la galaxie et en tous cas, un des rares à faire partie du cercle des intimes de l'Empereur. Sa corporation a de nombreux contrats exclusifs avec les armées impériales et elle est incontournable dans le Noyau, les Colonies et la Région d'Expansion.

\subsubsection{Zaltin et Xucphra}
Ces deux compagnies rivales sont basées sur la planète Thyferra dont les indigènes non humains, les Vratix, connaissent le secret permettant de fabriquer du Bacta de premier choix avec le meilleur rapport qualité-prix. Les dirigeants humains de Zaltin et Xucphra ont obtenu de l'Empire le monopole légal de ce médicament miracle. D'autres sources (clandestines) de Bacta de moindre qualité existent mais à elle toute seule, la planète Thyferra contrôle environ 95\% du marché. Ses dirigeants déjà richissimes avant la venue de l'Empire sont parvenus à multiplier les profits de manière incroyable.

\subsection{Institutions galactiques}
Le gouvernement de Palpatine n'est pas la seule organisation bureaucratique qui étend son influence sur la galaxie. Un certain nombre d'organisations semi-gouvernementales ou privées existent également, parfois depuis des millénaires et leur influence se fait encore sentir dans le quotidien de beaucoup d'êtres intelligents. Certaines de ces organisations cohabitent sans problème avec l'Empire, d'autres sont par contre dans le collimateur mais n'ont pas encore pu être démantelées ou réformées.

Voici les plus influentes.

\subsubsection{Le Bureau officiel des Services Stellaires (BoSS)}
Le Bureau officiel des Services Stellaires est l'institution la plus ancienne de l'histoire galactique qui soit encore active à grande échelle. La tâche du BoSS est considérable et incontournable dans une civilisation galactique digne de ce nom. C'est en effet cet organisme qui procède à l'immatriculation de tous les vaisseaux fabriqués légalement, à l'enregistrement des capitaines titulaires des licences officielles, etc\ldots en résumé, le BoSS est LE service qui permettra à un entrepreneur d'avoir les licences, titres de propriétés et immatriculations qui feront de lui un capitaine en règle avec les autorités. Les services du BoSS sont tels que ses bases de données contiennent sur chaque vaisseau enregistré tous les noms des propriétaires successifs et la liste des modifications ou réparations effectuées du moment que cela a été fait de manière légale. Ainsi, acheter un navire dument immatriculé est une opération sure car avant même de l'achat, le client peut obtenir du BoSS l'origine, l'âge et les différents travaux réalisés sur son futur navire (et certains ont plusieurs décennies de vol derrière eux).\\

Bien qu'une proportion non négligeable de capitaines procède à des voyages non enregistrés ou des modifications illégales et que certains n'aient que de faux papiers pour toute licence, les pilotes expérimentés savent bien que tôt ou tard, si on veut ne serait-ce que demeurer marginalement en règle, il faudra en passer par le BoSS qui possède un petit corps d'agents d'investigation extrêmement qualifiés pour tenir ses dossiers à jour.\\

Les employés du Bureau sont pour la presque totalité les descendants d'autres employés et le BoSS tient autant du service gouvernemental que du clan multiracial à l'échelle galactique. Il est extrêmement difficile de corrompre ses employés et quasiment impossible d'infiltrer leurs rangs mais cette garantie de probité fait que beaucoup de gens font confiance au BoSS alors qu'ils fuiraient un organisme gouvernemental.\\

Au début de son règne et à plusieurs reprises depuis cette époque, Palpatine a bien sûr essayé de faire du BoSS un autre rouage de l'Empire. Après qu'un certain nombre de yachts privés appartenant à des fonctionnaires impériaux aient été arraisonnés par les forces impériales parce que leurs titres de propriété ou leur immatriculation avaient été "effacés par erreur" des archives du BoSS, après que plusieurs corporations d'affrètement aient été obligées de suspendre leurs transports pour l'empire afin d'éviter que certains "problèmes d'immatriculation dus à des difficultés techniques" durent trop longtemps, le BoSS a finalement conservé le même statut avec l'Empire que celui dont il jouissait durant la République : il fournit certaines données publiques et accepte de donner des informations complémentaires si le demandeur est un organisme officiel et que sa demande est motivée. Les axes politiques du BoSS dans ce domaine sont encore difficiles à cerner mais il semble se satisfaire d'une prudente neutralité dans les affaires galactiques, neutralité "active" lorsque l'on se montre trop pressant à son égard. Le BoSS est présent dans toutes les capitales sectorielles de l'Empire ainsi que presque tous les mondes des régions centrales et même dans la plupart des systèmes des Bordures Moyenne et Extérieure. À terme, l'Empire compte bien remplacer le BoSS par sa propre organisation et les diverses agences impériales de maintien de l'ordre travaillent de plus en plus souvent en ce sens.

\subsubsection{L'Institut des Ingénieurs de Vaisseaux Stellaires (IIVS)}
Cet institut est sans conteste le meilleur organisme de formation en ce qui concerne les ingénieurs en astronautique, au point que même la très réputée Académie Impériale d'Ingénierie envoie ses propres diplômés faire un cursus à l'IIVS. Malgré des frais de formation hallucinants (15.000 Cr par trimestre), de nombreuses compagnies construisant des navires ou les utilisant font également appel à cet institut pour la formation de son personnel. Il est important de rappeler que les Ingénieurs ne sont pas des mécaniciens mais bien des experts capables d'envisager un navire dans son ensemble et à diriger une équipe technique incluant des spécialistes divers. Basé sur Coruscant, l'IIVS possède des campus dans les systèmes de Corellia, Sullust et Alderande entres autres.\\

À la fin de son cursus, l'étudiant de l'IIVS conna\^{i}t tous les aspects de la construction, du fonctionnement et de l'entretien des navires les plus répandus et même de modèles plus rares. Trouver du travail avec d'excellentes conditions n'est jamais un problème avec un diplôme de l'IIVS et ceux que l'on appelle "les Ingénieurs de l'Institut" deviennent assez souvent des concepteurs réputés. La majorité des diplômés de l'Institut, y compris ceux de la marine impériale, continuent à recevoir des  mises à jour de leur formation (compter 5000 crédits par an) et beaucoup rempilent occasionnellement à l'Institut pour approfondir leurs connaissances. C'est grâce à l'IIVS que durant des millénaires, la plupart des grandes compagnies d'astronautique ont fini par développer des normes techniques et des pièces compatibles. Les compagnies en question disposent bien évidemment de leurs propres cursus de formation mais ceux-ci sont généralement spécialisés et définis par la politique de l'entreprise et ses normes de conception internes alors que le cursus de l'IIVS a justement pour axe de former des gens polyvalents.

\subsubsection{L'Institut Médical Galactique (IMG)}
Dans une galaxie ou des milliers d'espèces arpentent des dizaines de milliers de mondes et traversent des centaines d'années lumières, les raisons de craindre une épidémie sont bien réelles. Tout d'abord par la quantité de denrées biologiques et d'êtres vivants qui transitent d'un monde à l'autre, ensuite par la multitude de différences biologiques qui font qu'une bactérie anodine ou même vitale pour une espèce peut s'avérer mortelle pour une autre\ldots l'Institut Médical Galactique est chargé de recenser les découvertes réalisées par tous les organismes médicaux de la galaxie afin de limiter autant que faire se peut les risques en assurant la distribution la plus large possible d'informations vitales. De nombreuses universités et d'innombrables hôpitaux sont membres de l'IMG et contribuent à la mise à jour de ses gigantesques banques de données, ce qui leur donne également le droit d'y accéder sans restriction. Avec le démantèlement de la partie civile de l'HoloNet par l'Empire, seules les demandes les plus urgentes sont désormais instantanément transmises et la politique de ségrégation raciale impériale, souvent accompagnée de blocus planétaires, a amené de nombreux mondes à abandonner leur participation à l'IMG. Certaines rumeurs parlent d'infections désastreuses qui se seraient propagées à cause de cela\ldots

\subsubsection{Le Service de Sauvetage et de Récupération}
Les navires noir et rouge de cet organisme sont connus de tous et rares sont les gens qui accepteraient d'ouvrir le feu sur eux, y compris parmi les pirates. Le SSR opère principalement dans le Noyau, les Colonies, la Région d'Expansion et la Bordure Intérieure mais il existe quelques bureaux locaux un peu partout le long des principales routes commerciales de la galaxie et divers organismes analogues de taille plus petites dans certains secteurs hors de sa portée. Les équipages du SSR réalisent souvent l'impossible pour porter secours aux naufragés, écarter les épaves dangereuses des routes commerciales, récupérer les capsules de sauvetage perdues et sauver les cargaisons précieuses. Bien qu'il soit soumis à l'autorité impériale suive toute la réglementation, les statuts du SSR définissent sa mission comme étant de porter secours à tout navire en détresse quelle que soit son origine, sauf s'il fait preuve d'agressivité à l'égard des sauveteurs.

\subsubsection{La Guilde des Marchands Corelliens}
Bien que fondée à l'origine par des humains de Corellia, la GMC est devenue une organisation multiraciale puisqu'elle accepte dans ses rangs tous les navires dont le capitaine ou un des membres d'équipage a des ascendants nés dans le système corellien et de solides références auprès des autres pilotes de la GMC. La Guilde des Marchands Corelliens est de fait une collection de confréries, cartels et associations de transporteurs de petite taille dont les activités couvrent toute la galaxie. L'Empire considère (sans doute avec raison) que la GMC est une pépinière de contestataires, d'escrocs et de sympathisants rebelles et les longues traditions antibureaucratiques de Corellia ont une part importante dans cette situation. Néanmoins, il ne peut interdire la GMC sous peine de se retrouver avec un embargo galactique sur les bras et doit se contenter de fermer une de ses branches à l'occasion en guise de coup de semonce. Bien que l'adhésion annuelle à la GMC ne soit pas donnée (10 000 crédits en moyenne et jusqu'à 100 000 pour ses membres les plus importants), elle assure partout où elle est présente plusieurs services pour ses membres : assistance juridique, garantie des tarifs de stationnement et de réparation aux astroports, messagerie, petites annonces, prêts en liquidités, etc\ldots la grande majorité de ses capitaines faisant preuve de beaucoup de solidarité entres eux.\\

Il existe un certain nombre d'associations et de fraternités semblables à la GMC, certaines tout aussi cosmopolites (comme la Confrérie des Spationautes Lantilliens par exemple) et quelques-unes ont une réputation encore plus douteuse (comme la Guilde Commerciale Klatooienne liée aux Hutts)  mais aucune n'a son influence et la plupart ne sont pas en mesure de fournir à leurs membres tous les services de la GMC. De par leur taille et leur rayonnement plus réduits, elles sont aussi davantage soumises aux exigences impériales et certaines sont d'ailleurs activement soutenues par l'Empire pour prendre les marchés détenus par la GMC.\\

En plus de cela, la politique isolationniste de Corellia a, depuis quelques années, quelque peu privé la GMC d'une partie de ses appuis financiers et politiques et la pression des grandes compagnies de transport (qui ont souvent des sympathies pro-impériales) a encore diminué sa puissance alors que l'Empire s'appuie de plus en plus sur les mégacorporations et que les petits transporteurs demeurent les victimes favorites des pirates de tous poils. Être membre de la GMC est aussi à l'occasion un bon moyen d'être harcelé par les Douanes Impériales ou les autorités portuaires et pas vraiment conseillé si l'on vit surtout d'activités illégales.

\subsubsection{La Très Honorable Guilde des Armuriers}
Cette organisation rassemble pratiquement tous les concepteurs d'armement de la galaxie, qu'il s'agisse de particuliers inventifs ou de corporations interstellaires. Sa fonction officielle est de garantir le respect des standards de prix et de qualité dans les chaines de vente ou les armureries qui acceptent d'arborer son emblème (un certificat de qualité en quelque sorte). On peut trouver les mêmes produits ailleurs, chez des revendeurs non affiliés,  mais leur prix et parfois leur conformité seront plus variables\ldots L'autre activité de la Guilde des Armuriers, pratiquement officielle d'ailleurs, est la vente aux enchères d'armes uniques, exotiques, rares ou archaïques. Pour ce faire, la Guilde a acheté la planète Epsi Nadir dans le Noyau ou elle tient périodiquement des sessions d'enchères. Même si elle ne vend que des "armes de collection" et procède immédiatement aux formalités nécessaires auprès de l'administration impériale, beaucoup de gens savent bien que certaines donations discrètes faites par des clients réguliers peuvent amener à l'occasion des formulaires gênants à "s'égarer". Le fait que certaines grandes compagnies affiliées à la Guilde soient les fournisseurs d'armement de l'Empire et que beaucoup de fonctionnaires impériaux ou de gradés militaires soient collectionneurs d'armes contribuent certainement à maintenir la tranquillité de la Très Honorable Guilde des Armuriers

\subsubsection{Les Guildes Commerciales Ithoriennes}
Le peuple Ithorien qui se déplace à bord de gigantesques vaisseaux-troupeaux claniques possède aussi des délégations commerciales dans toute la galaxie ou les représentants des clans vendent les produits agricoles des vaisseaux. Pour les Ithoriens, l'agriculture n'est pas un simple travail mais une tache religieuse sacrée et en dehors du strict nécessaire à leur consommation, la presque totalité de leur colossale production est vendue. Les prix pratiqués par les Ithoriens sont réputés honnêtes et raisonnables si l'on sait marchander correctement et la qualité de leur "travail" n'est plus à démontrer. Lorsque l'on se rend dans les petites maisons des Guildes Ithoriennes, véritables enclaves de verdure sous serre, il est possible de se procurer des racines, tubercules, feuilles, graines, fruits, légumes etc\ldots en faisant de bonnes affaires tout en permettant aux Ithoriens d'amortir leur investissement agricole de manière significative. Si on leur donne des spécifications précises et que l'on n'est regardant ni au temps, ni à la dépense, le clan propriétaire d'une Guilde peut facilement accepter de tenter la culture de nouvelles plantes et des rabais importants sont accordés aux clients qui amènent des échantillons de plantes exotiques inconnues aux vertus médicinales ou alimentaires.

\subsubsection{La Guilde Minière}
Cette organisation très étendue veille aux intérêts de plus de 300 000 compagnies minières de toutes tailles dans la galaxie, compagnies qui contrôlent environ 75\% de l'extraction, du raffinage et de la distribution des minerais sur le marché galactique. Le rôle essentiel de la Guilde Minière est de permettre aux membres de défendre leurs intérêts face au gouvernement central ou aux autorités locales, de s'accorder sur certains tarifs et ainsi de suite\ldots presque aussi ancienne que le BoSS, la Guilde Minière demande des droits d'entrée annuels monstrueux aux compagnies membres en échange desquels elle fournit les services de ses avocats, voire même de ses agents de sécurité lorsqu'une compagnie membre doit faire face à des attaques pirates. Les 25\% de l'économie minière qui ne sont pas contrôlées par les membres de la Guilde sont détenus par quelques mégacorporations non-affiliées, plusieurs gouvernements locaux, une multitude de petits entrepreneurs trop insignifiants pour être approchés par la Guilde et bien évidemment les compagnies nationalisées par l'Empire. Le siège social de la Guilde est sur Coruscant mais ses principaux membres sont en fait les grandes corporations qui avaient investi dans la Région d'Expansion et la Bordure Moyenne. Lorsque les ressources de la Région d'Expansion ont commencé à se raréfier, ces mêmes compagnies se tournèrent vers le Secteur Corporatif et par la suite dans le reste de la Bordure Extérieure. 


\subsection{Les Chasseurs de Primes}
Bien que la galaxie soit trop vaste pour qu'on puisse s'en passer, s'il est une catégorie d'individus détestés ce sont bien les Chasseurs de Primes. Aux yeux de la majorité des citoyens de l'Empire, les chasseurs de primes ne sont au mieux que des assassins qui louent leurs services pour retrouver des criminels et dont les actes sont souvent bien plus abominables que ceux reprochés aux personnes qu'ils traquent.\\

Cette situation a pour sources divers facteurs.\\

En premier lieu, demander la mise à prix d'un individu n'est normalement légal que si la demande est faite par une agence de maintien de l'ordre civile ou militaire reconnue par l'Empire, par le biais du BICE. En théorie, cela signifie qu'il faut avoir affaire à un criminel ou un suspect reconnu comme tel par la loi. Dans la pratique, de nombreux responsables impériaux abusent de leur autorité pour lancer des avis de recherches leur permettant de régler quelques petites affaires personnelles, souvent lucratives\ldots\\

En second lieu, un Chasseur de Primes est normalement un auxiliaire de police dûment enregistré et autorisé à opérer dans un secteur défini. Seul un Chasseur enregistré peut capturer/tuer une cible sans avoir de problèmes avec la loi. De fait, beaucoup de mondes ou l'Empire s'est montré brutal considèrent les Chasseurs comme des ordures parce qu'ils aident au maintien de l'ordre impérial\ldots
Ensuite, les citoyens respectables des mondes les plus contrôlés par l'Empire et ou sa politique est la mieux perçue ont une méfiance instinctive pour tout ce qui sort des normes, en particulier provenant de régions mal famées de la galaxie et armé jusqu'aux dents. Leur système juridique local n'a d'ailleurs la plupart du temps pas du tout besoin de chasseurs de primes.\\

Enfin, de nombreux chasseurs de primes sont connus pour accepter des contrats juteux qui n'ont rien de légal, notamment lorsque les grands patrons du crime mettent la tête de quelqu'un à prix par exemple\ldots "Assassinat" et "enlèvement avec séquestration" sont les termes juridiques qui s'appliquent le plus souvent à ce genre d'activités\ldots

\subsubsection{Licences de Chasseurs de Primes}
Le seul intérêt d'une licence est de donner le droit au Chasseur de se promener en armure de combat avec un fusil blaster en bandoulière et de s'en servir dans certaines limites légales ou encore de posséder un vaisseau spatial armé jusqu'aux dents et équipé de gadgets normalement interdits.\\

\begin{center}
	\begin{tabular}{|c|c|}
		\hline 
		\rule[-1ex]{0pt}{2.5ex} Validité de la licence & Co\^{u}t annuel \\ 
		\hline 
		\rule[-1ex]{0pt}{2.5ex} Système & 50 Cr \\ 
		\hline 
		\rule[-1ex]{0pt}{2.5ex} Secteur & 500 Cr \\ 
		\hline 
		\rule[-1ex]{0pt}{2.5ex} Région & 5 000 Cr \\ 
		\hline 
		\rule[-1ex]{0pt}{2.5ex} Galaxie & 50 000 Cr \\ 
		\hline 
	\end{tabular}
\end{center}

\subsubsection{La Guilde des Chasseurs de Primes}
La Guilde est une fraternité qui a connu son heure de gloire en symbolisant les meilleurs de la profession et qui, bien qu'encore très puissante, n'est plus que l'ombre d'elle-même. Elle remplit encore cependant ses principales missions aux yeux de ses membres.

\begin{itemize}
	\item Appartenir à la Guilde permet de faciliter les formalités vis à vis des autorités impériales pour devenir un chasseur de primes en règle : contrairement à un indépendant, les Chasseurs de la Guilde n'ont rien à débourser pour obtenir leur licence, la Guilde les obtenant facilement pour eux auprès des autorités et se payant sur leurs cotisations.
	\item La Guilde est souvent la première informée des bonnes affaires du moment et peut passer l'information moyennant une commission confortable sur la prime
	\item La Guilde possède un réseau galactique de contacts toujours prêts à faire une petite ristourne aux membres : armuriers, avocats, médecins, ingénieurs spatiaux, vendeurs de vaisseaux\ldots
	\item Le règlement de la Guilde interdit à ses membres de se tirer dessus ou de se mettre des bâtons dans les pattes. Lorsque plusieurs chasseurs affiliés visent la même proie, ils doivent partager. Bien sûr, vu le manque notoire d'honnêteté de beaucoup de Chasseurs, un certain nombre "d'accidents regrettables" surviennent parfois durant une poursuite mais il arrive occasionnellement que la Guilde se livre à une enquête approfondie sur une affaire et si jamais il y a preuve de meurtre d'un membre par un autre, le coupable voit alors sa tête mise à prix par l'organisation\ldots les règlements de comptes sont bien évidemment interdits dans les locaux de la Guilde. 
\end{itemize}

Pour tous ces services, la Guilde prélève automatiquement une "cotisation" de 10\% du montant de chaque prime encaissée par le Chasseur.\\

Au niveau des licences, les membres de la Guilde choisissent en début d'année l'étendue de la zone où ils souhaitent opérer et la Guilde fait aussitôt le nécessaire. En fin d'année, soit le montant des "cotisations" du chasseur est suffisant et on revoit avec lui la licence pour l'année suivante, soit il ne l'est pas\ldots\\

Dans ce cas, la Guilde conclut un arrangement avec le Chasseur pour qu'il débourse le reste. Dans l'intervalle, la Guilde lui retire la licence qu'elle lui a accordé et il ne bénéficie plus du statut de membre. Si sa dette est importante, la Guilde peut même décider de mettre sa tête à prix, les autorités se fichant pas mal de ce qui arrive à leurs précieux "auxiliaires".\\

Dans ce type particulier de mise à prix, la Guilde ne prélève aucune "cotisation" mais ne paye pas non plus des sommes monstrueuses puisqu'elle souffre déjà d'un manque à gagner\ldots La plupart des Chasseurs étant essentiellement dans le coup pour l'argent, cela entraine presque toujours les autres Chasseurs infructueux à courser leurs collègues mis à l'index pour régler leurs propres dettes avant qu'on ne leur fasse le même coup\ldots bien sûr, si la Guilde reçoit son argent (avec les éventuelles pénalités appropriées), la prime sur la tête du Chasseur est aussitôt levée.\\

Malgré sa puissance encore conséquente, la Guilde des Chasseurs de Primes attire de moins en moins de monde. Tout d'abord parce qu'une multitude d'autres organisations moins importantes mais plus jeunes et vigoureuses existent même si aucune ne peut prétendre opérer régulièrement à l'échelle galactique (la plupart ne délivrent pas de licences au-delà d'un ou deux secteurs). La Guilde préfère également n'admettre que des Chasseurs confirmés (généralement, il faut que le candidat ait déjà accumulé au moins 20.000 crédits de primes personnelles) alors que les autres organisations sont souvent moins exigeantes.\\

Enfin, la réputation de la Guilde est celle d'une organisation corrompue dont les membres acceptent parfois de relâcher les captifs moyennant rançon ou des contrats douteux (les Hutts et divers autres syndicats du crime sont réputés avoir des arrangements de longue date avec la Guilde). Dans les faits, les concurrents de la Guilde sont très probablement aussi malhonnêtes pour la plupart mais ils ont moins de grosses pointures célèbres dans leurs rangs et cela est donc moins visible.

\subsubsection{Les Indépendants}
Les Chasseurs indépendants, qu'ils opèrent en petits groupes ou en solo, ont une vie plus difficile sur un plan administratif. Tout d'abord, ils doivent payer de leur propre poche pour obtenir leur licence. Ils dépendent aussi considérablement de réseaux d'informateurs qu'ils doivent bâtir de leurs propres mains pour saisir les bons coups avant la Guilde. Enfin, et contrairement aux Chasseurs de la Guilde ou des autres organisations, les indépendants ne disposent pas d'une liste de gens à contacter ou de "partenaires" faciles à joindre. Tout ce qu'ils obtiennent résulte de leurs propres efforts. Ceux qui deviennent célèbres sont donc particulièrement dangereux\ldots

\subsubsection{Les Avis de Recherche}
Un avis de recherche légal peut concerner un seul monde, un système, un secteur, une région ou toute la galaxie. En plus de la prime promise à celui qui remplira le contrat, le dépôt d'un avis de recherche entraine le versement d'une taxe au BICE.
Légalement, il est interdit de pourchasser une cible au-delà des limites ou s'applique son avis de recherche. Seuls les fonctionnaires impériaux de rang suffisant (Moff, directeur sectoriel du BSI ou des Renseignements, Amiral de la marine\ldots) peuvent faire la demande d'un avis de recherche à l'échelle de la galaxie.\\

Toucher la prime peut se faire de deux manières

\begin{itemize}
	\item Dans le cadre d'une mise à prix légale, il faut ramener la cible (ou ses restes) à un bureau du BICE se trouvant dans la zone ou la cible peut être pourchassée légalement. (Dans le cadre d'une mise à prix galactique, n'importe quel bureau du BICE fera l'affaire). À défaut, on peut se rendre à une garnison militaire ou auprès des services d'un gouverneur impérial mais il y aura quelques délais avant le versement de la somme promise.
	\item Pour les autres primes (officieuses\ldots), il faut le plus souvent se rendre chez  le commanditaire avec ce qu'il demande et prendre garde à ne pas se faire doubler\ldots 
\end{itemize}

Bien évidemment, il est plus que recommandé d'avoir un exemplaire de l'avis de recherche avec soi lorsqu'on tente d'arrêter/tuer une cible afin d'éviter quelques complications légales (du genre inculpation d'homicide volontaire avec préméditation)\ldots avoir une licence de Chasseur en règle est bien évidemment obligatoire.\\

Comme dit plus haut, pour obtenir que le BICE accepte de valider la mise à prix de la tête de quelqu'un, il faut en théorie que la cible soit soupçonnée ou coupable d'un crime réprouvé par la loi impériale. Il faut également s'acquitter d'une taxe variable selon l'étendue couverte par la mise à prix.

\begin{itemize}
	\item Mise à prix sur un système : 20 Cr/système (15 systèmes maximum, pas forcément dans le même secteur)
	\item Mise à prix sur un secteur : 1000 Cr/secteur (tous les systèmes du secteur sont concernés, quel que soit leur nombre)
	\item Mise à prix sur toute la galaxie : seulement possible aux responsables impériaux de rang élevé ou avec leur permission,  aucune taxe\ldots
\end{itemize}

\subsubsection{Avis locaux et corporatistes}
Dans les limites de leur juridiction, de nombreuses agences de police locales ou corporatistes peuvent déposer des avis de recherche sans passer par le BICE. Ces avis ne sont pas valables au-delà des limites ou l'agence en question est autorisée à opérer. Dans ce cas, la réglementation appliquée par ces différentes entités ne les oblige pas à passer par le BICE (sauf pour les avis "Mort"). Les agences en question peuvent délivrer leurs propres licences locales à qui elles veulent et au prix qu’elles souhaitent. Elles sont également obligées de considérer qu'un titulaire d'une licence impériale de Chasseur peut agir à sa guise sur leur territoire et qu'il est habilité à y poursuivre une cible mise à prix par le pouvoir impérial dans cette juridiction. Exemple : un chasseur de prime autorisé à traquer une cible dans toute la Bordure Extérieure pourra également la poursuivre dans le Secteur Corporatif qui en fait techniquement partie bien qu'il s'agisse d'une entité légale distincte, que la cible ait été mise à prix par l'Empire ou le Secteur Corporatif. Par contre, un chasseur possédant uniquement une licence délivrée par le Secteur Corporatif ne pourra pas opérer légalement à l'extérieur de ses frontières\ldots

\subsubsection{Mort ou Vif}
La difficulté se pose lorsqu'il est spécifié qu'un individu doit être ramené vivant : provoquer sa mort empêchera non seulement le chasseur de toucher la prime promise mais peut éventuellement entrainer des complications légales (tuer un individu qui doit être capturé vivant demeure un meurtre sur le plan technique\ldots sauf en cas de légitime défense avérée), voire des représailles de la part du commanditaire (s'il a demandé qu'on lui ramène l'autre type vivant, a priori c'est qu'il ne lui servira à rien s'il est mort\ldots).\\

De même, un avis de recherche peut spécifier que le sujet doit être ramené vivant et intact. Dans le cadre des avis officiels, une cible devant être ramenée vivante sans précision particulière sera validée et la prime versée du moment qu'elle n'est pas mourante à l'arrivée au bureau local du BICE (les sévices et mutilations "durant une tentative d'évasion" sont généralement oubliés par les autorités, surtout après une petite donation discrète\ldots). Par contre, les avis officiels peuvent eux aussi spécifier que la prise doit être ramenée en bon état et le BICE se réserve le droit de ne verser qu'une partie de la prime promise si la prise ne correspond pas à l'état souhaité.\\

La taxe versée au BICE pour une mise à prix est la même que l'on souhaite obtenir une cible vivante en bonne santé, tout juste vivante ou morte. Officiellement, il faut un crime capital avéré pour justifier un avis "mort" que seul le BICE est autorisé à délivrer.\\

La plupart des avis délivrés pour des crimes classiques sont de type "Mort ou Vif" parce que les autorités sont conscientes que le Chasseur n'est pas toujours en mesure de capturer la cible vivante. Néanmoins, dans  ce genre de cas la prime promise pour une prise vivante est le plus souvent divisée par deux lorsque l'on ramène un cadavre (afin d'éviter une prime "Mort" qui n'en aurait pas les apparences puisque beaucoup de Chasseurs considèrent qu'un cadavre est moins encombrant\ldots).

\subsubsection{Le Club des Cent}
On désigne sous cette appellation tous les individus dont la tête a été mise à prix pour une somme de 100 000 crédits ou plus. Ils sont plusieurs centaines de milliers dans ce cas, peut être un million, ce qui n'est proportionnellement qu'une goutte d'eau dans l'océan des avis de recherche publiés. Que la mise à prix soit légale ou pas, la promesse d'une somme de 100 000 crédits sur votre tète (en état de fonctionner ou non) est un bon moyen de vous rendre célèbre dans tous les bouges et auprès de tous les flingueurs de la galaxie.

\subsection{La Frange}
Ce terme désigne la multitude d'individus et de groupes qui opèrent en marge de la loi, voire dans l'illégalité la plus totale. Il faut cependant distinguer les personnes qui agissent ainsi pour le profit de celles qui sont hors-la-loi pour des raisons idéologiques (l'Alliance Rebelle ou certains mouvements politiques et religieux par exemple) et qui ne font donc pas partie de la Frange à proprement parler.

\subsubsection{Cartels et Syndicats du Crime}
Une partie conséquente de la frange est constituée d'opérateurs indépendants : tueurs à gages, dealers, contrebandiers, chirurgiens marron, loubards, commerçants réalisant des opérations plus discrètes sous le comptoir, etc\ldots D’autres criminels sont organisés en gangs, en bandes et en groupes généralement impliqués dans un créneau particulier (abordés dans les chapitres qui suivent).
Bien sûr, derrière tous ces individus ou ces petites organisations, ce sont les grands syndicats du crime qui sont les véritables puissances de ce milieu si particulier. Personne n'est en mesure de savoir exactement ou commence et ou s'arrête l'influence de ces groupes aux intérêts tentaculaires et variés, qui dirigent souvent dans l'ombre la vie quotidienne d'opérateurs "indépendants" qui ne soupçonnent même pas leur existence. Les rumeurs exagérées et les faits contradictoires abondent mais certains noms reviennent périodiquement sur le tapis et méritent de s'y attarder.

\subsubsection{Le Soleil Noir}
Par bien des aspects, le Soleil Noir est une organisation qui demeure mystérieuse. Pour autant qu'on le sache, le Soleil Noir étend son influence un peu partout dans la galaxie et inclut dans ses rangs une multitude d'individus peu recommandables de toutes les races imaginables. Son organisation, sa direction, ses objectifs et l'étendue de son pouvoir demeurent inconnus et sont l'objet des rumeurs les plus folles. Il est arrivé par le passé que des criminels se revendiquent du Soleil Noir mais le fait est plutôt rare et on ne saurait dire si leurs prétentions étaient fondées ou s'ils ont mis ce nom en avant pour tenter d'intimider leurs adversaires ou leurs victimes. Le Soleil Noir est un paradoxe vivant en ce sens que les autorités connaissent son nom, qu'il dispose même d'un blason et que certains criminels prétendent parfois qu'ils en sont membres alors que dans le même temps, personne ne peut affirmer que tout cela est autre chose que de la poudre aux yeux. Dans les milieux bien informés au sein des autorités comme de la pègre, on se contente généralement de réponses évasives ou d'affirmations invérifiables lorsque le sujet est abordé.\\

Si le Soleil Noir existe bien en tant que cartel du crime, le peu d'indices donnent à penser qu'il ferait dans toutes les formes de délinquance et de criminalité envisageable, depuis le racket en passant par la contrebande, l'esclavage, l'escroquerie et la piraterie.

\subsubsection{Les Clans Hutts}
Contrairement au Soleil Noir, l'existence des syndicats du crime Hutts est non seulement indéniable mais également bien connue du public dans toute la galaxie. Depuis des millénaires, les Hutts trempent effectivement dans tout ce que l'on peut imaginer comme crime, délit, combine ou arnaque. Bien que les gouvernants présents et passés aient toujours prétendu contrôler le problème, beaucoup de gens pensent que sans les rivalités féroces qui les opposent, les clans de Nal Hutta auraient depuis longtemps la mainmise sur le crime galactique, voire pire\ldots depuis l'Espace Hutt, les grands seigneurs du crime comme Jabba, Durga et consorts, qui aiment à se faire appeler de titres flatteurs comme "Votre Énormité" ou "Son Enflure", étendent les tentacules et les ramifications de leurs syndicats sur des dizaines de milliers de mondes. Les Hutts sont universellement réputés pour leur caractère capricieux, leur cruauté et leur opportunisme. Contrairement à la légende populaire, la plupart des seigneurs du crime Hutts (y compris les célébrités comme Jabba) n'opèrent pas totalement seuls mais au sein de leur clan familial, le kajidic. Dans l'organisation du kajidic, les Hutts ont toujours le rôle dominant puisqu'ils en sont les seuls membres de droit, leurs employés des autres races étant à leurs yeux des exécutants voués aux besognes secondaires. Contrairement à de nombreuses organisations du crime, les kajidics sont donc des organisations ou l'on ne peut espérer atteindre le sommet à moins d'être un Hutt. Même le plus insignifiant des hutts exerce toujours des responsabilités au sein de son clan, qu'il s'agisse de superviser ses activités illégales ou celles qui sont parfaitement légitimes. Race éminemment égocentrique les Hutts aiment à s'entourer de mercenaires et hommes de main qui doivent se montrer à la fois compétents, prompts à obéir et assez malléables pour éviter de s'attirer la colère de leur patron. Les Hutts sont en effet réputés pour leur mépris de la vie d'autrui mais la puissance de leurs organisations rend souvent les compensations financières plus qu'adéquates pour ceux qui acceptent de les servir et qui survivent suffisamment longtemps. Les Hutts s'intéressent à tous les marchés mais ils sont particulièrement présents et en position dominante en ce qui concerne la contrebande d'épices.

\subsection{Les Esclavagistes}
Fléau qui a toujours existé dans la galaxie, l'esclavage prend de multiples formes dont certaines officiellement tolérées par l'Empire. Par exemple, un certain nombre de mondes non-humains sont considérés comme des "états-clients" de l'Empire et leur population entière est salariée auprès de corporations ou de branches de l'administration impériale. En apparence, les indigènes sont donc des employés comme les autres mais techniquement, un employé qui ne peut démissionner, qui ne peut choisir son employeur, qui ne peut pas renégocier son salaire et qui doit le dépenser dans des magasins accrédités pratiquant des prix sur lesquels il ne peut influer et fournissant des marchandises qu'il ne peut choisir n'est rien de plus qu'un esclave. Et sa situation est plutôt enviable par rapports à d'autres\ldots\\

Certaines races sont également, par habitude ou par tradition, esclavagistes et procèdent à des "recrutements" massifs pour se procurer de la main d'œuvre. La plupart du temps, elles ciblent des mondes voisins et des peuples spécifiques qui sont leurs victimes ou ennemis traditionnels, pour des raisons économiques, militaires, religieuses\ldots. L'Empire a mis le plus souvent le holà à de telles pratiques sous prétexte de faire respecter l'égalité de droits entre citoyens impériaux, quelle que soit leur origine raciale. Ces peuples ont donc adopté des systèmes d'indenture proches de la législation impériale qui leur permet ainsi d'avoir des esclaves "légaux" bien qu'il s'agisse au mieux de paravents juridiques extrêmement minces mais suffisants pour les autorités impériales.\\

Cependant, dans leur grande majorité, les esclavagistes sont des professionnels. Il existe en fait deux activités distinctes liées à l'esclavage : la capture et la distribution.\\

Les groupes les plus importants disposent de branches distinctes qui s'occupent de fournir la matière première (raids, endettement abusif, etc\ldots), de l'entretien et du triage des marchandises et enfin de leur revente à des particuliers, des corporations ou des états intéressés.\\

Les groupes de taille plus modeste (la grande majorité) se contentent généralement d'une seule facette des opérations. Quelques esclavagistes se vouent exclusivement à la capture de proies revendues en bloc à des spécialistes qui feront le triage mais la plupart sont en fait des intermédiaires qui sont en affaire à la fois avec divers fournisseurs (notamment des bandes de pirates ou des dictateurs peu scrupuleux) et avec les clients potentiels.\\

Enfin, de nombreux chasseurs de primes, petits barons du crime et autres individus peu recommandables font parfois affaire avec des esclavagistes pour se débarrasser de personnes encombrantes tout en faisant un petit bénéfice\ldots\\

L'attitude des esclavagistes envers leur "marchandise" varie considérablement selon qu'ils visent à satisfaire la demande en quantité ou en qualité. Lorsqu'ils disposent d'un vivier considérable d'esclaves interchangeables, ils ont tendance (et les acheteurs également) à ne pas faire dans le détail ce qui permet de vendre à bas prix des individus qui coutent moins cher que des droïdes spécialisés onéreux à entretenir\ldots à l'opposé, les esclaves recrutés de manière spécifique (races exotiques ou individus ayant des capacités spéciales par exemple), sont généralement traités un peu mieux car on les destine à un créneau ou la demande n'est pas élevée. Dans le meilleur des cas, un grand trust esclavagiste dispose d'une division chargée du suivi médical de ses marchandises, y compris de gérer leur progéniture éventuelle.\\

Au niveau clientèle, les esclavagistes comptent surtout sur certaines corporations peu scrupuleuses qui ont besoin de main d'œuvre bon marché. L'Empire est un client occasionnel mais comme c'est lui qui fabrique et qui applique les lois, il recrute généralement directement sa "main d'œuvre" grâce à ses tribunaux et à ses gouverneurs chargés "d'inciter" les populations indigènes à servir la machine de guerre impériale. De plus, les réseaux d'esclavagistes légaux versent leur taxe à l'empire lors de la vente de chaque esclave, comme il en va de toute marchandise passant par des circuits officiels. Ainsi, l'Empire accroit sa puissance à la fois par la souffrance qu'il inflige directement et par celle que d'autres infligent en accord avec ses lois.\\

L'autre clientèle des esclavagistes est constituée par des personnes assez riches pour satisfaire leurs caprices et qui cherchent des serviteurs d'un genre particulier liés à eux pour la vie et revendables. Objets de plaisir, cuisiniers, serviteurs, mineurs, ouvriers, artistes et artisans ou parfois même (dans le cas de certaines races d'esclaves à l'éthique guerrière appropriée) gardes du corps ou soldats. Comme on peut s'en douter, ces clients achètent rarement en gros à moins d'avoir des mines ou des cultures à faire entretenir à bas prix.

\subsubsection{Exemple de groupe esclavagiste : Les Zygériens}
Longtemps honni sous l'Ancienne République, ce groupe de la Bordure Extérieure a pu refaire surface avec la législation impériale. Bien qu'il existe une multitude d'associations, guildes, cartels et syndicats esclavagistes, dont certains opérant sur des centaines de systèmes, les Zygériens se taillent une certaine réputation dans ce domaine et une relative bienveillance des autorités par leur professionnalisme exacerbé. Ils opèrent dans les strictes limites des ordonnances impériales, ne massacrent pas les populations, veillent scrupuleusement sur la qualité de leurs "marchandises" et réduisent la violence au strict minimum.
Cette attitude leur permet de se placer au sommet dans leur créneau d'activité et écarte une bonne partie des tracasseries administratives même si du point de vue des victimes cela ne change pas grand-chose\ldots les méthodes des Zygériens sont tellement bien rodées que certains mondes comme Ryloth (le berceau de la race Twi'lek), qui pratiquent depuis longtemps la vente de certains de leurs ressortissants, préfèrent traiter de manière régulière avec les Zygériens plutôt que d'avoir à faire avec des maraudeurs d'autres organisations.\\

Dans la Bordure Extérieure, le Syndicat des Esclavagistes Zygériens est l'organisation la plus importante dans sa partie et les zygériens assurent toute la chaine commerciale de leur activité, depuis leurs groupes de capture spécialisés en passant par leurs centres de détention et de sélection pour aboutir à leurs courtiers et grossistes. Tous les esclaves vendus par les zygériens le sont avec des certificats de qualité et des dossiers de propriété parfaitement en règle qui détaillent les origines de l'esclave, sa formation et ses compétences éventuelles ainsi que son état de santé lors de la vente. Les zygériens sont parvenus à aseptiser à un tel point leurs transactions qu'ils en sont devenus presque "respectables" et tolérés sur certains mondes normalement opposés à l'esclavage sur le plan légal, peu de gens dans les systèmes ou ils opèrent prenant la peine de réfléchir à ce que l'esclavage implique du côté des victimes\ldots

\subsection{Les Pirates}
La piraterie a toujours été une activité risquée mais lucrative. L'Empire a adopté dans de nombreux secteurs une attitude de "tir à vue" en ce qui concerne les navires pirates (ou soupçonnés d'être pirates\ldots) ce qui n'empêche pas les criminels de se multiplier, principalement parce que l'espace connu est trop vaste pour pouvoir être régulièrement contrôlé. Les pirates réalisent leurs plus gros coups sur les grandes routes marchandes mais c'est aussi en ces occasions qu'ils courent les plus gros risques. Certains d'entre eux opèrent en tant que corsaires pour un gouvernement local en délicatesse avec ses voisins mais la grande majorité sont des renégats qui tueraient leur propre mère plutôt que de signer un formulaire des douanes impériales. Certains groupes de pirates ne comptent qu'une poignée d'hommes dans un vaisseau poussif mais d'autres incluent des milliers de combattants répartis sur une flottille d'engins meurtriers\ldots de même, les tactiques des pirates vont de l'assaut frontal pur et simple aux embuscades minutieusement préparées. Certains parmi les groupes les plus importants ont même quelques victoires contre l'Empire à leur crédit.\\

Comme on peut s'y attendre, les pirates agissent le plus souvent à proximité d'une planète, d'un corps céleste ou d'un phénomène naturel qui oblige les vaisseaux de passage à naviguer à vitesse subluminique. Il existe aussi diverses manières de bloquer temporairement une route hyperspatiale et ainsi forcer un vaisseau à revenir dans l'espace normal pour tomber dans les griffes des pirates : explosions nucléaires, détournements de comètes, installation d'un blocus de plusieurs navires sur une route très étroite, projecteurs de perturbations électromagnétiques, etc\ldots méthodes créant un obstacle imprévu sur un plan de navigation et que les navordinateurs ne pourront pas gérer sans revenir dans l'espace normal pour faire le point.\\

La plupart des pirates sont aussi en rapport avec des réseaux d'esclavagistes. Ils revendent ainsi les équipages et passagers qui ne peuvent pas être rapidement rançonnés. Certaines parmi les grandes compagnies de transport incluent dans les contrats de leurs officiers une clause de rançonnement et les plus généreuses en font même bénéficier les autres gradés ou les ingénieurs de bord. Pour les simples hommes d'équipage ou les passagers sans réelle richesse, les perspectives sont nettement plus sombres. Il arrive parfois qu'un groupe de pirates dont les effectifs ont été sérieusement entamés propose aux prisonniers de choisir entre l'esclavage ou rejoindre les rangs de leurs attaquants et certains comptent même majoritairement des engagés de force qui ont fini par adopter leur nouveau style de vie. A l'opposé, il existe un grand nombre de groupes qui ne recrutent que des membres d'une certaine race ou même seulement les adeptes d'une religion exotique particulière\ldots enfin, une petite minorité de pirates essaye de traiter les équipages avec un minimum de dignité, entres autres en les abandonnant à bord des capsules de sauvetage de leur navire capturé ou en les déposant sur une planète habitable, voire même civilisée, moyennant une rançon des plus symboliques. Ces groupes agissent le plus souvent ainsi pour des raisons tactiques (on aura plus tendance à combattre jusqu'à la mort contre des pirates réputés pour ne pas faire de quartiers ou vendre tous leurs prisonniers que contre des attaquants qui acceptent de traiter correctement les survivants de leur assaut) mais parfois également pour des raisons idéologiques.

\subsubsection{Les Corsaires}
Les corsaires sont des pirates agissant pour le compte d'un gouvernement ou d'une corporation. En fait, leur activité n'est pas plus légale que s'ils travaillaient au service d'un clan Hutt mais généralement, leur employeur attend d'eux qu'ils s'attaquent à des cibles bien précises et pas à tout ce qui passe à portée de tir. La plupart des corsaires ont donc passé un accord avec leur employeur sous la forme d'une Lettre de Marque. La Lettre de Marque définit quelles sont les cibles que veut faire attaquer l'employeur et quel pourcentage des biens saisis il laisse aux corsaires en guise de rémunération. En échange, le dit employeur accorde généralement des remises importantes lorsque les corsaires doivent faire réparer leur navire et peut même accepter de les dissimuler dans un lieu discret ou de leur fournir des navires et des armes. Par rapport aux pirates classiques qui disposent rarement d'un port sûr et pas toujours de matériel de qualité, la condition de corsaire peut s'avérer intéressante sur un plan financier. Généralement, les corsaires savent que leur avenir dans ce domaine dépend de leur réputation et ils prennent garde à ne pas "braconner" en sautant sur des navires qu'ils n'ont pas été engagés pour poursuivre mais il y a bien évidemment un certain nombre d'exceptions\ldots\\

Aux yeux de la loi, les corsaires sont de toute manière des pirates à part entière et il n'y a guère que leurs employeurs qui soient prêts à fermer les yeux sur certaines de leurs petites affaires, et encore\ldots en raison de la puissance considérable de sa Marine, l'Empire n'a aucun intérêt officiel à sponsoriser des corsaires mais certains hauts responsables ou gouverneurs locaux peuvent avoir des raisons plus\ldots personnelles de le faire.

\subsubsection{Les ports-francs}
La plupart des pirates ne disposent pas d'une base fixe et leur navire est leur seul foyer. Mais même pour les groupes qui ont la chance d'avoir un pied-à-terre difficile à localiser, il est indispensable à un moment ou un autre de faire le plein, remplacer le matériel détruit, écouler les marchandises volées, etc\ldots En dehors de quelques contrebandiers complaisants qui font de la livraison à domicile, la seule solution (bien plus pratique d'ailleurs) est de se rendre dans un port-franc.\\

Les ports-francs sont indispensables à l'exercice des activités des pirates mais aussi de nombreux mercenaires et contrebandiers. Il s'agit d'astroports le plus souvent clandestins ou les capitaines pourront faire réparer leur navire, laisser l'équipage se détendre, prendre contact avec des receleurs pour vendre ce qu'ils ont dans leurs soutes ou passer commandes de pièces particulières et occasionnellement recruter du personnel pour remplacer les pertes au combat ou augmenter leurs effectifs.\\

Généralement, les ports-francs sont situés sur des routes peu connues ou dissimulés dans des champs d'astéroïdes, des nébuleuses, etc\ldots bien qu'il existe un certain nombre d'astroports que les autorités locales ignorent délibérément moyennant quelques arrangements discrets et tant que les choses demeurent assez peu visibles. Les tarifs de ce que l'on peut acheter ou vendre dans un port-franc sont généralement à la limite de l'escroquerie flagrante. En effet, leurs propriétaires savent que la plupart du temps ils sont les seuls dans le coin à proposer leurs services ou que les autres ports-francs sont tout aussi difficiles à localiser. De même, bien que de nombreux ports-francs soient des endroits ou la loi du blaster est la seule que l'on respecte, certains disposent de milices extrêmement bien organisées et même de navires de défense destinés à calmer les ardeurs des clients et à éviter les règlements de compte dans leur voisinage.

\subsubsection{Exemple de groupe pirate : Les Survivants de Khuiumin}
Les Survivants de Khuiumin sont tout ce qui reste d'un des plus puissants groupe pirate de la Bordure Extérieure. Opérant autrefois depuis le système Khuiumin, leurs effectifs de plus de 8000 hommes étaient répartis dans une petite escadre de combat capable de vaincre n'importe quel convoi : 28 corvettes, plus de 50 cargos légers et yachts modifiés et environ 70 chasseurs stellaires. Il y a quelques années, deux stars destroyers impériaux, le Bombard et le Croisé furent chargés de briser cette bande qui devenait un peu trop arrogante au gout des autorités.
Malgré un rapport des forces relativement en faveur des pirates, les impériaux utilisèrent leur orgueil pour les mener à une bataille perdue d'avance et pulvérisèrent l'armada pirate dont seuls une corvette et quelques chasseurs réchappèrent du désastre, avec des effectifs n'atteignant pas les 300 hommes. Depuis, les Survivants de Khuiumin ont adopté un profil nettement plus bas pour reconstituer leurs forces. En plus de leur activité de piraterie, ils recherchent avidement les deux stardestroyers et leurs capitaines pour des raisons très faciles à deviner.

\subsection{Les Slicers}
Malgré le démantèlement partiel du HoloNet, il existe encore beaucoup de réseaux de communication planétaires, interplanétaires et interstellaires. Une catégorie de criminels bien précise, les Slicers, s'est vouée à forcer les protections informatiques de ces réseaux pour collecter et revendre les informations de valeur qu'ils contiennent. La plupart des Slicers sont farouchement individualistes mais il existe aussi des groupes, fraternités et cellules d'espionnage organisés autour de leurs activités. Et celles-ci sont loin d'être évidentes :\\

La puissance des systèmes informatiques est telle que l'usage d'une machine individuelle ne mène généralement pas à grand-chose dès que l'on tente de s'attaquer à des bases de données un peu protégées.\\

La situation est compliquée par le fait que certains mondes ont des lois très laxistes alors que d'autres sont extrêmement restrictifs et que les points d'accès instantanés depuis un autre système stellaire sont faciles à circonscrire : le Holonet (en lui-même physiquement et logiciellement très bien protégé) et l'éventuel réseau subspatial local qui, s'il existe, n'est pas non plus généralement accessible au grand public. De plus, un grand nombre de systèmes stellaires n'ont aucun de ces deux moyens de communication instantanée à leur disposition.\\

Grosso modo, les pirates informatiques doivent donc opérer dans les limites d'une planète ou de ses lunes. S'ils disposent d'un navire spatial ou de son équipement de communication subspatial, ils peuvent espérer joindre des planètes proches, voire les systèmes avoisinants, à condition d'avoir un délai dû à la distance nul ou extrêmement réduit. De plus, si on met de côté les réseaux de communication standardisés des institutions, gouvernements et corporations interstellaires, le reste n'est a priori destiné qu'à un usage local et rarement compatible avec des protocoles développés pour un autre usage local.\\

Les bases de données de l'Empire et des autres organisations galactiques publiques ou privées sont bien évidemment des cibles fort tentantes puisque leurs protocoles internes sont standardisés et qu'elles disposent souvent de l'accès au Holonet. En théorie, quelqu'un qui arriverait à pénétrer dans le bureau d'un Gouverneur Impérial, à utiliser ses codes d'accès et à faire passer son portable boosté pour le terminal d'origine pourrait aller vider des banques de données impériales à l'autre bout de la galaxie mais dans la pratique, il faudrait qu'il puisse disposer de dizaines d'autorisations et d'encryptages que même ses logiciels ne pourront pas tous simuler et qui sont régulièrement mis à jour. Le simple fait d'envoyer son nom d'un système à l'autre prendrait des heures de travail dans le meilleur des cas.
Il est bien plus économique, moins risqué et moins difficile de cibler ses activités sur le réseau planétaire qui peut s'avérer très conséquent.\\

Ainsi, la minorité des Slicers les plus réputés sont des gens le plus souvent sans adresse fixe ou qui possèdent un navire. Ils se rendent sous un prétexte ou un autre dans un système stellaire pour faire quelques opérations et soit s'y installent pour devenir des spécialistes locaux, soit bougent rapidement vers un autre système pour recommencer. Rien que les frais initiaux d'investissement (navire spatial, programmes de slicing performants) et de maintien à niveau question logiciel mettent hors course 99\% des candidats à cette lucrative activité.
De fait, beaucoup de programmeurs intéressés par l'accès illégal préfèrent donc en rester à un niveau purement local ou doubler leur activité de pirate avec une activité de chiffreur, réalisant (ou perçant) sur mesures les codes du gouvernement local, d'une corporation, voire d'un syndicat du crime ou d'un réseau clandestin\ldots\\

Les Slicers sont fondamentalement divisés en deux catégories : ceux qui accèdent aux informations pour les redistribuer ou les revendre et ceux qui y accèdent afin de les détruire ou les modifier. Dans l'absolu, la frontière entre les deux est extrêmement ténue et la plupart des Slicers ne ressentent pas cette division et font surtout en fonction de leurs goûts personnels, de leur matériel et des opportunités. Malheureusement pour les Slicers, rares sont ceux qui parviennent à exercer cette activité assez longtemps pour s'enrichir. En effet, la civilisation galactique dispose en matière d'intelligence artificielle de gigantesques ressources qui permettent au gouvernement planétaire le plus insignifiant de mettre en place une politique de contrôle de ses bases de données gérée par une demi-douzaine d'employés et une cinquantaine de droïdes spécialisés, actifs en permanence et parfaitement suffisants à l'échelle d'un monde de plusieurs milliards d'habitants. Les institutions bancaires, les grandes compagnies interstellaires, les services de police etc\ldots disposent de ressources encore plus colossales à cet égard. Ainsi, l'accès à l'information est difficile mais surtout l'ajout, l'effacement ou la modification de données sont rendues très délicates parce que les droïdes vérificateurs procèdent à des comparaisons multiples permanentes qui atteignent parfois des centaines de milliers de fichiers par minute.

\subsection{Exemple de Slicers : les réseaux Bothans}
La réputation d'espions intergalactiques de premier plan des indigènes de Bothawui n'est plus à faire et tient en partie à leurs groupes de Slicers. Les Slicers bothans peuvent être collectivement considérés comme le sommet de la profession, non pas en tant qu'individus doués mais en tant qu'organisation de renseignement électronique officieuse opérant à l'échelle de la galaxie. Les Clans Unis de Bothawui sponsorisent un grand nombre d'opérateurs en solo ou de petits groupes et possèdent des services de collecte et de traitement des données aux capacités impressionnantes.\\

Lorsqu'ils comptent procéder à une collecte de longue durée ou une surveillance, les Bothans utilisent bon nombre de corporations, missions diplomatiques et particuliers comme couverture à leurs activités et ils font toujours en sorte que leurs agents passent par des intermédiaires d'autres races lorsqu'il s'agit d'opérations en dehors de leur sphère d'influence ou lorsqu’un contact direct est nécessaire. Ces intermédiaires ignorent tous des objectifs de leurs employeurs et les longues traditions bothanes en matière d'intrigue permettent de cloisonner efficacement les informations qui leur sont fournies.\\

Ainsi, les bothans eux-mêmes collectent depuis Bothawui ou d'autres mondes un certain nombre d'informations et utilisent un vaste réseau d'hommes de main pour assurer les facettes de leur activité les plus susceptibles de permettre de remonter jusqu'à eux. Sur Bothawui et les autres mondes leur appartenant, les bothans utilisent également leurs Slicers pour rendre plus étanches leurs bases de données et les contrôler.\\

Des organisations comme le BSI ou les Renseignements Impériaux se sont largement inspirées des techniques bothanes pour monter des opérations sur des mondes non-humains ou leurs agents seraient trop facilement reconnaissables. Ainsi, ils procèdent souvent par le biais d'opérateurs d'autres races et d'un groupe de Slicers afin de constituer des cellules de renseignement permanentes sur les mondes non-humains ou l'Empire est peu apprécié.

\subsection{Les Contrebandiers}
Dans la grande majorité de la galaxie, les mégacorporations de transport interstellaires monopolisent avec leurs lignes régulières de gigantesques cargos l'acheminement des biens et des personnes. Il y a cependant des quantités incroyables de pilotes indépendants qui ravitaillent les mondes les plus isolés ou assurent des conditions de transport convenant à des clients ou des cargaisons spécifiques, voire qui sont les seuls à offrir des marchandises exotiques venant de coins perdus ou les grandes compagnies ne s'aventurent jamais pour des raisons de rentabilité. En fait, sans cette multitude d'indépendants et de petites compagnies limitées à deux ou trois navires de petite taille, la civilisation galactique ne serait jamais parvenue à s'étendre au cours des millénaires.\\

Les activités des indépendants les amènent souvent à se tourner vers la contrebande dont il existe plusieurs formes. Aux yeux de la plupart des gens, un contrebandier transporte par nature des substances illégales mais cela n'est pas forcément vrai. Il existe en effet trois raisons essentielles de faire appel à un contrebandier pour transporter discrètement une cargaison :
\begin{itemize}
	\item[Pour des raisons légales] Généralement, cela concerne la technologie militaire, certaines substances toxiques, les esclaves et diverses denrées ou substances précieuses et d'accès restreint. Les restrictions légales peuvent concerner l'ensemble du territoire de l'Empire qui s'en réserve le monopole ou interdit purement et simplement le commerce et la détention des marchandises en question. Il peut aussi s'agir de marchandises réservées à certains groupes (agences de police, corporations accréditées) ou certains marchés légaux mais surveillés (marché des esclaves et marché de l'épice par exemple).  De même, une marchandise peut être parfaitement légale à certains endroits mais bannie par les autorités locales ailleurs (exemple : certains sels marins que l'on trouve dans de nombreux océans de la galaxie ont des propriétés hallucinogènes et toxiques sur le peuple des Arconas et leur commerce est donc interdit dans leur espace\ldots).
	\item[Pour des raisons économiques] Le transport ou le commerce de ces marchandises est légal mais fait l'objet de lourdes taxes impériales ou locales (voire même impériales ET locales\ldots) et ceux qui font appel aux contrebandiers considèrent que les pertes et risques du circuit du marché noir valent le coup par rapport aux taxes à payer. Ou encore, l'offre et la demande sont telles que le prix public est monstrueusement élevé par rapport à la nécessité qu'en ont les acheteurs.
	\item[Pour des raisons politiques] Une des tactiques "douces" de l'Empire pour faire plier un monde dont il veut récupérer les infrastructures consiste à établir un blocus total de son espace et à paralyser ainsi son commerce. Bien que de nombreuses planètes puissent survivre sur le plan alimentaire à un tel blocus, leur survie économique est loin d'être aussi vraisemblable et pour peu qu'elles soient obligés d'importer massivement certains aliments, équipements essentiels (systèmes environnementaux par exemple) ou médicaments indispensables, leurs dirigeants se retrouvent vite devant un choix très simple : céder totalement à l'Empire ou tenter de négocier une reddition en jouant la montre et en passant par des contrebandiers particulièrement audacieux. 
\end{itemize}

La plupart des contrebandiers disposent de moteurs puissants, de navires maniables, d'un peu de crédits de côté pour quelques pots de vins et de beaucoup d'audace. En effet, même si l'Empire n'a ni les moyens, ni l'intention de contrôler tous les vaisseaux qui traversent l'espace, les Douanes, la Marine, et un tas d'agences de police locales, compliquent quand même pas mal les choses. Des senseurs longue portée et des systèmes passifs ou actifs permettant de tromper les scanners aident sensiblement les contrebandiers mais une fouille intégrale d'un navire est une expérience que beaucoup d'entre eux ont malheureusement faite et la majorité ont dû dépenser jusqu'à leur dernier crédit en pots de vin pour ne pas terminer leur vie derrière des barreaux.\\

Les Douanes impériales ont leur comptant d'officiers qui font du rentre-dedans mais sont en fait prêts à fermer les yeux sur des infractions mineures ou même plus sérieuses si l'on a les bonnes références. Il ne faut cependant pas généraliser et également tenir compte des officiers de la Marine Impériale qui ont quant eux la réputation d'être notablement moins corrompus.\\

La vie d'un contrebandier n'est pas non plus exempte de risques en dehors des autorités puisque même lorsqu'il transporte une cargaison légale, il n'est pas à l'abri des hasards des voyages spatiaux ainsi que des pirates.
De plus, les vaisseaux spatiaux ne sont pas donnés et les modifier afin de battre de vitesse les navires de patrouille est très onéreux. De même, les contrebandiers changent souvent la signature du transpondeur de leur navire afin d'échapper aux recherches et cette opération est loin d'être anodine. Enfin, un certain nombre de clients recourent parfois au meurtre afin de récupérer leur cargaison et acquérir en même temps un vaisseau en état de marche, le tout gratuitement.\\

À l'origine, la plupart des contrebandiers n'ont pas commencé leur carrière dans cette perspective mais dans celle de petits transporteurs en règle avec la loi. Cependant, les frais engendrés par les permis, taxes de stationnement, acquisition et entretien (voire optimisation) d'un navire et les salaires de l'équipage ont amené la plupart de ces pilotes à emprunter des sommes d'argent conséquentes à des gens moins regardants que les banques : les usuriers. D'autres se sont vus "offrir" leur navire par des petites sociétés respectables dissimulant des activités plus marginales.\\

Dans un cas comme dans l'autre, le capitaine indépendant doit un certain nombre de faveurs et/ou d'argent à des individus pas très recommandables et pour les rembourser (ou leur rendre leurs bienfaits\ldots), le transport de marchandises de contrebande s'avère presque toujours obligatoire. Ceux qui parviennent à survivre assez longtemps obtiennent généralement des bénéfices plus élevés que prévu à l'origine et la plupart finissent par se tourner de leur plein gré vers la contrebande, y compris celle impliquant les cargaisons les plus douteuses sur un plan moral (esclaves, armes bactériologiques\ldots).\\

Les contrebandiers moins doués ou moins chanceux terminent généralement au fond d'une prison, atomisés par un navire impérial ou encore pourchassés jusqu'à la mort par leurs clients parce qu'ils ont perdu trop de cargaisons ou ne sont pas parvenus à rembourser leurs dettes assez vite\ldots\\

Il existe un grand nombre de petites confréries et associations informelles de contrebandiers mais la plupart d'entre eux sont des individualistes par nature (dans le cas contraire, ils préfèrent entrer directement au service d'un seigneur du crime et renoncer à une liberté trop souvent théorique en échange d'un emploi dangereux mais stable et où ils bénéficient d'une certaine "protection" vis à vis des autorités et de certains pirates\ldots). Il n'y  a donc pas de grands cartels de la contrebande ou pour être plus exact, les contrebandiers ne contrôlent pas vraiment les grands trafics de la galaxie. En effet, ils ne sont dans le fond que les exécutants ou les sous-traitants d'organisations bien plus puissantes et étendues qui préfèrent faire appel à des pilotes indépendants pour assurer l'essentiel de leurs transports ce qui leur offre le double avantage de maintenir une façade de respectabilité tout en n'ayant pas à amortir les frais d'un navire et de son équipage employé à temps plein alors que cela n'est pas toujours nécessaire\ldots les grands syndicats du crime sont particulièrement vigilants et guettent tout signe indiquant que leurs contrebandiers seraient en train de s'unir dans leur dos, ce qui poserait à la fois des problèmes d'ordre disciplinaire et surtout d'ordre financier (les groupes de pression obtiennent de meilleurs tarifs que les individus isolés\ldots).\\

Contrebandier est donc une profession particulièrement appropriée pour les individualistes qui veulent faire leurs preuves parce que c'est exactement ce que l'on attend d'eux : qu'ils parviennent à faire ce qu'ils affirment et peut-être qu'à la longue s'ils sont assez malins, exigeants, diplomates et surtout rentables, ils parviendront à fixer le prix de leurs services à un niveau satisfaisant\ldots dans le cas contraire, ils sont partis au mieux pour lécher les bottes de leurs commanditaires et au pire pour être descendus dans une ruelle sombre ou atomisés en particules avec leur navire.\\

Tout comme les pirates et les mercenaires, les contrebandiers font grand usage des ports-francs et il leur arrive aussi de faire partie intégrante de l'économie de telles installations (ce sont eux après tout qui transportent les pièces et aliments nécessaires à la survie d'une communauté illégale\ldots).

\subsection{Les Mercenaires}
Alors que l'Empire tente d'exercer un farouche contrôle sur les forces armées des gouvernements sous sa domination et sur les moyens de défense accessibles à ses citoyens, il n'a curieusement jamais tenté d'empêcher la prolifération d'une multitude de sections, brigades, confréries, régiments et légions mercenaires de tous types et de toutes races. L'inconvénient de cette situation est qu'un certain nombre de ces organisations se livrent également à diverses exactions, voire à la piraterie ou à l'esclavagisme pratiquement au vu de tous. Néanmoins, du point de vue du pouvoir impérial, les mercenaires ont leurs raisons d'être :

\begin{itemize}
	\item L'Empire a parfois besoin de sous-traiter certaines opérations militaires sans lien apparent avec l'Ordre Nouveau et qui demandent un niveau de professionnalisme et de polyvalence certain. Bien que redoutables individuellement ou en petits groupes, les Chasseurs de Primes n'ont pas dans leur grande majorité de formation militaire et encore moins la logistique et l'équipement d'une brigade de soldats de métier.
	\item En infiltrant des agents du BSI et des Renseignements dans de nombreuses organisations mercenaires, les impériaux parviennent à se tenir au courant de la plupart des ventes d'armes significatives dans le milieu, des contrats passés avec des organisations hostiles (comme l'Alliance Rebelle\ldots) et des plans de bataille de certains mondes récemment contactés et peu enclins à se soumettre de leur plein gré\ldots
	\item En cas de grosses difficultés locales, un Gouverneur ou un Moff avec quelques liquidités à sa disposition peut toujours faire appel à des soldats professionnels sans avoir à solliciter des renforts qui peuvent s'avérer indispensables ailleurs au même moment\ldots une fois la crise passée, les mercenaires se retrouvent disponibles pour de nouveaux contrats et coutent néanmoins moins cher que l'entrainement et l'entretien de troupes à plein temps. (Même si dans son ensemble l'Empire est un véritable gouffre financier sur le plan militaire, les responsables territoriaux sont parfois nettement moins libres dans ce domaine en fonction de l'importance stratégique de leur secteur ou des ressources dont ils peuvent disposer par le biais des taxes imposées aux populations et industries locales\ldots). 
\end{itemize}


\subsubsection{Exemple de groupe mercenaire : le Premier Régiment Mobile Soleil}
Cette unité opère principalement dans la Bordure Extérieure. C'est un régiment mercenaire spécialisé dans les missions de type "recherche et destruction". Contrairement à de nombreux groupes mercenaires qui se donnent des titres fantaisistes, le Premier Mobile Soleil est effectivement un véritable régiment d'infanterie aérotransportée selon les normes de l'Empire. Ses effectifs fluctuent donc autour de 3530 hommes (dont environ 2558 combattants) et 130 véhicules à répulseurs.\\

Bien que de nombreuses planètes ou corporations puissent se permettre de louer les services de l'un des quatre bataillons qui composent le Premier Mobile Soleil, seul l'Empire s'est avéré capable de financer régulièrement des missions impliquant l'ensemble de la force mercenaire (même les grandes corporations n'ont pas souvent besoin d'un tel appui militaire).\\

Le Premier Mobile Soleil est ainsi utilisé par plusieurs Moffs de la Bordure Extérieure lorsque des problèmes locaux monopolisent leurs ressources militaires classiques ou qu'il faut s'adjoindre une force de frappe significative. L'Empire étant son principal client et n'étant pas trop regardant sur les méthodes employées, le régiment a une réputation plus que méritée d'unité brutale et impitoyable et certains médias l'accusent d'un grand nombre d'atrocités officiellement commises par des "pirates et maraudeurs de l'Espace Sauvage".

\subsection{Les Récupérateurs}
Toute civilisation produit des déchets industriels, que ce soit par son activité économique, par l'évolution technologique qui rend obsolète certains équipements ou par la guerre qui produit son lot d'épaves et de ruines. La civilisation galactique, qui n'est ni plus ni moins que l'agrégat d'une multitude de cultures et d'industries, produit elle aussi des milliards de milliards de tonnes de rebuts, de pièces usagées et de vaisseaux hors d'état.\\

Qu'il s'agisse de fouiller une décharge planétaire, de scruter l'espace d'un secteur réputé dangereux pour la navigation ou de roder parmi les ruines d'un champ de bataille, une multitude de petits groupes sont parvenus à créer leur propre créneau économique en récupérant et en revendant de la ferraille, des appareillages réparés ou bricolés, des armes encore capables de tirer à peu près droit, des navires spatiaux plus ou moins capables de voler et ainsi de suite.\\

Qu'il s'agisse d'un clan Jawa qui erre parmi les dunes de Tatooine en quête de matériel abandonné par les colons ou d'un groupe de charognards qui suit les armées impériales en campagne, les Récupérateurs ont souvent fort à faire avec la concurrence, parfois de manière très violente. En effet, même si la quantité de "matières premières" est souvent incroyable, une bonne partie est totalement impossible à revendre ou nécessite trop de travail pour s'avérer rentable. En plus de cette concurrence, les autorités locales ou impériales gardent à l'œil ce genre d'activité qui pourrait permettre de fournir en armes ou en matériel des dissidents en puisant dans les déchets des autorités\ldots de plus, la "récupération" est légalement désignée dans la plupart des systèmes sous un terme beaucoup moins flatteur : pillage.\\

Les Récupérateurs sont donc des gens prompts à saisir les bonnes opportunités, rapides à dispara\^{i}tre et le plus souvent pas trop regardants sur l'origine de leur marchandise. Lorsque les affaires ne sont pas terribles, certains parmi les plus opportunistes de ces charognards n'hésitent pas à virer dans le banditisme et la piraterie à mi-temps.\\

Généralement, la qualité de ce que les Récupérateurs revendent est très moyenne, voire assez souvent en dessous de tout. Les navires, droïdes ou armes pratiquement intacts sont plus souvent l'exception que la norme. La plupart du temps, les appareillages récupérés et revendus ont subi tellement de modifications que leur durée de vie peut en être dramatiquement affectée. Il n'est pas rare qu'ils ne soient même plus capables de réaliser certaines de leurs fonctions essentielles. Il arrive même qu'une machine ou un véhicule tombe en panne entre le moment où le client l'achète et celui où il tente de s'en servir pour la première fois.\\

Les plus grands groupes de Récupérateurs ont mis en place de grands marchés ou des réseaux de revente très complexes afin de fourguer leur camelote à toute la galaxie. En effet, malgré un manque de fiabilité chronique, il n'est pas rare que des particuliers dans le besoin ou des mondes au niveau technologique faible soient prêts à tenter leur chance (ou à se faire escroquer) avec du matériel de récupération.\\

\subsubsection{Exemple de groupe de Récupérateurs – les Squibs}
La race des Squibs (des bipèdes à fourrure bleue ou rouge dont le faciès rappelle celui d'un renard) est sans aucun doute l'exemple le plus incroyable de toute l'histoire de la Récupération. Négociateurs avisés et nomades compulsifs, les Squibs sont en effet unis sous la bannière du Consortium Commercial Squib, une corporation à l'existence tout ce qu'il y a de plus légale. Tous les Squibs sont membres de droit du Consortium (mais certains ne travaillent qu'occasionnellement pour lui) et celui-ci est même parvenu à décrocher de juteux contrats avec certaines planètes, plusieurs autres corporations et même, dans certains secteurs, avec l'Empire. Les Squibs agissent dans le cadre de ces contrats selon une procédure très simple "vous balancez vos ordures ou vous voulez, nous les ramassons et vous ne déboursez rien". Bien évidemment, les Squibs font leur possible pour signer des contrats  leur permettant de récupérer un maximum de choses revendables (les contrats de la Marine Impériale sont particulièrement recherchés). Ils retapent et réparent ensuite leurs acquisitions avant de les revendre à des clients intéressés après d'intenses négociations (un Squib considère comme preuve de folie ou de sénilité une transaction réalisée sans passer quelques heures ou quelques jours à se jeter des arguments alambiqués à la figure\ldots).\\

Bien que les Squibs tiennent à maintenir leur réputation à un niveau nécessaire mais suffisant (leur corporation ne figure même pas dans les registres officiels de certaines planètes qui font appel à elle depuis des siècles) pour leurs affaires, des rumeurs prétendent que certains négociants Squibs sont parvenus à plusieurs reprises à revendre du matériel impérial délaissé dans un secteur au gouvernement (tout aussi impérial) d'un secteur voisin. Il serait éventuellement, toujours selon la rumeur, possible de faire appel au Consortium pour des missions de pillage, pardon, de récupération spécifiques (par exemple, l'Alliance Rebelle serait parvenue ainsi à récupérer à posteriori des quantités importantes d'équipement qu'elle avait dû abandonner sur certains champs de batailles ou ses forces avaient été vaincues par les impériaux).

\subsection{Cultes et églises}
Officiellement, l'Empire se moque pas mal des croyances religieuses de ses membres. Dans la pratique, il est souhaitable de ne jamais faire passer ses convictions avant sa fidélité envers le régime impérial\ldots bien évidemment, les tenants de certaines philosophies sont pourchassés à vue (les Chevaliers Jedi par exemple) alors que d'autres entretiennent parfois d'excellents rapports avec le régime politique central, quel qu'il soit.\\

Parmi les millions de races intelligentes qui peuplent la galaxie connue, presque autant de religions ont fait surface durant la lente évolution d'innombrables civilisations. La plupart ne se sont jamais étendues au-delà d'un monde, voire d'un continent. Un grand nombre ont disparu ou ont changé et se sont transformées au point de devenir méconnaissables. Certaines ont doucement végété dans la sphère d'influence d'un peuple unique alors que d'autres ont lancées des croisades qui ont embrasé des dizaines de mondes. Une quantité incroyable de religions n'ont sans doute qu'un seul adepte, prophète auto-proclamé en quête de convertis alors qu'un nombre conséquent est parvenu à créer des états théocratiques ou les prêtres administrent la vie de leurs fidèles à tous points de vue.\\

On peut cependant à titre d'exemples tirer de cette sarabande galactique quelques religions qui ont depuis un certain temps un retentissement à l'échelle de l'espace connu et dont le crédo a été adopté par biens des races sur bien des mondes ou qui ont marqué l'histoire de manière plus\ldots originale.

\subsubsection{L'Église du Grand Cercle Sacré}
Cette foi contemplative est basée sur la méditation et la recherche de l'harmonie. Toutes les espèces intelligentes y sont admises. Le Grand Cercle est basé sur la planète Monastère dans la Bordure Intérieure. Il se tient soigneusement à l'écart de la politique galactique depuis des millénaires et sa neutralité est bien connue. D'ailleurs, Monastère fut précisément choisie par les dirigeants du Grand Cercle de l'époque parce qu'elle était totalement à l'écart des routes commerciales existantes durant cette période ou la Bordure Intérieure était encore mal connue et simplement appelée "la Bordure". Le crédo du Grand Cercle est que si tous les êtres vivants de la galaxie parviennent d'eux-mêmes à rechercher l'harmonie intérieure selon les principes enseignés par l'Église, le "grand cercle" de la vie, qui englobe toute la galaxie, sera réalisée et une nouvelle forme d'existence supérieure sera accessible à tous. On peut considérer le Grand Cercle comme une idéologie pacifiste et progressiste puisque la notion de croisade ou de conversion forcée est à l'antithèse de cette idée centrale de communion des âmes. Le Grand Cercle rejette toute notion de violence même pour l'autodéfense ou la protection d'autrui.\\

En dehors de Monastère, le Grand Cercle a des chapitres un peu partout, presque uniquement des temples-monastères dont les prêtres assurent l'entretien et cultivent les petites plantations. Divers ateliers ou services rémunérés mais bon marché à destination du circuit économique local (viticulture, artisanat, enseignement primaire, dispensaires\ldots) complètent cela et assurent l'autonomie financière de la plupart des chapitres. Les prêtres du Grand Cercle portent d'amples robes jaune safran et le symbole de leur foi est un pendentif en forme d'anneau doré. Lorsqu'ils prononcent une bénédiction ou se signent par piété, les adeptes de cette foi ont coutume de faire un geste rapide de la main droite (ou ce qui en tient lieu) afin de former un cercle vertical.

\subsubsection{L'Église de la Dualité Cosmique}
Le principe directeur de cette foi est que toute chose possède son contraire et que rien n'est jamais vraiment "blanc" ou "noir". Les bonnes intentions et les émotions négatives cohabitent en chacun de nous et un individu n'est entier que s'il parvient à assurer une cohabitation pacifique à l'intérieur de lui-même. Dans l'absolu, aucune cause aussi valable puisse-t-elle paraitre n'est vraiment bonne ou mauvaise puisque les causes résultent des idées d'être duels dans lesquels le bien, le mal, la lumière, l'obscurité, la raison et l'instinct cohabitent plus ou moins. Le Pancréateur qui est la source de toute chose est également double par nature. Il faut se garder de voir les choses de manière tranchée et encore plus d'agir dans ce sens\ldots la Dualité Cosmique est souvent considérée comme une foi passéiste et conservatrice ou le but final est de maintenir le statu quo entre les différentes forces et pulsions qui tiraillent l'individu et les civilisations afin de préserver un état d'équilibre idéal. Comme on peut le penser, durant les milliers d'années de son existence, l'Église de la Dualité Cosmique a souvent considéré la philosophie Jedi qui sous-tendait en partie l'Ancienne République comme quelque chose de foncièrement "déséquilibrant".\\

Les prêtres de la Dualité Cosmique sont généralement vêtus de gris. Le blason du culte (souvent porté en médaillon ou sur une bague) représente un carré séparé en deux par une diagonale allant vers la droite, la partie supérieure du blason étant blanche, l'autre noire.\\

L'Église de la Dualité Cosmique n'a pas de centre administratif, bien que les congrégations des Mondes du Noyau soient celles qui y détiennent le pouvoir au sein du Conseil Exécutif. Une quinzaine de grandes congrégations assument ainsi la direction du culte par le biais de réunions semestrielles organisées sur le monde de l'une d'elles à tour de rôle. La relative proximité des mondes du Noyau rend cela aisé mais assure aussi la mainmise administrative et idéologique des congrégations dirigeantes, la plupart des responsables de congrégations plus éloignées ne pouvant assister aussi fréquemment aux sessions du conseil.
La nature même de la Dualité fait que le culte est actuellement divisé en ce qui concerne l'Empire qui est (surtout dans le Noyau) un garant de stabilité tout en étant politiquement extrêmement orienté (hors, aucune cause aussi valable soit-elle n'a vraiment d'importance selon le dogme).\\

L'Église de la Dualité Cosmique possède une multitude de temples, d'abbayes, d'écoles mais subsiste essentiellement par les dons des fidèles et quelques propriétés foncières judicieusement acquises.

\subsubsection{L'Unicité}
Cette "religion" a été il y a peu publiquement dénoncée comme une escroquerie galactique abominable. Ses "prêtres" sont tous membres de la race des T'landa t'il, des cousins lointains des Hutts. Une faculté des mâles de l'espèce leur permet de générer des ondes de plaisir intense qui affectent les femelles de leur espèce mais également (de manière bien plus efficace) la plupart des races humanoïdes de la galaxie qui développent en plus très souvent une forte dépendance à cette impulsion. En bâtissant un faux dogme et en parvenant à convaincre des gens crédules que cette sensation physique (baptisée "Exultation" pour les besoins de l'arnaque) était en fait une manifestation de la communion avec "l'Unique", les prêtres d'Ylesia eurent un grand succès durant une dizaine d'années. Des milliers de gens de toutes les espèces se précipitèrent vers Ylesia pour demeurer auprès des prêtres et "communier" chaque jour. Les Tlanda t'il firent travailler leurs adeptes dans des usines clandestines de raffinage d'épices, les récompensant à coups de séances d'exultation et de charabia mystique. Après quelques temps, la plupart des pèlerins, mal nourris et totalement dépendants de leur séance quotidienne de "communion" devenaient physiquement trop fragiles pour les travaux délicats (le raffinage de l'épice demande beaucoup de dextérité manuelle et se fait le plus souvent aux infrarouges pour éviter d'activer certaines épices sensibles à la lumière). Ils étaient alors revendus à des trafiquants d'esclaves les distribuant à divers réseaux ou l'on pouvait les faire travailler jusqu'à leur mort qui ne tardait pas à survenir par épuisement ou contrecoup du manque (les mines de Kessel et un certain nombre d'établissements "de loisirs" de la Bordure Extérieure ont selon certaines sources recruté pas mal de main d'œuvre de cette manière).\\

À la suite de rumeurs, de plusieurs raids menés par des groupes anti-esclavagistes sur Ylesia et des efforts d'anciens pèlerins parvenus à surmonter leur dépendance et à échapper à l'esclavage, les candidats à l'exultation devinrent de moins en moins nombreux et actuellement, seuls les vrais désespérés ou les gens les plus crédules des planètes reculées sont susceptibles de fournir de la main d'œuvre à ce réseau. Il existe malheureusement un grand nombre de "cultes" de taille plus réduite qui utilisent divers moyens à des fins analogues.

\subsubsection{Les Trines}
La doctrine Trine est plus une philosophie qu'une religion bien qu'elle s'appuie sur un concept mystique : le droit du sang. Officiellement, elle a complètement disparu aujourd'hui.\\

Les Trines apparurent il y a environ 1000 ans avant l'époque actuelle au sein de la noblesse galactique. Dans la presque totalité des civilisations ou ils existent, les nobles ont presque toujours à un moment ou un autre considéré qu'ils étaient la classe dominante par la grâce divine et que cela était donc l'ordre naturel des choses que cette situation perdure indéfiniment. Le mouvement Trine fut fondé par une minorité de nobles qui à l'époque considérait que ce genre d'attitude et l'inexorable décadence qui frappe toute civilisation trop stagnante ne pouvaient que provoquer des problèmes. Les Trines tentèrent de ressusciter l'idée (toujours présente dans la noblesse mais rarement appliquée) que ce pouvoir sur les masses devait s'accompagner de devoirs et de responsabilités. Selon les adeptes du Trine, la noblesse ne peut demeurer dominante que si elle remplit les rôles qui lui sont traditionnellement fixés, à savoir non seulement diriger mais aussi protéger et juger. Les Trines montrèrent aisément que la plupart des membres de leur caste avaient abandonné ces devoirs aux Chevaliers Jedi tout en continuant à user et abuser de leurs droits.\\

Le terme de "Trine" vient du fait que selon la doctrine de ces nobles, la noblesse doit se reposer sur trois vertus cardinales sans lesquelles elle n'est rien : l'Honneur, la Justice et la Compassion. Sans ces vertus, un noble n'est pas un "noble" mais juste un profiteur ou un tyran et il n'est donc pas digne de la place qui lui est échue.
\begin{description}
	\item[L'Honneur] Un noble a des devoirs et se doit d'avoir un comportement digne et intègre. On doit pouvoir se fier à sa parole et à ses promesses. Il doit être fier mais aussi modeste. Il doit honorer la mémoire de ses ancêtres et la défendre parce qu'il leur doit son statut. Que ce soit pour défendre son domaine, sa réputation ou ses sujets, le noble doit agir avec courage mais ne doit pas faire preuve ni de témérité, ni d'inconscience car cela nuirait  à son image dans le meilleur des cas et pourrait avoir des conséquences bien plus sinistres\ldots
	\item[La Justice] Un noble doit rendre justice à ceux qui sont sous son autorité et ne doit jamais oublier qu'il n'est pas non plus au-dessus des lois qu'il applique ou qu'il énonce. Il peut être un noble, il n'est pas un dieu et il n'est pas responsable de ce statut qui résulte plus des hasards de la naissance et des agissements de ses ancêtres que de ses propres accomplissements. Il doit donc agir au mieux dans l'intérêt de tous puisqu'il a été "décidé " qu'il ferait partie de la noblesse et que son titre seul ne lui donne aucune compétence ou clairvoyance particulière.
	\item[La Compassion] La dignité et la loi ne sont rien sans la connaissance des autres, de leurs besoins, de leurs souffrances, en particulier de ceux qui sont sous l'autorité du noble. Parce qu'il est né dans un milieu privilégié, le noble ne doit pas oublier que cet état n'est pas donné à la plupart et il doit veiller à défendre ceux qui sont plus démunis, moins instruits et moins bien lotis par la vie que lui. Il ne doit pas faire preuve de cruauté et se rappeler constamment qu'en cherchant à satisfaire un caprice bénin, il peut causer de la souffrance à autrui.
\end{description}

Le mouvement Trine survint durant une phase de l'histoire ou la noblesse était des plus apathiques. Il provoqua pendant près d'un siècle un renouveau d'intérêt de la part des jeunes nobles pour la politique, l'armée, les grandes œuvres sociales et les hauts faits chevaleresques. Un certain nombre d'entre eux partirent dans des croisades idéalistes, aidèrent les Jedi, distribuèrent leurs biens et plus généralement causèrent pas mal de problèmes dans les anciennes familles\ldots finalement, une bonne partie de ces jeunes gens idéalistes périt dans des aventures rocambolesques, se rangea ou finit par gouter au vin de l'amertume. Avec cela, certaines familles parmi les plus influentes n'y allèrent pas de main morte pour tenir leurs héritiers dans le droit chemin (bannissements, emprisonnements, modifications testamentaires\ldots) et en l'espace de cent cinquante ans, les Trines ne furent plus qu'un souvenir. Certaines familles de planètes comme Alderande ou Chandrila portent encore le médaillon de cristal en forme de triangle et se réunissent pour financer des œuvres charitables ainsi que pour quelques cérémonies commémorant les héros de leurs lignées mais c'est tout ce qui reste de ce mouvement.

\subsubsection{Les Cultes de Dim-U}
Cette religion est si ancienne que l'on en a oublié l'origine exacte et qu'elle a connue tant d'avatars qu'il serait impensable d'imaginer une autorité centrale unifiant cette foi. Pratiquement toutes les races comptent des adeptes du Dim-U. Selon les préceptes communs aux multiples courants de cette religion, l'univers est la création d'une entité bénigne et toutes les souffrances que supportent les mortels ne sont que le résultat (direct ou métaphysique) de leurs actions. Les adeptes du Dim-U considèrent que le créateur universel a pour animal sacré le Bantha. En effet, de toutes les espèces animales qui ont été transportées sur d'autres mondes par les colons de bien des races durant des millénaires, c'est le Bantha qui est demeuré le plus répandu et le plus utile. Le Bantha survit sous pratiquement tous les climats, son capital génétique extrêmement stable et en même temps adaptable lui épargne les mutations les plus radicales qui mènent si souvent à des impasses génétiques ou à des aberrations monstrueuses. Sa chair et son lait sont comestibles par la plupart des espèces sous une forme brute ou traitée, sa force et son endurance le rendent précieux aux colons. Il est très résistant à un grand nombre de bactéries et de toxines. Lorsqu'il est domestiqué, il est d'une douceur et d'une fidélité à toute épreuve et ses excréments eux-mêmes peuvent fertiliser des terrains très variés pour que l'on y fasse pousser bien des plantes.\\

Animal sacré de leur culte et symbole matériel de leur divinité, les adeptes du Dim-U sont comme on peut le penser très affectueux et respectueux envers les banthas. Sur le plan local, les rituels et les préceptes détaillés ainsi que l'organisation du culte varient beaucoup, allant de doux pacifistes un peu foldingues à des communautés rigides aux traditions enracinés dans de multiples textes sacrés. Pourtant, les différents cultes de Dim-U ont plus d'une chose en commun et les heurts sont presque inconnus. Tout d'abord, les adeptes du Dim-U ont tendance à observer minutieusement le comportement des banthas, domestiques ou sauvages, pour y trouver des présages et des révélations. Ils sont les meilleurs experts de la galaxie concernant ces animaux. Les communautés Dim-U ainsi que les adeptes solitaires sont presque toujours non-violents et pacifistes. Nombre d'entre eux participent activement à divers programmes de colonisation afin d'emmener avec eux leurs animaux sacrés et multiplier ainsi les mondes ou l'on trouve des Banthas. Plus des trois quarts des églises, chapitres, cultes, assemblées ou conclaves du Dim-U ont adopté le bâton de marche orné d'une sculpture représentant une tête de Bantha comme symbole sacré. \\

Sur les mondes agricoles ou les colonies dans l'installation desquelles les Banthas jouèrent un rôle essentiel, leurs adeptes sont souvent considérés comme des sages ou au minimum des gens de bon conseil. Sur les colonies vivant essentiellement du commerce ou de ressources minérales, on a tendance à les considérer comme de gentils naïfs et comme on peut s'en douter, leur foi est pratiquement absente de tous les mondes ultra-urbanisés et industrialisés. Cet état de fait et l'absence de hiérarchie centrale font que l'Empire se préoccupe rarement des cultes du Dim-U qui non seulement son plutôt passéistes mais aussi n'ont aucune influence auprès des mondes les plus importants.

\subsection{Les Empoisonneurs Malkites}
Cette fraternité d'empoisonneurs est particulière à bien des égards. En effet, contrairement à de nombreux groupes qui entrainent des spécialistes de l'empoisonnement, les Malkites louent très rarement leurs services en tant qu'organisation mais préfèrent former d'autres personnes à leur art.
Les Empoisonneurs Malkites existent depuis des millénaires et personne n'est jamais parvenu à démanteler cette organisation. Bien qu'il existe un grand nombre d'experts dans leur "spécialité" et qu'il soit incontestable que les forces impériales détiennent les meilleures armes bactériologiques et biochimiques à grande capacité de destruction, les Malkites sont réputés pour être les meilleurs à la fois dans la conception des produits et dans le perfectionnement des techniques d'utilisation.\\

Bien qu'ils soient capables de développer des armes toxiques de classe militaire, les Malkites préfèrent de beaucoup l'élaboration de toxines plus subtiles, dont les effets vont du simple malaise à la mort instantanée en passant par d'autres alternatives comme une agonie douloureuse ou un effet cyclique. Ils possèdent une palette presque infinie de poisons et développent sans cesse de nouveaux produits.
En fait, il existe deux catégories de Malkites : les Frères et les Adeptes. Les Frères Malkites sont les membres à plein temps de la fraternité malkite et leurs objectifs sont à la fois de perfectionner leurs techniques et d'assurer la survie de l'organisation. Ce sont eux qui examinent minutieusement les candidats qui souhaitent apprendre les techniques secrètes des Malkites pour pouvoir ensuite s'en servir afin de s'enrichir. Bien qu'ils fassent de juteux bénéfices en assurant la formation des candidats retenus, les Malkites se voient plutôt comme des artistes et non des commerçants.\\

Les Adeptes quant à eux sont en fait les "clients" des Malkites : ils payent très cher pour se voir enseigner les secrets malkites et selon leurs moyens, un tel apprentissage peut aller de quelques mois à plusieurs années.\\

Par la suite, les Adeptes demeurent libres d'utiliser leurs talents comme ils le souhaitent avec cependant deux restrictions importantes :\\

\begin{itemize}
	\item Il leur est interdit d'enseigner les techniques secrètes à quelqu'un d'autre. Seuls les Frères en ont le droit. Ils peuvent recommander un candidat mais les Frères sont également les seuls à décider de son admission.
	\item Lorsque les Malkites sollicitent leur aide de manière ponctuelle, les Adeptes sont tenus d'assister l'organisation du moment que cette assistance n'excède pas une durée de trois semaines standard par année calendaire. 
\end{itemize}

Afin de s'assurer que les Adeptes remplissent bien leurs obligations, les Frères Malkites disposent de plusieurs moyens qui ont prouvé leur efficacité. Entres autres, ils consentent d'importantes remises à des élèves en cours de formation pour traquer les renégats et font également appel aux Adeptes tenus par leur engagement pour éliminer leurs confrères indélicats. Les Malkites peuvent également compter sur plusieurs groupes d'assassins, chasseurs de primes et tueurs à gages qui accueillent en leur sein un Adepte et sont tout prêts à "rendre service" (sans savoir forcément à qui).\\

Dans l'imaginaire collectif, le terme d'Empoisonneurs Malkites désigne donc en fait les nombreux adeptes de toutes les races qui ont étudié auprès des maitres empoisonneurs et se sont vus remettre en guise de diplôme une trousse spéciale pour exercer leur art. Cette trousse contient à la base un échantillon assez vaste de composants permettant de fabriquer aisément un grand nombre de poisons malkites connus mais à l'efficacité éprouvée.\\

Un grand nombre d'Adeptes y ajoutent quelques composés personnels et procèdent à des recherches pour élaborer des variantes correspondant à leurs besoins ou à leurs gouts. Les plus ingénieux disposent en fait de trousses entièrement customisées mais aussi d'un bon laboratoire secret.\\

Il peut arriver qu'au lieu de solliciter les services d'un Adepte, les Frères lui demandent de partager ses découvertes avec eux, ce qui peut leur permettre si elles sont vraiment intéressantes de perfectionner leur art ainsi que la formation des futurs adeptes.\\

On sait très peu de choses sur la formation à laquelle participent les Adeptes des Malkites. Tous sont amenés en secret sur un monde où se déroule leur formation. Les Malkites peuvent également dispenser des enseignements plus martiaux (combat à mains nues ou à l'arme blanche, techniques d'infiltration\ldots) mais ce type de cursus est très onéreux et toujours dispensé en complément de la formation "de base" qui vise à produire des spécialistes es poisons.\\

On raconte également que durant leur entrainement, les candidats sont soumis à plusieurs épreuves afin de tester leur vigilance et leurs capacités à identifier les toxines et leurs antidotes. La philosophie malkite repose sur un principe simple : celui qui survit à plusieurs tentatives d'empoisonnement par sa vigilance ou par sa maitrise des secrets malkites est le plus à même de devenir un empoisonneur de première classe\ldots\\

La planète où résident les Frères est appelée Malkii mais il est très probable qu'il s'agisse d'une fausse appellation dissimulant un monde possédant une autre dénomination officielle. Au cours des siècles, quelques Adeptes (qui ont tous rapidement péri par la suite) ont parfois révélé des bribes d'informations sur ce monde. On sait à partir des constellations visibles dans le ciel qu'il se trouve très certainement dans la région des Colonies. Il possède une lune, son soleil est de couleur blanc-jaune et le relief  de la région ou se déroule l'entrainement des Adeptes est fait de collines venteuses et de massifs forestiers.\\

Ces quelques détails permettent de déterminer avec certitude la localisation du monde en question, à condition de savoir lequel choisir parmi les 1145 planètes situées dans cette région de l'espace et dont les conditions locales correspondent à celles de Malkii.\\

La nature du centre d'entrainement reste également mal définie puisque l'on a évoqué d'anciennes ruines, des monastères, des camps de plein air, des galeries souterraines et même des faubourgs industriels.\\
Il se peut qu'il existe plusieurs "planète Malkii" ou même que l'organisation soit en fait mobile et déménage constamment d'un monde à l'autre parmi les 1145 susceptibles de l'accueillir. La nature de la formation, le peu de ressources fixes nécessaires et le fait que les effectifs d'une "promotion" d'Adeptes dépassent rarement la douzaine de personnes pour environ moitié moins de Frères chargés de l'encadrement rend cette possibilité nettement plus vraisemblable que si l'on s'intéressait à une structure de type criminelle ou terroriste (rebelle\ldots) classique.\\

À l'issue de leur formation, les adeptes qui ont survécu sont à nouveau transportés loin de "Malkii", se voient remettre leur trousse d'empoisonneur et sont individuellement déposés dans la campagne ou sur un spatioport discret.\\

La rumeur veut également qu'après avoir laissé les Adeptes agir à leur guise pendant plusieurs années, les Frères accordent plus d'attention à ceux qui se montrent les plus créatifs, afin de leur offrir ensuite la possibilité de rejoindre leurs rangs. On ignore tout des promesses qui sont faites à ces élus ou des avantages qui vont avec leur nouveau statut mais il est facile de deviner, ou au moins de supposer, ce qui attend ceux qui dédaignent l'offre qui leur est faite\ldots

\subsection{Le Réseau Action Justice}
Cette organisation qui vit le jour quatre ans après la fondation de l'Empire incarne parfaitement ce que la propagande impériale met sous l'étiquette "terrorisme rebelle". Les membres du Réseau sont des purs et durs qui considèrent que l'Alliance est trop "modérée" car elle ne s'en prend qu'à des cibles militaires.
La doctrine du Réseau Action Justice est la suivante : l'Empire œuvre afin de persuader les masses qu'en lui obéissant elles seront en sécurité alors qu'en se rebellant elles risquent une répression féroce. Donc, la peur l'emporte sur la raison et les peuples de la galaxie ont échangé leur liberté contre "la sécurité" parce qu'ils redoutent que les choses puissent être pires.\\

Le but du Réseau est donc de montrer que l'Empire malgré tous ses efforts ne peut pas garantir la sécurité dans la galaxie. Si ni les rafles, ni les purges, ni les bombardements ne peuvent empêcher des gens résolus de continuer à frapper l'Empire, alors les populations finiront par se rendre compte qu'en se laissant persécuter, elles n'obtiennent rien si ce n'est d'autres persécutions.\\

La méthode d'action du Réseau est donc aussi radicale que son idéal : en semant la terreur, il espère amener les gens à ne plus redouter la puissance de l'Empire mais celle du Réseau qui peut continuer à frapper à sa guise malgré la répression. À choisir leur camp. Une fois qu'il sera devenu évident pour tout le monde que la répression ne fait qu'ajouter des victimes de plus et retarder l'inévitable révolte contre l'Empire, les populations finiront par agir dans l'espoir de mettre à bas le régime impérial, pour retrouver ainsi la paix et la liberté.\\

Comme toute organisation terroriste qui se respecte, le Réseau est organisé en cellules compartimentées et possède de nombreuses caches d'armes et sympathisants. L'organisation a vu le jour dans les Colonies et possède donc des branches dans les régions centrales de la galaxie mais très peu de membres au-delà de la Bordure Médiane. Sur certains mondes pro-impériaux, elle est même la seule véritable organisation rebelle et l'Empire n'hésite pas pour des raisons de propagande à attribuer ses méfaits à l'Alliance pour mieux démontrer que la Rébellion n'est bel et bien qu'un ramassis de tueurs impitoyables.\\

Les membres du RAJ préfèrent agir de manière aussi violente et spectaculaire que possible et là ou un commando de l'Alliance ciblerait un gouverneur ou une base militaire, ils n'hésitent pas à faire sauter des bâtiments publics ou à mitrailler des fonctionnaires impériaux. Les attentats-suicides dans les transports en commun sont également au nombre de leurs méthodes. Leur but est bel et bien d'inspirer aux populations une terreur encore plus grande que celle de l'Empire afin de montrer aux gens à quel point leur soumission est sans objet puisqu'elle ne leur garantit pas la sécurité.\\

Comme on s'en doute, le RAJ n'a que peu de sympathisants sur les mondes ou il opère. Personne ne souhaite sauter en même temps que des soldats impériaux sous prétexte qu'on mange tous les jours dans le même restaurant. Les agents du Réseau sont de véritables forcenés, ultra-entrainés et motivés qui feraient des recrues de choix pour l'Alliance s'ils n'étaient pas aussi radicaux et fanatisés. Les rapports entre le Réseau et l'Alliance n'ont d'ailleurs jamais été bons et il est même arrivé occasionnellement que l'Alliance juge nécessaire de neutraliser une commande du RAJ plutôt que de le laisser commettre une atrocité que la propagande impériale pourrait récupérer à son compte.

\subsection{Le Bha'lir Noir}
Les origines exactes de cette organisation criminelle sont assez mal connues et l'on ne peut que s'en référer à la légende que colportent ses membres.
Il y a des millénaires, une association de contrebandiers corelliens décida d'adopter comme mascotte le félin connu sous le nom de Bha'lir, un animal à la fois féroce quand il chasse et calme quand on le laisse tranquille. Ces individus finirent par établir un port secret sur une petite planète de la Bordure Extérieure ou ils nouèrent des contacts fructueux avec les indigènes. Les habitants de ce monde, Socorro, possédaient eux aussi des racines corelliennes et une éthique basée sur une forme d'honneur et de respect mutuel extrêmement développée.\\

C'est l'association des contrebandiers des étoiles et de leurs cousins qui errent encore dans les déserts de sable noir de Socorro qui fit d'un petit cartel minable une organisation réputée dans toute la galaxie. Réputée à la fois pour sa puissance qui s'étend sur plus de soixante-dix systèmes stellaires et pour son code de conduite rigide qui lui donne une respectabilité que bien peu d'autres contrebandiers peuvent prétendre égaler.\\

Fortement inspiré par les Socorrans, le Bha'lir Noir est tout entier imprégné de son code de conduite qui comprend quatre points principaux :

\begin{description}
	\item[Le Bha'lir Noir ne profite pas de la faiblesse des autres] On leur fournit ce qu'ils veulent et on doit le leur faire payer au juste prix. Les contrebandiers travaillent pour le bénéfice et tout travail risqué mérite une récompense appropriée. Mais on ne "gonfle" pas les prix en inventant des complications inexistantes ou en faisant exprès de s'attirer des problèmes. On conclut un accord et on s'y tient.
	\item[Pas de groupe sans échanges, pas d'échanges sans réciprocité] Tout ce qui est donné doit pouvoir être rendu. Celui qui offre son aide doit pouvoir réclamer de l'aide en retour. Ainsi, les membres du Bha'lir s'entraident et veillent les uns sur les autres. De même, ils n'oublient jamais leurs amis en dehors de l'organisation.
	\item[La parole d'un homme passe par la Voie de la Rétribution] Celui qui trompe un membre du Bha'lir doit être puni. Tout membre du Bha'lir qui trahit un des quatre points du code doit également être puni. Le Bha'lir Noir n'oublie jamais ses ennemis et ceux qui ont cessé d'être ses amis. Les ma\^{i}tres de l'organisation décident du châtiment approprié et c'est une preuve d'honneur et de respect envers l'organisation que de se porter volontaire pour exécuter la sentence.
	\item[Le respect que l'on obtient dépend du respect que l'on donne] La violence est inutile quand la menace suffit, la menace inutile quand le dialogue suffit et le dialogue inutile quand l'autre a compris tout seul ce qu'il devait faire. Il faut toujours respecter l'autre en fonction du respect qu'il vous accorde. Cela ne veut pas dire se montrer naïf ou crédule mais simplement qu'il faut agir de manière déterminée : il faut parfois faire des choses terribles mais uniquement parce qu'il n'y a pas d'autre moyen de procéder. Les envies personnelles et le plaisir n'ont rien à voir avec la manière dont on fait les choses.
\end{description}


Avec de tels principes, le Bha'lir Noir s'est attiré de nombreux ennemis parmi les groupes criminels : le Soleil Noir, les Kajidics Hutts et la plupart des groupes pirates ou esclavagistes n'ont que faire de contrebandiers qui prétendent hypocritement veiller à défendre leur "honneur" et à préserver celui de leurs partenaires.\\

Le fait est que les membres du Bha'lir Noir prennent très au sérieux leur code. Tous les membres qui trahissent un de ces quatre principes est susceptible d'être traqué impitoyablement par les autres et de se voir châtié selon les exigences du Tribunal, l'instance suprême de l'organisation. Le Bha'lir Noir gère ses propres problèmes de discipline et contrairement à la conception répandue selon laquelle les voleurs n'ont pas d'honneur, les autorités de bien des mondes ont appris depuis longtemps que la parole d'un émissaire de l'organisation avait bel et bien de la valeur.\\

Le Bha'lir Noir n'est pas une organisation charitable, ni pacifiste. Ses membres ne sont pas des philanthropes et ne font ni crédit, ni cadeau. Leur principe fondateur est simple. Quand on vit dans le crime, il n'existe que trois manières de s'en sortir : être plus fourbe que les autres, être plus fort que les autres ou être plus digne de confiance que les autres.\\

Le Bha'lir Noir ne veut pas rivaliser en fourberie ou en force avec ses concurrents. Tout au moins, il garde une bonne partie de sa fourberie et de sa force pour les occasions ou on souhaite lui nuire. Mais c'est parce que les membres de l'organisation savent qu'ils peuvent compter les uns sur les autres qu'ils connaissent un succès qui fait bien des envieux.\\

Le Bha'lir Noir s'est montré peu sensible aux changements politiques dans l'histoire récente de la Galaxie. Empire ou République, il y a toujours des choses qu'il est interdit de transporter. Donc, il y a toujours du travail pour les vaisseaux du Bha'lir Noir.\\

Les activités de l'organisation tournent essentiellement autour du transport illégal de cargaisons interdites ou théoriquement frappées de lourdes taxes. Le Bha'lir Noir possède également plusieurs établissements de jeu, des organismes de crédit et diverses sociétés écrans qui lui permettent de blanchir l'argent de ses gains illégaux et d'y mêler des bénéfices tout à fait propres. Une partie importante de ces bénéfices ne revient pas à l'organisation mais est directement reversée à divers hauts responsables des autorités en place dans les systèmes qui intéressent l'organisation, afin de "huiler les rouages". Celle-ci a même des intérêts au cœur de l'Espace Hutt, malgré l'inimitié féroce qui l'oppose aux seigneurs du crime venus de Nar Shadda.\\

Il y a peu d'activités que le Bha'lir Noir refuse d'envisager : l'esclavage, la prostitution, le trafic de drogues dures, l'assassinat sur commande sont incompatibles avec les traditions qui découlent du Code. Le but est de fournir aux autres quelque chose contre lequel ils sont prêts à verser beaucoup d'argent. Pas de les transformer en bêtes de somme ou les réduire à la ruine ou à la dépendance afin de pouvoir les vendre ainsi que leurs familles.\\

\subsection{Les Derviches Seyugi}
Six cent ans avant la fin de l'Ancienne République, plusieurs mondes du Noyau furent terrorisés par les mystérieux assassins aux manteaux rouges qui se faisaient appeler les Derviches Seyugi. Les cultes, guildes et sectes d'assassins avaient toujours fleuri parmi les anciennes et souvent décadentes civilisations du Noyau mais les Seyugi faisaient partie de cette minorité de groupes visiblement dotés de pouvoirs occultes. De fait, les Derviches possédaient une aptitude limitée envers la Force qu'ils avaient entièrement focalisée sur le mouvement et le corps à corps. Dépourvus des pouvoirs sensoriels ou télékinétiques des Jedi ou des adeptes du Côté Obscur, les Seyugi étaient par contre devenus de véritables machines à tuer dont les redoutables aptitudes aux arts martiaux étaient démultipliées par leur usage spécialisé de la Force.\\

Un Derviche au sommet de son art et armé d'une simple vibro-lame pouvait aisément anéantir une escouade de soldats entra\^{i}nés armés de blasters s'il parvenait à arriver au contact et même les meilleurs lutteurs ou spécialistes du close-combat ne pouvaient espérer vaincre un Seyugi combattant à mains nues.
N'éprouvant aucun intérêt pour les considérations mystiques et fondamentalement désireux d'amasser des crédits, les Seyugi se firent rapidement conna\^{i}tre comme des tueurs à gages particulièrement chers mais aussi particulièrement efficaces puisqu'un seul homme infiltré dans un navire ou un immeuble et dépourvu d'armes pouvait aisément éliminer tous les autres occupants. On ne sut jamais qui avait fondé cette secte ni d'où elle était originaire mais elle suscita beaucoup de peur et d'animosité.\\

Comme il en va souvent avec les groupes dotés d'aptitudes à la Force, les Seyugi finirent par poser suffisamment de problèmes pour qu'à l'issue de quelques incidents isolés l'Ordre Jedi finisse par agir. Ils furent alors impitoyablement traqués et capturés ou éliminés.\\

Un chapitre des Seyugi basé sur la planète Recopia dans le Noyau comprit que les Jedi utilisaient la Force pour les repérer lorsqu'ils se servaient de leurs pouvoirs. Désespérés, les chefs de ce groupe qui comptait encore un bon millier d'adeptes dissimulèrent leur petite forteresse en se faisant passer pour de simples moines d'un ordre contemplatif. Quelques dizaines de Seyugi parmi les novices furent chargés de veiller à maintenir cette mascarade pendant que les centaines d'autres étaient placés en congélation carbonique et dissimulés dans un caveau secret sous le monastère dont l'architecture fut rapidement modifiée.
Quelques pots de vins et menaces plus tard, on avait oublié jusqu'à la présence d'un fortin Seyugi sur Recopia et près d'un siècle après leurs premières apparitions publiques, les Derviches Seyugi avaient cessé d'exister.\\

Choisir des disciples encore novices pour dissimuler le secret des Seyugi fut une erreur car pendant plusieurs générations, ces hommes durent s'immerger totalement dans leur rôle, attendant patiemment que les Jedi les oublient. Les archives et manuels des Derviches furent dissimulés dans des alcôves secrètes et les novices eurent de plus en plus de mal à transmettre ce qu'ils savaient au fur et à mesure que les plus anciens périssaient sans révéler ou se trouvaient certaines alcôves. Leur fausse religion attira des croyants en tous genres et finit par devenir une réalité pour les faux moines.\\

Finalement, lorsque Palpatine prit le pouvoir les descendants des Seyugi avaient oublié leurs origines et la présence de leurs frères congelés. Le petit monastère de Mallif Cove abritait désormais un groupe de moines excentriques, parlant à profusion des "fluctuations dans le champ d'énergie universel" et aux rituels aussi bizarres que futiles. Les Inquisiteurs Impériaux ne trouvèrent aucune trace de la Force parmi les moines vêtus de vert dont les croyances s'avérèrent inoffensives car ils n'avaient même pas d'influence sur les autres communautés de leur planète. Plusieurs descentes eurent lieu et certains moines furent accusés de sympathies rebelles et exécutés mais personne ne jugea nécessaire d'éradiquer un mouvement aussi ridicule par sa taille que par ses croyances.\\

Les moines de Mallif Cove poursuivirent leur vie comme à l'accoutumée et personne n'imagina jamais que sous leurs pieds dormaient depuis cinq siècles un bon millier de tueurs d'élite adeptes de la Force.\\

Attendant patiemment dans le froid de la carbonite que quelqu'un les réveille\ldots

\clearpage

\appendix

\section{Les Notes Personnelles de Pénombre sur la Chronologie}

D'une façon générale, j'ai conservé l'étendue du calendrier même si j'ai souvent trouvé le gigantisme de Star Wars un peu abusif à cet égard (par exemple : la galaxie découvre les blasters portables il y a 25.000 ans et malgré toutes les espèces et races intelligentes, malgré toutes les guerres qui s'écoulent dans l'intervalle,  les armes à feu individuelles sont sensiblement les mêmes au moment des films...)

\subsection{La Technologie Hyperspatiale}
Par rapport aux chronologies approuvées par Lucas, je me suis permis une divergence majeure à ce sujet.
Si on se fie aux sources approuvées et aux cartes galactiques "officielles", le fait que les Hutts et Xym le Despote aient pu s'affronter (- 27.000 avant l'Ere Impériale) est curieux vu qu'aucun des deux ne possédait le vol plus rapide que la lumière - découvert presque 2000 ans après leur conflit - alors que leurs empires respectifs sont séparés par des milliers d'années lumières - les Hutts de l'époque contrôlant déjà leur espace actuel et le berceau de l'empire de Xym se trouvant dans l'Hégémonie de Tion, dans les parages de l'actuel Secteur Corporatif... 
Par contre, les anciennes races brièvement mentionnées au tout début de la chronologie possédaient le secret des Portes Stellaires d'après le Star Wars Adventure Journal et on peut concevoir que Xym et les Hutts aient utilisé les dernières reliques de cette technologie sans vraiment la comprendre...tout au moins, c'est la seule explication qui me semble pouvoir être insérée dans le reste du contexte... le temps que les moteurs hyperspatiaux soient d'usage courant et que la République voie le jour, les dernières Portes ont été détruites ou sont tombées en panne ce qui explique pourquoi elles furent oubliées et pourrait également expliquer le fait que les humains apparemment originaires du Noyau aient aussi  des civilisations tout aussi anciennes dans l'Hégémonie de Tion ainsi que divers autres coins de la galaxie. En fait, avant l'hyperpropulsion, il est fort probable que Xym utilisa  les rares Portes à sa disposition pour aller dans de nouveaux secteurs (il semble d'après l'Adventure Journal que les Portes avaient des destinations fixes pré-programmées et non modifiables) et de là construisaient des bases avancées et des vaisseaux subluminiques pour s'emparer des étoiles les plus proches....
Dernier détail : les corelliens ne sont PAS les inventeurs de la technologie hyperspatiale, il s'agit d'une race extragalactique de passage qui leur a vendu ce secret avant de repartir et de retomber dans l'oubli. Mais bon, nous parlons bien de la chronologie approuvée par l'Empire...

\subsection{Les Anciennes Races }
Je reviendrai dessus plus tard mais pour l'instant, il suffit de dire que d'après les sources dont je dispose, on peut en identifier au moins trois qui existent encore à l'époque des films : les Gree, les Columni et les Biths. Toutes trois sont pratiquement exsangues et l'étaient apparemment déjà  avant la naissance de la République. Les Gree connaissaient le secret des Portes Stellaires et on peut supposer que les autres aussi. \\

Dans la chronologie officielle, l'évolution des Hutts et des Humains semble simultanée alors que les Hutts sont souvent par ailleurs mentionnés  comme étant une race relativement ancienne très méprisante envers les espèces plus jeunes actuellement sur le devant de la scène, notamment les humains... j'ai donc modifié les dates et raccourcie l'ancienneté de l'humanité qui passe de 1 million à 500.000 ans. Les Hutts sont donc plus jeunes que les "anciennes races" mais leur évolution est antérieure à celle des humains et de nombreuses autres espèces. 

\subsection{L'Espèce Humaine}
Son statut et le mystère de ses origines (non résolu) feront l'objet d'une page spéciale dans la rubrique à venir sur les races intelligentes de la galaxie.  

\subsection{Les Siths}
Bien qu'ils aient eu un impact considérable dans l'histoire galactique, je ne les ai pas mentionnés dans la chronologie impériale pour deux raisons essentielles : 
\begin{itemize}
	\item bien que Sith lui-même, Palpatine se moque bien que la galaxie sache de quoi il retourne du moment qu'elle lui obéit. De plus, la Force n'a aucune existence officiellement reconnue par l'Empire et les propagandistes s'acharnent à effacer toute trace de son existence, alors parler d'un ordre mystique voué au mal...
	\item l'ordre Jedi doit être déconsidéré à tout prix parce qu'il y a encore des milliards de gens assez vieux pour avoir rencontré un Chevalier avant leur extermination. Pour ce faire, il vaut mieux tordre les évènements afin de le présenter tout au long de l'histoire comme une organisation divisée et faillible plutôt que comme l'ennemi d'un ordre voué à la conquête, les Sith... ainsi, les Jedi apparaissent à la fois comme des protecteurs mais aussi des gens corruptibles, longtemps divisés par des querelles internes qui ont rejailli sur toute la galaxie et ont fait des centaines de millions de victimes... de cette manière, même leur rôle dans les Guerres Cloniques (évènement encore très présent dans les mémoires) peut-être réécrit comme il convient à Palpatine : un ordre corrompu comme la République qui n'a pas hésité à sacrifier ses éléments les plus intègres afin de maintenir son pouvoir encore quelques décades, jusqu'à l'arrivée du sauveur en personne : l'Empereur. 
\end{itemize}


\subsection{Les Clones dans la Galaxie}
Si l'on se fie aux romans, les Guerres Cloniques ont été un véritable traumatisme et ont notamment provoqué un fort mouvement d'hostilité envers les êtres clonés et les cloneurs. Personnellement, j'ai tendance à considérer les choses de la manière suivante : un conflit à l'échelle de la Galaxie demande d'énormes ressources et rapidement, autant la République que la Confédération durent faire face à des besoins gigantesques. Finalement, les deux adversaires multiplièrent les centres de clonage afin d'obtenir en quelques semaines des légions entières de soldats. Les critères de sélection des "donneurs génétiques", les aptitudes plus ou moins limitées des cloneurs dépassés par la demande, les copies artisanales de leur technologie et les pressions économiques et technologiques inhérentes à une production de masse en temps de guerre ont certainement mené à la conception de séries de clones plus ou moins défectueuses, instables, voire même rebelles.\\

Il y eut donc non seulement des massacres perpétrés par des soldats clonés dans des conflits délimités (vous savez, les "pertes collatérales inévitables"...) mais surtout des mutineries ou des boucheries réalisées par des groupes défectueux.  Il y eut donc certainement des purges à l'encontre des cloneurs, les populations civiles étant horrifiées par la perspective de voir leurs propres armées se livrer à des atrocités contre elles.\\

Dans les films et même une bonne partie des romans, l'origine exacte des Stormtroopers impériaux demeure obscure mais de fortes présomptions donnent à penser qu'il s'agit également de clones (bien qu'on fasse plusieurs références à l'endoctrinement forcé sur Carida et ailleurs de jeunes gens réquisitionnés par l'Empire). Chaque MJ est libre de voir comment il considère la chose mais là encore, mon idée personnelle est la suivante : Palpatine a effectivement des armées de clones à sa disposition. Il a su durant le conflit récupérer les meilleurs cloneurs tandis que les autres étaient pourchassés par les foules et il a probablement même contribué à leur extermination afin de conserver le monopole de cette technologie. Sa cadence de production n'est certainement pas aussi élevée qu'elle pourrait l'être mais il doit s'assurer que tous les clones lui sont parfaitement dévoués et ne sont pas défectueux. Ils sont réalisés non pas à partir d'un donneur unique mais de plusieurs centaines de donneurs afin d'augmenter leur diversité et leur adaptabilité tout en conservant leur nature de pièces organiques interchangeables.\\

Ses armées clonées instillent la terreur dans l'esprit des citoyens sans qu'il soit besoin de révéler leur vraie nature mais de nombreux individus ont des doutes ou même des preuves concernant la nature des Stormtroopers.\\

Sous l'ancienne république, de telles preuves auraient certainement amené à interdire que la production des stormtroopers continue mais nous sommes sous l'Empire... le Sénat est bâillonné et ses membres déjà bien trop occupés à empêcher qu'on annexe purement et simplement leurs mondes pour tenter de partir en croisade contre les armées même qui peuvent débarquer pour les anéantir et qui ont permis de restaurer "la paix". Les médias sont censurés, les importuns déportés ou exécutés par des gens qui eux ne sont pas des clones. Palpatine sait qu'il peut compter sur un tas de serviteurs  pour veiller à ce que la rumeur lui donne encore plus de pouvoir que la vérité.\\

Car dans le fond, ceux qui sont au courant savent bien que sans ces clones, sans la dictature impériale, la galaxie vivrait encore dans l'anarchie la plus totale. Mieux vaut des bavures et des génocides dans la Bordure Extérieure, mieux vaut des races non-humaines déportées et exterminées qu'une nouvelle guerre civile dans les systèmes "civilisés". Mieux vaut la paix du tyran que la guerre... dans le fond, n'est-il pas préférable que les mondes du Noyau et des autres régions "importantes" qui ont tant souffert des Guerres Cloniques profitent d'une certaine prospérité même si le prix en est l'oppression généralisée et la violence des forces impériales dans les régions périphériques ?  Les stormtroopers sont peut-être des clones mais ils font régner l'ordre. Et l'ordre doit continuer à régner si l'on veut éviter de nouvelles atrocités. Quant à ceux qui refusent cette réalité, finalement, ils n'ont que ce qu'ils méritent.\\

Voilà comment les faibles, les hypocrites et les ambitieux pourraient justifier leur soutien à l'Ordre Nouveau et leur silence sur bien des choses... y compris sur les clones.

\end{document}
