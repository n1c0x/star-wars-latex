\documentclass{article}
\usepackage[french]{babel}
\usepackage[T1]{fontenc}
\usepackage[left=2cm,right=2cm,top=2cm,bottom=2cm]{geometry}
\usepackage{fancyhdr}
\pagestyle{fancy}
\chead{One-Roll Towns}
\usepackage{multicol}

\usepackage{fontspec}
\defaultfontfeatures{Ligatures=TeX}
\setmainfont[Mapping=tex-text]{Sitka Display}
\usepackage[small,sf,bf]{titlesec}

\usepackage{graphicx}
\usepackage{xcolor}
\usepackage{sectsty}

\definecolor{DarkGreen}{HTML}{384d3e}
\definecolor{PureWhite}{HTML}{FFFFFF}
\definecolor{DarkRed}{HTML}{6e272d}
\definecolor{DarkGold}{HTML}{a48e3b}

\sectionfont{\color{DarkGreen}}
\subsectionfont{\color{DarkRed}}
\subsubsectionfont{\color{DarkGold}}

\begin{document}

\title{\vspace{-0.5cm}{\Huge One-Roll Towns} \vspace{-1cm}}

\date{}

\maketitle

\section*{Utilisation}
Tout d'abord, vous pouvez n’imprimer que la page suivante. Cette page n'est pas nécessaire pour jouer.
Ce système est excellent pour générer des "Villes à la volée". Les deux parties de cette phrase sont importantes :

\begin{itemize}
	\item il n’est pas utilisables pour les bourgs (qui sont moins utiles dans une aventure car ils manquent de nourriture et de logement), mais il peut être étendu facilement comme décrit dans la section \textbf{Taille} ci-dessous. Pour une petite structure comme un bourg, il peut être plus approprié de ne lancer que quelques d6 comme détaillé dans la section \textbf{Population}.
	\item "à la volée" décrit l'autre élément important de ce système ; il est conçu pour créer une ville là où il n'y en a pas. Il peut aider à concevoir des endroits importants où le groupe est destiné à aller, mais c'est surtout pour créer ces endroits à le groupe se rend. Idéalement, il répond à toutes les questions d'un groupe qui ne fait que passer, rien de plus. Pour un regard plus approfondi sur la ville parce que le groupe NE PARTIRA PAS, voir la section \textbf{Social et politique}.
\end{itemize}

\section*{Systèmes additionnels}

\subsection*{Taille}
Pour les colonies plus importantes, lancez à nouveau les dés. Avec plusieurs jets de dés --- ou plusieurs ensembles de dés --- un bourg devient un donjon, et un donjon devient une ville. Au fur et à mesure de sa croissance, un magasin de vêtements peut devenir un quartier de l'habillement d'une grande ville.

\subsection*{Population}
Pour cartographier des détails supplémentaires sur la population de la ville, y compris l'emplacement des maisons et des informations sur les habitants, on peut utiliser des D6 standards. Ils peuvent être achetés en blocs de 36d6 et plus, et sont également utiles comme jetons représentant des ennemis, etc. En déposant une poignée sur votre carte après avoir documenté les principales caractéristiques de la ville, vous obtiendrez le plan des maisons et le nombre d'habitants de chacune. Une bonne règle empirique est \textbf{12d6} pour un bourg, \textbf{24d6} pour un donjon et \textbf{36d6} pour une ville.

\subsection*{Social et politique}
Pour beaucoup de bourgs, le groupe ne fait que passer. Ils ne se soucient pas de son histoire ou de sa gouvernance, et seulement à l'occasion ils s'enquièrent des défenses de la Ville. Pour cela, j'ai écrit un outil séparé utilisant le moteur Blunderbuss.

\section*{Conclusion}
Un système élégant pour aider les MJ à générer un Steading, qui répond aux questions les plus courantes des joueurs, par ordre de priorité.

\begin{description}
	\item [Utilisation par les MJ]Le système évitera de supplanter les paramètres communément déterminés par le MJ en ce qui concerne le cadre qu'il imagine pour l'aventure. Par conséquent, bien que de nombreux joueurs s'enquièrent immédiatement de la composition raciale d'une ville, elle a été omise de cet outil afin de laisser cela entre les mains du MJ.
	\item [Elégance] Peu de dés, résultats faciles à lire, faciles à transporter et à documenter.
	\item [Questions fréquentes] Où puis-je trouver de la nourriture ? Où puis-je me reposer ? Quelle est la taille de la ville ? Quelles sont les races ici ?
	\item [Priorité] Les joueurs veulent généralement connaître d'abord le plan, puis où trouver de la nourriture et un logement, puis d'autres endroits intéressants en ville, puis passer aux questions de défense et de caractéristiques sociopolitiques. 
\end{description}

\clearpage


Un système simple pour générer des villes à la volée. Lancez un jeu de dés standard (d4, d6, d8, 2d10, d12, d20) tous en même temps. Sur une feuille de papier vierge, tracez les emplacements des dés exactement aux endroits où ils sont tombés. Une fois que vous avez entourés les dés, trouvez chaque valeur correspondante sur les tables suivantes et placez-les sur votre carte là où le dé est tombé.  Outils supplémentaires à la page 3.


\begin{multicols}{2}
	\section*{D4 -- Grand-place}
	\begin{enumerate}
		\item Puits
		\item Feu de joie
		\item Marché libre/Bazar
		\item Statue ou sanctuaire
	\end{enumerate}
	\section*{D8 -- Nourriture}
	\begin{enumerate}
		\item Gibier/poisson sauvage abondant
		\item Citoyen généreux avec de la nourriture
		\item Jardin potager public
		\item Repas - partage communautaire 
		\item Barbecue à fosse ouverte
		\item Fumoir
		\item Marché en ligne
		\item Taverne
	\end{enumerate}
	\section*{2D10 -- Magasins}
	\begin{enumerate}
		\item Magasin général
		\item Alchimiste/Herboriste/Médecin
		\item Forgeron (Armure, Armes, Outils)
		\item Charpentier (bateaux, bâtiments, chariots)
		\item Vêtements (communs, fins)
		\item Enchanteur 
		\item Souffleur de verre
		\item Maroquinerie (Armure, Sellerie)
		\item Ecuries
		\item Produits exotiques (Tapis \& Tissus, Bijoux, Parfums)
	\end{enumerate}
	\section*{D6 -- Hébergement}
	\begin{enumerate}
		\item Déboisement relativement sécuritaire près de la ville
		\item Citoyen généreux avec de la place
		\item Camping en plein air
		\item Bâtiment de rechange (grange, maison vide)
		\item Pavillon Communal
		\item Auberge
	\end{enumerate}
	\section*{D12 -- Economie}
	\begin{enumerate}
		\item Carrefour commercial
		\item Agriculture
		\item Bétail
		\item Docks/Ports
		\item Ferry/Pont Majeur
		\item Pêche
		\item Lieu sacré/Source de pouvoir magique
		\item Moulin
		\item Mine
		\item Carrefour commercial
		\item Défense (caserne, défense d'un emplacement stratégique ou d'une route, poste de garde, entraînement)
		\item Industrie de production (chantiers navals, sidérurgie)
	\end{enumerate}
	\section*{D20 -- Bâtiments supplémentaires}
	\begin{enumerate}
		\item Tour de Sorcier (Actif, Abandonné)
		\item Université (Arcane, Barde, Érudit)
		\item École d'instruction au combat
		\item Église / Sanctuaire / Temple
		\item Fosse de combat
		\item Jardins suspendus
		\item Hall de guilde (Artisanat, Commerce, Combat, Voleurs)
		\item Bibliothèque/dépôt de connaissances
		\item Phare ou tour de guet
		\item Hippodrome (chiens, chevaux)
		\item Ruines (Château, cathédrale, Ports)
		\item Entrée de grotte scellée
		\item Conséquences d’escarmouche (ville voisine, horde envahissante, bête en furie)
		\item Loge Spirituelle
		\item Équipements permanents de châtiments corporels (potence, gibet, billot)
		\item Théâtre/Amphithéâtre
		\item Une rivière la traverse
		\item Intégrée dans une colline ou un versant d'une montagne
		\item Construite dans un Canyon ou un Ravin
		\item Entourée de forêt et de nature sauvage
	\end{enumerate}
\end{multicols}


\end{document}
