\documentclass{article}
\usepackage[french]{babel}
\usepackage[T1]{fontenc}
\usepackage[left=2cm,right=2cm,top=2cm,bottom=2cm]{geometry}
\usepackage{fancyhdr}
\pagestyle{fancy}
\chead{Désastres}
\usepackage{multicol}

\usepackage{fontspec}
\defaultfontfeatures{Ligatures=TeX}
%\setmainfont[Mapping=tex-text]{Sitka Display}
\usepackage[small,sf,bf]{titlesec}

\usepackage{graphicx}
\usepackage{xcolor, colortbl}
\usepackage{sectsty}

\definecolor{DarkGreen}{HTML}{384d3e}
\definecolor{PureWhite}{HTML}{FFFFFF}
\definecolor{DarkRed}{HTML}{6e272d}
\definecolor{DarkGold}{HTML}{a48e3b}

\sectionfont{\color{DarkGreen}}
\subsectionfont{\color{DarkRed}}
\subsubsectionfont{\color{DarkGold}}

\begin{document}

\title{\vspace{-0.5cm}{\Huge Désastres} \vspace{-1cm}}

\date{}

\maketitle

\renewcommand{\arraystretch}{1.4}

\begin{center}
	\begin{tabular}{|p{1.5cm}|p{15cm}|}
		\hline 
		\cellcolor{DarkRed} {\large \textcolor{PureWhite}{\textbf{Roll}}} & \cellcolor{DarkRed} {\large \textcolor{PureWhite}{\textbf{Despair}}} \\
		\hline 
		01 -- 10 & L’arme est à court de munitions. \\
		\hline
		11 -- 20 & Changement environnemental entraînant un dé/plusieurs dés de malus, ex. : les lumières s’éteignent, une tempête de sable soudaine, perte de gravité, un tremblement de terre, de la fumée. \\
		\hline
		21 -- 30 & Ajouter un sbire supplémentaire au combat. \\
		\hline
		31 -- 40 & Le PJ trébuche et fait tomber son arme, qui atterrit aux pieds d’un ennemi hors de sa portée. \\
		\hline
		41 -- 50 & L’arme s’enraye, les munitions doivent être remplacées, l’arme subit des dégâts (briseur 1). \\
		\hline
		51 -- 60 & Ajouter un nouveau groupe de sbires au combat. \\
		\hline
		61 -- 65 & Changement environnemental extrême entraînant des dégâts aux PJ (test de Résilience pour annuler/réduire l’effet), ex. : une explosion, un incendie soudain, un éboulement, l’effondrement d’une plateforme/d’un bâtiment, une décompression brutale de l’atmosphère d’un vaisseau. \\
		\hline
		66 -- 70 & Une nouvelle faction arrive, ex. : police, impériaux, CorSec, etc., créant un combat à trois camps. \\
		\hline
		71 -- 80 & Une voie d’avancée ou de retraite pour les PJ est bloquée, ex. : un turbolift tombe en panne, un pont s’effondre, des portes se ferment. \\
		\hline
		81 -- 90 & L’arme se brise et doit être réparée, elle subit également des dégâts (briseur 3). \\
		\hline
		91 -- 99 & Ajouter un nouvel adversaire Rival. \\
		\hline
		100/00 & Lancez deux fois et combinez les résultats. \\
		\hline
	\end{tabular}
\end{center}

\end{document}
