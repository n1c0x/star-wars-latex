\documentclass{article}
\usepackage[french]{babel}
\usepackage[T1]{fontenc}
\usepackage[left=2cm,right=2cm,top=2cm,bottom=2cm]{geometry}
\usepackage{fancyhdr}
\pagestyle{fancy}
\chead{Suggestion de lieux pour villes et villages}
\usepackage{multicol}

\usepackage{fontspec}
\defaultfontfeatures{Ligatures=TeX}
%\setmainfont[Mapping=tex-text]{Sitka Display}
\usepackage[small,sf,bf]{titlesec}

\usepackage{graphicx}
\usepackage{xcolor}
\usepackage{sectsty}

\definecolor{DarkGreen}{HTML}{384d3e}
\definecolor{PureWhite}{HTML}{FFFFFF}
\definecolor{DarkRed}{HTML}{6e272d}
\definecolor{DarkGold}{HTML}{a48e3b}

\sectionfont{\color{DarkGreen}}
\subsectionfont{\color{DarkRed}}
\subsubsectionfont{\color{DarkGold}}

\begin{document}

\title{\vspace{-0.5cm}{\Huge Suggestion de lieux pour villes et villages} \vspace{-1cm}}

\date{}

\maketitle

\begin{itemize}
	\item Voici des suggestions pour une inspiration de globale
	\item Une ville ne devrait pas essayer d'inclure tous les lieux énumérés.
	\item On suppose qu'une catégorie plus grande pourrait inclure plusieurs des emplacements suggérés de catégories plus petites.
	\item Les endroits importants au cœur de l'aventure devraient avoir 2 ou 3 caractéristiques fantastiques/mémorables ajoutées par le MJ.
\end{itemize}

\section*{Petit village, colonie ou avant-poste}
\begin{multicols}{2}
	\begin{itemize}
		\item Articles de rareté 1 à 3
		\item Bazar des contrebandiers
		\item Magasin de ferraille ou de récupération
		\item Cantina
		\item Peu de véhicules à louer (s'il y en a)
		\item Petit avant-poste militaire ou base de reconnaissance
		\item Installation de communication à moyenne portée
		\item Premiers secours
		\item Poste de gendarme
		\item Cabane d'ermite
		\item Tour de guet
		\item Palissade et/ou remparts
		\item Plate-forme d'atterrissage et station de ravitaillement pour les navires de silhouette 1 -- 3
		\item Baie de véhicules pour véhicules de silhouette 1 -- 3
	\end{itemize}
\end{multicols}

\section*{Grand village ou petite ville}
\begin{multicols}{2}
	\begin{itemize}
		\item Articles de rareté 1 à 5
		\item Marché souterrain pour les articles soumis à des restrictions
		\item Nid d'épices
		\item Taudis et/ou quartier(s) de sans-abri et/ou "villes de tentes".
		\item Système(s) de transport en commun de base
		\item Entreprises de location de véhicules
		\item Banques et caissiers droïdes mobiles
		\item Petites et moyennes installations HoloNet (navigation/observation)
		\item Installations pour navires et véhicules de silhouette 1 -- 4
		\item Hangar
		\item Aire d'atterrissage
		\item Garage / Baie pour véhicules terrestres
		\item Bâtiment pour la religion ou la réflexion
		\item Temple
		\item Église
		\item Observatoire stellaire
		\item Bibliothèque / médiathèque
		\item Retraite dans la nature
		\item Galeries d'art
		\item Commissariat de police
		\item Garnison militaire ou base de taille moyenne
		\item Multiples cantinas, tapas, buffets et restaurants
		\item Hôtel(s)
		\item Des écoles, peut-être une université
		\item Usines à petite et moyenne échelle
		\item Petit hôpital ou clinique
		\item Cybernétique de base
		\item Un ou deux quartiers chics
		\item Pourrait avoir un casino
		\item Installation de communication à plus longue portée
		\item Murs de la ville avec portes anti-souffle
		\item Passerelles avec ponts escamotables
		\item Rues à plusieurs niveaux avec ascenseurs à turbine (et sans garde-corps)
		\item Carnavals, foires ou cirques itinérants
	\end{itemize}
\end{multicols}

\section*{Grande ville portuaire spatiale}
\begin{multicols}{2}
	\begin{itemize}
		\item Articles de rareté 1 à 10
		\item Marché souterrain des articles à usage restreint : très probable
		\item Grand port spatial pouvant accueillir des vaisseaux de silhouette 1 à 5 (ou même plus grands)
		\item Plusieurs pistes d'atterrissage
		\item Hangars multiples
		\item Plusieurs garages de véhicules terrestres et baies de réparation
		\item Tours d'amarrage ou anneaux d'amarrage orbital pour les plus grands navires
		\item Plusieurs dépôts de carburant
		\item Systèmes de transport public avancés
		\item Installations HoloNet avancées (y compris la radiodiffusion)
		\item Multiples commissariats de police
		\item Grands complexes pour la religion ou la réflexion
		\item Cathédrales ou méga-églises
		\item Les temples palatiaux
		\item Mausolées monumentaux
		\item Parcs / espaces verts
		\item Les pièges à touristes
		\item Sanctuaire ou zoo pour animaux
		\item Musées
		\item Holo-parcs
		\item Vastes complexes commerciaux à l'intérieur et à l'extérieur
		\item Restaurants chics
		\item Casinos
		\item Hippodromes de course
		\item Fosse de gladiateur
		\item Salles de spectacle (danse, musique, comédie)
		\item Jardins de sculptures
		\item Fontaines animées
		\item Spectacles lumineux
		\item Pont d'observation / tours d'observation
		\item Établissements d'enseignement multiples à tous les niveaux d'apprentissage
		\item Stade de bal Nuna-Ball ou autres arénas sportifs
		\item Grands hôpitaux et cliniques multiples
		\item Cybernétique de luxe
		\item Installations de communication avancées
		\item Districts urbains clairement délimités (souvent divisés par classe de revenu)
		\item Palais, Demeures, Villas, etc.
		\item Banlieues de la classe moyenne
		\item Quartier d'affaires ou "centre-ville
		\item Dédale de gangs et/ou grands ghettos
		\item Grande(s) base(s) militaire(s) - armée et marine
		\item District(s) industriel(s)
		\begin{itemize}
			\item Laboratoires de recherche
			\item Industrie lourde
			\item Usines
			\item Chantiers navals
			\item Entrepôts et quais de chargement
		\end{itemize}
		\item Pensez verticalement : à la fois en hauteur et en sous-sol.
		\item Générateur(s) de bouclier(s) à l'échelle de la ville
		\item Murs rideaux concentriques avec tours et portes anti-souffle / portails / ponts escamotables
		\item Passerelles pour piétons / promenades dans le ciel
		\item Couches sur couches de trafic plus rapide se divisent en autoroutes aériennes.
		\item Zones protégées contre les rayons
		\item Batteries défensives
		\item Vastes fosses à ciel ouvert dans les puits d'entretien, puits de ventilation colossaux, etc.
		\item Bâtiments et structures "animés" qui se transforment ou se déplacent, se translatent ou se reconfigurent d'une manière ou d'une autre.
		
	\end{itemize}
\end{multicols}

\end{document}
