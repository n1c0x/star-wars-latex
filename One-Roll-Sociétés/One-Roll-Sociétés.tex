\documentclass{article}
\usepackage[french]{babel}
\usepackage[T1]{fontenc}
\usepackage[left=2cm,right=2cm,top=2cm,bottom=2cm]{geometry}
\usepackage{fancyhdr}
\pagestyle{fancy}
\chead{One-Roll Sociétés}
\usepackage{multicol}
\usepackage{enumitem}

\usepackage{fontspec}
\defaultfontfeatures{Ligatures=TeX}
\setmainfont[Mapping=tex-text]{Sitka Display}
\usepackage[small,sf,bf]{titlesec}

\usepackage{graphicx}
\usepackage{xcolor}
\usepackage{sectsty}

\definecolor{DarkGreen}{HTML}{384d3e}
\definecolor{PureWhite}{HTML}{FFFFFF}
\definecolor{DarkRed}{HTML}{6e272d}
\definecolor{DarkGold}{HTML}{a48e3b}

\sectionfont{\color{DarkGreen}}
\subsectionfont{\color{DarkRed}}
\subsubsectionfont{\color{DarkGold}}


\begin{document}

\title{\vspace{-0.5cm}{\Huge One-Roll Sociétés} \vspace{-1cm}}

\date{}

\maketitle


\section*{Utilisation}
Tout d'abord, vous ne pouvez probablement imprimer que les pages 2 et 3. Cette page n'est pas nécessaire pour jouer directement.
Bien sûr, vous pouvez développer ce concept comme vous le souhaitez : des trios de dés rapprochés au lieu de simples paires, des paires multiples d'éléments alliés, ou traiter un seul dé éloigné comme un groupe sectaire ou un groupe dissident. Gardez à l'esprit que les joueurs peuvent seulement être en mesure de suivre quelques-unes de ces affiliations et litiges à la fois - vous avez le plan complet, ils n'ont que des indices. Des systèmes supplémentaires - l'axe de l'histoire et l'axe de la hiérarchie - s'ajouteront à la société que vous avez créée, mais ne sont pas nécessaires (surtout si le groupe est déjà sur le point de quitter la ville au moment où vous écrivez tout cela). Néanmoins, il peut être utile pour faire mûrir une colonie que vous avez déjà construite.


\section*{Systèmes additionnels}

\subsection*{Axe historique}
Pour aller encore plus loin, lisez les dés comme une ligne du temps de gauche à droite --- les éléments de l'extrême gauche viennent en premier, tandis que ceux de l'extrême droite sont de nouveaux développements. Les \textbf{Affiliations} ou les \textbf{Différends} peuvent être lus de la même manière. (ex : \textbf{Croyances} : \textit{Dévotion à un Dieu neutre} à gauche pourrait signifier que la ville a été fondée pour la liberté religieuse. \textbf{Gouvernance} : \textit{Mandarinat} à droite pourrait signifier une révolution politique récente ou imminente.)

\subsection*{Axe hiérarchique}
Alors que l'Axe historique peut engendrer une complexité fascinante dans une petite ville, l'axe hiérarchique s'avère plus utile dans les grandes villes. En plus de lire les dés de gauche à droite, lisez-les de haut en bas. Les dés en haut représentent les éléments affiliés à la haute société - soit une caste distincte, soit simplement la crème de la crème - et les dés en bas représentent les éléments inférieurs de la société. Vous pouvez diviser les dés en castes individuelles ou traiter le tableau de dés comme un spectre de disparité de classe. (ex : \textbf{Crime} : \textit{Organisé} au sommet pourrait indiquer un réseau d'esclavage secret dirigé par des élites sociales. \textbf{Crime} : \textit{Organisé} en bas pourrait indiquer une cellule d'assassins de gamins de rue.)


\clearpage

Un système de création de sociétés. Lancez un jeu de dés standard (d4, d6, d8, 2d10, d12, d20) et documentez leurs valeurs dans les tableaux ci-dessous pour générer les éléments de la société. Notez leur proximité les uns des autres. Les deux dés les plus proches représentent les éléments alliés ; les deux qui sont les plus éloignés les uns des autres représentent les éléments de différends. Exemples et outils supplémentaires à la page 3. 


\section*{D4 -- Défenses}
\begin{enumerate}
	\item \textbf{Milice :} Les citoyens entrainés combattent les menaces/crimes
	\item \textbf{Armée régulière :} Soldats professionnels et organisés
	\item \textbf{Champion :} Un seul défenseur notable
	\item \textbf{Aucune défense militaire}
\end{enumerate}

\section*{D6 -- Commerce}
\begin{enumerate}
	\item \textbf{Économie de troc :} Pas d'argent, seulement du commerce
	\item \textbf{Désespéré :} Les difficultés du commerce peuvent se traduire par des prix bas et une faible intégrité chez les commerçants.
	\item \textbf{Forte fiscalité :} Les prix des commerçants augmentés à leur tour
	\item \textbf{Ingérence :} Gouvernance, guildes ou groupes religieux
	\item \textbf{Nouvelle route commerciale :} Voyageurs exotiques et douanes
	\item \textbf{Traditionnelle :} Coutumes strictes, préjugés contre certains groupes fondés sur la race ou la croyance
\end{enumerate}

\section*{D8 -- Criminalité}
\begin{enumerate}
	\item \textbf{Violente :} Même un meurtre n'est pas rare
	\item \textbf{Col blanc :} Escrocs et marchands tricheurs
	\item \textbf{Gouvernance corrompue :} Lois prédatrices, procès inéquitables
	\item \textbf{Peines sévères :} Peines violentes, bannissement
	\item \textbf{Application de la loi corrompue :} Corruption, fausses accusations
	\item \textbf{Organisée :} Le Parti n'est pas le seul gang en ville
	\item \textbf{Supprimée :} Des forces de l'ordre compétentes, des gouvernants justes
	\item \textbf{Évitée :} Personne n'est associé à des criminels connus.
\end{enumerate}

\section*{2D10 -- Croyances majeures}
\begin{multicols}{2}
	\renewcommand{\labelitemi}{}
	\begin{itemize}
		\item 10. En colère contre \ldots
		\item 20. Se rebelle contre \ldots
		\item 30. Culpabilisé par \ldots
		\item 40. Révolution vers \ldots
		\item 50. Punir les adeptes de la \ldots
		\item 60. Apathique envers \ldots
		\item 70. En attente de \ldots
		\item 80. Pieux envers \ldots
		\item 90. Peur de \ldots
		\item 100. En croisade pour \ldots
	\end{itemize}
	\begin{enumerate}
		\item \ldots Athéisme/Agnosticisme
		\item \ldots Magie
		\item \ldots Puissance
		\item \ldots Richesse
		\item \ldots L'adoration des aînés
		\item \ldots Un bon Dieu
		\item \ldots Un Dieu maléfique
		\item \ldots Un Dieu neutre
		\item \ldots Un Dieu légitime
		\item \ldots Un Dieu chaotique
	\end{enumerate}
\end{multicols}

\section*{D12 -- Gouvernance}
\begin{enumerate}
	\item \textbf{Arcanocratie} : Le dirigeant ou la classe dirigeante est magique
	\item \textbf{Anarchie} : Il n'y a pas de lois. Si quelqu'un fait quelque chose que vous n'aimez pas, répondez comme vous le souhaitez.
	\item \textbf{Règle du Conquérant} : Celui qui a battu le dernier chef
	\item \textbf{Conseil} : Un groupe prend des décisions ensemble
	\item \textbf{Démocratie} : Chacun vote directement sur chaque question
	\item \textbf{Dictature} : Décision par la force
	\item \textbf{Allégeance} : La ville doit allégeance à l'étranger
	\item \textbf{Mandarinat} : Exige des tests ou des essais pour statuer
	\item \textbf{Monarchie} : Monarque unique avec succession de la lignée sanguine
	\item \textbf{Oligarchie} : Un groupe d'individus divisent le pouvoir
	\item \textbf{République} : Souverain élu
	\item \textbf{Théocratie} : Dieu(s) fait la loi. Il peut y avoir ou non du clergé pour interpréter les souhaits de Dieu.
\end{enumerate}

\section*{D20 -- Élément culturel important }
\begin{multicols}{2}
\begin{enumerate}
	\item \textbf{Nourriture} : Normes élevées, respect pour les bons cuisiniers
	\item \textbf{Musique} : Musique constante ; la plupart jouent d'un instrument
	\item \textbf{Peinture} : Art coloré sur des bâtiments/objets/personnes
	\item \textbf{Écriture} : Auteur(s), calligraphie, poésie généralisée
	\item \textbf{Mode} : S'habiller pour le conformisme et l'expression de soi
	\item \textbf{Body Art} : Tatouages, piercings, coiffure, maquillage/peinture
	\item \textbf{Drame} : Narration, discours poétique, menteurs talentueux.
	\item \textbf{Architecture} : Monuments, bâtiments uniques
	\item \textbf{Vacances} : Célébrations et événements fréquents
	\item \textbf{Danse} : Le mouvement est important dans les célébrations, les traditions, la cour, les transactions, le respect.
	\item \textbf{Tribalisme} : Société séparée en groupes distincts
	\item \textbf{Duel} : La plupart des différends sont réglés dans le cadre d'un combat formel.
	\item \textbf{Symbologie} : Marques proéminentes de croyance ou de faction
	\item \textbf{Toxicomanie} : Abus d'alcool, d'aliments et d'autres substances
	\item \textbf{Gloutonnerie} : Abus de nourriture, de boisson ou de plaisir
	\item \textbf{Avidité} : Le commerce malhonnête, les jeux de hasard, l'égoïsme
	\item \textbf{Fierté} : Confiant, facilement offensé
	\item \textbf{Désespoir} : Pessimisme généralisé
	\item \textbf{Colère} : Facilement provoquée, impatiente de se battre
	\item \textbf{Paresse} : Paresse, rythme de vie lent
\end{enumerate}
\end{multicols}

\section*{Exemples}

\subsection*{Affiliations}
\begin{itemize}
	\item \textbf{Criminalité} $\Rightarrow$ \textit{Col blanc (2)} près de \textbf{Croyances} $\Rightarrow$ \textit{Pieux envers (80) un Dieu neutre (8)} pourrait indiquer un clergé corrompu. 
	\item \textbf{Gouvernance} $\Rightarrow$ \textit{Théocratie (12)} près de \textbf{Défense} $\Rightarrow$ \textit{Milice (1)} pourrait indiquer que la participation à la milice est considérée comme un signe de dévouement à la religion qu'elle sert.
\end{itemize}

\subsection*{Différends}
\begin{itemize}
	\item \textbf{Gouvernance} $\Rightarrow$ \textit{Allégence (7)} éloigné de \textbf{Défense} $\Rightarrow$ \textit{Armée régulière (2)} pourrait indiquer une rébellion naissante. 
	\item \textbf{Gouvernance} $\Rightarrow$ \textit{Conseil (4)} éloigné de \textbf{Commerce} $\Rightarrow$ \textit{Ingérence (4)} pourrait indiquer que les guildes locales se battent contre le conseil pour le pouvoir.
\end{itemize}

\clearpage

\section*{Conclusion}
Un système élégant permettant aux MJ de mettre du muscle sur les os d'une ville. Une fois que la géographie a été établie, ce système devrait tenir compte des autres qualités d'une petite ville qui lui donnent son caractère. Il doit être agnostique quant à la méthode utilisée pour créer une ville.

\begin{description}
	\item [Pour les MJ] Bien que les éléments introduits par ce système aient un effet plus actif sur l'intrigue d'une aventure, il ne sera pas rigide au point d'interférer profondément avec le canon de la campagne.
	\item [Elégance] Peu de dés, résultats faciles à lire, faciles à transporter et à documenter.
	\item [Muscle] Il ne s'agira pas d'établir une géographie, des gens ou des lieux existants, ou des choses à trouver ; il s'agira de mettre ces pièces en mouvement. Il devrait rendre le monde vivant, non pas pour qu'il ne se sente pas vivant, mais pour qu'il génère réellement une histoire lointaine et contemporaine. Il portera sur la façon dont les sociétés se déplacent : la politique, la société et l'activité dans le cadre d'une géographie établie.
	\item [Face au joueur] Il y a de nombreuses qualités dans une ville avec lesquelles les joueurs n'interagissent pas et dont ils ne se soucient pas. Irrigation, systèmes d'écriture, etc. Chaque élément sur les tables reflétera un élément de la société qui peut avoir un impact direct sur les joueurs. C'est pour cette raison qu'on a changé les tables de la criminalité et du commerce - trop d'échanges commerciaux 
	\item [Autres Qualités] Race, gouvernance, défenses, religion, mouvements et factions (bien que les factions puissent devenir un système complètement différent).
	\item [Agnostique] Il n'annulera aucune des caractéristiques du système Bones/Tarrasque, et s'appliquera également aux Villes créées sans le système One-Roll.
\end{description}

\end{document}
