\documentclass{article}
\usepackage[french]{babel}
\usepackage[T1]{fontenc}
\usepackage[left=2cm,right=2cm,top=2cm,bottom=2cm]{geometry}
\usepackage{fancyhdr}
\pagestyle{fancy}
\chead{Créer des PNJ ayant une personnalité}
\usepackage{multicol}

\usepackage{fontspec}
\defaultfontfeatures{Ligatures=TeX}
\setmainfont[Mapping=tex-text]{Sitka Display}
\usepackage[small,sf,bf]{titlesec}

\usepackage{graphicx}
\usepackage{xcolor}
\usepackage{sectsty}

\definecolor{DarkGreen}{HTML}{384d3e}
\definecolor{PureWhite}{HTML}{FFFFFF}
\definecolor{DarkRed}{HTML}{6e272d}
\definecolor{DarkGold}{HTML}{a48e3b}

\sectionfont{\color{DarkGreen}}
\subsectionfont{\color{DarkRed}}
\subsubsectionfont{\color{DarkGold}}

\begin{document}

\title{\vspace{-0.5cm}{\Huge Créer des PNJ ayant une personnalité} \vspace{-1cm}}

\date{}

\maketitle

Ce guide offre un moyen rapide de générer des PNJ intéressants avec des personnalités à part entière. Les personnalités sont basées sur la théorie de Myers-Briggs, et chaque personnage est défini par 5 attributs : \textbf{esprit}, \textbf{énergie}, \textbf{nature}, \textbf{tactique} et \textbf{identité}. De plus, il y a aussi une table pour les bizarreries de personnalité.

\section*{Pour commencer}
Commencez par lancer les 7 dés : 1d4, 1d6, 1d8, 1d100, 1d12, 1d20. Pour chaque dé (ou paire, pour d100), consultez les listes ci-dessous. Au fur et à mesure que vous écrivez chaque attribut, pensez à la façon dont ils interagissent les uns avec les autres, et le PNJ prendra vie.

\subsection*{1d4 : Esprit (Comment ils interagissent avec le monde)}

\begin{description}
	\item[1 --- 2 : Introverti (I)] Le PNJ préfère les activités solitaires et essaie d'échapper aux stimuli extérieurs. Ils ont tendance à éviter les tavernes, les discussions politiques, les foules et préfèrent rester seuls puisqu'ils sont plus à l'aise seuls. Le PNJ ne donnera pas facilement des informations et il est difficile de se lier d'amitié.
	\item[3 --- 4 : Extraverti (E)] Le PNJ préfère les activités de groupe et a tendance à être plus enthousiaste et proactif. Ils aiment s'engager avec d'autres personnes et se mettre au défi, ainsi que les autres. Le PNJ interagira facilement avec les joueurs, se connectera avec eux et appréciera les commentaires des autres.
\end{description}

\subsection*{1d6 : Energie (Comment ils voient le monde)}

\begin{description}
	\item[1 --- 3 : Observateur (S)] Le PNJ est très pratique, pragmatique et terre à terre. Ils se concentrent sur le présent et sont doués pour traiter les faits. Ils favorisent ceux qui ont fait leurs preuves et savent se concentrer sur une chose à la fois.
	\item[4 --- 6 : Intuitif (N)] Le PNJ est ouvert d'esprit et curieux, s'appuyant sur l'imagination, les idées et les possibilités. Ils aiment la nouveauté et sont prêts à renoncer à la commodité, au confort et à la prévisibilité en échange de l'excitation apportée par l'exploration.
\end{description}

\subsection*{1d8 : Nature (Comment ils prennent des décisions)}

\begin{description}
	\item[1 --- 4 : Réfléchi (T)] Le PNJ met l'accent sur l'objectivité et la rationalité, privilégiant la logique sur les émotions. Pour eux, l'efficacité est plus importante que la coopération et ils comptent sur leur tête plutôt que sur leur cœur. Le PNJ a tendance à soumettre et à outrepasser leurs sentiments avec une logique rationnelle.
	\item[5 --- 8 : Sentimental (F)] Le PNJ est plus empathique et moins compétitif, se concentrant sur l'harmonie sociale et la coopération. Ils sont compatissants, sensibles et émotifs. Le PNJ cherche des solutions qui rendent tout le monde plus heureux, et ils valorisent les principes et les idéaux plus que le succès.
\end{description}

\subsection*{1d12 : Tactique (Comment ils abordent les stratégies)}

\begin{description}
	\item[1 --- 6 : Jugement (J)] Le PNJ est minutieux et organisé. Ils apprécient la clarté, la prévisibilité, la structure et la planification. Ils placent leurs devoirs et leurs responsabilités au-dessus de tout, ont une éthique de travail rigoureuse et sont stricts lorsqu'il s'agit de maintien de l'ordre. L'alignement du PNJ aura tendance à être loyal.
	\item[7 --- 12 : Prospection (P)] Le PNJ est très bon pour improviser, être flexible et garder ses options ouvertes. Ils ont tendance à se concentrer davantage sur ce qui les rend heureux que sur ce qu'on attend d'eux, et ils sont toujours à la recherche d'occasions et d'options. L'alignement du PNJ aura tendance à être chaotique.
\end{description}

\subsection*{1d20 : Identité (Ce qu'ils ressentent pour eux-mêmes)}

\begin{description}
	\item[1 --- 10 : Affirmatif (-A)] Le PNJ est sûr de lui, d'humeur égale et résistant au stress. Ils refusent de trop s'inquiéter et ne se donnent pas trop de mal pour atteindre leurs objectifs. Le PNJ sera probablement content et satisfait.
	\item[11 --- 20 : Turbulent (-T)] Le PNJ est gêné et sensible au stress. Ils sont susceptibles d'éprouver un large éventail d'émotions et d'être motivés par le succès, perfectionnistes et désireux de s'améliorer. Le PNJ voudra probablement faire ses preuves.
\end{description}


\subsection*{1d100 : Bizarreries}

\begin{multicols}{2}
	\begin{enumerate}
		\item Se frotte fréquemment les mains comme si elles étaient constamment froides.
		\item Ne parle qu'en chuchotant à voix basse
		\item Parle toujours à un volume plus élevé que nécessaire, comme si tout le monde autour d'eux était malentendant
		\item Renifle et tousse tous les quelques mots
		\item N'appelle les gens que par leur nom complet
		\item Se lèche lèvres lentement avant de dire quelque chose
		\item S'inflige de la douleur pour démontrer son dévouement à sa foi ou pour punir ses péchés
		\item Fait des blagues inappropriées dans les moments graves
		\item Toujours d'accord avec tout le monde, même si cela signifie se contredire (c.-à-d. être d'accord avec les deux côtés d'un argument entendu)
		\item Il doit fermer les yeux pour se concentrer sur les souvenirs les plus insignifiants
		\item Se moque de ses propres pets et aime l’odeur
		\item Incapable de donner sa propre opinion, s'en remettant toujours aux autres. Évite les questions sur leur point de vue
		\item Il a toujours un bon mot à dire sur tout le monde. Au premier abord, cela peut sembler une qualité attachante, mais avec le temps, cela semble de moins en moins sincère
		\item Fait tourner une brindille autour de leurs doigts de façon pratiquement constante (comme certaines personnes le font avec des crayons)
		\item Tressaille/grimace dès que quelqu'un mentionne des nains
		\item A un titre long, et EXIGE qu'il soit utilisé chaque fois qu'ils sont mentionnés
		\item Fixe au loin jusqu'à ce que ce soit à leur tour de faire quelque chose
		\item Doivent toucher les portes, les poteaux et les torches qu'ils croisent
		\item Trop attaché aux biens d'autrui
		\item Se tord et se tend constamment le cou, faisant des bruits de craquement
		\item Toujours à regarder par-dessus leur épaule, comme si on cherchait quelqu'un pour les suivre
		\item Éblouit et plisse les yeux pendant les conversations lorsqu'une personne décrit quelque chose en détail, comme si elle essayait de discerner si elle dit la vérité par la force de la volonté, même si le détail est banal
		\item Renifle fort, le nez en l'air, quand quelqu'un en dessous de son poste lui parle et ne lui parle qu'avec dédain
		\item Agit avec une extrême déférence à l'égard de ceux qui sont au-dessus d'eux
		\item Toujours prêchant leur foi, quelle que soit la conversation, ils parviennent toujours à l'adapter à leur religion
		\item Leurs yeux ne clignent pas à l'unisson
		\item Essayer constamment de garder les dernières mèches de cheveux sur leur tête pour couvrir le reste de leur calvitie
		\item Remue les sourcils
		\item réprimande et gronde les gens, même pour les plus petites infractions
		\item Converse avec quelqu'un qu'il est le seul à pouvoir voir ou entendre, en lui demandant son opinion sur des décisions importantes
		\item Trouve la joie dans sa propre douleur
		\item Toujours des commentaires audibles sur la météo lorsque vous sortez à l'extérieur
		\item Commence à répondre à toutes les questions avec "hmm."
		\item Il compte avec ses doigts, même pour de grosses sommes
		\item Éternue toujours deux fois lorsqu'il commence à pleuvoir
		\item Se cure le nez quand ils pensent que personne ne regarde (ils le sont totalement)
		\item Ils fixent un peu trop longtemps
		\item Ils lancent continuellement une incantation bénigne lorsqu'ils sont nerveux
		\item Ils ont tendance à radoter à propos de tout ce dont ils parlent
		\item Leur voix tonnante prend le dessus de toute conversation à laquelle ils participent
		\item Ils ont tendance à interrompre les gens
		\item Ils se tiennent toujours debout ou s'assoient dos aux murs et face à la porte
		\item Ils analysent tout le monde pour voir s'ils mentent
		\item Ils disent qu'ils préfèrent glaner autant d'informations que possible
		\item Commence la plupart des conversations par de cruels blagues personnelles
		\item Il rit comme un fou au combat. Plus l'ennemi est fort, plus les rires sont grands
		\item Glousse nerveusement en essayant de mentir, ou quand ils savent que quelqu'un d'autre l'est
		\item Oublie l'espace personnel et est trop affectueux
		\item Elle se craque la mâchoire en faisant un bruit fort
		\item Sent comme un fruit en particulier
		\item Fixe le toit d'un regard vide et marmonne du charabia
		\item Relie tout à un événement historique
		\item Il semble toujours avoir des démangeaisons et gratte son entre-jambe quand il pense que personne ne regarde
		\item Mange constamment mais ne semble jamais prendre du poids
		\item Baisse les yeux quand quelqu'un d'autre que lui parle
		\item N'approuve que les conversations régimentaires. Si quelqu'un semble s'exprimer de façon inopportune, il sera réprimandé. Attend que chaque réponse ait un début, un milieu et une fin
		\item A toujours une tête comme à la sortie du lit
		\item A un rire horrible et strident
		\item Surutilise le nom des gens dans la conversation, surtout celui de la personne à qui il s'adresse
		\item Ne s'adresse pas aux autres par leur nom, sauf en cas d'absolue nécessité
		\item Très émotionnels avec leur visage entier, surtout leurs sourcils
		\item Il a une expression d'impassibilité en tout temps
		\item A une courte phrase d'accroche qu'ils utilisent pour entrer ou sortir d'une conversation
		\item Donne des surnoms à leurs amis
		\item Donne des surnoms à leurs ennemis
		\item Inspirera avec un bref son "oui" tous les quelques respirations pour signifier qu'ils sont toujours à l'écoute
		\item Utilise plusieurs gestes de la main pour encadrer son visage tout en parlant
		\item Il continue d'essayer de raconter des blagues, mais il se met à rire avant de finir la chute
		\item Sur-explique tout
		\item Est raciste 
		\item Est toujours courbé comme s'il essayait de prendre moins de place.
		\item Tressaille chaque fois que la violence est mentionnée
		\item Agit avec agressivité envers les personnes agressives, mais avec modération avec les autres
		\item Aime jouer constamment avec quelque chose
		\item Lors d'une conversation, ils regardent toujours légèrement par-dessus l'épaule de l'autre personne
		\item Il se souvient de chaque visage de chaque personne qu'il croise
		\item Évitez de regarder tout le monde dans les yeux, surtout les étrangers
		\item Fléchit la main de son arme lorsqu'il est frustré ou en colère
		\item Grince des dents quand il est en colère
		\item Ne laisse jamais une question sans réponse et ne laisse jamais un mensonge sans réponse
		\item Mâche la lèvre en mentant
		\item Compte méticuleusement les pièces de monnaie
		\item Compte sur leurs doigts
		\item Mord n'importe quelle pièce de monnaie pour s'assurer qu'elle est réelle
		\item Peser n'importe quelle pièce de monnaie sur une balance pour s'assurer qu'elle est réelle
		\item Agit nerveusement alors qu'il dit la vérité
		\item Appelle une divinité spécifique dans les moments difficiles. Le nom de ce dieu unique n'est cependant réservé qu'aux choses vraiment sérieuses, et n'est souvent entendu que dans le contexte d'une bataille, d'une blessure grave ou d'un coup de malheur
		\item Joue instinctivement avec les animaux de compagnie d'autres personnes
		\item S’entraine dans la cour tous les jours à l'aube
		\item Admet toujours ses fautes. Trop confiant et peu disposé à se livrer à des tactiques sournoises
		\item S'essuie le nez qui coule sur la manche ou la main, puis veut toujours terminer la conversation par une poignée de main, une étreinte ou une claque
		\item S'arrête souvent au milieu d'une phrase lorsqu'il parle pour rassembler ses pensées. En conséquence, est souvent interrompu par des personnages bavards, et prendra offense
		\item S'agite avec une intensité croissante au fur et à mesure que l'interaction s'éternise, comme s'il fallait aller aux toilettes (et c'est peut-être le cas !)
		\item Lèche fréquemment les lèvres
		\item Retire toujours ses cheveux devant les yeux, enroule la moustache autour du doigt, se frotte le cuir chevelu ou s’essuie le front
		\item Achète toujours une tournée pour tous dans l'établissement lorsqu'il arrive dans une taverne pour la première fois
		\item Touche une babiole particulière lorsqu'elle a besoin de chance
		\item Agit avec étourdissement lorsqu'il parle aux aventuriers, en utilisant souvent des mots comme " merveilleux " et " incroyable " à l'écoute de leurs récits
		\item Il a une dette de vie envers quelqu'un et lui est fanatiquement loyal
		\item Fait tout son possible pour apprendre les capacités de ses camarades, les meilleurs pour maximiser leur efficacité tactique
	\end{enumerate}
\end{multicols}

\section*{Exemples}
\subsection*{Belmus}
Belmus est un nain mâle. Pour sa personnalité nous avons comme jets :
\begin{itemize}
	\item 1d4 : 2
	\item 1d6 : 1
	\item 1d8 : 6
	\item 1d12 : 6
	\item 1d20 : 8
	\item 1d100 : 95 
\end{itemize}

Cela signifie que sa personnalité est \textbf{ISFJ-A}. C'est un \textbf{introverti} qui préfère rester seul, plutôt que de s'engager avec d'autres personnes. Il est aussi \textbf{observateur}, attaché aux faits. Cela signifie qu'il évite d'interagir avec des étrangers et que les aventuriers peuvent avoir de la difficulté à lui parler. Il pense que les aventuriers sont des rêveries insensées, des adolescents qui n'ont pas grandi et qui ne préfèrent pas interagir avec eux.\\

Ses décisions sont fondées sur ses \textbf{sentiments} et il apprécie le sens de la communauté de sa ville. Il croit que tout le monde devrait faire sa part, connaître sa place, et tout ira bien. Encore une fois, cela le rend encore plus méfiant à l'égard des étrangers et des gens qui pourraient nuire à cet équilibre.\\

Belmus est soigné et organisé, et il croit au travail acharné, puisque sa tactique est de \textbf{juger}. Il est aussi \textbf{assertif} et sûr de lui. Il aime la vie telle qu'elle est, et il ne passe pas trop de temps à penser à ses actions ou choix passés, puisque ce qui est fait est fait. Le convaincre de quelque chose sera extrêmement difficile.\\

Le nain a une bizarrerie amusante : il peigne toujours sa barbe avec un peigne à os, et quand il a fini, il laisse le peigne attaché à sa barbe.

\subsection*{Trisfiel}
Trisfiel est un elfe femme. Pour sa personnalité nous avons comme jets :
\begin{itemize}
	\item 1d4 : 2
	\item 1d6 : 4
	\item 1d8 : 1
	\item 1d12 : 6
	\item 1d20 : 11
	\item 1d100 : 47
\end{itemize}
Sa personnalité est donc \textbf{INTJ-T}. Comme Belmus, Trisfiel est une \textbf{introvertie}, mais elle est aussi \textbf{intuitive}. Elle aime le plein air, rester loin de son village pendant des heures dans les bois, curieuse du monde dans lequel elle vit, essayant de mieux le comprendre sans les sons et les odeurs accablants de son village.\\

Sa nature est la \textbf{pensée}, et c'est pour cela qu'elle ne s'intéresse pas beaucoup à son village ni à ses voisins. Elle pense que l'attachement aux autres est insensé, et elle prévoit de partir bientôt pour faire ses preuves auprès de ses parents et de sa famille.\\

Comme Belmus, elle \textbf{juge}. Dans son cas, cela se manifeste par un plan clair (et un plan de secours) pour sa vie. Elle pratique ses cantrips toute seule dans les bois, espérant un jour pouvoir fréquenter l'Université de Magie de la capitale. Si elle échoue au test, elle prévoit devenir une aventurière et apprendre la magie toute seule.\\

L'identité de Trisfiel est \textbf{turbulente}. Elle est toujours à la recherche de moyens de se surpasser, et elle est impatiente d'améliorer ses capacités à lancer des sorts.\\

Enfin, sa bizarrerie est qu'elle glousse nerveusement lorsqu'elle essaie de mentir ou lorsqu'elle sait que quelqu'un d'autre le fait. Cela l'a amenée à s'inquiéter plus d'une fois lorsque ses parents lui ont demandé pourquoi elle passait tant de temps seule dans la forêt, car ils n'approuvent pas qu'elle devienne une utilisatrice de magie.


\subsection{Roklo}
Roklo est un kobold homme. Pour sa personnalité nous avons comme jets :
\begin{itemize}
	\item 1d4 : 3
	\item 1d6 : 5
	\item 1d8 : 1
	\item 1d12 : 11
	\item 1d20 : 4
	\item 1d100 : 42
\end{itemize}

Cela signifie que sa personnalité est \textbf{ENTP-A}. Étant un extraverti\textbf{extraverti}, Roklo aime la vie de meute dans son village de Kobold, et particulièrement l'interaction avec les visiteurs et les étrangers. Il pousse toujours sa meute à chasser des proies plus grosses et à créer des pièges plus mortels et plus vicieux. Son énergie est \textbf{intuitive}, ce qui l'encourage à améliorer leurs défenses en créant de nouveaux pièges complexes.\\

Comme sa nature est la \textbf{pensée}, Roklo privilégie la logique et ne se soucie pas trop de la coopération. Il préfère suivre ses propres idées sur la façon de créer des pièges et de capturer les aventuriers qui s'approchent trop près de leur village, même si dans le passé cela a causé quelques problèmes. Combiné au fait que sa tactique est la \textbf{prospection}, Roklo essaie constamment de nouvelles façons de contourner les règles mises en place par sa meute lorsque ses projets ont échoué dans le passé ("Pas de feu dans les pièges ? Bien, j'utiliserai de la lave !").\\

Son identité est affirmée, il est donc confiant dans sa capacité à faire face aux situations difficiles. Ce qui s'est passé vit dans le passé, et même si certaines de ses idées n'ont pas bien fonctionné, il est sûr qu'il fait ce qu'il faut.\\

Enfin, Roklo se tient toujours debout ou assis, dos aux murs, face à la porte. Il a développé cette habitude après avoir été attaqué par un membre de sa meute qui est tombé dans un de ses nouveaux pièges infalsifiables. Mieux vaut prévenir que guérir !

\section*{Références}
\begin{itemize}
	\item https://www.16personalities.com/articles/our-theory
	\item  https://www.reddit.com/r/d100/comments/a02kis/lets\_build\_d100\_character\_mannerisms\_and\_quirks/
	\item https://www.fantasynamegenerators.com/
\end{itemize}





\end{document}
