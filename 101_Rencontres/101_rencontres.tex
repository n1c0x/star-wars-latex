\documentclass{article}
\usepackage[french]{babel}
\usepackage[T1]{fontenc}
\usepackage[left=2cm,right=2cm,top=2cm,bottom=2cm]{geometry}
\usepackage{fancyhdr}
\pagestyle{fancy}
\chead{Rencontres}
\usepackage{multicol}

\usepackage{fontspec}
\defaultfontfeatures{Ligatures=TeX}
%\setmainfont[Mapping=tex-text]{Sitka Display}
\usepackage[small,sf,bf]{titlesec}

\usepackage{graphicx}
\usepackage{xcolor}
\usepackage{sectsty}

\definecolor{DarkGreen}{HTML}{384d3e}
\definecolor{PureWhite}{HTML}{FFFFFF}
\definecolor{DarkRed}{HTML}{6e272d}
\definecolor{DarkGold}{HTML}{a48e3b}

\sectionfont{\color{DarkGreen}}
\subsectionfont{\color{DarkRed}}
\subsubsectionfont{\color{DarkGold}}

\begin{document}

\title{\vspace{-0.5cm}{\Huge Rencontres} \vspace{-1cm}}

\date{}

\maketitle

\section*{1 --- 20 : Urbain}
\begin{enumerate}

	\item Un chasseur de primes Rodien passe près de vous. Il vous détaille rapidement du regard, vous évalue, puis continue son chemin.
	\item Un quasi-Humain vous bouscule et s’excuse aussitôt. Vous entendez d’autres personnes dans le secteur murmurer que c’est l’individu qu’un assassin Defel recherche.
	\item Un Devaronien bien habillé s’approche de vous et demande la direction de l’armurerie la plus proche.
	\item Un large groupe de Duros en uniforme sort d’un restaurant. Ils montent dans un landspeeder et s’éloignent en direction de la zone d’amarrage stellaire la plus proche.
	\item Un groupe de forces de l’ordre locales est conduit par un pisteur Gotal. Ils entrent dans le bâtiment où vous alliez entrer. En approchant, deux agents postés à l’extérieur vous informent que le bâtiment est désormais fermé et que vous devrez revenir plus tard.
	\item Une barge antigrav se déplace lentement dans la rue. Elle transporte une foule tapageuse et lourdement armée, rassemblée autour d’un énorme Hutt. Les gardes du corps Gamorréens vous fixent froidement alors que la barge passe devant vous.
	\item Un transport impérial sur répulseurs passe devant vous. Il porte les marques d’un vaisseau de prisonniers ou d’esclaves. Vous entendez un hurlement provenant de l’intérieur, puis le silence. Par la vitre arrière, vous apercevez au moins une douzaine de Jenets en uniforme gris et entièrement entravés.
	\item Un Ortolan assis au pied d’un bâtiment vous demande si vous avez de la monnaie.
	\item Une escouade de Stormtroopers s’approche de vous et exige de voir vos papiers d’identité.
	\item Un contremaître humain dirige un petit groupe d’Ossans dans la construction d’un bâtiment voisin de celui que vous vous apprêtez à entrer. Un gros élément de charpente tombe sur l’un des Ossans, le blessant et l’immobilisant au sol.
	\item Des secouristes s’occupent de plusieurs blessés dans un accident de landspeeder impliquant trois véhicules. Deux des victimes sont manifestement décédées. Une dispute éclate entre certaines personnes impliquées et d’autres individus présents. Les forces de l’ordre ne sont pas encore arrivées.
	\item Une Togorienne fixe un monument situé sur une place centrale. Elle grogne en vous voyant passer.
	\item Votre véhicule cale brusquement au milieu de la route. Un Verpine bricoleur, étrangement présent dans le secteur, s’approche et vous propose de réparer votre véhicule. Cela sent le piège…
	\item Un gang de swoops traverse les rues à vive allure sur leurs Nebulon-Q Swoops. Peu après, vous voyez les forces de l’ordre locales filer dans la même direction.
	\item Un Whiphid passe devant vous, monté sur une énorme créature reptilienne. Vous êtes à peu près certain qu’aucun des deux n’est originaire de cette planète, et tous deux paraissent totalement déplacés. Les passants leur laissent un large passage.
	\item Un groupe d’humains armés conduit un Abyssin enchaîné dans une ruelle latérale. Vous les voyez frapper à une porte, puis tous entrer dans le bâtiment.
	\item Un Aqualish, encerclé par plusieurs Stormtroopers, se dispute avec eux à un coin de rue. L’Aqualish fait un geste brusque vers sa ceinture, et deux Stormtroopers l’abattent aussitôt. Les soldats commencent alors à exhorter vivement les passants à circuler.
	\item Un commerçant Arcona se tient près d’un véhicule ouvert à un coin de rue. En passant, il vous demande si vous cherchez des articles particuliers. Lorsque vous approchez, il soulève un tissu de velours recouvrant une boîte noire cubique d’environ un demi-mètre de côté. La boîte vous semble étrangement familière.
	\item Un transport impérial passe lentement en vol stationnaire près de vous. Alors qu’il entre dans une intersection, le véhicule explose violemment, projetant des éclats partout. La plupart des passants tentent de se mettre à l’abri, mais bien trop tard. Un Bothan vous frôle avec un sourire en coin et marmonne quelque chose dans sa barbe.
	\item Deux Mon Calamari et un humain s’approchent de vous. L’humain hoche la tête en attendant, comme si vous deviez faire quelque chose. Après quelques secondes, il vous regarde et dit : « On va rester plantés là ou on y va ? »
\end{enumerate}

\section*{21 --- 40 : Spatial}
\begin{enumerate}
	\setcounter{enumi}{20}
	\item Une petite station spatiale se dresse dans la zone que vous traversez. Elle semble de nature scientifique, peut-être un poste d’observation. L’installation est alimentée en énergie mais aucun signe apparent de mouvement ou de vie n’est détecté à bord.
	\item Les capteurs de votre vaisseau détectent brièvement la balise de détresse d’un autre vaisseau. Avant que vous puissiez en déterminer l’emplacement exact, la balise se tait.
	\item Vous êtes contraint de naviguer autour d’un vaste champ de débris spatiaux. Ceux-ci semblent provenir d’un vieux cargo, peut-être un Mobquet Medium Cargo Hauler. L’épave paraît relativement récente.
	\item Une escadrille de chasseurs TIE poursuit un cargo léger Starlight. Les deux TIE de tête tirent plusieurs salves, désactivant les moteurs du cargo. Une transmission est envoyée à votre vaisseau vous avertissant de quitter la zone et de ne pas interférer.
	\item Une ligne de six cargos Zuraco avance lentement en vitesse subluminique, se dirigeant vers la planète la plus proche.
	\item Un Croiseur Interdicteur impérial se dresse devant vous. Vous apercevez un Croiseur de luxe SoroSuub 200 en vol stationnaire près de l’Interdicteur. Une Navette d’assaut impériale de classe Gamma semble être en train d’accoster le yacht de luxe, tandis qu’une escadrille de chasseurs TIE patrouille dans les environs.
	\item Une Frégate des douanes impériales vous notifie de son intention de fouiller votre vaisseau. Elle vous ordonne d’arrêter votre vaisseau, de couper les moteurs et de vous préparer à être abordé.
	\item Une Frégate des douanes impériales lourdement modifiée fonce vers votre vaisseau. Une voix extraterrestre grésille dans votre canal de communication : le Kajidic Nikto prend possession de votre vaisseau. Vous vous souvenez de ce nom, associé à un navire pirate qui rôde dans la zone.
	\item Un Patrouilleur impérial de système traverse la zone à grande vitesse. Vos capteurs détectent qu’il scanne brièvement votre vaisseau avant de poursuivre sa route.
	\item Une patrouille de Rangers Trianii suit la trajectoire de votre vaisseau. La force comprend six vaisseaux de patrouille RX4. Après avoir imité votre trajectoire vers votre destination pendant environ une minute, la patrouille change de cap et s’éloigne.
	\item Un vaisseau de patrouille intra-système IR-3F local semble retourner vers la planète la plus proche. Vous pouvez constater des dégâts modérés sur sa coque. Les trois tourelles turbolaser restantes balaient activement la zone.
	\item Un Yacht spatial Aavman Extravagance 11-S croise votre route. Il est escorté par deux chasseurs Toscan 8-Q.
	\item Une grande quantité de débris provenant d’un champ d’astéroïdes voisin perturbe les capteurs de votre vaisseau. Pour passer en hyperespace, vous devrez contourner les débris, ce qui prendra environ trente minutes. Pendant ce temps, il est probable que vos capteurs ne fonctionnent pas correctement.
	\item Un large groupe de chasseurs « Y-TIE Ugly » se dirige vers vous. Ils n’ont pas encore tenté de communiquer, mais vous savez que ces vaisseaux sont généralement associés à des mercenaires et à de petits seigneurs du crime. Ça n’augure rien de bon…
	\item Vous observez un petit objet dérivant sur le flanc tribord de votre vaisseau. Une rapide inspection révèle qu’il s’agit d’une combinaison spatiale blindée (Merr-Sonn Weapons – Superior Boarding Armor). Elle semble flotter lentement dans l’espace. Un balayage des capteurs indique que l’individu à l’intérieur est…
	\item Un cargo moyen HT-2200 lourdement modifié semble exploiter une ceinture d’astéroïdes locale. Le canon ventral a été remplacé par un large outil minier. À votre connaissance, l’exploitation de cette portion de la ceinture d’astéroïdes est illégale.
	\item Une escadrille de chasseurs TIE entre dans votre secteur. Ils vous demandent de vous identifier et d’indiquer votre destination.
	\item Vous apercevez un Destroyer Stellaire de classe Victory au loin. Le transpondeur l’identifie comme étant le Basilisk II. Deux escadrilles de TIE l’escortent, et vous pouvez distinguer plusieurs autres vaisseaux dans la zone. Vos capteurs détectent qu’il tente de scanner votre vaisseau à longue portée.
	\item Une perturbation gravitationnelle secoue votre vaisseau. Les voyants d’alerte clignotent et une petite explosion retentit à l’arrière. Aussi soudainement qu’elle a commencé, la perturbation cesse. Vous feriez mieux de vérifier les dégâts subis par votre vaisseau. En jetant un coup d’œil par le cockpit avant d’aller inspecter, vous apercevez… un autre vaisseau en approche ?
	\item Une grande station spatiale est visible au loin. De petits vaisseaux intra-planétaires effectuent des allers-retours. Une communication de la station apparaît sur votre écran de commande. Elle semble être une sorte d’annonce, mais dans une langue que vous ne comprenez pas.
\end{enumerate}

\section*{41 --- 60 : Régions sauvages}
\begin{enumerate}
	\setcounter{enumi}{40}
	\item Un grand groupe d’oiseaux prédateurs tournoie au-dessus de vous. Soit quelque chose vient de mourir dans le secteur, soit quelque chose va bientôt le faire.
	\item Des bruits mécaniques de pas et de grincements se font entendre au loin. Un AT-ST passe à proximité, ignorant votre présence. L’articulation du genou gauche est endommagée. Le moteur d’entraînement gronde de façon incontrôlable. De lourdes marques de brûlures recouvrent le cockpit.
	\item Des hurlements résonnent au loin. Puis d’autres, en réponse, venant d’une direction différente. Vous remarquez alors une faible piste traversant la zone. Une carcasse de grand animal gît à côté du sentier. Cela n’augure rien de bon…
	\item Vous arrivez sur une zone où une excroissance sphérique incrustée d’environ cent mètres de diamètre émerge du sol. Plusieurs larges trous semblent servir d’entrées menant sous terre. Un fort bourdonnement émane des tunnels, de plus en plus fort au fil des secondes.
	\item Un petit groupe d’humanoïdes montés sur des bêtes de somme passe lentement à proximité. Leurs vêtements, armes et technologies paraissent très primitifs : sans doute des autochtones. Ils semblent parfaitement indifférents à votre présence.
	\item Vous entendez des tirs de blaster et les cris d’un grand animal. Au loin, vous repérez un speeder avec trois Humains près d’une créature indigène. Ce sont probablement des braconniers.
	\item Le sol se met à trembler sous vos pieds. Une fissure s’ouvre devant vous, laissant s’échapper de la vapeur brûlante. La température grimpe rapidement et il devient difficile de respirer. Avez-vous pensé à emporter un respirateur ?
	\item Vous découvrez une grande antenne radar pointée vers le ciel. De nombreuses caisses reliées au dispositif jonchent la zone. Cela ressemble à un appareil de collecte de données.
	\item Vous arrivez dans une vaste plaine soigneusement nivelée et dégagée. Elle semble servir de zone d’atterrissage pour des vaisseaux de petite ou moyenne taille. Elle a été utilisée très récemment.
	\item Vous découvrez l’épave d’un grand astronef, environ soixante-dix mètres de long. Après inspection, il s’agit d’un vieux vaisseau d’exploration Star Cab. La végétation qui l’a envahi laisse penser qu’il est là depuis des décennies.
	\item Vous tombez sur un gigantesque puits d’un kilomètre de diamètre. De nombreuses espèces y exploitent les ressources à l’aide de divers équipements. Des « chefs de fosse » dirigent les travailleurs. Un Humain en uniforme impérial supervise les opérations.
	\item Vous découvrez ce qui semble être un campement d’expédition archéologique. Il n’y a visiblement personne depuis plusieurs jours.
	\item Vous tombez sur une ferme isolée, en plein milieu de nulle part. L’endroit paraît presque anormal ici. Un fermier Ithorien laboure ses champs sur un tracteur antigrav.
	\item En traversant une zone reculée, vous trouvez un étrange bâtiment creusé dans une falaise. Des glyphes uniques, semblant représenter une carte stellaire, ornent l’extérieur. C’est apparemment une crypte d’une famille Duros.
	\item Trois Gotals sur des swoops lourdement modifiés passent devant vous à vive allure. Impossible de savoir s’ils poursuivent quelque chose… ou fuient.
	\item Vous localisez un ancien dépôt impérial abandonné. Il semble inutilisé depuis longtemps. Les stocks ont été retirés ou pillés récemment. Mais il semble qu’une créature énorme y réside désormais.
	\item Vous rencontrez un ermite Ubese accompagné d’une horde de droïdes. Il prétend que vous êtes sur sa concession minière et exige que vous partiez. Malheureusement, l’endroit est très proche de votre destination.
	\item Vous surprenez un groupe d’esclavagistes humains chargeant des autochtones dans un transport. Certains gisent déjà morts. Si les esclavagistes vous remarquent, vous pourriez être les prochains.
	\item Un grand troupeau d’herbivores indigènes broute au loin. Un tir de blaster retentit : la panique les fait foncer droit sur vous.
	\item Vous découvrez une base impériale mobile, installée depuis plusieurs jours. Vous comptez au moins quatre douzaines d’Imperiaux avec des véhicules de soutien. Ils ne semblent pas avoir remarqué votre présence.
\end{enumerate}

\section*{61 --- 70 : Urbain}
\begin{enumerate}
	\setcounter{enumi}{60}

	\item Un droïde roule jusqu’à vous et attire votre attention. Il vous désigne un Gand lourdement armé de l’autre côté de la rue. Le Gand semble être un chasseur de primes. Le droïde vous informe que cet individu vous cherche.
	\item Un Quarren s’approche de vous en marmonnant dans sa langue natale. Quelques instants plus tard, il s’effondre, mort, à vos pieds. La police locale débouche au coin de la rue et marche rapidement dans votre direction.
	\item Un pisteur Shistavanen mène un groupe d’Humains patibulaires vers une taverne locale. Ils s’alignent devant le bâtiment et déclenchent une pluie de tirs de blaster à l’intérieur. Quelques secondes plus tard, le Shistavanen part d’un côté et les Humains de l’autre.
	\item Un Sullustéen vous aborde à un coin de rue. Il vous propose de réparer votre équipement, vos droïdes ou votre vaisseau pour une somme dérisoire.
	\item Un joueur Teltior vous invite à prendre place à sa table de sabacc. Mise d’entrée : cent crédits seulement. Les autres joueurs présents vous accueillent avec un sourire un peu trop appuyé…
	\item Un marchand Bimm vous aborde et vous assure posséder de nombreux artefacts rares. Il vous invite dans sa boutique, persuadé d’avoir quelque chose qui vous intéressera.
	\item Deux Devaroniens et un Chiss entrent dans un bar. Le premier demande au Chiss ce qui ne va pas. Le Chiss répond qu’il est un peu triste. Le second Devaronien lâche : « Oh, tu te sens bleu ? » Le Chiss sort son blaster, abat les deux Devaroniens et s’en va.
	\item Un noble Falleen et sa suite traversent la ville. Ils s’arrêtent pour vous demander où se trouve l’hôtel le plus prestigieux. Avant que vous ne répondiez, l’un des Falleen suggère plutôt que vous les y conduisiez personnellement. Étrangement, vous avez envie d’accepter.
	\item Un Klatooinien vous aborde et déclare que son maître Hutt veut vous parler. Il ne prendra pas non pour réponse.
	\item Un droïde d’astrogation défectueux surgit et commence à tirer avec un petit pistolet blaster. Quelques instants plus tard, il explose en projetant des éclats de métal. Était-ce vraiment une panne… ou un attentat ?
\end{enumerate}

\section*{71 --- 80 : Spatial}
\begin{enumerate}
	\setcounter{enumi}{70}

	\item Vous captez un faible signal de balise de détresse en bordure d’une ceinture d’astéroïdes. En allant enquêter, un cargo léger SoroSuub Nesst lourdement modifié surgit et fonce vers vous. C’est un piège !
	\item Une panne totale affecte toutes les communications de la zone. Impossible de dire s’il s’agit d’un phénomène naturel ou d’une interférence volontaire.
	\item Une grande flotte marchande Nalroni se prépare à passer en hyperespace. Lorsque tous les vaisseaux sautent, un seul cargo lourd Action IV reste en arrière, apparemment en difficulté de navigation. Il est sans défense : aucun armement externe.
	\item Un vieux cargo de classe Dynamique dérivant dans l’espace vous contacte. Ses occupants affirment manquer de carburant. En échange d’un peu du vôtre, ils offrent des données précises de navigation pour votre carte stellaire.
	\item Vous trouvez une sonde impériale flottant près d’une balise de navigation. Elle est intacte mais inactive. Ces sondes valent cher et contiennent beaucoup de données… mais leur récupération est punie de mort par l’Empire.
	\item Vous croisez un transport TL-1800 se déplaçant très lentement en cercle. Pas de signe de vie à bord, mais le vaisseau est intact. Par contre, les senseurs détectent un nombre excessif de droïdes à l’intérieur.
	\item Votre vaisseau traverse une tempête électrique atmosphérique. Des éclairs frappent la coque. Vous ignorez si des dégâts ont été causés… il faudra vérifier dès que possible.
	\item Une sonde sphérique se dirige droit sur vous. Ses capteurs balayent intensivement votre vaisseau.
	\item Une immense créature organique dérive dans l’espace, près de votre trajectoire. Vous n’avez jamais rien vu de tel. Mieux vaut éviter le contact : elle est assez grande pour avaler votre vaisseau.
	\item Alors que vous tentez d’entrer en orbite, le contrôle orbital vous informe que seuls les vaisseaux autorisés peuvent accéder au spatioport. Aucune explication ne vous est donnée.
\end{enumerate}

\section*{81 --- 90 : Étrange}
\begin{enumerate}
	\setcounter{enumi}{80}

	\item Vous vous réveillez dans une pièce inconnue, faible, désorienté, à moitié aveugle. Vous réalisez que vous souffrez de la maladie de l’hibernation. Plusieurs blocs de carbonite ouverts se trouvent dans la pièce.
	\item Vous découvrez un petit cube dans la soute de votre vaisseau. Chaque fois que vous vous en approchez, un vertige violent vous fait presque perdre connaissance.
	\item Vous tombez sur une station spatiale dans un endroit improbable. Vous êtes certain qu’aucune station ne devrait se trouver ici. Elle ne semble pas pouvoir se déplacer. Il y a des signes de vie à bord, mais aucune réponse à vos appels.
	\item Vous rencontrez un droïde qui prétend pouvoir manipuler la Force.
	\item Le dernier souvenir que vous avez est d’avoir entré des coordonnées pour atterrir sur votre planète de destination. Vous vous réveillez en hyperespace, vers une destination inconnue.
	\item Vous découvrez un vaisseau identique au vôtre… mais endommagé. Les registres montrent que c’est votre vaisseau venu du futur, et que ce qui s’est produit arrivera d’ici quelques jours.
	\item Vous trouvez ce qui semble être un vide spatial. Les senseurs ne détectent rien de particulier. Mais une puissante force gravitationnelle attire tout vers cette zone. Un trou noir, peut-être ?
	\item Vous trouvez un corps cryogéniquement congelé dans un conteneur. Sa silhouette est exactement la vôtre. Le sosie est tellement parfait que c’en est troublant.
	\item Votre vaisseau croise un astronef au design totalement inconnu. Bien que vous le voyiez visuellement, il n’apparaît pas sur vos senseurs. Alors que vous vous en approchez, il disparaît brutalement.
	\item Vous découvrez un Humain effondré dans le cockpit d’un landspeeder. Il est inconscient, tenant un blaster dans une main et un papier dans l’autre. Le document tombe au sol : c’est une affiche de recherche… avec VOTRE portrait.
\end{enumerate}

\section*{91 --- 101 : Force}
\begin{enumerate}
	\setcounter{enumi}{90}
	\item Vous tombez sur un Twi’lek mort à l’entrée d’une ruelle. Selon vos estimations, il est mort récemment… sans doute tué par un sabre laser.
	\item Un enfant des rues (quasi-Humain) vous interpelle. Après avoir vérifié que personne ne regarde, il vous montre son trésor : un holocron Jedi. Il est prêt à le vendre… au bon prix.
	\item Un Humain en robe sombre vous aborde et vous demande qui vous êtes. Sans comprendre pourquoi, vous répondez immédiatement. Il vous ordonne ensuite de déposer vos armes et de le suivre. Et étrangement, vous obéissez.
	\item Vous ressentez une perturbation dans la Force… même si vous n’êtes pas sensible à la Force.
	\item Vous entrez dans un bâtiment. Un quasi-Humain s’approche et vous dit que vous n’êtes pas le bienvenu. Il écarte son manteau, révélant un sabre laser à sa ceinture. D’autres silhouettes se tiennent derrière lui, prêtes à intervenir.
	\item Un véhicule perd le contrôle et fonce vers un groupe de passants. L’un d’eux lève les mains : le véhicule s’immobilise en plein vol avant de se poser au sol. L’individu disparaît aussitôt dans une ruelle.
	\item Dans un conteneur de cargaison, vous trouvez une capsule de stase portative. À l’intérieur repose un alien inconnu, figé dans l’hibernation, tenant un sabre laser contre sa poitrine. Les inscriptions extérieures sont incompréhensibles, mais vous déchiffrez une mention : « Ne pas ouvrir. »
	\item Un Tarasin s’avance vers vous. Il vous dit que la Force l’a conduit jusqu’à vous et que vous êtes en grand danger. Des tirs de blaster et des cris éclatent aussitôt derrière un bâtiment. Le Tarasin vous dit : « Nous devons partir. Vite. »
	\item Dans un bar, un Humain ivre vous confie qu’il fut jadis le plus grand Jedi de la galaxie. Pour le prouver, il désigne une bouteille derrière le comptoir. Celle-ci tremble, chute et se brise. Le barman râle en nettoyant, tandis que l’homme vous adresse un clin d’œil avant de boire une gorgée.
	\item Un duel éclate en pleine place publique entre deux Jedi. Leurs sabres laser s’entrechoquent, attirant la foule : certains fuient, d’autres restent, fascinés.
	\item Un Zabrak s’approche et s’arrête à distance. Il sort lentement un sabre laser de sa ceinture et l’active. Dans son autre main, des arcs d’électricité jaillissent.
\end{enumerate}

\end{document}
