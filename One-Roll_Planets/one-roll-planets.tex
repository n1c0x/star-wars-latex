\documentclass{article}
\usepackage[french]{babel}
\usepackage[T1]{fontenc}
\usepackage[left=2cm,right=2cm,top=2cm,bottom=2cm]{geometry}
\usepackage{fancyhdr}
\pagestyle{fancy}
\chead{One-Roll Planets}
\usepackage{multicol}

\usepackage{fontspec}
\defaultfontfeatures{Ligatures=TeX}
\setmainfont[Mapping=tex-text]{Sitka Display}
\usepackage[small,sf,bf]{titlesec}

\usepackage{graphicx}
\usepackage{xcolor}
\usepackage{sectsty}

\definecolor{DarkGreen}{HTML}{384d3e}
\definecolor{PureWhite}{HTML}{FFFFFF}
\definecolor{DarkRed}{HTML}{6e272d}
\definecolor{DarkGold}{HTML}{a48e3b}

\sectionfont{\color{DarkGreen}}
\subsectionfont{\color{DarkRed}}
\subsubsectionfont{\color{DarkGold}}

\begin{document}

\title{\vspace{-0.5cm}{\Huge One-Roll Planets} \vspace{-1cm}}

\date{}

\maketitle

Utilisez ce tableau comme point de départ. Posez des questions aux joueurs pour l'étoffer : Qui, quoi, quand, où, pourquoi. Extrapoler et embellir leurs réponses, s'inspirer de la fiction existante et l'utiliser pour peupler le monde avec des personnes, des objets et des complications qui correspondent à votre contexte. Et si un lancer ne vous va pas, changez-le !


\begin{multicols}{2}
	\section*{D6 -- Population}
	\begin{enumerate}
		\item Monoculture établie
		\item Cosmopolite 
		\item Criminels/Pirates/Rebelles
		\item Tribal/Régressif/Sauvage
		\item Avant-poste - Militaire/Scientifique/Commercial
		\item Colons
	\end{enumerate}
	\section*{D8 -- Types de terrain}
	\begin{enumerate}
		\item Forêt/Jungle
		\item Montagne/Collines
		\item Marécage
		\item Monde Jardin
		\item Désert/Gelé/Toxique - Terrain vague
		\item Grottes souterraines/Mines
		\item Monde aquatique
		\item Géante gazeuse
	\end{enumerate}
	\section*{D10 -- Atouts}
	\begin{enumerate}
		\item Ressource de haute qualité 
		\item Tourisme - culture ou nature uniques
		\item Des formes de vie uniques attirent les chercheurs
		\item Ruines et technologies extraterrestres
		\item Une délicatesse est récoltée ici
		\item Un lieu d'apprentissage réputé
		\item Une atmosphère vivifiante/une météo/une ressource qui ne peut être exportée 
		\item Faune dangereuse pour la chasse
		\item Une position stratégique
		\item Usines de fabrication massives
	\end{enumerate}
	\section*{D4 -- État de droit}
	\begin{enumerate}
		\item Très dangereux
		\item Dangereux
		\item Plutôt sécurisé
		\item Sécurisé
	\end{enumerate}
	\section*{D20 -- Idées d’aventures}
	\begin{enumerate}
		\item Quarantaine
		\item Réfugiés
		\item Guerre civile
		\item Invasion
		\item Tyrannie/ Exploitation
		\item Pénurie
		\item Ruée vers l'or
		\item Catastrophe imminente
		\item Un festival sauvage et exotique
		\item Abandonné
		\item Découverte capitale
		\item Météo étrange
		\item Champ de bataille
		\item Pèlerinage
		\item Faune/flore hostile
		\item Terrorisme
		\item Prise de contrôle d'une entreprise 
		\item Corruption
		\item Une nouvelle religion
		\item C'est si agréable ici. pourquoi partir ? Rester
	\end{enumerate}
\end{multicols}
\clearpage

\section*{D100 -- Nom}
\begin{multicols}{4}
	\begin{enumerate}
		\item Craka V 
		\item New Yellowstone 
		\item New Alexandria 
		\item Fotti Prime 
		\item Astarte 
		\item I'Tedai 
		\item Chi-You 
		\item Phoebe 
		\item Ch'Deni 
		\item Kazi
		\item Hezitis 
		\item Giveria 
		\item Cholion 
		\item Nulrade 
		\item Duwei 
		\item Leanus 
		\item Dorscind's World 
		\item Goiturn 
		\item Bryke 
		\item 1A4 Vinda RO
		\item Dyton
		\item Sihnon
		\item Higgins' Moon
		\item Ariel
		\item Londinium
		\item Liann Jiun
		\item Santo
		\item Triumph
		\item Three Hills
		\item Hera
		\item Lambda V
		\item Morloo IV
		\item Omid IV
		\item Gule IV
		\item Horki II
		\item Arcturus VI
		\item Brako VI
		\item Mu Arae VI
		\item Menkalinan VII
		\item Poxu II
		\item Hrane
		\item Siono
		\item Kote
		\item Gerte
		\item Yedin
		\item Palmary
		\item Zathru
		\item Axus
		\item Calfuu
		\item Kidia	
		\item Lungor
		\item Munei
		\item Ekak
		\item Otaw
		\item Olok
		\item Anein
		\item Lonei
		\item Tsunei
		\item Eytaw
		\item Malu
		\item Sihnon
		\item New Melbourne
		\item Bernadette
		\item New Canaan
		\item Lazarus
		\item Parth
		\item Paquin
		\item St. Albans
		\item Iota Felis III
		\item Maenali VI
		\item Cassiopeiae IV
		\item Mega Cerberi
		\item Zorgi III
		\item Regulus Prime
		\item Pegasi III
		\item Pleione IV
		\item Theta Carinae
		\item Sagittae VI
		\item Gana
		\item Nara
		\item Beyscrim
		\item Bora
		\item Anosh
		\item Aros
		\item Myto
		\item Parmea
		\item Ablis
		\item Tala
		\item Chostrastea
		\item Sevozuno
		\item Kallilia
		\item Imiq
		\item Roabos
		\item Euwei
		\item Lluxetis
		\item Vaipra
		\item Comia UT5
		\item Thonoe 142
		\item Mercedes Creon
		\item Thalida
	\end{enumerate}
\end{multicols}

\section*{Exemple : Poxu II(40)}
\begin{description}
	\item [État de droit] 2 -- Dangereux
	\item [Population] N.C.
	\item [Type de terrain] 1 -- Forêt/Jungle  
	\item [Culture] 8 -- Zélotes religieux 
	\item [Atouts] 8 -- Faune dangereuse pour la chasse
	\item [Idées d’aventure] 8 -- Catastrophe imminente
\end{description}

Le mandat théocratique de Poxian a chassé le grand Upox à travers les forêts vierges de Poxu II pendant des millénaires. Leur économie est soutenue par la vente de permis de chasse aux milliardaires de toute la galaxie. Ils se méfient des braconniers et ne permettent le débarquement d'aucun vaisseau qui n'a pas acheté de permis Upox. Récemment, la surchasse de l’Upox a conduit à une pénurie dramatique des créatures et toute l'économie de Poxian vacille. La colère est courte et les doigts sont près de la gâchette. Il faudra être très habile pour réussir un atterrissage sur Poxu II et en sortir indemne.

\end{document}
