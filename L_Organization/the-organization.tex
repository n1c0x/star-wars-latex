\documentclass{article}
\usepackage[french]{babel}
\usepackage[T1]{fontenc}
\usepackage[left=2cm,right=2cm,top=2cm,bottom=2cm]{geometry}
\usepackage{fancyhdr}
\pagestyle{fancy}
\chead{L'Organisation -- Fonder son propre syndicat criminel}
\usepackage{multicol}
\usepackage{enumitem}

\usepackage{fontspec}
\defaultfontfeatures{Ligatures=TeX}
%\setmainfont[Mapping=tex-text]{Sitka Display}
\usepackage[small,sf,bf]{titlesec}

\usepackage{graphicx}
\usepackage{xcolor}
\usepackage{sectsty}

\definecolor{DarkGreen}{HTML}{384d3e}
\definecolor{PureWhite}{HTML}{FFFFFF}
\definecolor{DarkRed}{HTML}{6e272d}
\definecolor{DarkGold}{HTML}{a48e3b}

\sectionfont{\color{DarkGreen}}
\subsectionfont{\color{DarkRed}}
\subsubsectionfont{\color{DarkGold}}


\begin{document}

\title{\vspace{-0.5cm}{\Huge L'Organisation -- Fonder son propre syndicat criminel} \vspace{-1cm}}

\date{}

\maketitle

\section*{L'Organisation}
L'Organisation est le parapluie sous lequel les différentes équipes opèrent et est entièrement dirigée par les joueurs. Ils constituent le Cercle Intérieur de l'Organisation, et tout ce qui profite directement à l'Organisation peut être considéré comme leur profit. Les équipes travaillent pour l'Organisation, et l'Organisation est à leurs côtés.

\section*{Équipes}
Au fur et à mesure que l'Organisation gagne en notoriété, les joueurs peuvent gagner, recruter ou simplement embaucher des équipes qui travailleront en dessous d'eux, leur donnant une part de leurs business en échange d'une protection contre les concurrents et l'accès à leurs ressources. Chaque équipe est dirigée par un lieutenant et possède une compétence de base et, éventuellement, une compétence secondaire, ce qui en rendra certaines mieux adaptées à différents types de travail que d'autres. Les quatre compétences sont : Social, Technique, Combat et Pilotage. Chaque équipe a également une note de fidélité qui représente sa volonté de continuer à travailler pour l'Organisation.

\section*{Lieutenants}
Les équipes sont dirigées par des lieutenants, des PNJ mémorables auxquels les joueurs ont confié un navire et un équipage, qui déterminent les compétences essentielles de l'équipe qu'ils dirigent.  Leur équipe commence avec une "caractéristique" de base de 3 et 1 rang dans la compétence associée. Chaque membre supplémentaire de l'équipe qui possède la même compétence améliore une fois le classement de l'équipe dans cette compétence, jusqu'à un maximum de cinq. Lorsqu'un membre de l'équipe recrutée possède une compétence différente, il peut soit être élevé au rang de lieutenant d'une nouvelle équipe, soit se joindre à une équipe existante pour créer sa compétence secondaire et permettre aux membres possédant l'une ou l'autre compétence de rejoindre l'équipe.\\

En plus de diriger des équipes, les lieutenants apportent habituellement une compétence unique à la table qui peut être activée une fois par scénario, et un emploi disponible de façon permanente qu'ils peuvent occuper. Dans certains cas, leurs compétences peuvent être améliorées, mais c'est au GM de décider précisément comment et à quel coût. Par exemple, Aeon Kelrian est un pilote avec un penchant pour le Grand Theft Speeder et un don étrange pour être au bon endroit au bon moment. Ainsi, une fois par scénario, un Point de Destin peut être retourné par les joueurs pour avoir demandé à Aeon de planquer un speeder Silhouette 2 près de leur emplacement qui cessera de fonctionner après la scène actuelle. En lui fournissant un garage ou en la faisant entrer dans les bonnes grâces d'un atelier local, cette habileté pourrait être améliorée pour permettre au véhicule de persister pendant une scène supplémentaire, d'être jusqu'à la Silhouette 3, ou même d'être un aérospeeder.\\

Le travail qu'ils fournissent n'est généralement pas particulièrement difficile et, en tant que tel, il n'est pas très bien rémunéré, mais il est toujours à leur disposition. Ce travail peut être amélioré pendant le jeu narratif, tout comme la compétence du lieutenant, ce qui le rend plus difficile, mais permet en même temps un meilleur paiement.

\section*{Loyauté}
La Valeur de Loyauté d'une équipe est initialement fixée à 2, mais peut être augmentée jusqu'à 5 ou réduite à 0 par des avantages ou des menaces, ou par un jeu narratif. Si les joueurs consacrent du temps et de l'énergie à aider un membre de l'équipe avec leurs problèmes personnels, ils peuvent être récompensés par une augmentation de la loyauté de cette équipe. De même, si les joueurs maltraitent ou négligent leur équipe, la loyauté de cette équipe peut diminuer en conséquence. La loyauté détermine en premier lieu le montant de la rémunération que l'Organisation perçoit à l'achèvement d'un travail. Le pourcentage des suppressions d'emplois de l'Organisation augmentera de 10\% pour chaque rang dans la cote de loyauté de l'équipe, jusqu'à un maximum de 50\%. Par exemple, la part de l'Organisation d'une équipe ayant un coefficient de loyauté de 2 représenterait 20\% de leur prise totale. Les crédits non payés aux joueurs sont utilisés par l'équipe pour leurs propres besoins.\\

Sous le stress de l'échec et des obligations croissantes, la loyauté d'une équipe peut, en de rares occasions, être mise à l'épreuve. Dans ces cas, l'équipe doit effectuer un contrôle de loyauté moyen. Si l'équipe échoue à ce test, ou si son taux de loyauté tombe à 0, elle se dissout immédiatement ou devient malhonnête, emportant avec elle tout actif sous son contrôle direct. Prenez soin des vôtres, ou ils pourraient se retourner contre vous\ldots

\section*{Réputation }
Les joueurs, à quelques rares exceptions près, font la réputation de l'Organisation. Les emplois qu'ils occupent personnellement auront un impact direct sur la position de leur organisation dans tous les milieux. Prendre des chemins honnêtes pendant le jeu, travailler en étroite collaboration avec les forces de l'ordre et lutter contre l'élément criminel d'une manière relativement honnête augmentera leur cote de légitimité, ajoutera un dé de fortune à tout emploi honnête pris par leurs équipes et augmentera une fois la difficulté des contrôles pour les emplois illégaux. De même, le crime et la méchanceté augmenteront leur cote d'hors-la-loi, ajoutant un dé de fortune aux emplois illégaux et augmentant une fois la difficulté des emplois honnêtes. Il s'agit de deux voies distinctes, qui peuvent être influencées positivement ou négativement indépendamment l'une de l'autre. Il est donc possible d'avoir à la fois une difficulté accrue et une amélioration pour tous les emplois s'ils ne s'installent pas dans un style ou l'autre.\\

Outre leur réputation générale, les emplois qu'occupent les équipes peuvent affecter la réputation de l'Organisation auprès d'autres factions. Si une équipe travaille sur un travail qui cible une organisation différente, sa réputation auprès de cette faction se détériorera de façon appropriée (comme le détermine le GM), que le travail réussisse ou échoue.

\section*{Travail hors-la-loi}
Chaque travail hors-la-loi génère une chaleur égale au total des dés de difficulté, de défi et d’échec du travail, ce qui représente l'attention négative qu'ils ont obtenue des forces de l’ordre. La chaleur se détériore naturellement de 1 par arc narratif, tant qu'il n'y a pas de chaleur supplémentaire. A 5 Chaleurs, l'équipe perd toutes les Chaleurs accumulées et obtient la cote Recherché. Cela peut se produire plusieurs fois, jusqu'à Recherché 5, et toute chaleur supplémentaire provenant d'un travail qui les rend Recherchés déborde. Les équipages recherchés améliorent la difficulté de tous les emplois une seule fois par cote Recherché, quel qu'en soit le type, et ne peuvent perdre leur statut Recherché que par les joueurs qui paient leur prime, soit 2 000 crédits par grade de Recherché, ou par temps de service. Tout équipage qui purge sa peine ne peut pas faire un travail pour un Arc égal à son niveau Recherché, et plusieurs peines purgées peuvent (à la discrétion du MJ) menacer la Loyauté d'une équipe.

\section*{Missions}
Au début d'un arc narratif, les joueurs peuvent envoyer n'importe quelle équipe disponible pour réaliser une mission. Il peut s'agir du suivi de pistes et d’indices qui n’intéressent pas les joueurs, ou à envoyer des équipes pour trouver leur propre travail dans la galaxie, représenté par le deck de carte de mission.
Chaque carte indique le nom de l'emploi, sa cible, sa légalité ou non, la difficulté, les compétences requises, une brève description narrative de ce qui est en jeu, et la récompense. Certains emplois exigent des compétences particulières ; si une compétence particulière est indiquée, l'équipe doit l'avoir comme compétence principale ou secondaire afin d'occuper cet emploi, et elle utilisera cette valeur de compétence pour calculer le nombre de dés. Si plusieurs compétences sont énumérées, l'une ou l'autre peut être utilisée pour accomplir le travail. La récompense est déterminée par la difficulté du travail et son danger en utilisant la formule suivante :
\[
(1 000 \times \mathit{nb\_dés\_difficulté}) + (2 000 \times \mathit{nb\_dés\_défi}) + (1 000 \times \mathit{nb\_compétences\_exigées}) + (500 \times \mathit{chaleur})
\]

Au début d'un arc narratif, vous pouvez tirer 1 carte par joueur de la pioche en plus de celles qu'ils ont gagnées avec les avantages des emplois précédents ou pendant le jeu. Il s'agit là de rumeurs et d'opportunités qui promettent d'aboutir à un travail rémunéré. Chaque équipe est affectée à un poste et se voit attribuer un atout (voir ci-dessous), ou obtient une permission à terre. Les équipes en permission à terre peuvent changer librement leur composition, et les lieutenants en permission à terre peuvent faire un contrôle de fidélité facile pour réduire la chaleur accumulée de 1 par succès non annulé. Les avantages et les menaces peuvent être dépensés comme d'habitude.\\

A l'issue de l'Arc, chaque équipe fera un jet d'adresse pour déterminer son succès et ses conséquences éventuelles. En cas de succès, tout symbole de succès non annulé (moins 1) augmente la prise totale de l'emploi de 10\%, et les avantages et les menaces peuvent être dépensés normalement, ou en puisant dans la liste ci-dessous. En cas d’échec, l'équipage est temporairement mis hors service après avoir manqué à sa mission. Tout symbole de menace non annulé (moins 1) nécessitera un coût de 10\% de la récompense initiale du travail pour remettre cette équipe en service (ce qui représente la nécessité de payer les réparations, l'entretien et le salaire des employés embauchés). Par exemple, sur un contrôle réussi avec un total de 3 symboles de réussite \includegraphics[height=\fontcharht\font`\B]{../_img/result_succes_success}, le job paiera 20\% de plus. De même, si la vérification échouait avec 3 symboles d'échec \includegraphics[height=\fontcharht\font`\B]{../_img/result_echec_failure}, l'équipe exigerait le paiement de 20\% de la prise initiale du travail pour retourner à la pleine fonction.
Alternativement, si votre campagne est riche en argent et pauvre en objets, vous pouvez remplacer les récompenses en argent par du matériel, des épices, des accessoires, des informations précieuses ou même des petits véhicules. Dans ces cas, le paiement n'est pas modifié par d'autres succès, et l'Organisation bénéficie de la totalité de la récompense qu'elle peut utiliser comme bon lui semble.

\section*{Destin}
Les équipes ne sont pas les héros de cette histoire et n'ont donc pas de destin à accomplir. Les points de Destin ne peuvent pas être utilisés par l'un ou l'autre camp pour affecter le résultat d'une mission.

\section*{Aide}
Les joueurs peuvent fournir de l’aide aux équipes pour améliorer leurs chances de succès, soit sous forme d'aide matérielle, soit sous forme de crédits froids et durs. L'aide matérielle appropriée au travail à faire équivaut à un dé de fortune - une caisse pleine de blasters n'est pas d'une grande utilité dans un travail social, mais peut être exactement ce dont on a besoin dans un travail de combat. De plus, les joueurs peuvent dépenser 500 crédits pour donner à l'équipe un dé de fortune \includegraphics[height=\fontcharht\font`\B]{../_img/dice_blue}, jusqu'à 3 dés pour 1 500 crédits.

\section*{Exemple}
\begin{itemize}
	\item Profil d'organisation : Le Fang
	\item Cercle intérieur : S'tuun, Terras, Qevut et Nagnuhx
	\item Légalité : 1, Illégalité : 1
	\item Profil d'équipe : Sarlacc 1 (Cracheur de feu)
	\item Lieutenant : Daro Blunt
	\item Spécial : A l'écoute du sol. Daro peut tirer 2 cartes supplémentaires lorsqu'il envisage un emploi.
	\item Emploi : Social du contrebandier : Illégal, Difficulté : \includegraphics[height=\fontcharht\font`\B]{../_img/dice_purple} \includegraphics[height=\fontcharht\font`\B]{../_img/dice_purple}, Social, 3 000 CR.
	\item Loyauté : 3
	\item Chaleur : 3
	\item Recherché : 0
	\item Compétences de base : Social (\includegraphics[height=\fontcharht\font`\B]{../_img/dice_yellow} \includegraphics[height=\fontcharht\font`\B]{../_img/dice_yellow} \includegraphics[height=\fontcharht\font`\B]{../_img/dice_yellow} : Daro, JB-L9, V3-PO)
	\item Compétence secondaire : Combat (\includegraphics[height=\fontcharht\font`\B]{../_img/dice_green} \includegraphics[height=\fontcharht\font`\B]{../_img/dice_green} - R4-W9)
\end{itemize}

\subsection*{Carte de job tirée : Des sujets pour le bon docteur}
\textit{Un chirurgien esthétique renommé cherche à se lancer dans la neurochirurgie et a besoin de sujets d'essai pour son approche expérimentale. Vous trouverez sûrement des "bénévoles"\ldots}

\subsection*{Modificateurs} Le Fang donne à Daro de faux identifiants pour le faire passer pour un recruteur de bonne réputation, lui donnant un dé de Fortune, et 500 crédits. De plus, lors de son dernier emploi, il a gagné un dé de Fortune grâce aux avantages dépensés. Enfin, il recevra un dé de fortune grâce à la réputation des Fang pour le travail illégale, mais il devra aussi améliorer la difficulté une fois à cause de l’aspect légitime de la mission.

\subsection*{Dés utilisés}
\begin{itemize}
	\item Positifs
	\begin{itemize}
		\item 3 Dés de Ma\^{i}trise (\includegraphics[height=\fontcharht\font`\B]{../_img/dice_yellow} \includegraphics[height=\fontcharht\font`\B]{../_img/dice_yellow} \includegraphics[height=\fontcharht\font`\B]{../_img/dice_yellow})
		\item 5 Dés de Fortune (\includegraphics[height=\fontcharht\font`\B]{../_img/dice_blue} \includegraphics[height=\fontcharht\font`\B]{../_img/dice_blue} \includegraphics[height=\fontcharht\font`\B]{../_img/dice_blue} \includegraphics[height=\fontcharht\font`\B]{../_img/dice_blue} \includegraphics[height=\fontcharht\font`\B]{../_img/dice_blue})
	\end{itemize}
	\item Négatifs
	\begin{itemize}
		\item 3 Dés de Difficulté (\includegraphics[height=\fontcharht\font`\B]{../_img/dice_purple} \includegraphics[height=\fontcharht\font`\B]{../_img/dice_purple} \includegraphics[height=\fontcharht\font`\B]{../_img/dice_purple})
		\item 1 Dé de Défi (\includegraphics[height=\fontcharht\font`\B]{../_img/dice_red})
	\end{itemize}
\end{itemize}

\subsection*{Résultats}
\begin{itemize}
	\item 2 Succès (\includegraphics[height=\fontcharht\font`\B]{../_img/result_succes_success} \includegraphics[height=\fontcharht\font`\B]{../_img/result_succes_success})
	\item 4 Avantages (\includegraphics[height=\fontcharht\font`\B]{../_img/result_avantage_advantage} \includegraphics[height=\fontcharht\font`\B]{../_img/result_avantage_advantage} \includegraphics[height=\fontcharht\font`\B]{../_img/result_avantage_advantage} \includegraphics[height=\fontcharht\font`\B]{../_img/result_avantage_advantage})
\end{itemize}

Le mystérieux médecin obtient ses sujets, et pour leur qualité (et leur quantité) il accepte de payer 10\% de plus de l'emploi pour un total de 7 700 crédits. L'un des "sujets" impressionne Daro par ses grâces sociales au point qu'il décide qu'elle vaut plus pour son équipage qu'un autre sujet pour le Docteur (3 Avantages \includegraphics[height=\fontcharht\font`\B]{../_img/result_avantage_advantage} \includegraphics[height=\fontcharht\font`\B]{../_img/result_avantage_advantage} \includegraphics[height=\fontcharht\font`\B]{../_img/result_avantage_advantage}). Elle lui donne également des indices sur une occasion d'affaires qui pourrait l'intéresser plus tard (1 Avantage \includegraphics[height=\fontcharht\font`\B]{../_img/result_avantage_advantage}). Cependant, alors qu'il transporte sa malheureuse cargaison humaine dans un port spatial, son navire est identifié et un mandat d'arrêt est lancé contre lui (+4 chaleur l'amène à Recherché 1, Chaleur 2).
Le Fang gagne 2 310 CR du travail, le Social de l’équipe de Daro s'est amélioré à \includegraphics[height=\fontcharht\font`\B]{../_img/dice_yellow} \includegraphics[height=\fontcharht\font`\B]{../_img/dice_yellow} \includegraphics[height=\fontcharht\font`\B]{../_img/dice_yellow} \includegraphics[height=\fontcharht\font`\B]{../_img/dice_green}, et ils peuvent tirer une carte supplémentaire au début du prochain Arc. Mais Daro est Recherché maintenant, rendant tous les travaux futurs un peu plus dangereux jusqu'à ce qu'il ait le mandat résolu.

\section*{Utilisation des Avantages ou Menaces dans les jets de l’organisation}
Chaque succès supplémentaire dans un travail augmente la prise totale de 10\%.

\subsection*{1 Avantage \includegraphics[height=\fontcharht\font`\B]{../_img/result_avantage_advantage}}
\begin{itemize}
	\item Nous avons entendu une rumeur pendant que nous étions sur le terrain : l'équipe découvre des informations utiles à l'Organisation sur le terrain, en détaillant potentiellement de nouveaux contacts, de nouveaux emplois ou de nouvelles pistes. Ceci peut être utilisé à des fins narratives, ou l'équipe peut tirer une tâche supplémentaire de la pile des tâches.
	\item Sous les scanners : Prenez une chaleur de moins du travail. Peut être sélectionné plusieurs fois (jusqu'à un minimum de 0).
\end{itemize}

\subsection*{2 Avantages \includegraphics[height=\fontcharht\font`\B]{../_img/result_avantage_advantage} \includegraphics[height=\fontcharht\font`\B]{../_img/result_avantage_advantage}}
\begin{itemize}
	\item Je n'aurais pas pu y arriver sans vous : augmentez votre part de 10\% dans ce travail.
	\item Faire des vagues : Ajoutez un dé de fortune (bleu) au prochain travail de cette équipe.
\end{itemize}

\subsection*{3 Avantages \includegraphics[height=\fontcharht\font`\B]{../_img/result_avantage_advantage} \includegraphics[height=\fontcharht\font`\B]{../_img/result_avantage_advantage} \includegraphics[height=\fontcharht\font`\B]{../_img/result_avantage_advantage}}
\begin{itemize}
	\item Un informateur dans la poche : Vous avez obtenu une faveur d'un sympathique marchand d'information qui vous propose des pistes sur le travail le mieux adapté à vos compétences. Tirez le double du nombre de cartes au début de votre prochain arc.
	\item Donner un appel aux RH : L'équipe a recruté un nouveau membre, en améliorant soit ses compétences de base, soit ses compétences secondaires une fois de manière permanente. S'il n'y a pas de compétence secondaire, le nouveau membre le déterminera.
\end{itemize}

\subsection*{Triomphe \includegraphics[height=\fontcharht\font`\B]{../_img/result_triomphe_triumph}}
\begin{itemize}
	\item Mon genre de travail : Ajouter un dé bleu à la compétence principale ou secondaire de l'équipe pour tous les lancers futurs pour ce genre d'emploi.
	\item Nous sommes avec vous, Patron : Augmentez la Fidélité de cette équipe de 1.
\end{itemize}
\subsection*{Double triomphe \includegraphics[height=\fontcharht\font`\B]{../_img/result_triomphe_triumph} \includegraphics[height=\fontcharht\font`\B]{../_img/result_triomphe_triumph}}
\begin{itemize}
	\item Avantage inattendu : Doublez la récompense de cet emploi.
	\item Vous connaissez notre nom : La réputation de l'Organisation pour ce genre de travail augmente de 1.
\end{itemize}

\subsection*{1 Menace \includegraphics[height=\fontcharht\font`\B]{../_img/result_menace_threat}}
\begin{itemize}
	\item Tu crois qu'ils nous ont vus ? : C'est ce qu'ils ont fait. Ajoutez un dé d’Infortune (\includegraphics[height=\fontcharht\font`\B]{../_img/dice_black}) au prochain travail de l'équipe.
	\item Enchevêtrements impériaux : L'équipe devient négligente et attire l’attention. Prendre 1 chaleur supplémentaire pour le travail. Cette option peut être sélectionnée plusieurs fois.
\end{itemize}

\subsection*{2 Menaces \includegraphics[height=\fontcharht\font`\B]{../_img/result_menace_threat} \includegraphics[height=\fontcharht\font`\B]{../_img/result_menace_threat}}
\begin{itemize}
	\item Même moi je me fais aborder : L'équipage a dû larguer une partie de sa cargaison. Réduisez votre récompense de 10\% par rapport à ce travail (ceci peut être sélectionné plusieurs fois et peut réduire votre récompense  à 0), OU perdez une aide fournie à l'équipe pour ce travail.
	\item Est-ce qu'il a l'air d'un skug ? L'un des autres syndicats a fait la lumière sur l'implication de votre équipe dans le travail et a envoyé des agents de police pour obtenir des excuses. Perdre temporairement un membre d'équipage du prochain emploi pendant qu'il récupère.
\end{itemize}

\subsection*{3 Menaces \includegraphics[height=\fontcharht\font`\B]{../_img/result_menace_threat} \includegraphics[height=\fontcharht\font`\B]{../_img/result_menace_threat} \includegraphics[height=\fontcharht\font`\B]{../_img/result_menace_threat}}
\begin{itemize}
	\item Hors de notre élément : Ajoutez un dé d’Infortune (\includegraphics[height=\fontcharht\font`\B]{../_img/dice_black}) à tous les jets futurs sur ce genre de travail.
	\item Où étiez-vous, Kriff ? : Le taux de fidélité de l'équipe diminue de 1.
\end{itemize}

\subsection*{Désastre \includegraphics[height=\fontcharht\font`\B]{../_img/result_desastre_despair}}
\begin{itemize}
	\item Karabast ! Nous sommes dans une situation délicate : L'équipe est en danger et a besoin d'une opération de sauvetage immédiate (et difficile) sous peine de se perdre.
	\item Ils sont venus par derrière : Un membre de l'équipe est définitivement perdu.
\end{itemize}

\subsection*{Double Désastre \includegraphics[height=\fontcharht\font`\B]{../_img/result_desastre_despair} \includegraphics[height=\fontcharht\font`\B]{../_img/result_desastre_despair}}
\begin{itemize}
	\item L'argent n'était pas suffisant : L'équipe doit immédiatement faire un contrôle de loyauté. Si la vérification est un échec, l'équipe abandonne votre organisation et prend avec elle tous les biens actuellement sous son contrôle.
	\item Le percepteur des impôts arrive : L'Empire envoie un chasseur de primes pour collecter les impôts qu'il a éludés. L'Organisation obtient une obligation de prime de groupe de 10.
\end{itemize}

\end{document}
