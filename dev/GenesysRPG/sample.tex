\documentclass{book}

\usepackage[background]{Genesys}
\geometry{paperheight=13in}
\begin{document}

\chapter{Sample Document}

This is a sample document for the \emph{Genesys} \LaTeX\ package. Please see below for the various commands.

\section{Dice}

All dice types and symbols have their own commands:

\begin{multicols}{2}

\begin{itemize}[noitemsep,nolistsep]
\item \verb|\BoostDie| produces \BoostDie
\item \verb|\AbilityDie| produces \AbilityDie
\item \verb|\ProficiencyDie| produces \ProficiencyDie
\item \verb|\SetbackDie| produces \SetbackDie
\item \verb|\DifficultyDie| produces \DifficultyDie
\item \verb|\ChallengeDie| produces \ChallengeDie
\item \verb|\Advantage| produces \Advantage
\item \verb|\Success| produces \Success
\item \verb|\Triumph| produces \Triumph
\item \verb|\Threat| produces \Threat
\item \verb|\Failure| produces \Failure
\item \verb|\Despair| produces \Despair
\end{itemize}

\end{multicols}

\section{Tables}

Tables are easy to use with the \verb|GenesysTable| environment. If you're using the \verb|\begin{table}| command to add a \verb|\caption{}| to the table that you add the \verb|[H]| optional argument or else the table will float to the nearest open space (the \verb|\begin{table}[H]| tells \LaTeX\ to put the table \textbf{right here}).

\begin{table}[H]
\caption{Sample Table}
\begin{GenesysTable}{l X}
Heading & Long Heading\\
Table line one & with the second column in blue!\\
And here's & the second line, with white background!\\
Last line & again in blue\\
\end{GenesysTable}
\end{table}

\section{Characters}

When you are making stat blocks for NPCs, be sure to use the \verb|\Characteristics| command which takes 6 arguments, once for each characteristic. \verb|\Characteristics{1}{3}{2}{2}{2}{2}| grants:

\vspace{1em}

\Characteristics{1}{3}{2}{2}{2}{2}

\vspace{1em}


Lastly, we have the derived numbers: soak, WT and ST. Use the \verb|\Derived| command, with two arguments?one for the title and the second for the number. For Melee/Ranged defense, we use \verb|\DerivedSplit| with 5 arguments: title, first number, second number, first subtitle and second subtitle. Using \verb|\Derived{Soak}{4}| and \verb|\DerivedSplit{Defense}{2}{0}{Melee}{Ranged}|, for instance, gives us:

\vspace{1em}

\hspace*{\fill}\Derived{Soak}{4}\qquad\DerivedSplit{Defense}{2}{0}{Melee}{Ranged}\hspace*{\fill}

\section{Talents}

There is now a \verb|\Talent| command that takes 4 arguments. \verb|\Talent{talent name}{tier}{activation}{ranked?}|. 

\verb|\Talent{Grit}{1}{Passive}{Yes}| would give you: 

\Talent{Grit}{1}{Passive}{Yes}



\end{document}